По теме выпускной квалификационной работы были подготовлены следующие публикации:
\begin{enumerate}
\item Золотин А.А., Шляк А.В., Тулупьев А.Л. Пропагация виртуального стохастического свидетельства в алгебраических байесовских сетях: алгоритмы и уравнения // Труды VII Всероссийской Научно-практической Конференции Нечеткие Системы, Мягкие Вычисления и Интеллектуальные Технологии (НСМВИТ-2017). Т. 2. С. 96--107.
\item  Шляк А.В., Золотин А.А. Тулупьев А.Л. Пропагация виртуального свидетельства в алгебраических байесовских сетях: алгоритмы и их особенности // Всероссийская научная конференция по проблемам информатики (СПИСОК-2017). (Санкт-Петербург, 25-27 апреля 2017 г.). Санкт-Петербург: СПбГУ, 2017. С.450--457.
\item Шляк А.В. Задачи матрично-векторной формализации
обработки виртуального свидетельства в алгебраической
байесовской сети // Школа-семинар по искусственному интеллекту: сборник научных трудов. Тверь: ТвГТУ, 2018. C. 76--82.
\end{enumerate}
Также комплекс программ был зарегистрирован в Роспатент:
\begin{enumerate}
\item Шляк А. В., Золотин А.А., Тулупьев А.Л. Algebraic Bayesian Net\-work Virtual Evidence Propagators, Version 01 for CSharp (AlgBN VE Propagators cs.v.01). (Свидетельство). Свид. о гос. регистрации программы для ЭВМ. Рег. № 2018612634(21.02.2018). Роспатент.
\end{enumerate}