\documentclass[14pt]{matmex-diploma-custom}

\usepackage{tocvsec2}
\usepackage{amssymb}
\usepackage{listings}
\usepackage{xcolor}
\usepackage[font=small,labelsep=period]{caption}
%%% \usepackage{concrete}
% Calligraphy letters
% use: \mathcal
\usepackage[mathscr]{eucal}
% Algorithms
% use: \begin{algorithm}
\usepackage{algorithm}
% use: \begin{algorithmic}
\usepackage[noend]{algpseudocode}
% Maths
% use: for correct control sequence
\usepackage{amsmath}
% Centreing
% use: \centering
\usepackage{varwidth}
% First indent
% use: automatically
\usepackage{indentfirst}

\definecolor{bluekeywords}{rgb}{0,0,1}
\definecolor{greencomments}{rgb}{0,0.5,0}
\definecolor{redstrings}{rgb}{0.64,0.08,0.08}
\definecolor{xmlcomments}{rgb}{0.5,0.5,0.5}
\definecolor{types}{rgb}{0.17,0.57,0.68}

\lstset{
    inputencoding=utf8x, 
    extendedchars=false, 
    keepspaces = true,
    language=[Sharp]C,
    captionpos=b,
   frame=lines, % Oberhalb und unterhalb des Listings ist eine Linie
    showspaces=false,
    showtabs=false,
    breaklines=true,
    showstringspaces=false,
    breakatwhitespace=true,
    basicstyle=\linespread{0.8}
    escapeinside={(*@}{@*)},
    commentstyle=\color{greencomments},
    morekeywords={partial, var, value, get, set},
    keywordstyle=\color{bluekeywords},
    stringstyle=\color{redstrings},
    basicstyle=\small\ttfamily,
}


\renewcommand{\lstlistingname}{Листинг}

\newtheorem{Th}{Теорема}[section]
\newtheorem{Def}{Определение}[section]
\newtheorem{Lem}{Утверждение}[subsection]

\newcommand{\underdot}[1]{\mathop{#1}\limits_{\cdot}}

\newenvironment{Proof} % имя окружения
{\par\noindent{\bf Доказательство.}} % команды для \begin
{\hfill$\scriptstyle\blacksquare$} % команды для \end



\newcommand{\Pc}{\mathbf{P}_\mathrm{c}}
\newcommand{\Pq}{\mathbf{P}_\mathrm{q}}
\newcommand{\In}{\mathbf{I}_n}
\newcommand{\Jn}{\mathbf{J}_n}
\newcommand{\Qmatr}{\mathbf{Q}}
\newcommand{\V}{\mathbf{V}}
\renewcommand{\G}{\mathbf{G}}
\renewcommand{\T}{\mathbf{T}}
\newcommand{\evidenceNumbers}{\langle i, j \rangle}
\newcommand{\Tij}{\T^{\evidenceNumbers}}
\newcommand{\TijTilda}{\widetilde{\T}^{\evidenceNumbers}}
\newcommand{\QijTild}{\widetilde{\Qmatr}^{\evidenceNumbers}}
\newcommand{\VijTilda}{\widetilde{\V}^{\evidenceNumbers}}
\newcommand{\rij}{\mathbf{r}^{\evidenceNumbers}}
\newcommand{\rijTilda}{\widetilde{\mathbf{r}}^{\evidenceNumbers}}
\newcommand{\ev}{\mathrm{ev}}
\newcommand{\Gind}{\mathrm{GInd}}
\newcommand{\Selector}{\mathbf{s}}
\newcommand{\sij}{\mathbf{s}^{\evidenceNumbers}}
\newcommand{\sijTilda}{\widetilde{\mathbf{s}}^{\evidenceNumbers}}
\newcommand{\Pd}{\mathbf{P}_\mathrm{d}}
\newcommand{\Ln}{\mathbf{L}_n}
\newcommand{\On}{\mathbf{F}_n}
\newcommand{\Kn}{\mathbf{K}_n}
\renewcommand{\M}{\mathbf{M}}
\newcommand{\Mij}{\M^{\evidenceNumbers}}
\newcommand{\dij}{\mathbf{d}^{\evidenceNumbers}}
\newcommand{\dijTilda}{\widetilde{\mathbf{d}}^{\evidenceNumbers}}
\newcommand{\MijTilda}{\widetilde{\mathbf{M}}^{\evidenceNumbers}}
\newtheorem{theorem}{Теорема}[section]
\newtheorem{definition}[theorem]{Определение}
 \setcounter{tocdepth}{2}
 \usepackage{lastpage}
 \usepackage{mathtools}
\begin{document}
% Год, город, название университета и факультета предопределены,
% но можно и поменять.
% Если англоязычная титульная страница не нужна, то ее можно просто удалить.
\filltitle{ru}{
    chair              = {Фундаментальная информатика и информационные технологии\\ Информационные технологии},
    title              = {Алгебраические байесовские сети:\\ синтез глобальных структур и алгоритмы логико-вероятностного вывода (проектная работа)},
    type               = {bachelor},
    position           = {студента},
    group              = 14.Б08-мм,
    author             = {Анна Викторовна Шляк},
    supervisorPosition = {проф. каф. инф., д. ф.-м. н., доц.},
    supervisor         = {Тулупьев А.\,Л.},
    reviewerPosition   = {проф. каф. инф., д. ф.-м. н., доц. },
    reviewer           = {Фильченков А. А.},
    chairHeadPosition  = {?},
    chairHead          = {?},
    faculty            = {Математико-механический факультет},
    city               = {Санкт-Петербург},
    year               = {2018}
}
\filltitle{en}{
    chair              = {Fundamental informatics and information technologies \\ Information technologies},
    title              = {Algebraic Bayesian networks: \\global structure synthesis and probabilistic-logic inference algorithms(project work)},
        type               = {bachelor},
    author             = {Anna Shliak},
    supervisorPosition = {Prof. Computer Science Department, Dc. Sc. in Math, Assoc. Prof.},
    supervisor         = {Alexander Tulupyev},
    reviewerPosition   = {As. Prof. Computer Technology Department, Ph. D. Sc. in Math},
    reviewer           = {Andrey Filchenkov},
    chairHeadPosition  = {?},
    chairHead          = {?},
}
\maketitle
\tableofcontents
\section*{Введение}
\underline{Актуальность темы.} В современном мире существует необходимость анализировать и обрабатывать большие объемы информации, однако данных не всегда бывает достаточно и могут появляться различные неопределенности, что значительно осложняет задачу обработки. Подход к обработке знаний с неопределенностью предлагает класс вероятностных графических моделей~(ВГМ), к которым относятся алгебраические байесовские сети~(АБС), байесовские сети доверия~(БСД), марковские сети и многие другие. Зависимости между данными в ВГМ задаются с помощью графов, а степень неопределенности данных характеризуется оценками вероятностей. ВГМ находят широкое применение в различных областях, например, байесовские сети доверия, родственные классу АБС, рассматриваемому в данной работе, используются в медицине~\cite{99}, оценке рисков~\cite{98}, области финансов~\cite{97} и других областях~\cite{96}. 

Алгебраические байесовские сети введены профессором В. И. Городецким~\cite{95, 94} и, благодаря многолетним исследованиям, активно развиваются и также занимают свое место в классе вероятностных графических моделей. АБС состоит из набора случайных элементов, соответствующих высказываниям, которым приписаны оценки вероятностей, и структуры связи между указанными высказываниями, которую представляют в виде графа. 

В АБС используется принцип декомпозиции знаний на фрагменты знаний~(ФЗ). Формирование оценок истинности и обработка данных с неопределенностью в АБС основана на локальном и глобальном логико"=вероятностном выводе~\cite{93}. Новая информация, поступающая во фрагменты знаний, основана на логико-вероятностной модели свидетельств~\cite{109}. Для представления фрагмента знаний в АБС описаны три математические модели фрагмента знаний, построенные над идеалами конъюнктов, идеалами дизъюнктов или множествами пропозиций-квантов~\cite{121}. 
 
 \underline{Степень разработанности темы исследований.} 
 Актуальной задачей является матрично-векторное описание алгоритмов логико"=вероятностного вывода в АБС, потому что матрично"=векторные операции относятся к классическому математическому инструментарию, упрощают программную реализацию и понимание работы алгоритмов, а также предоставляют возможности для дальнейшего исследования математической модели, например, исследование чувствительности уравнений. На сегодняшний день предложены матрично"=векторные уравнения для алгоритмов локального логико"=вероятностного вывода~\cite{91}, однако в них остаются элементы функционального вычисления~(функция $\Gind$). Матрично"=векторный подход для глобального вывода также нуждается в доработке~\cite{70}. Параллельно развитию теории разрабатывается комплекс программ для работы с АБС~\cite{89}, реализующий в себе  алгоритмы ЛВВ, однако, не включающий в себя на данный момент алгоритмы глобального логико"=вероятностного вывода, в частности, в нем отсутствует реализация алгоритма распространения виртуального свидетельства между двумя фрагментами знаний.

Таким образом, \underline{объектом исследования} являются алгебраические байесовские сети,
а \underline{предметом исследования} --- матрично-векторная формализация алгоритма распространения виртуального свидетельства между двумя фрагментами знаний в АБС.

\underline{Целью} данной выпускной квалификационной работы является автоматизация глобального логико-вероятностного вывода в алгебраических
байесовских сетях, а именно алгоритмов распространения виртуального свидетельства между двумя фрагментами знаний. 
Для достижения поставленной цели решаются следующие \underline{задачи}:
\begin{enumerate}
        \item  Развить матрично"=векторную формализацию алгоритмов распространения виртуального свидетельства для различных моделей фрагментов знаний со скалярными оценками;
        \item Предложить матрично"=векторную формализацию для функции $\Gind$;
        \item Осуществить интеграцию и реинженеринг алгоритмов локального логико"=вероятностного вывода для альтернативных моделей ФЗ в рамках комплекса программ;
    \item Реализовать алгоритмы распространения виртуального свидетельства в рамках комплекса программ;
    \item Провести вычислительные эксперименты по распространению виртуальных свидетельств и написать документацию.
\end{enumerate}

\underline{Научная новизна.} 
В данной выпускной квалификационной работе бакалавра предложена матрично-векторная формализация алгоритмов распространения виртуального свидетельства для ФЗ, построенных над идеалами дизъюнктов и наборами пропозиций-квантов, в частности, сформулирована и доказана теорема о матрично-векторном формировании матрицы перехода от вектора вероятностей элементов идеала дизъюнктов к вектору вероятностей элементов виртуального свидетельства, доказана теорема о пропагации виртуального свидетельства между двумя ФЗ, построенными над множествами квантов, введена матрица перехода от вектора вероятностей дизъюнктов к вектору вероятностей квантов. Также получена матрица проекции $\G$, заменяющая функцию $\Gind$, в частности, введены понятия характеристического вектора конъюнкта, дизъюнкта и кванта. 

\underline{Теоретическая и практическая значимость исследования.} Теоретическая значимость работы заключается в создании базы для развития матрично"=векторного подхода в глобальном ЛВВ. Практическая значимость заключается в создании основы для реализации алгоритмов распространения свидетельства во все ФЗ сети, решения задачи визуализации работы с АБС, а также для проведения вычислительных экспериментов.

\underline{Методы исследования.} Для решения поставленных задач в данной области понадобилось изучить методические материалы, поставить проблему, проанализировать ее, спроектировать несколько возможных вариантов решения. Затем были выбраны подходящие средства и технологии программирования, связанные с языком реализации~(C\#), средой разработки~(Rider), сервисом для совместной разработки~(BitBucket). После чего были проведены вычислительные и программные эксперименты с целью обоснования корректности  полученного решения. В качестве методов решения используются теоретические методы~(анализ предыдущих результатов, формализация алгоритмов) и методы объектно"=ориентированного программирования~(ООП).


\underline{На защиту выносятся следующие положения:} 
\begin{enumerate}
\item Матрично"=векторные уравнения распространения виртуального свидетельства между двумя фрагментами знаний для следующих моделей фрагмента знаний: идеал дизъюнктов, множество пропозиций квантов;
\item Матрично"=векторная формализация для функции $\Gind$;
\item Программная реализация алгоритмов распространения виртуального свидетельства для каждой из трех моделей.
\end{enumerate}

\underline{Достоверность полученных результатов.} 
Достоверность предложенных в работе результатов обеспечена корректным применением методов исследования, подтверждена вычислительными экспериментами и примерами работы программы. Полученные результаты не противоречат известным результатам других авторов.

\underline{Апробация результатов.} Результаты исследования докладывались на следующих научных мероприятиях:
\begin{enumerate}
\item Всероссийская научная конференция по проблемам информатики
СПИСОК~(Санкт-Петербург, апрель 2017);
\item VII-Всероссийская научно"=практическая конференция НСМВИТ-2017~(Санкт"=Петербург, июль 2017);
\item Научный семинар лаборатории теоретических и междисциплинарных проблем информатики в СПИИРАН~(Санкт-Петербург, ноябрь 2017).
\end{enumerate}

\underline{Публикации.} Результаты работы вошли в две публикации, была зарегистрирована одна заявка в Роспатенте.

\underline{Сведения о личном вкладе автора.}
Постановка цели и задач, а также выносимые на защиту результаты получены лично автором. 

Личный вклад А.В. Шляк в публикациях с соавторами характеризуется следующим образом: в ~\cite{9} --- краткое описание существующих алгоритмов и примеры вычисления апостериорных вероятностей после пропагации виртуального свидетельства, в ~\cite{70} --- постановка задач формализации глобального ЛВВ.

\underline{Структура и объем работы.} Работа состоит из введения, четырех
глав, заключения, списка используемой литературы и двух приложений. Общий объем работы составляет \pageref{LastPage} страниц. Список используемой литературы содержит 35 источников.

Во введении описана актуальность темы исследования и степень ее разработанности, сформулированы цель и задачи работы и представлены выносимые на защиту результаты.

Первая глава носит обзорный характер и состоит из пяти разделов. В первом разделе описаны общие сведения об АБС. Во втором разделе представлены основные инструменты для работы с АБС. В третьем разделе дан обзор существующих библиотек. В четвертом разделе представлен краткий обзор комплекса программ, в рамках которого осуществлялась разработка, а также приведены и обоснованы  цели и задачи работы. В пятом разделе представлены выводы по данной главе.

Вторая глава описывает основные теоретические результаты, используемые в данной работе. В первом разделе представляются основные элементы теории АБС. Во втором разделе даются основные определения. В третьем разделе рассматриваются виды локального логико"=вероятностного вывода, а в четвертом разделе --- глобальный логико"=вероятностный вывод. В пятом разделе перечислены недостатки имеющихся алгоритмов.

Третья глава содержит основные теоретические результаты, полученные в данной работе. В первом разделе дается краткий обзор полученных результатов. Во втором разделе предложена матрично"=векторная интерпретация функции $\Gind$, а в третьем разделе --- матрично"=векторная формализация матрицы перехода. В четвертом разделе представлены матрично-векторные уравнения распространения виртуального свидетельства для каждой из моделей ФЗ. В пятом разделе описаны выводы по главе.

Четвертая глава описывает программную реализацию алгоритмов из второй и третьей глав. Первый раздел описывает используемые программные технологии. Второй раздел описывает архитектуру комплекса программ. В третьем разделе описаны основные реализованные классы и методы, а в четвертом разделе приведены примеры их использования. В пятом разделе представлены выводы по текущей главе. 

В заключении содержатся итоги работы и дальнейшие перспективы исследования.

В приложении А содержатся вычислительные эксперименты по работе предложенных алгоритмов, а в приложении В --- список публикаций по теме работы.

\small{
Эта работа является частью более широких инициативных проектов, выполняющихся в лаборатории теоретических и междисциплинарных проблем информатики СПИИРАН под руководством А.Л.~Тулупьева; кроме того, разработки были частично поддержаны грантами РФФИ 15-01-09001-a~--- «Комбинированный логико"=вероятностный графический подход к представлению и обработке систем знаний с неопределенностью: алгебраические байесовские сети и родственные модели», 18-01-00626~--- «Методы представления, синтеза оценок истинности и машинного обучения в алгебраических байесовских сетях и родственных моделях знаний с неопределенностью: логико-вероятностный подход и системы графов».}

\normalsize
Работа выполнялась в рамках общей проектной работы вместе с А.А. Золотиным, А.Е. Мальчевской, А.И. Березиным.

\section{Автоматизация алгоритмов вывода в алгебраической байесовской сети}
\subsection{Введение}
Алгебраические байесовские сети относятся к классу вероятностных графических моделей и являются эффективным инструментом для обработки и представления знаний с неопределенностью~\cite{184}.     

 Будем считать, что знания формируют эксперты в определенной предметной области. Эксперты задают оценки вероятностей утверждениям, образующим базу знаний, и характеризуют связи между утверждениями с помощью оценок вероятностей. Из-за этого и возникает неопределенность знаний.

 Представим ситуацию, что экспертам необходимо охарактеризовать несколько утверждений и связи между ними. Пусть характеризуемые утверждения будут атомарными. Так как число взаимосвязей с ростом числа утверждений будет растет экспоненциально,  эффективность работы с этими утверждениями будет быстро падать.
 
 Отличительной особенностью алгебраических байесовских сетей является подход к декомпозиции области знаний на фрагменты знаний~\cite{93}: разобьем множество утверждений на подмножества, называемые фрагментами знаний, и будем характеризовать утверждения в каждом фрагменте знаний отдельно. Это позволит использовать быстрые алгоритмы обработки данных, не предъявляя серьезных требований к вычислительным мощностям. Также этот подход удобен при характеризации высказываний, потому что при ограниченном наборе элементов необходимо охарактеризовать связи лишь между несколькими высказываниями. 
 
 Таким образом, АБС состоит из фрагментов знаний и структуры связей между ними. На данный момент описаны три математические модели фрагмента знаний: идеал конъюнктов, идеал дизъюнктов и множество квантов~\cite{121}. Фрагмент знаний строится над алфавитом, где элементы алфавита соответствуют характеризуемым утверждениям. В данной работе будет рассматриваться вторичная структура АБС~\cite{87}.
 
 
\subsection{Инструменты для работы с алгебраическими байесовскими сетями}
Рассмотрим предоставляемые АБС инструменты для работы с данными, которые содержат неопределенности.

Для того чтобы охарактеризовать утверждение, эксперт может дать оценку вероятности истинности данного утверждения. Оценка может быть скалярной или интервальной. АБС позволяют задавать и обрабатывать как и скалярные, так и интервальные оценки вероятностей~\cite{184}. В данной работе в основном будут рассматриваться скалярные оценки вероятности истинности.

Задачи обработки данных в АБС решаются с помощью аппарата логико-вероятностного вывода~\cite{121}. Логико-вероятностный вывод делится на локальный и глобальный~\cite{51, 93}. Алгоритмы локального вывода работают с отдельным фрагментом знаний, а глобального --- со всей сетью. В данной работе будет кратко рассмотрен локальный ЛВВ и более подробно глобальный вывод. Логико-вероятностный вывод позволяет нам оценить вероятность пропозициональной формулы, основываясь на имеющихся уже оценках в сети, а также позволяет нам получить новые оценки элементов АБС при поступлении новых обуславливающих данных, называемых свидетельствами~\cite{84, 284}.
 
 Также существует аппарат для поддержки и проверки различных видов непротиворечивости в АБС, для того чтобы оценки вероятностей элементов в ФЗ не противоречили друг другу и были согласованы между собой. Подробнее с ним можно познакомиться в ~\cite{184, 121}.
 
\subsection{Библиотеки для работы с алгебраическими байесовскими сетями}
Развитием теории АБС активно занимается научный коллектив ТиМПИ СПИИРАН, в разное время включавший в себя А.Л. Тулупьева, А.В. Сироткина, А.А. Фильченкова и других исследователей. Были опубликованы монографии ~\cite{109, 1, 85}, диссертации~\cite{84, 184, 284}, статьи, посвященные тематике АБС и байесовских сетей в целом~\cite{94}.

Вместе с теорией создавались также библиотеки для работы с АБС, так в 2009 году была создана библиотека AlgBN Modeller j.v.01 на языке Java~\cite{124, 123, 122}. В данной библиотеке есть структуры для хранения фрагментов знаний, машины локального логико-вероятностного вывода и проверки непротиворечивости, а также машины глобального логико-вероятностного вывода. Однако библиотека позволяет осуществить глобальный логико-вероятностный вывод только для АБС, состоящей из ФЗ, построенных над идеалами конъюнктов с заданными оценками вероятностей их элементов.

Позже была создана библиотека AlgBN KPB Reconciler cpp.v.01, разработанная на языке C++~\cite{81}. В этой библиотеке также есть средства для осуществления локального логико"=вероятностного вывода и проверки непротиворечивости, а также структуры для хранения фрагментов знаний, однако алгоритмы глобального логико-вероятностного вывода в ней не реализованы.

Таким образом, в обеих библиотеках отсутствует полноценный функционал для проведения глобального логико-вероятностного вывода в АБС.
\subsection{Цели и задачи исследования}
Так как теория алгебраических байесовских сетей активно развивается, в 2015 году возникла потребность начать разработку комплекса программ AlgBN Math Library~\cite{89, 50}, представляющего собой библиотеку для работы с АБС. Так как алгоритмы ЛВВ приобретают матрично-векторную форму, позволяющую упростить программную реализацию, было принято решение о создании новых библиотек для работы с АБС, использующих современные матрично-векторные алгоритмы. На основе данных библиотек также создается приложение, визуализирующее АБС и позволяющее проводить вывод в сети. Комплекс программ реализован на языке C\# и платформе .Net, являющимися одними из популярнейших современных технологий в разработке программного обеспечения. 

На данный момент в комплексе программ реализованы структуры для хранения фрагментов знаний и АБС, алгоритмы локального логико-вероятностного вывода, а также алгоритмы проверки АБС и ФЗ на непротиворечивость. Реализации алгоритмов глобального логико-вероятностного вывода еще нет.

Что касается теории глобального логико"=вероятностного вывода, существующие алгоритмы обладают рядом недостатков~\cite{70}, к 
которым относится присутствие операций, не относящихся к матрично"=векторным вычислениям, которые усложняют алгоритмы и их программную реализацию.

Таким образом, целью данной выпускной квалификационной работы является автоматизация глобального логико-вероятностного вывода в алгебраических
байесовских сетях, а именно алгоритмов распространения виртуального свидетельства между двумя фрагментами знаний в АБС. 
Для достижения поставленной цели решаются следующие \underline{задачи}:
\begin{enumerate}
        \item  Развить матрично"=векторную формализацию алгоритмов распространения виртуального свидетельства для различных моделей фрагментов знаний с скалярными оценками;
        \item Предложить матрично"=векторную формализацию для функции $\Gind$;
        \item Осуществить интеграцию и реинженеринг алгоритмов локального логико"=вероятностного вывода для альтернативных моделей ФЗ в рамках комплекса программ;
    \item Реализовать алгоритмы распространения виртуального свидетельства в рамках комплекса программ;
\item Провести вычислительные эксперименты по распространению виртуальных свидетельств и написать документацию.
\end{enumerate}
\subsection{Выводы по главе}
В данной главе были рассмотрены основные понятия, используемые в теории алгебраических байесовских сетей и логико"=вероятностном вывода в АБС. Рассмотрены существующие программные реализации, их основные возможности и недостатки. Сформулированы задачи и цель данной выпускной квалификационной работы.

\section{Элементы теории алгебраических байесовских сетей}\
\subsection{Введение}
В данной главе введем основные математические определения и понятия, используемые в теории АБС, и дадим описание операций логико"=вероятностного вывода в АБС. Введем определения фрагмента знаний и алгебраической байесовской сети. Рассмотрим матрично"=векторные уравнения локального апостериорного вывода для каждой из трех моделей и стохастического свидетельства. Также представим имеющийся результат для глобального логико"=вероятностного вывода над идеалами конъюнктов. Более подробное описание введенных понятий, алгоритмов и обоснование формулировок можно найти в ~\cite{ 76, 91, 84,  184, 109, 1, 122, 284, 70}.

\subsection{Математические модели фрагмента знаний и алгебраической байесовской сети}
Зафиксируем алфавит $\mathcal{A} = \{x_i\}^n_{i=0}$ --- конечное множество атомарных пропозициональных формул~(атомов). Над алфавитом определим идеал конъюнктов, идеал дизъюнктов и множество квантов.
\begin{definition}[\cite{1}]
Идеал конъюнктов $C_\mathcal{A} $ --- это множество вида 
\begin{math}
\{x_{i_1} \wedge x_{i_2}\wedge ... \wedge x_{i_k} | \; 0 \leq i_1 < i_2 < ... < i_k \leq  n - 1, 0 \leq k \leq n\}
\end{math}.
\end{definition}

\begin{definition}[\cite{1}]
Квант над алфавитом $\mathcal{A} = \{x_i\}^n_{i=0}$ ---  это конъюнкция, которая для любого атома алфавита содержит либо этот атом, либо его отрицание.
\end{definition}

\begin{definition}[\cite{1}]
Литерал $\tilde{x}_i$ обозначает, что на его месте в формуле может стоять либо $x_i$, либо его отрицание $\bar{x}_i$.
\end{definition}

\begin{definition}[\cite{1}]
	Множество квантов $Q_\mathcal{A} $ --- это множество всех комбинаций вида $\{\tilde{x}_0\tilde{x}_1...\tilde{x}_{n-1}\}$.
\end{definition}

\begin{definition}[\cite{184}]
Идеал дизъюнктов $C_\mathcal{A} $ --- это множество вида 
\begin{math}
\{x_{i_1} \vee x_{i_2}\vee ... \vee x_{i_k} | \; 0 \leq i_1 < i_2 < ... < i_k \leq  n - 1, 0 \leq k \leq n\}
\end{math}.
\end{definition}

Введем правило нумерации на множестве конъюнктов и на множестве квантов.

Каждому кванту $\tilde{x}_0\tilde{x}_1...\tilde{x}_{n-1}$ поставим в соответствие двоичную запись, в которой на $i$-м месте будет стоять $1$, если $i$-тый литерал означен положительно, и $0$ иначе. Аналогично занумеруем конъюнкты: каждому конъюнкту $x_{i_1}x_{i_2}...x_{i_k}$ поставим в соответствие сумму $2^{i_1}+2^{i_2}+...+2^{i_k}$. Тогда, если представить полученную сумму в виде двоичной записи и дополнить лидирующими нулями до $n$ знаков, $i$-тый атом будет входить в конъюнкт тогда и только тогда, когда $i$-тый бит числа равен $1$. Таким образом, получится биективное отображение множества квантов на множество конъюнктов.

Для множества дизъюнктов правило нумерации будет аналогичным правилу для множества конъюнктов, следовательно, существует биективное отображение множества квантов и на множество дизъюнктов.

Теперь введем вектор вероятностей элементов идеала конъюнктов $\Pc$, вектор вероятностей элементов идеала дизъюнктов $\Pd$ и   вектор вероятностей элементов множества квантов $\Pq$: 
\begin{equation*}
\Pc = \begin{pmatrix}
	    p(c_0)  \\ p(c_1) \\ \vdots \\  p(c_{2^n-1})
      \end{pmatrix}, \,
\Pq = \begin{pmatrix}
	    p(q_0)  \\ p(q_1) \\ \vdots \\  p(q_{2^n-1})
      \end{pmatrix}, \,
\Pd = \begin{pmatrix}
	    p(d_0)  \\ p(d_1) \\ \vdots \\  p(d_{2^n-1})
      \end{pmatrix}. 
\end{equation*}

$\Pc$ и $\Pq$ связаны между собой соотношениями $\Pc = \Jn\Pq$ и $\Pq = \In\Pc$, где матрицы перехода $\In$ и $\Jn$ получены с помощью следующих рекуррентных соотношений~\cite{109}:

\begin{equation*}
    \In = \mathbf{I}_1 \otimes ... \otimes \mathbf{I}_{n-1} = \mathbf{I}_1^{[n]},
\end{equation*}
где $\mathbf{I}_1 = \begin{pmatrix*}[r] 1 & -1 \\ 0 & 1 \end{pmatrix*}$.

\begin{equation*}
    \Jn = \mathbf{J}_1 \otimes ... \otimes \mathbf{J}_{n-1} = \mathbf{J}_1^{[n]},
\end{equation*}
где $\mathbf{J}_1 = \begin{pmatrix} 1 & 1 \\ 0 & 1 \end{pmatrix}$, $\otimes$ обозначает кронекерово произведение матриц.

Вектор вероятностей элементов идеала дизъюнктов связан с $\Pc$ и $\Pq$ следующими соотношениями~\cite{76}:

\begin{equation}\label{pdtopq}
\Pq = \Ln(\mathbf{1} -\Pd),
\end{equation} 
где $\Ln = \begin{pmatrix*}[r] 0 & 1 \\ 1 & -1 \end{pmatrix*}^{[n]}$.

\begin{equation}
\Pc = \Kn(\mathbf{1} -\Pd),
\end{equation}
где $\Kn = \begin{pmatrix*}[r] 1 & 0 \\ 1 & -1 \end{pmatrix*}^{[n]}$.

Для удобства будем обозначать $\Pd^{\prime} = \mathbf{1} -\Pd$ и далее работать с вектором $\Pd^{\prime}$.

Дадим определение фрагмента знаний для трех моделей в случае, когда оценки вероятностей элементов скалярные.

\begin{definition}[\cite{1}]
Фрагмент знаний $\mathscr{C}$ над идеалом конъюнктов со скалярными оценками --- это пара вида $(C, \,p)$, где $C$ --- идеал конъюнктов, $p$ --- функция из $C$  в интервал $[0;1]$.
\end{definition}

\begin{definition}[\cite{1}]
	Фрагмент знаний со скалярными оценками непротиворечив тогда и только тогда, когда $\In\Pc \geq \mathbf{0}$.
\end{definition}

\begin{definition}[\cite{121}]
Фрагмент знаний $\mathscr{C}$ над идеалом дизъюнктов со скалярными оценками --- это пара вида $(C, \,p)$, где $C$ --- идеал дизъюнктов, $p$ --- функция из $C$  в интервал $[0;1]$.
\end{definition}

\begin{definition}[\cite{121}]
Фрагмент знаний $\mathscr{C}$ над множеством квантов со скалярными оценками --- это пара вида $(Q, \,p)$, где $Q$ --- множество квантов, лежащее в основе ФЗ, $p$ --- функция из $Q$  в интервал $[0;1]$.
\end{definition}

В качестве функции $p$ можно использовать вероятность истинности пропозиций.
\begin{definition}[\cite{1}]
Алгебраическую байесовскую сеть $\mathcal{N}$ определим как набор фрагментов знаний: $\mathcal{N}^{\circ} = \{\mathscr{C}_i\}_{i=1}^{n}$.
\end{definition}
\subsection{Локальный логико"=вероятностный вывод}
\subsubsection{Задачи локального логико"=вероятностного вывода}
Локальный логико"=вероятностный вывод делится на априорный и апостериорный~\cite{ 184, 121, 109}. Задачей локального априорного вывода является построение оценки истинности пропозициональной формулы, заданной над тем же алфавитом $\mathcal{A}$, что и данный фрагмент знаний $\mathscr{C}$. 
Для описания локального апостериорного вывода введем понятие свидетельства.
\begin{definition}[\cite{184}]
Под свидетельством мы понимаем новые «обуславливающие» данные, которые поступили во фрагмент знаний, и с учетом которых нам требуется пересмотреть все (или некоторые) оценки.  Для обозначения свидетельства будут использоваться угловые скобки --- $\langle ...\rangle$.
\end{definition}

Локальный апостериорный вывод решает 2 задачи: во-первых, оценить вероятности истинности свидетельства при данных оценках вероятности истинности элементов фрагмента знаний, и, во-вторых, оценить условные вероятности истинности элементов фрагмента знаний, предполагая, что свидетельство истинно.
Свидетельства бывают детерминированными, стохастическими и неточными~\cite{184, 109}. В данной работе в основном будут рассматриваться стохастические свидетельства, которые можно трактовать как набор детерминированных свидетельств в сочетании с заданным на них распределением вероятности.
\begin{definition}[\cite{184}]
	Детерминированное свидетельство --- это предположение, что один или несколько атомов получили конкретное означивание. 
\end{definition}

Детерминированное свидетельство обозначим $\langle i,j\rangle$, где $i$ --- индексы положительно означенных атомов, $j$ --- индексы отрицательно означенных атомов.	
\begin{definition}[\cite{184}]
Стохастическое свидетельство --- предположение о том, что над $C^\prime$ --- подыдеале $C$, задан непротиворечивый фрагмент знаний со скалярными оценками,  который определяет вероятности истинности элементов соответствующего подыдеала. Данное свидетельство обозначается $\langle (C^\prime, \Pc) \rangle$.
\end{definition}
\begin{definition}[\cite{184}]
	Неточное свидетельство --- это предположение о том, что над $C^\prime$ --- подыдеале $C$,  задан непротиворечивый фрагмент знаний с интервальными оценками, который определяет вероятности истинности элементов соответствующего подыдеала. Данное свидетельство обозначается $\langle (C^\prime, \Pc^-, \Pc^+) \rangle$.
\end{definition}

\subsubsection{Локальный апостериорный вывод над конъюнктами}

Приведем решение первой и второй задач локального апостериорного вывода, когда  фрагмент знаний $(C, \Pc)$ над идеалом конъюнктов $C$ содержит скалярные оценки и в него поступило стохастическое свидетельство $(C^{\ev}, \Pc^{\ev})$. 

Решение первой задачи~\cite{91}:
\begin{equation} \label{conjStochFirst}
  p(\langle C^{\ev}, \Pc^{\ev} \rangle) =\sum_{i=0}^{2^{n^\prime}-1}  (\mathbf{r}^{\langle \Gind(i,m),\,\Gind(2^{n^\prime}-1-i,m)\rangle},\Pc) \In\Pc^{a}[i],
\end{equation} 
где $\rij = \otimes^{0}_{k=n-1}\rijTilda_k$, \\
\begin{math}
    \rijTilda_k = 
    \begin{cases}
        \mathbf{r}^+ \text{, если $x_k$ входит в $c_i$,}\\
        \mathbf{r}^- \text{, если $x_k$ входит в $c_j$,}\\
        \mathbf{r}^\circ  \text{, иначе;}
    \end{cases}
\end{math}\\
\begin{math}
    \mathbf{r}^+ = \begin{pmatrix} 0 \\ 1 \end{pmatrix},
    \mathbf{r}^- = \begin{pmatrix*}[r] 1 \\ -1 \end{pmatrix*},
    \mathbf{r}^\circ = \begin{pmatrix} 1 \\ 0 \end{pmatrix}
\end{math},\\
$\Gind(i,m)$ --- функция, которая по индексу наибольшего элемента $C^{\ev}$ в алфавите $\mathcal{A}$ и индексу конъюнкта в алфавите $\mathcal{A}^{\ev}$ возвращает индекс соответствующего конъюнкта в алфавите $\mathcal{A}$.

Решение второй задачи апостериорного вывода находится по формуле~\cite{91}:
\begin{equation} \label{conj1}
\Pc^a=\sum_{i=0}^{2^{n^\prime}-1}\dfrac{\T^{\langle \Gind(i,m),\,\Gind(2^{n^\prime}-1-i,m)\rangle }\Pc}{(\mathbf{r}^{\langle \Gind(i,m),\,\Gind(2^{n^\prime}-1-i,m)\rangle},\Pc)}\In\Pc^{\ev}[i],
\end{equation} 
где $\Pc^a$ --- вектор апостериорных вероятностей истинности элементов данного фрагмента знаний,\\
\begin{math}
    \Tij = \TijTilda_{n-1}\otimes \TijTilda_{n-2} \otimes ...\otimes \TijTilda_{0}
\end{math}, \\
\begin{math}
    \TijTilda_k = 
    \begin{cases}
        \T^+ \text{, если $x_k$ входит в $c_i$,}\\
        \T^- \text{, если $x_k$ входит в $c_j$,}\\
        \T^\circ  \text{, иначе;}
    \end{cases}
\end{math} \\ 
\begin{math}
    \T^+ = \begin{pmatrix} 0 & 1 \\ 0 & 1 \end{pmatrix},
    \T^- = \begin{pmatrix*}[r] 1 & -1 \\ 0 & 0 \end{pmatrix*},
\T^\circ = \begin{pmatrix} 1 & 0 \\ 0 & 1 \end{pmatrix}.
\end{math}

\subsubsection{Локальный апостериорный вывод над дизъюнктами}

Приведем имеющиеся результаты для локального ЛВВ для модели ФЗ, построенной над идеалами дизъюнктов~\cite{76, 49}. Рассмотрим решение первой и второй задач  локального апостериорного вывода для стохастических свидетельств.

Решение первой задачи~\cite{49}:
\begin{equation}\label{dis1}
  p(\langle C^{\ev}, \Pd^{\prime \ev} \rangle) =\sum_{i=0}^{2^{n^\prime}-1}  (\mathbf{d}^{\langle \Gind(i,m),\,\Gind(2^{n^\prime}-1-i,m)\rangle},\Pd^{\prime}) \Ln\Pd^{\prime a}[i],
\end{equation}
где
$\dij = \otimes^{0}_{k=n-1}\dijTilda_k$,\\
\begin{math}
    \dijTilda_k = 
    \begin{cases}
        \mathbf{d}^+ \text{, если $x_k$ входит в $c_i$,}\\
        \mathbf{d}^- \text{, если $x_k$ входит в $c_j$,}\\
        \mathbf{d}^\circ  \text{, иначе;}
    \end{cases}
\end{math} \\ 
\begin{math}
    \mathbf{d}^+ = \begin{pmatrix*}[r] 1 \\ -1 \end{pmatrix*},
    \mathbf{d}^- = \begin{pmatrix} 0 \\ 1 \end{pmatrix},
    \mathbf{d}^\circ = \begin{pmatrix} 1 \\ 0 \end{pmatrix},
\end{math}\\
$\Pd^\prime = \mathbf{1} - \Pd$,\\
$\Gind(i,m)$ --- функция, которая по индексу наибольшего элемента $C^{\ev}$ в алфавите $\mathcal{A}$ и индексу дизъюнкта в алфавите $\mathcal{A}_{\ev}$ возвращает индекс соответствующего дизъюнкта в алфавите $\mathcal{A}$.

Решение второй задачи~\cite{49}:
\begin{equation}\label{dis2}
\Pd^{\prime a} =\sum_{i=0}^{2^{n^\prime}-1}\dfrac{\M^{\langle \Gind(i,m),\,\Gind(2^{n^\prime}-1-i,m)\rangle }\Pd^{\prime}}{(\mathbf{d}^{\langle \Gind(i,m),\,\Gind(2^{n^\prime}-1-i,m)\rangle},\Pd^{\prime})}\Ln\Pd^{\prime \ev}[i],
\end{equation}
где 
\begin{math}
    \Mij = \otimes^{0}_{k=n-1}\MijTilda_k
\end{math},\\
\begin{math}
    \MijTilda_k = 
    \begin{cases}
        \M^+ \text{, если $x_k$ входит в $c_i$,}\\
        \M^- \text{, если $x_k$ входит в $c_j$,}\\
        \M^\circ  \text{, иначе;}
    \end{cases}
\end{math} \\ 
\begin{math}
    \M^+ = \begin{pmatrix*}[r] 1 & -1 \\ 0 & 0 \end{pmatrix*},
    \M^- = \begin{pmatrix} 0 & 1 \\ 0 & 1 \end{pmatrix},
    \M^\circ = \begin{pmatrix} 1 & 0 \\ 0 & 1 \end{pmatrix}
\end{math}\\
и $\Pd^{\prime , \evidenceNumbers} = \mathbf{1} - \Pd^{\evidenceNumbers}$.

\subsubsection{Локальный апостериорный вывод над квантами}
Рассмотрим решение первой и второй задачи апостериорного локального ЛВВ, когда во фрагмент знаний со скалярными оценками, построенный над множеством пропозиций-квантов, пришло стохастическое свидетельство. Подробнее про локальный ЛВВ над квантами можно прочитать в~\cite{74}.

Решение первой задачи~\cite{74}:
\begin{equation} \label{quantsFirst}
  p(\langle C^{\ev}, \Pq^{\ev} \rangle) =\sum_{i=0}^{2^{n^\prime}-1}  (\Selector^{\langle \Gind(i,m),\,\Gind(2^{n^\prime}-1-i,m)\rangle},\Pq) \Pq^{a}[i],
\end{equation} 
где вектор-селектор $\sij = \otimes^{0}_{k=n-1}\sijTilda_k$, \\
\begin{math}
    \sijTilda_k = 
    \begin{cases}
        \Selector^+ \text{, если $x_k$ входит в $c_i$,}\\
        \Selector^- \text{, если $x_k$ входит в $c_j$,}\\
        \Selector^\circ  \text{, иначе;}
    \end{cases}
\end{math} \\
и 
\begin{math}
    \Selector^+ = \begin{pmatrix} 0 \\ 1 \end{pmatrix},
    \Selector^- = \begin{pmatrix} 1 \\ 0 \end{pmatrix},
    \Selector^\circ = \begin{pmatrix} 1 \\ 1 \end{pmatrix}
\end{math},\\
$\Gind(i,m)$ --- функция, которая по индексу кванта $i$ в алфавите, над которым построено свидетельство, и индексу $m$ наибольшего элемента $\Pq^a$ в исходном алфавите сопоставляет индекс свидетельства с индексом множества квантов поступившего свидетельства в исходном алфавите.

Теперь рассмотрим решение второй задачи~\cite{74}. Пусть дан фрагмент знаний $(C, \Pq)$ со скалярными оценками и стохастическое свидетельство $(C^{\ev}, \Pq^{\ev})$.

Решение второй задачи апостериорного вывода находится по формуле:
\begin{equation} \label{quants1}
 \Pq^{\evidenceNumbers}=\sum_{i=0}^{2^{n^\prime}-1}\dfrac{{\Selector^{\langle \Gind(i,m),\,\Gind(2^{n^\prime}-1-i,m)\rangle }}\circ \Pq}{(\Selector^{\langle \Gind(i,m),\,\Gind(2^{n^\prime}-1-i,m)\rangle},\Pq)}\Pq^{\ev}[i],
\end{equation}
где $\circ$ обозначает произведение Адамара~(операция покомпонентного произведения векторов одинаковой размерности).


\subsection{Глобальный апостериорный логико"=вероятностный вывод}
Рассмотрим связную ациклическую алгебраическую сеть~\cite{284} со скалярными оценками во всех фрагментах знаний. Предположим, что в один из фрагментов знаний поступило стохастическое свидетельство. Задачей глобального апостериорного вывода является распространение влияния этого свидетельства~(пропагация свидетельства) во все фрагменты знаний сети. Схема глобального апостериорного вывода подробно описана в~\cite{93}.
 Алгоритм пропагации состоит из 3 шагов: 
\begin{enumerate}
	\item  Пропагация свидетельства во фрагмент знаний, в который оно пришло, и оценка апостериорных вероятностей его элементов;
	\item Формирование виртуального свидетельства;
	\item Пропагация виртуального свидетельства в соседний фрагмент знаний.
\end{enumerate}

Аналогичным образом свидетельство пропагируется далее, пока не будут переозначены оценки элементов во всех фрагментах знаний.

Виртуальным свидетельством называется пересечение двух фрагментов знаний~(сепаратор), также являющийся фрагментом знаний~\cite{51, 93}. Так как после переозначивания оценок в начальном фрагменте знаний, конъюнкты, принадлежащие сепаратору, имеют новые оценки, потому что принадлежат начальному фрагменту знаний, но с другой стороны они также принадлежат соседнему фрагменту знаний и имеют другие оценки, то сепаратор можно рассмотреть как новую информацию, поступившую в соседний фрагмент знаний. 

Таким образом, на втором шаге алгоритма из вектора вероятностей начального фрагмента знаний необходимо выделить вектор значений, принадлежащих обоим фрагментам знаний, и принять его за новое свидетельство, которое на третьем шаге пропагируется в соседний фрагмент знаний, оценки которого необходимо переозначить.

Для того чтобы выделить необходимые оценки с помощью матрично-векторных вычислений, требуется сформировать матрицу перехода от вектора оценок вероятностей элементов ФЗ к вектору оценок вероятностей виртуального свидетельства.

Виртуальные свидетельства могут быть двух видов: стохастическое и неточное, потому что пересечение двух фрагментов знаний всегда является фрагментом знаний, но может содержать как и скалярные, так и интервальные оценки вероятностей.

Остановимся подробнее на втором шаге алгоритма и рассмотрим формирование и распространение виртуального свидетельства из одного фрагмента знаний в другой, когда они построены над идеалами конъюнктов. Считаем, что вектор $\Pc^{1, a}$ содержит апостериорные оценки вероятностей, потому что в него ранее было распространено влияние какого-то свидетельства. Распространим влияние этого свидетельства в соседний фрагмент знаний с оценками $\Pc^2$.

Рассмотрим, как выглядит матрица перехода $\Qmatr$ от вектора оценок вероятностей элементов ФЗ к вектору оценок вероятностей виртуального свидетельства. Матрица будет размерности $m \times n$~($m$ --- длина вектора $\Pc^{\ev}$,  $n$ --- длина вектора $\Pc^{1, a}$). Элементы матрицы определяются по следующему правилу~\cite{51, 9, 70}:
\begin{equation*}
    \Qmatr[i,j] = 
    \begin{cases}
        1 \text{, если $\Pc^{1, a}[j] = \Pc^{\ev}[i]$,}\\
        0 \text{, иначе;}\\
    \end{cases}
\end{equation*}

Под $\Pc^{1, a}[i]$ и $\Pc^{\ev}[i]$ подразумеваются сами конъюнкты, а не значения вероятностей. Затем нужно умножить матрицу на $\Pq^{1, a}$. Таким образом, $\Pc^{\ev} = \Qmatr\Jn\Pq^{1, a} = \Qmatr\Pc^{1, a}$.

Алфавит, над которым построено свидетельство, можно найти как $\mathcal{A}_{\ev} = \mathcal{A}_1 \cap \mathcal{A}_2$, где $\mathcal{A}_1$ --- алфавит первого фрагмента знаний, $\mathcal{A}_2$ --- алфавит второго фрагмента знаний.

Таким образом, пропагировать виртуальное свидетельство из одного ФЗ в другой можно с помощью следующего уравнения~\cite{51}:
\begin{equation} \label{conjglob1}
    \Pc^{2,a}=\sum_{i=0}^{2^{n^\prime}-1}\dfrac{\T^{\langle \Gind(i,m),\,\Gind(2^{n^\prime}-1-i,m)\rangle }\Pc^2}{(\mathbf{r}^{\langle \Gind(i,m),\,\Gind(2^{n^\prime}-1-i,m)\rangle},\Pc^2)}\In\Pc^{\ev}[i],
\end{equation}
где $\Pc^{\ev} = \Qmatr\Pc^{1, a}$, $\Pc^{1, a}$ и $\Pc^2$ --- векторы вероятностей элементов идеалов конъюнктов первого и второго фрагментов знаний, $\Pc^{2,a}$ --- вектор апостериорных вероятностей элементов идеала конъюнктов второго фрагмента знаний.

Матрицу перехода $\Qmatr$ можно также сформировать с помощью матрично"=векторных операций, а именно через кронекерово произведение:

\begin{theorem}[\cite{70}]
Вектор оценок виртуального свидетельства $\Pc^{\ev}$ можно вычислить как $\Pc^{\ev} = \Qmatr\Pc^{a,1}$, где $\Qmatr = \QijTild_{n-1} \otimes \QijTild_{n-2} \otimes ... \otimes \QijTild_{0}$,
 $\QijTild_k = \begin{cases}
\Qmatr^+ \text{, если $x_k$ входит в $\mathcal{A}^{\ev}$,} \\
\Qmatr^- \text{, иначе;}
\end{cases}$,
\begin{math}
    \Qmatr^+ = \begin{pmatrix} 1 & 0 \\ 0 & 1 \end{pmatrix},
    \Qmatr^- = \begin{pmatrix} 1 & 0 \end{pmatrix},
\end{math}\\ 
$x_k$ --- $k$-тый элемент в $\mathcal{A}_1$,  $\mathcal{A}_1$ --- алфавит, над которым построен $\Pc^{1,a}$, $\mathcal{A}^{\ev}$ --- алфавит, над которым построен $\Pc^{\ev}$.
\end{theorem}


\subsection{Выводы по главе}
В главе были даны основные определения и понятия, используемые в теории алгебраических байесовских сетей. Представлены определения фрагмента знаний, алгебраической байесовской сети. Рассмотрен логико"=вероятностный вывод в АБС: его виды и задачи, решаемые с помощью ЛВВ, а также основные теоретические результаты, которые ложатся в основу решения задач, рассматриваемых в данной работе.

Стоит отметить, что предложенный в данной главе подход обладает рядом недостатков~\cite{70}, а именно содержит в себе операции, не относящиеся к матрично"=векторным вычислениям. К таким операциям относится функция $\Gind$, для которой ранее не была предложена матрично"=векторная формализация. 

Таким образом, ставится задача улучшить имеющийся результат и заменить функцию $\Gind$ матрично"=векторным аналогом. Также имеющийся результат относится только к одной модели ФЗ и не подходит для двух других моделей, поэтому ставится задача описания алгоритмов глобального ЛВВ для альтернативных моделей фрагмента знаний.


\section{Матрично"=векторная  формализация глобального логико"=вероятностного вывода}\label{cap3}
\subsection{Введение}
В данной главе рассмотрим алгоритм распространения виртуального свидетельства между двумя фрагментами знаний для трех математических моделей ФЗ и получим соответствующие матрично"=векторные уравнения. 

Для того чтобы улучшить результат для модели идеала конъюнктов, представленный в предыдущей главе, и получить уравнения для модели идеала дизъюнктов и модели множества квантов, рассмотрим матрично-векторную формализацию функции $\Gind$. Для этого введем понятие характеристического вектора конъюнкта, дизъюнкта и кванта и матрицы проекции $\G$. Также рассмотрим матрично-векторное формирование матрицы перехода $\V$ от вектора вероятностей элементов идеала дизъюнктов к вектору вероятностей элементов виртуального свидетельства и приведем поясняющие примеры, введем матрицу перехода от вектора дизъюнктов к вектору квантов. Кроме этого докажем теорему, предлагающую матрично"=векторный подход к пропагации виртуального свидетельства в случае пропозиций-квантов. Используя полученные результаты, приведем получившиеся матрично-векторные уравнения для каждой из трех моделей фрагмента знаний. 
\subsection{Матрично"=векторная интерпретация для функции $\Gind$}
Рассмотрим функцию $\Gind(i,m)$ в контексте математической модели фрагмента знаний, заданного идеалом конъюнктов. Для двух других моделей формализация будет аналогичной, потому что сама функция $\Gind(i,m)$ определяется аналогичным образом~\cite{74, 49}, а алфавиты для всех трех моделей одинаковы.

Напомним, что  $\Gind(i,m)$ --- функция, которая по индексу $m$ наибольшего элемента $C^{\ev}$ в алфавите $\mathcal{A}$ и индексу конъюнкта $i$ в алфавите $\mathcal{A}_{\ev}$ возвращает индекс соответствующего конъюнкта в алфавите $\mathcal{A}$~\cite{84}. Так как индексы конъюнктов пронумерованы согласно строго заданному правилу, то у каждого индекса есть свой строго заданный номер, который можно рассмотреть в двоичной системе счисления или как характеристический вектор из нулей и единиц. Аналогично можно рассмотреть индексы квантов и дизъюнктов. 

\begin{definition}
Характеристический вектор конъюнкта --- это вектор из нулей и единиц, в котором единица на $i$-том месте соответствует единице на $i$-том месте в двоичной записи индекса конъюнкта, а ноль на $i$-том месте --- нулю.
\end{definition}

Будем обозначать характеристический вектор  конъюнкта с индексом $i$ как  $\chi^{i}$. Аналогичным будет обозначение для характеристического вектора дизъюнкта или кванта. Так как в данном разделе рассматриваются ФЗ над идеалами конъюнктов, будем далее рассматривать характеристический вектор конъюнкта.

Можно заметить, что функция $\Gind$ проецирует индекс, соответствующий конъюнкту детерминированного свидетельства, на другой индекс, соответствующий эквивалентному данному конъюнкту ФЗ, построенного над алфавитом $\mathcal{A}$. Вместо индексов можно взять соответствующие характеристические векторы и построить проекцию одного на другой.


Пусть $\chi^{i}$ --- характеристический вектор $i$-того конъюнкта свидетельства в алфавите  $\mathcal{A}_{\ev}$, и  пусть $\chi^{j}$ --- характеристический вектор соответствующего ему $j$-того конъюнкта в алфавите  $\mathcal{A}$. $m$ --- мощность алфавита $\mathcal{A}_{\ev}$, $n$ --- мощность алфавита $\mathcal{A}$.

Тогда $\Gind(i,m)$ можно заменить матрицей проекции вектора $\chi^{i}$ на $\chi^{j}$ по следующему правилу:
\begin{equation}\label{G}
\G[i,j] = \begin{cases}
1 \text{, если $\mathcal{A}[n -1-i] = \mathcal{A}_{\ev}[m - 1 -j]$,} \\
0 \text{, иначе};
\end{cases} 
\end{equation}

Такая матрица будет размера $n \times m$, где $n$ --- мощность алфавита $\mathcal{A}$ и $m$ --- мощность алфавита $\mathcal{A}_{\ev}$.

Чтобы получить характеристический вектор конъюнкта $\chi^{j}$ в алфавите $\mathcal{A}$, нужно будет домножить вектор $\chi^{i}$ слева на матрицу $\G$: $\chi^{j} = \G\chi^{i}$.

По построению матрица $\G$ будет содержать в себе нулевые строки, соответствующие элементам алфавита $\mathcal{A}$, которые не входят в алфавит $\mathcal{A}_{\ev}$. Единица на j-той позиции в ненулевой строке i соответствует атому $x_i \in \mathcal{A}$, эквивалентному атому $x_j \in \mathcal{A}_{\ev}$. Умножение нулевых строк на характеристический вектор будет давать нулевой элемент в результирующем векторе,  умножение ненулевых строк даст единицу, если элемент алфавита входит в конъюнкт, и ноль иначе. Таким образом, матрица $\G$ проецирует характеристический вектор конъюнкта, построенного над алфавитом $\mathcal{A}_{\ev}$ на характеристический вектор конъюнкта над алфавитом $\mathcal{A}$.

\subsubsection{Пример использования матрично-векторной интерпретации функции $\Gind$}
Приведем поясняющий пример. Пусть $\mathcal{A} =  \{x_1, x_2, x_3, x_4\}$ и $\mathcal{A}_{\ev} =  \{x_2, x_4\}$. Построим матрицу проекции $\G$ по формуле \ref{G}:

\begin{math}
\G[i,j] = \begin{cases}
1 \text{, если $\mathcal{A}[n -1-i] = \mathcal{A}_{\ev}[m - 1 - j]$,} \\
0 \text{, иначе};
\end{cases}
= \begin{pmatrix}
1 & 0 \\ 0 & 0 \\ 0 & 1\\ 0 & 0
\end{pmatrix}.
\end{math}

Пусть $\chi^{1} = \begin{pmatrix} 0 \\ 1 \end{pmatrix}$ и  соответствует конъюнкту $x_2$ свидетельства. Найдем характеристический вектор индекса конъюнкта $x_2$ в алфавите $\mathcal{A}$, домножив вектор $\chi^{1}$ на матрицу $\G$:

\begin{math}
\begin{pmatrix}
1 & 0 \\ 0 & 0 \\ 0 & 1\\ 0 & 0
\end{pmatrix}
\begin{pmatrix} 0 \\ 1 \end{pmatrix} 
= \begin{pmatrix} 0\\ 0 \\  1 \\ 0 \end{pmatrix}.
\end{math}

Видно, что искомый индекс получен.

\subsubsection{Уравнения для решения второй задачи апостериорного вывода}


Заменим в уравнениях \ref{conj1} и \ref{quants1}, приведенных во второй главе, функцию $\Gind$ на матрицу проекции $\G$. 

Уравнение \ref{conj1} для конъюнктов приобретет следующий вид:
\begin{equation} \label{conjG}
    \Pc^a=\sum_{i=0}^{2^{n^\prime}-1}\dfrac{\T^{\langle \G\chi^{i} ,\,\G\chi^{2^{n^\prime}-1-i} \rangle }\Pc}{(\mathbf{r}^{\langle  \G\chi^{i},\,\G\chi^{2^{n^\prime}-1-i} \rangle},\Pc)}\In\Pc^{\ev}[i].
\end{equation} 

Уравнение \ref{quants1} для квантов будет следующим:
\begin{equation} \label{quantsG}
 \Pq^{\evidenceNumbers}=\sum_{i=0}^{2^{n^\prime}-1}\dfrac{{\Selector^{\langle \G\chi^{i} ,\, \G\chi^{2^{n^\prime}-1-i}\rangle }}\circ \Pq}{(\Selector^{\langle\G \chi^{i} ,\, \G\chi^{2^{n^\prime}-1-i}\rangle},\Pq)}\Pq^{\ev}[i].
\end{equation}

Уравнение \ref{dis2} для дизъюнктов заменится следующим:
\begin{equation}\label{dis3}
 \Pd^{a, \prime} =\sum_{i=0}^{2^{n^\prime}-1}\dfrac{\M^{\langle  \G\chi^{i},\,\G\chi^{2^{n^\prime}-1-i} \rangle  }\Pd^{\prime}}{(\mathbf{d}^{\langle \G\chi^{i} ,\, \G\chi^{2^{n^\prime}-1-i}\rangle },\Pd^{\prime})}\Ln\Pd^{\prime \, \ev}[i].
\end{equation}
\subsection{Формализация формирования матрицы перехода}
\subsubsection{Теорема о построении матрицы перехода}
Рассмотрим фрагменты знаний, построенные над идеалами дизъюнктов. Аналогично случаю для конъюнктов~\cite{51, 9}, построим вектор вероятностей элементов виртуального свидетельства. В виртуальное свидетельство будут входить элементы, стоящие на пересечении двух фрагментов знаний.

Сначала введем матрицу перехода от вектора вероятностей квантов к вектору вероятностей дизъюнктов. Воспользуемся приведенным во второй главе выражением ~\ref{pdtopq} для матрицы перехода $\Ln$ от вектора $\Pd^{\prime}$ к вектору $\Pq$. Так как $\Ln$ имеет обратную матрицу, то домножим выражение для перехода от вектора $\Pd^{\prime}$ к вектору $\Pq$ на обратную матрицу и получим: 
\begin{equation*}
 \Ln^{-1} \Pq  = \mathbf{1} -\Pd,
\end{equation*}
где $\Ln^{-1} = \begin{pmatrix} 1 & 1 \\ 1 & 0 \end{pmatrix}^{[n]}.$

Обозначим $\On = \Ln^{-1}$. Перепишем выражение:
\begin{equation*}\mathbf{1} -\Pd = \On \Pq, \end{equation*} где $\On = \begin{pmatrix} 1 & 1 \\ 1 & 0 \end{pmatrix}^{[n]}.$

Теперь построим матрицу перехода $\V$ от вектора вероятностей  элементов идеала дизъюнктов к вектору вероятностей элементов виртуального свидетельства. Выделим из вектора $\Pd^{\prime 1,a}$, содержащего новые апостериорные оценки, элементы, принадлежащие виртуальному свидетельству.

 От вектора вероятностей квантов можно перейти к вектору вероятностей дизъюнктов с помощью матрицы перехода: $\Pd^{\prime 1,a} = \On\Pq^{1,a}$. При умножении $i$-той строки $\On[i]$ на $\Pq^{1,a}$, получается $i$-тый элемент вектора $\Pd^{\prime 1,a}$. Если выделить в отдельную матрицу строки $\On[i]$, выделяющие из $\Pd^{\prime 1,a}$ элементы виртуального свидетельства, то при умножении эту матрицы на $\Pq^{1,a}$, получился нужный нам вектор оценок вероятностей.

Для того чтобы выделить строки $\On[i]$, нужно домножить $\On$ слева на матрицу $\V$ размерности $m \times n$~($m$ --- длина вектора $\Pd^{\prime \, \ev}$,  $n$ --- длина вектора $\Pd^{\prime 1,a}$). Элементы матрицы определим по следующему правилу:
\begin{equation*}
    \V[i,j] = 
    \begin{cases}
        1 \text{, если $\Pd^{\prime 1,a}[j] = \Pd^{\prime \, \ev}[i]$,}\\
        0 \text{, иначе;}
    \end{cases}
\end{equation*}

При этом под $\Pd^{\prime 1,a}[i]$ и $\Pd^{\prime \, \ev}[i]$ подразумеваются сами дизъюнкты, а не значения вероятностей. Затем нужно умножить матрицу на $\Pq^{1,a}$. Таким образом, $\Pd^{\prime \, \ev} = \V\On\Pq^{1,a} = \V\Pd^{\prime 1,a}$.

Алфавит, над которым построено свидетельство, можно найти как $\mathcal{A}_{\ev} = \mathcal{A}_1 \cap \mathcal{A}_2$, где $\mathcal{A}_1$ --- алфавит первого фрагмента знаний, $\mathcal{A}_2$ --- алфавит второго фрагмента знаний.

Теперь покажем, что матрицу перехода $\V$ можно построить через кронекерово произведение, и обоснуем предложенное построение.

\begin{theorem}\label{disq}
Вектор оценок виртуального свидетельства $\Pd^{\prime \ev}$ можно вычислить как $\Pd^{\prime \ev} = \V \Pd^{\prime a,1}$, где $\V = \VijTilda_{n-1} \otimes \VijTilda_{n-2} \otimes ... \otimes \VijTilda_{0}$ ,
 $\VijTilda_k = \begin{cases}
\V^+ \text{, если $x_k$ входит в $\mathcal{A}^{\ev}$,} \\
\V^- \text{, иначе;}
\end{cases}$, 
\begin{math}
    \V^+ = \begin{pmatrix} 1 & 0 \\ 0 & 1 \end{pmatrix},
    \V^- = \begin{pmatrix} 1 & 0 \end{pmatrix},
\end{math}\\
$x_k$ --- $k$-тый элемент в $\mathcal{A}_1$,  $\mathcal{A}_1$ --- алфавит, над которым построен $\Pd^{\prime 1,a}$, $\mathcal{A}^{\ev}$ --- алфавит, над которым построен $\Pd^{\prime \ev}$.
\end{theorem}
\begin{Proof}
Для упрощения доказательства дополним матрицу $\V^-$ строкой из нулей и будем рассматривать построение матрицы $\V$ через матрицы $\V^+$ и $\V^{\prime-}$, где $\V^{\prime-} = \begin{pmatrix} 1 & 0 \\ 0 & 0 \end{pmatrix}$. Тогда искомая матрица будет квадратной и диагональной по построению и будет содержать строки из нулей. 

Также можно заметить, что $i$-тая единица на диагонали означает, что оценка на $i$-том месте в векторе вероятностей входит в виртуальное свидетельство. Ноль же означает, что оценка не относится к вектору виртуального свидетельства. 

Обоснуем, что позиции нулей и единиц при таком построении действительно показывают, входит ли оценка в виртуальное свидетельство.

Рассмотрим на каких позициях диагонали матрицы $\V$ будут нули.
По построению каждое произведение Кронекера увеличивает размерность искомой матрицы вдвое.
Можно заметить, что нули могут появиться только при участии в произведении матрицы $\V^{\prime-}$, в которой есть единственный $0$, находящийся на диагонали.

Получается, что по построению матрица $\V^{\prime-}$, соответствующая условию $x_m \not \in \mathcal{A}^{\ev}$, в
произведении Кронекера даст нули на всех позициях $\V[i,i]$, где
$i\&2^{m+1} = 2^{m+1}$, где $\&$ обозначает операцию побитового И. Множество таких чисел соответствует множеству чисел, в двоичной записи которых на месте $m$ стоит единица~(Нумерация мест в двоичной записи и нумерация атомов в алфавите начинается с 0). Сравнив с правилом нумерации дизъюнктов, получим, что все такие числа соответствуют номерам дизъюнктов, в которых присутствует атом с номером $m$. 

Следовательно, участие матриц $\V^{\prime-}$ в кронекеровом произведении занулит вероятности в результирующем векторе $\Pd^{\ev}$ всех дизъюнктов $d_m$, для которых не выполняется условие $\forall x_k \in d_m \, x_k \in \mathcal{A}_{\ev}$. А умножение $\Pd^{1,a}$ на столбец матрицы с единицей на диагонали как раз выделит в результирующий вектор нужную оценку вероятности.

Осталось разобраться с размерностью результирующего вектора $\Pd^{\ev}$. Он получается одинаковой размерности с исходным вектором $\Pd^{ 1,a}$ и содержит в себе нулевые элементы, возникающие, как раз, из-за видоизменения матрицы $\V^-$. Тогда при замене матрицы  $\V^{\prime-}$ на $\V^-$ получим вектор $\Pd^{\ev}$ нужной размерности. Так как $\Pd^{\prime}$ линейно выражается через $\Pd$, то результат будет корректным и для векторов $\Pd^{\prime 1,a}$  и $\Pd^{\prime \ev}$.
\end{Proof}

\subsubsection{Пример построения матрицы перехода}
 Рассмотрим на примере формирование матрицы $\V$ с помощью предложенного матрично-векторного алгоритма. Пусть даны два фрагмента знаний над алфавитами $\mathcal{A}_1 = \{x_1, x_2, x_3\}$ и $\mathcal{A}_2 = \{x_1, x_3, x_4\}$. Будем считать, что первый фрагмент знаний содержит апостериорные оценки после пропагации свидетельства и нужно пропагировать свидетельство дальше из первого ФЗ во второй. 
Выделим элементы, относящиеся к виртуальному свидетельству, построив матрицу перехода $\V$:
\begin{equation*}
\Pd^{\prime 1} =  \begin{pmatrix}
1 \\ p(\overline{x}_1) \\ p(\overline{x}_2) \\ p(\overline{x}_2\overline{x}_1) \\ p(\overline{x}_3) \\ p(\overline{x}_3\overline{x}_1) \\ p(\overline{x}_3\overline{x}_2) \\ p(\overline{x}_3\overline{x}_2\overline{x}_1)
\end{pmatrix}, 
\Pd^{\prime 2} =  \begin{pmatrix}
1 \\ p(\overline{x}_1) \\ p(\overline{x}_3) \\ p(\overline{x}_3\overline{x}_1) \\ p(\overline{x}_4) \\ p(\overline{x}_4\overline{x}_1) \\ p(\overline{x}_4\overline{x}_3) \\ p(\overline{x}_4\overline{x}_3\overline{x}_1)
\end{pmatrix},
\Pd^{\prime \ev} =  \begin{pmatrix}
1 \\ p(\overline{x}_1) \\ p(\overline{x}_3) \\ p(\overline{x}_3\overline{x}_1)
\end{pmatrix},
\end{equation*}
$\mathcal{A}_{\ev} = \mathcal{A}_1 \cap \mathcal{A}_2 = \{x_1, x_3\}$.
 
 Тогда $\V = \V^+ \otimes \V^- \otimes \V^+ = 
 \begin{pmatrix} 1 &0 \\ 0 &1 \end{pmatrix} \otimes
  \begin{pmatrix} 1 &0  \end{pmatrix} \otimes
 \begin{pmatrix} 1 &0 \\ 0 & 1 \end{pmatrix} 
 = \\
 \begin{pmatrix} 1 &  0 & 0 & 0 & 0 &  0 & 0 &  0 & 0 \\
                 0 & 1 & 0 & 0 & 0 & 0 & 0 & 0 & 0 \\
                 0 & 0 & 0 & 0 & 1 & 0 & 0 & 0 & 0 \\
                 0 & 0 & 0 & 0 & 0 & 1 & 0 & 0 & 0 \end{pmatrix} $.
                 
                 
Видно, что единицы на $i,j$-тых местах соответствуют выполнению равенства $\Pd^{\prime a,1} [i]= \Pd^{\prime \ev }[j]$, то есть таким построением была получена действительно матрица перехода от вектора $\Pd^{\prime a,1} $ к вектору $\Pd^{\prime \ev }$.

\subsection{Уравнения глобального логико"=вероятностного вывода}
\subsubsection{Матрично-векторное уравнение пропагации виртуального свидетельства над идеалом конъюнктов}

Перепишем приведенное во второй главе уравнение \ref{conjglob1} для распространения виртуального свидетельства между двумя фрагментами знаний с учетом описанной выше матрично"=векторной формализации функции $\Gind$. Заменим функцию $\Gind$ на матрицу проекции $\G$, тогда уравнение примет следующий вид:
\begin{equation} \label{conjglob2}
    \Pc^a=\sum_{i=0}^{2^{n^\prime}-1}\dfrac{\T^{\langle  \G\chi^{i},\,\G\chi^{2^{n^\prime}-1-i} \rangle }\Pc}{(\mathbf{r}^{\langle  \G\chi^{i},\,\G\chi^{2^{n^\prime}-1-i} \rangle},\Pc)}\In\Pc^{\ev}[i],
\end{equation} 
где \begin{math}\G[i,j] = \begin{cases}
1 \text{, если $\mathcal{A}_2[n -1-i] = \mathcal{A}_{\ev}[n^{\prime} - 1  - j]$,} \\
0 \text{, иначе;}
\end{cases}
\end{math}
\\ $n^{\prime}$ --- размерность алфавита $\mathcal{A}_{\ev}$, $n$ --- размерность алфавита $\mathcal{A}_2$.

\subsubsection{Матрично-векторное уравнение пропагации виртуального свидетельства над идеалом дизъюнктов}

Рассмотрим алгоритм пропагации виртуального свидетельства из одного фрагмента знаний в другой. Воспользуемся теоремой \ref{disq} и  вычислим $\Pd^{\prime \, \ev} = \V\Pd^{\prime}$.

Далее, основываясь на уравнении \ref{dis3} и подставив в него выражение для вычисления вектора вероятностей элементов виртуального свидетельства, а также заменив функцию $\Gind$ матрицей $\G$, получим, что пропагировать виртуальное свидетельство из одного ФЗ с апостериорными оценками в соседний ФЗ можно с помощью следующего уравнения:
\begin{equation}\label{disGlob}
    \Pd^{\prime \, 2,a}=\sum_{i=0}^{2^{n^\prime}-1}\dfrac{\M^{\langle\G\chi^{i},\, \G \chi^{2^{n^\prime}-1-i} \rangle }\Pd^{\prime \, 2}}{(\mathbf{d}^{\langle \G \chi^{i},\,\G\chi^{2^{n^\prime}-1-i} \rangle},\Pd^{\prime \, 2})}\Ln\Pd^{\prime \, \ev}[i],
\end{equation}
где \begin{math}\G[i,j] = \begin{cases}
1 \text{, если $\mathcal{A}_2[n -1-i] = \mathcal{A}_{\ev}[n^{\prime} - 1 - j]$,} \\
0 \text{, иначе;}
\end{cases}
\end{math}
\\ $n^{\prime}$ --- размерность алфавита $\mathcal{A}_{\ev}$, $n$ --- размерность алфавита $\mathcal{A}_2$.

\subsubsection{Матрично-векторное уравнение пропагации виртуального свидетельства над множеством квантов}

Рассмотрим распространение виртуального свидетельства для случая, когда фрагменты знаний построены над множеством пропозиций"=квантов со скалярными оценками. Для того чтобы получить необходимое уравнение, необходимо научиться находить значения оценок вероятностей виртуального свидетельства $\Pq^{\ev}$. Под виртуальным свидетельством будем понимать фрагмент знаний, содержащий оценки вероятностей для квантов, согласованных как и с первым, так и со вторым фрагментом знаний.
 \begin{theorem}
Пропагировать виртуальное свидетельство из одного фрагмента знаний во второй, когда ФЗ построены над множествами пропозиций-квантов, можно с помощью следующего уравнения:
\begin{equation}\label{quantsGlob}
  \Pq^2=\sum_{i=0}^{2^{n^\prime}-1}\dfrac{{\Selector_2^{\langle  \G_{\mathcal{A}_2, \mathcal{A}_{\ev}} \chi^{i},\,\G_{\mathcal{A}_2, \mathcal{A}_{\ev}}\chi^{2^{n^\prime}-1-i} \rangle }}\circ \Pq^2}{(\Selector_2^{\langle  \G_{\mathcal{A}_2, \mathcal{A}_{\ev}}\chi^{i},\, \G_{\mathcal{A}_2, \mathcal{A}_{\ev}}\chi^{2^{n^\prime}-1-i}\rangle},\Pq^2)}(\Selector_1^{\langle  \G_{\mathcal{A}_1, \mathcal{A}_{\ev}}\chi^{i},\,\G_{\mathcal{A}_1, \mathcal{A}_{\ev}}\chi^{2^{n^\prime}-1-i} \rangle},\Pq^1),
\end{equation}
где $\Pq^1$ --- вектор апостериорных оценок первого ФЗ, а $\Pq^2$ --- вектор оценок второго ФЗ, куда нужно пропагировать свидетельство, $n^\prime$ --- мощность алфавита свидетельства $\mathcal{A}^{\ev}$, $m$ --- мощность алфавита $\mathcal{A}_1$, а $n$ --- мощность алфавита $\mathcal{A}_2$,\\
\begin{math}
\G_{\mathcal{A}_2, \mathcal{A}_{\ev}}[i,j] = 
\begin{cases}
1 \text{, если $\mathcal{A}_2[n-1 -i] = \mathcal{A}_{\ev}[n^\prime -1- j]$,} \\
0 \text{, иначе;}
\end{cases}
\end{math}\\
\begin{math}\G_{\mathcal{A}_1, \mathcal{A}_{\ev}}[i,j] = \begin{cases}
1 \text{, если $\mathcal{A}_1[m -1-i] = \mathcal{A}_{\ev}[n^\prime -1- j]$,} \\
0 \text{, иначе;}
\end{cases}
\end{math}
\end{theorem}
\begin{Proof}
Рассмотрим задачу нахождения оценок вероятностей элементов виртуального свидетельства $\Pq^{\ev}$. Можно заметить, что вектор $\Pq^{\ev}$ будет содержать оценки вероятностей для квантов, согласованных как и с первым, так и со вторым фрагментом знаний.

Алфавит, над которым построено свидетельство, можно найти как $\mathcal{A}_{\ev} = \mathcal{A}_1 \cap \mathcal{A}_2$, где $\mathcal{A}_1$ --- алфавит первого фрагмента знаний, $\mathcal{A}_2$ --- алфавит второго фрагмента знаний.

Таким образом, виртуальное свидетельство состоит из множества пропозиций-квантов, построенных над алфавитом $\mathcal{A}_{\ev}$. Для каждого кванта необходимо посчитать его вероятность, используя апостериорные оценки из первого фрагмента знаний. Это является первой задачей апостериорного вывода, если мы рассматриваем квант как детерминированное свидетельство $\evidenceNumbers$. Решение первой задачи апостериорного вывода выглядит следующим образом~\cite{74}:
\begin{equation*}
p(\langle i, j \rangle) = (\Selector^{\evidenceNumbers},\Pq^1).\end{equation*}

Значит, воспользуемся матрицей проекции $\G$ и получим значение вероятности каждого $i$-того кванта виртуального свидетельства:
\begin{equation}\label{quantsVirt}
\Pq^\ev[i] =(\Selector^{\langle  \G_{\mathcal{A}_1, \mathcal{A}_{\ev}} \chi^{i},\, \G_{\mathcal{A}_1, \mathcal{A}_{\ev}} \chi^{2^{n^\prime}-1-i}\rangle},\Pq^1),
\end{equation}
где $n^\prime$ --- мощность алфавита свидетельства $\mathcal{A}^{\ev}$, а $m$ --- мощность алфавита $\mathcal{A}_1$
и $\G_{\mathcal{A}_1, \mathcal{A}_{\ev}}[i,j] = \begin{cases}
1 \text{, если $\mathcal{A}_1[m -1-i] = \mathcal{A}_{\ev}[n^\prime - 1 -j]$,} \\
0 \text{, иначе;}
\end{cases}$

 В нижнем индексе матрицы $\G$ явно укажем алфавиты, участвующие в ее формировании, чтобы в дальнейшем различать различно сформированные матрицы в одном уравнении. У вектора-селектора также добавим нижние индексы $1$ и $2$, указывающие, элементы какого алфавита используются при его построении. 
 
Далее воспользуемся уравнением \ref{quantsG} для решения второй задачи апостериорного вывода для стохастического свидетельства, подставим в него выражение \ref{quantsVirt} и получим искомое уравнение. 
\end{Proof}
\subsection{Выводы по главе}
В данной главе была предложена матрично"=векторная формализация для функции $\Gind$ и предложены матрично"=векторные уравнения распространения виртуального свидетельства между двумя соседними фрагментами знаний для двух моделей ФЗ: идеал дизъюнктов, множество пропозиций-квантов. Была сформулирована и доказана теорема о формировании матрицы перехода от вектора оценок элементов ФЗ над идеалом дизъюнктов к вектору оценок виртуального свидетельства, а также доказана теорема о пропагации виртуального свидетельства между двумя ФЗ, построенными над множествами квантов. Введен характеристический вектор конъюнкта, дизъюнкта,  кванта и матрица проекции $\G$, матрица перехода от вектора вероятностей дизъюнктов к вектору вероятностей квантов. Также приведены уравнения локального ЛВВ и уравнение пропагации виртуального свидетельства между двумя ФЗ над идеалами конъюнктов после замены функции $\Gind$  на матрицу $\G$.


\section{Программная реализация}
\subsection{Введение}
В предыдущих главах были описаны основные имеющиеся и полученные теоретические результаты, а в данной главе будет рассмотрена программная реализация. Будет описана архитектура существующего комплекса программ, представлены основные классы и методы для реализации алгоритмов глобального логико-вероятностного вывода для различных моделей ФЗ, а также приведены примеры работы программы.

Полученная в ходе работы программная реализация разработана на языке C\#, в качестве среды разработки использовалась среда JetBrains Rider~\cite{68}, которая обладает меньшей функциональностью в отличие от Visual Studio, но отличается удобством и быстротой работы. Для совместной работы использовался репозиторий на BitBucket~\cite{111} и система контроля версий Git. Для тестирования использовалась библиотека NUnit~\cite{67}, а для работы с матрицами использовалась библиотека Math.Net Numerics~\cite{66}.
\subsection{Архитектура программного комплекса}
Комплекс программ AlgBN Math Library~\cite{89} представляет собой библиотеку для автоматизации логико"=вероятностного вывода в АБС, в которой уже реализованы структуры данных для хранения фрагмента знаний и АБС, а также алгоритмы локального логико-вероятностного вывода и проверки непротиворечивости. Фрагменты знаний можно создавать с бинарными, скалярными и интервальными оценками вероятностей. Рассмотрим наиболее важные структуры, имеющиеся в библиотеке.

 Фрагмент знаний, представленный идеалом конъюнктов с скалярными оценками представлен классом ScalarConjKP, с интервальными оценками --- IntervalConjKP, с бинарными --- BinaryConjKP. Для создания экземпляра любого из этих классов нужно передать в конструктор глобальный индекс ФЗ и массив оценок вероятностей элементов ФЗ. Глобальный индекс --- это число, единицы в двоичной записи которого соответствуют номерам элементов алфавита, над которыми построен ФЗ. Аналогичные классы есть для двух других моделей ФЗ: для дизъюнктов --- BinaryDisjKP, ScalarDisjKP и IntervalDisjKP, для квантов --- BinaryQuantKP, ScalarQuantKP и IntervalQuantKP.

Структура Propagator состоит из классов, позволяющих пропагировать детерминированное, стохастическое или неточное свидетельство в ФЗ с скалярными или интервальными оценками. Каждый из таких классов содержит методы propagate для пропагации свидетельства и getResult для возвращения результата пропагации. Каждая реализация отвечает за пропагацию одного из типов свидетельства в ФЗ с одним видом оценок. Например, StochasticScalarConjunctsLocalPropagator отвечает за пропагацию стохастического свидетельства в ФЗ с скалярными оценками. Структура Propagator позволяет работать с двумя моделями ФЗ --- идеал конъюнктов и идеал дизъюнктов. Для третьей модели функциональность еще не реализована, а алгоритмы, работающие с идеалами дизъюнктов, требуют доработки.

За проверку локальной непротиворечивости отвечает структура In\-ferrer. Структура MatrixTransform реализует матрицы перехода между различными моделями ФЗ. Для решения задач линейного программирования используется библиотека на C++ lp\_solve55, за обращение к библиотеке отвечает класс LP. Кроме того, есть структура Alphabet для работы с алфавитом.

Также в комплексе есть программный модуль ABNGlobal, отвечающий за глобальную непротиворечивость. Так как пропагация виртуального свидетельства относится к глобальному ЛВВ, то расширим эту часть библиотеки своей реализацией. 

Реализация будет использовать существующие структуры для локального ЛВВ. Так как существующие алгоритмы позволяют работать с интервальными оценками, то реализация алгоритмов пропагации виртуального свидетельства будет разработана как и для ФЗ с интервальными оценками, так и для неточных виртуальных свидетельств. Существующую реализацию~\cite{4, 65} алгоритмов  локального логико"=вероятностного вывода для квантов перенесем на существующие в комплексе программ структуры хранения, адаптируем и затем переиспользуем при реализации алгоритма пропагации виртуального свидетельства для данной модели. Реализации Propagator для дизъюнктов также адаптируем и усовершенствуем с целью дальнейшего переиспользования.



\subsection{Описание классов и методов}
Для того чтобы реализовать алгоритмы распространения виртуального свидетельства, была расширена функциональность абстракций фрагмента знаний. Соответственно, для интерфейсов каждой модели~(идеал конъюнктов, идеал дизъюнктов, множество квантов) в случае скалярных и интервальных оценок был добавлен статический класс с тремя методами расширения: \begin{enumerate}
    \item IsIntersect --- проверяет, пересекаются ли между собой два фрагмента знаний, и возвращает true, если пересекаются, и false иначе;
    \item IsEqual --- позволяет сравнивать два фрагмента знаний между собой с заданной точностью и возвращает true, если ФЗ эквивалентны, и false иначе;
    \item GetSeparator --- возвращает пересечение двух фрагментов знаний, если пересечение не пусто, а в случае, если ФЗ не пересекаются, возвращается пустой фрагмент знаний.
\end{enumerate}

Всего было создано 6 таких статических классов.

Рассмотрим подробнее работу метода GetSeparator. В случае конъюнктов и дизъюнктов основную работу по выделению элементов сепаратора делает следующий цикл:

\begin{lstlisting}[caption = Цикл для выделения элементов виртуального свидетельства]
for (int index = 0; index <= KPGlobalIndex; index++)
{
    if ((index & KPGlobalIndex) == index)
    {
        if ((index & separatorGlobalIndex) == index)
        {
            probability[separatorIndex] =
                firstKP.GetPointEstimate(probIndex);
            probIndex++;
            separatorIndex++;
        }
        else
        {
            probIndex++;
        }
    }
}
\end{lstlisting}

Стоит отметить, что при программной реализации алфавит удобнее задавать с помощью глобального индекса. Глобальный индекс сепаратора можно получить с помощью операции побитового И глобальных индексов первого и второго фрагментов знаний, по этому принципу работает метод IsIntersect. Заметим, что количество единиц в глобальном индексе соответствует мощности алфавита. 

Данный цикл также использует операцию побитового И. Он проверяет все возможные числа index, которые могут соответствовать элементам вектора вероятностей, это числа от 0 до KPGlobalIndex --- глобального индекса первого ФЗ. Первое условие проверяет это соответствие, второе условие аналогично проверяет, соответствует ли индекс элементу сепаратора. Если условия верны, то значение вероятности сохраняется в массив вероятностей probability. Индексы текущих элементов в векторе вероятностей фрагмента знаний и векторе вероятностей сепаратора задаются с помощью переменных probIndex и separatorIndex соответственно.
 
 Отметим, что данный цикл реализует функциональность матриц перехода  $\Qmatr$ и $\V$, а сами матрицы строить не нужно, потому что, благодаря глобальным индексам, при реализации сразу понятно, какие элементы формируют виртуальное свидетельство.
 
 Так как у модели ФЗ, представленной квантами, виртуальное свидетельство содержит не просто оценки элементов на пересечении, а согласованные оценки, то тут алгоритм построения виртуального свидетельства будет другим. 
 
 Рассмотрим реализацию метода GetSeparator для построения виртуального свидетельства со скалярными оценками.  Вспомогательный метод GetQuantIndexes возвращает вектор из глобальных индексов соответствующих квантов, составляющих ФЗ.  В цикле для каждого индекса кванта из сепаратора считается его вероятность как вероятность детерминированного свидетельства с помощью метода GetEvidenceProbability объекта класса DeterministicScalarQuantLocalPropagator.
 
 \begin{lstlisting}[caption = Метод построения виртуального видетельства со скалярными оценками над квантами]
public static ScalarQuantKP_I GetSeparator(ScalarQuantKP_I firstKP, long separatorGlobalIndex)
{
    var numberOfElements = Convert
        .ToString(separatorGlobalIndex, 2)
        .Count(c => c == '1');
    var numberOfAtoms = (int)Math.Pow(2, numberOfElements);
    var probability = new double[numberOfAtoms];
    var sepAtomGlobalIndexes = GetQuantIndexes(numberOfAtoms, separatorGlobalIndex);
    var propagator = DeterministicScalarQuantLocalPropagator
    	.Instance(firstKP);
            
    for (int index = 0; index < sepAtomGlobalIndexes.Length; index++)
    {
        var binaryQuantKp = new BinaryQuantKP(separatorGlobalIndex);
        binaryQuantKp.SetQuant(sepAtomGlobalIndexes[index]);
        probability[index] = propagator.evidenceProbaility(binaryQuantKp);
    }
    return new ScalarQuantKP(separatorGlobalIndex, probability);
}
\end{lstlisting}

Кроме того, был создан интерфейс IVirtualEvidencePropagator и 6 его реализаций для каждой модели ФЗ в случае скалярных оценок и в случае интервальных оценок.

Дадим краткое описание классов, реализующих алгоритмы распространения виртуального свидетельства для модели ФЗ, представленной идеалом конъюнктов. 

 VirtualEvidenceScalarConjPropagator --- класс, отвечающий за следующие действия: пропагация стохастического виртуального свидетельства и пропагация неточного виртуального свидетельства во фрагмент знаний со скалярными оценками. Соответственно, класс пропагатора содержит 2 метода для двух видов свидетельства. 
 
 Метод PropagateStochasticEvidence используется в случае стохастического свидетельства. Метод принимает 2 параметра: ScalarConjKP\_I firstKP --- первый фрагмент знаний, ScalarConjKP\_I secondKP --- второй фрагмент знаний, в который нужно пропагировать виртуальное свидетельство. В этом методе сначала создается виртуальное свидетельство, далее оно пропагируется с помощью объекта propagator класса  StochasticScalarConjunctsLocalPropagator и его экземплярных методов propagate и getResult. 
 \begin{lstlisting} [caption = Реализация метода PropagateStochasticEvidence]
 public ScalarConjKP_I PropagateStochasticEvidence(ScalarConjKP_I firstKP, ScalarConjKP_I secondKP)
{
    var evidence = firstKP.GetSeparator(secondKP);
    var propagator = new StochasticScalarConjunctsLocalPropagator(secondKP);
    propagator.propagate(evidence);
    return propagator.getResult();
}\end{lstlisting}
 
Второй метод PropagateImpreciseEvidence нужен для пропагирования неточного виртуального свидетельства во второй фрагмент знаний, содержащий скалярные оценки. Пропагирование виртуального свидетельства осуществляется аналогично с помощью методов реализованного ранее класса ImpreciseScalarConjunctsLocalPropagator.
 \begin{lstlisting} [caption = Реализация метода PropagateImpreciseEvidence]
public IntervalConjKP_I PropagateImpreciseEvidence(IntervalConjKP_I firstKP, ScalarConjKP_I secondKP)
{
    var evidence = firstKP.GetSeparator(secondKP);
    var propagator = new ImpreciseScalarConjunctsLocalPropagator();
    propagator.setPattern(secondKP);
    propagator.propagate(evidence);
    return propagator.getResult();
}
\end{lstlisting}

 Второй класс VirtualEvidenceIntervalConjPropagator  отвечает за распространение виртуального  свидетельства из первого фрагмента знаний во второй ФЗ с интервальными оценками. Содержит в себе два аналогичных метода для пропагации свидетельства в двух различных ситуациях.
 
 Метод PropagateStochasticEvidence используется для распространения стохастического виртуального свидетельства в ФЗ с интервальными оценками. Метод принимает два параметра: ScalarConjKP\_I firstKP --- первый фрагмент знаний, IntervalConjKP\_I secondKP --- второй фрагмент знаний, в который нужно пропагировать виртуальное свидетельство.  Внутри метода создается виртуальное свидетельство и далее оно пропагируется с помощью методов объекта propagator ранее реализованного класса  StochasticIntervalConjunctsLocalPropagator.
\begin{lstlisting}[caption = Реализация метода PropagateStochasticEvidence]
public IntervalConjKP_I PropagateStochasticEvidence(ScalarConjKP_I firstKP, IntervalConjKP_I secondKP)
{
    var evidence = firstKP.GetSeparator(secondKP);
    var propagator = new StochasticIntervalConjunctsLocalPropagator();

    propagator.setPattern(secondKP);
    propagator.propagate(evidence);

return propagator.getResult();
}
\end{lstlisting}

Второй метод PropagateImpreciseEvidence нужен для пропагирования неточного виртуального свидетельства из первого ФЗ во второй. Пропагирование виртуального свидетельства осуществляется аналогично с помощью методов объекта propagator ранее реализованного класса ImpreciseIntervalConjunctsLocalPropagator.
\begin{lstlisting} [caption = Реализация метода PropagateImpreciseEvidence]
public IntervalConjKP_I PropagateImpreciseEvidence(IntervalConjKP_I firstKP, IntervalConjKP_I secondKP)
{
    var evidence = firstKP.GetSeparator(secondKP);
    var propagator = new ImpreciseIntervalConjunctsLocalPropagator();

    propagator.setPattern(secondKP);
    propagator.propagate(evidence);

    return propagator.getResult();
}\end{lstlisting}

Аналогичные классы реализованы для фрагментов знаний, построенных над идеалами дизъюнктов: VirtualEvidenceScalarDisjPropagator и VirtualEvidenceIntervalDisjPropagator, а также классы для ФЗ, построенных над множествами квантов: VirtualEvidenceScalarQuantPropagator и VirtualEvidenceIntervalQuantPropagator.  Каждый класс содержит методы PropagateStochasticEvidence и PropagateImpreciseEvidence для пропагации одного из видов виртуального свидетельства.

	   
\subsection{Примеры работы алгоритмов}
Приведем примеры использования методов реализаций интерфейса IVirtualEvidencePropagator. Рассмотрим примеры, когда фрагменты знаний содержат скалярные оценки, и в первый фрагмент знаний приходит стохастическое свидетельство. Распространим влияние свидетельства в первый ФЗ, сформируем виртуальное свидетельство и пропагируем его во второй ФЗ. Для остальных возможных ситуаций использование будет аналогичным.
\subsubsection{Пример пропагации стохастического виртуального свидетельства в ФЗ с скалярными оценками над идеалом конъюнктов}
Пусть даны два фрагмента знаний над алфавитами $\mathcal{A}_1 = \{x_1, x_2\}$ и  $\mathcal{A}_2 = \{x_2, x_3\}$ и в первый фрагмент знаний поступило стохастическое свидетельство над алфавитом $\mathcal{A}_{\ev} = \{x_1\}$. Пусть ФЗ и свидетельство имеют следующие оценки вероятностей: \\ 
\begin{equation*}
\Pc^1 =  \begin{pmatrix}
1 \\ p(x_1) \\ p(x_2) \\ p(x_2x_1)
\end{pmatrix} = \begin{pmatrix}
1 \\ 0.5\\ 0.7\\ 0.3
\end{pmatrix}, 
\Pc^2 =  \begin{pmatrix}
1 \\ p(x_2) \\ p(x_3) \\ p(x_3x_2)
\end{pmatrix} = \begin{pmatrix}
1 \\ 0.3\\ 0.7\\ 0.1 
\end{pmatrix},
\end{equation*}
\begin{equation*}
\Pc^{\ev} = \begin{pmatrix}  1 \\ p(x_1)	\end{pmatrix} = \begin{pmatrix}
1 \\ 0.4
\end{pmatrix}.
\end{equation*}

После пропагации свидетельства оценки вероятностей элементов первого фрагмента знаний, вычисленные с помощью уравнения~\ref{conjG}, будут следующими:
\begin{equation*}
\Pc^{1,a} = \begin{pmatrix}
1 \\  0.4 \\ 0.72 \\ 0.24
\end{pmatrix}.
\end{equation*}

Виртуальное свидетельство будет построено над алфавитом $\mathcal{A}_{\ev} = \mathcal{A}_1 \cap \mathcal{A}_2 = \{x_1, x_2\} \cap \{x_2, x_3\} = \{x_2\}$ и будет содержать следующий вектор оценок вероятностей элементов:
\begin{equation*}
\Pc^{\ev} = \begin{pmatrix}  1 \\ p(x_2)	\end{pmatrix} = 
\begin{pmatrix} 1 \\ 0.72 \end{pmatrix}.
\end{equation*}

После пропагации виртуального свидетельства во второй фрагмент знаний с помощью уравнения~\ref{conjglob2}, оценки второго ФЗ получаются следующими:
\begin{equation*}
\Pc^{2,a} = \begin{pmatrix}
1 \\ 0.72 \\ 0.48 \\ 0.24
\end{pmatrix}.
\end{equation*}

Данный результат пропагации можно получить с помощью следующего фрагмента кода:

\begin{lstlisting}[caption = Пример пропагации виртуального свидетельства]
// Creating KPs and evidence
var firstKP = new ScalarConjKP(Convert.ToInt64("011", 2), new[] { 1, 0.5, 0.7, 0.3 });
var secondKP = new ScalarConjKP(Convert.ToInt64("110", 2), new[] { 1, 0.3, 0.7, 0.1 });
var evidence = new ScalarConjKP(Convert.ToInt64("001", 2), new[] { 1, 0.4 });
// Creating propagators for stochastic and virtual evidences
var localPropagator = new StochasticScalarConjunctsLocalPropagator();
var virtualEvPropagator = new VirtualEvidenceScalarConjPropagator();

// Propagating stochasic evidence in fisrt KP
localPropagator.setPattern(firstKP);
localPropagator.propagate(evidence);
var firstKPwithAposterioriEst = localPropagator.getResult();

// Propagating virtual evidence in second KP
var secondKPwithAposterioriEst = virtualEvPropagator
    .PropagateStochasticEvidence(firstKPwithAposterioriEst, secondKP);
\end{lstlisting}

Вывод на консоль можно осуществить с помощью следующего кода:
\begin{lstlisting}[caption = Вывод результатов на консоль]
Console.WriteLine("Probability estimates in first KP after propagation:");
firstKPwithAposterioriEst
    .GetPointEstimate()
    .ToList()
    .ForEach(est => Console.Write(" " + est));
 
Console.WriteLine("Probability estimates in second KP after propagation:");
secondKPwithAposterioriEst
    .GetPointEstimate()
    .ToList()
    .ForEach(est => Console.Write(" " + est));
\end{lstlisting}

Результаты вывода на консоль:
\begin{lstlisting}[caption = Результаты]
Probability estimates in first KP after propagation:
 1, 0,4 0,72 0,24
Probability estimates in second KP after propagation:
1, 0,72, 0,48 0,24
\end{lstlisting}
\subsubsection{Пример пропагации стохастического виртуального свидетельства в ФЗ с скалярными оценками над идеалом дизъюнктов}
Рассмотрим фрагменты знаний над алфавитами $\mathcal{A}_1 = \{x_1, x_2\}$ и  $\mathcal{A}_2 = \{x_2, x_3\}$, построенные над идеалами дизъюнктов. Программная реализация принимает на вход и возвращает в качестве результата ФЗ, содержащие вектор вероятностей $\Pd$, в то же время как приведенные в работе формулы работают с векторами $\Pd^{\prime}$, поэтому приведем в данном примере векторы вероятностей как и $\Pd$, так и $\Pd^{\prime}$. Считаем, что в первый фрагмент знаний поступило стохастическое свидетельство над алфавитом $\mathcal{A}_{\ev} = \{x_1\}$. Пусть ФЗ и свидетельство имеют следующие оценки вероятностей: \\ 
\begin{equation*}
\Pd^{\prime 1} =  \begin{pmatrix}
1 \\ p(\overline{x}_1) \\ p(\overline{x}_2) \\ p(\overline{x}_2\overline{x}_1)
\end{pmatrix} = \begin{pmatrix}
1 \\ 0.5\\ 0.3\\ 0.1
\end{pmatrix}, 
\Pd^1 = \mathbf{1} - \Pd^{\prime 1} =  \begin{pmatrix}
0 \\ 0.5\\ 0.7\\ 0.9
\end{pmatrix}, 
\end{equation*}
\begin{equation*}
\Pd^{\prime 2} =  \begin{pmatrix}
1 \\ p(\overline{x}_3) \\ p(\overline{x}_3) \\ p(\overline{x}_3\overline{x}_2)
\end{pmatrix} = \begin{pmatrix}
1 \\ 0.7\\ 0.3\\ 0.1 
\end{pmatrix},
\Pd^2 = \mathbf{1} - \Pd^{\prime 2} =  \begin{pmatrix}
0 \\ 0.3\\ 0.7\\ 0.9 \end{pmatrix},
\end{equation*}
\begin{equation*}
\Pd^{\prime \ev} = \begin{pmatrix}  1 \\ p(\overline{x}_1)	\end{pmatrix} = \begin{pmatrix}
1 \\ 0.6
\end{pmatrix},
\Pd^{\ev} = \mathbf{1} - \Pd^{\prime \ev} =  \begin{pmatrix}
0 \\ 0.4 \end{pmatrix}.
\end{equation*}

После пропагации свидетельства оценки вероятностей элементов первого фрагмента знаний, вычисленные с помощью уравнения~\ref{dis3}, будут следующими:
\begin{equation*}
\Pd^{\prime 1,a} = \begin{pmatrix}
1 \\  0.6 \\ 0.28 \\ 0.12
\end{pmatrix},
\Pd^{1,a}= \mathbf{1} - \Pd^{\prime 1,a} = \begin{pmatrix}
0 \\  0.4 \\ 0.72 \\ 0.88
\end{pmatrix}.
\end{equation*}

Виртуальное свидетельство строится над алфавитом $\mathcal{A}_{\ev} = \mathcal{A}_1 \cap \mathcal{A}_2 = \{x_1, x_2\} \cap \{x_2, x_3\} = \{x_2\}$ и содержит следующий вектор оценок вероятностей элементов:
\begin{equation*}
\Pd^{\prime \ev} = \begin{pmatrix}  1 \\ p(\overline{x}_2)	\end{pmatrix} = 
\begin{pmatrix} 1 \\ 0.28 \end{pmatrix},
\Pd^{\ev} = \mathbf{1} - \Pd^{\prime \ev} = \begin{pmatrix} 0 \\ 0.72 \end{pmatrix}.
\end{equation*}

C помощью уравнения~\ref{disGlob} распространим виртуальное свидетельство во второй фрагмент знаний и получим апостериорные оценки вероятностей:
\begin{equation*}
\Pd^{\prime 2,a} = \begin{pmatrix}
1 \\ 0.28 \\ 0.52 \\ 0.04
\end{pmatrix},
 \Pd^{2,a}  = \mathbf{1} - \Pd^{\prime 2,a} = \begin{pmatrix}
0 \\ 0.72 \\ 0.48 \\ 0.96
\end{pmatrix}.
\end{equation*}

Данный результат пропагации можно получить с помощью следующего фрагмента кода:
\begin{lstlisting}[caption = Пример пропагации виртуального свидетельства]
// Creating KPs and evidence
var firstKP = new ScalarDisjKP(Convert.ToInt64("011", 2), new[] { 0, 0.5, 0.7, 0.9 });
var secondKP = new ScalarDisjKP(Convert.ToInt64("110", 2), new[] { 0, 0.3, 0.7, 0.9 });
var evidence = new ScalarDisjKP(Convert.ToInt64("001", 2), new[] { 0, 0.4 });
// Creating propagators for stochastic and virtual evidences
var localPropagator = new StochasticScalarDisjLocalPropagator(firstKP);
var virtualEvPropagator = new VirtualEvidenceScalarDisjPropagator();
// Propagating stochastic evidence in first KP
var firstKPwithAposterioriEst = localPropagator.PropagateEvidence(evidence); 
// Propagating virtual evidence in second KP
var secondKPwithAposterioriEst = virtualEvPropagator
    .PropagateStochasticEvidence(firstKPwithAposterioriEst, secondKP);
\end{lstlisting}

Вывод на консоль можно осуществить с помощью следующего кода:
\begin{lstlisting}[caption = Вывод результатов на консоль]
Console.WriteLine("Probability estimates in first KP after propagation:");
firstKPwithAposterioriEst
    .GetPointEstimate()
    .ToList()
    .ForEach(est => Console.Write(" " + est));

Console.WriteLine("Probability estimates in second KP after propagation:");
secondKPwithAposterioriEst
    .GetPointEstimate()
    .ToList()
    .ForEach(est => Console.Write(" " + est));
\end{lstlisting}

Результаты вывода на консоль:
\begin{lstlisting}[caption = Результаты]
Probability estimates in first KP after propagation:
 0 0,4 0,72 0,88
Probability estimates in second KP after propagation:
 0 0,72 0,48 0,96
\end{lstlisting}
\subsubsection{Пример пропагации стохастического виртуального свидетельства в ФЗ со скалярными оценками над множеством квантов}
Возьмем два фрагмента знаний над алфавитами $\mathcal{A}_1 = \{x_1, x_2\}$ и  $\mathcal{A}_2 = \{x_2, x_3\}$. Пусть в первый фрагмент знаний поступило стохастическое свидетельство над алфавитом $\mathcal{A}_{\ev} = \{x_1\}$. Будем считать, что ФЗ и свидетельство построены над множеством квантов и имеют следующие оценки вероятностей: 
\begin{equation*}
\Pq^1 =  \begin{pmatrix}
p(\overline{x}_2\overline{x}_1)\\ p(\overline{x}_2x_1) \\ p(x_2\overline{x}_1) \\ p(x_2x_1)
\end{pmatrix} = \begin{pmatrix}
0.1\\ 0.2\\ 0.4\\ 0.3
\end{pmatrix},
\Pq^2 =  \begin{pmatrix}
p(\overline{x}_3\overline{x}_2)\\ p(\overline{x}_3x_2)\\ p(x_3\overline{x}_2) \\ p(x_3x_2)
\end{pmatrix} = \begin{pmatrix}
0.1 \\ 0.2\\ 0.6\\ 0.1 
\end{pmatrix},
\end{equation*}
\begin{equation*}
\Pq^{\ev} = \begin{pmatrix}   p(\overline{x}_1) \\ p(x_1)	\end{pmatrix} = \begin{pmatrix}
0.6 \\ 0.4
\end{pmatrix}.
\end{equation*}

После пропагации свидетельства оценки вероятностей элементов первого фрагмента знаний, вычисленные с помощью уравнения~\ref{quantsG}, будут следующими:
\begin{equation*}
\Pq^{1,a} = \begin{pmatrix}
0.12 \\  0.16 \\ 0.48\\ 0.24
\end{pmatrix}.
\end{equation*}

Виртуальное свидетельство будет построено над алфавитом $\mathcal{A}_{\ev} = \mathcal{A}_1 \cap \mathcal{A}_2 = \{x_1, x_2\} \cap \{x_2, x_3\} = \{x_2\}$ и будет содержать следующий вектор оценок вероятностей квантов:
\begin{equation*}
\Pq^{\ev} = \begin{pmatrix}  p(\overline{x}_2) \\ p(x_2)	\end{pmatrix} = 
\begin{pmatrix} 0.28 \\ 0.72 \end{pmatrix}.
\end{equation*}

Пропагируем виртуальное свидетельство во второй фрагмент знаний с помощью уравнения~\ref{quantsGlob} и получим следующие оценки вероятностей:
\begin{equation*}
\Pq^{2,a} = \begin{pmatrix}
0.04 \\ 0.48 \\ 0.24 \\ 0.24
\end{pmatrix}.
\end{equation*}

Данный результат пропагации можно получить с помощью следующего фрагмента кода:

\begin{lstlisting}[caption = Пример пропагации виртуального свидетельства]
// Creating KPs and evidence
var firstKP = new ScalarQuantKP(Convert.ToInt64("011", 2), new[] { 0.1, 0.2, 0.4, 0.3 });
var secondKP = new ScalarQuantKP(Convert.ToInt64("110", 2), new[] { 0.1, 0.2, 0.6, 0.1 });
var evidence = new ScalarQuantKP(Convert.ToInt64("001", 2), new[] { 0.6, 0.4 });

// Creating propagators for stochastic and virtual evidences
var localPropagator = StochasticScalarQuantLocalPropagator.Instance(firstKP);
var virtualEvPropagator = new VirtualEvidenceScalarQuantPropagator();

// Propagating stochasic evidence in fisrt KP
var firstKPwithAposterioriEst = localPropagator.PropagateEvidence(evidence);
            
// Propagating virtual evidence in second KP
var secondKPwithAposterioriEst = virtualEvPropagator
    .PropagateStochasticEvidence(firstKPwithAposterioriEst, secondKP);
\end{lstlisting}

Вывод на консоль можно осуществить с помощью следующего кода:
\begin{lstlisting}[caption = Вывод результатов на консоль]
Console.WriteLine("Probability estimates in first KP after propagation:");
firstKPwithAposterioriEst
    .GetPointEstimate()
    .ToList()
    .ForEach(est => Console.Write(" " + est));
 
Console.WriteLine("Probability estimates in second KP after propagation:");
secondKPwithAposterioriEst
    .GetPointEstimate()
    .ToList()
    .ForEach(est => Console.Write(" " + est));
\end{lstlisting}

Результаты вывода на консоль:
\begin{lstlisting}[caption = Результаты]
Probability estimates in first KP after propagation:
 0,12 0,16 0,48 0,24
Probability estimates in second KP after propagation:
 0,04 0,48 0,24 0,24
\end{lstlisting}

\subsection{Выводы по главе}
Программная реализация, представленная в данной главе, реализует алгоритмы пропагирования виртуального свидетельства для различных моделей ФЗ, содержащих как скалярные, так и интервальные оценки. В главе описаны основные классы и методы и приведены примеры их использования.

\section*{Заключение}
\underline{Итогами}  квалификационной работы являются программная реализация алгоритма распространения виртуального свидетельства между двумя фрагментами знаний для различных моделей ФЗ в рамках существующего комплекса программ, а также:
\begin{enumerate}
    \item  Матрично"=векторная интерпретация алгоритма пропагации виртуального свидетельства между двумя фрагментами знаний: сформулированы матрично"=векторные уравнения для ФЗ, построенных над множествами квантов и над идеалами дизъюнктов, в частности, доказана теорема формирования матрицы перехода от вектора оценок вероятностей элементов к вектору виртуального свидетельства для модели идеала дизъюнктов и теорема о пропагации виртуального свидетельства между двумя ФЗ, построенными над множествами квантов, предложена матрица перехода от вектора вероятностей дизъюнктов к вектору вероятностей квантов; 
    \item Матрично"=векторная интерпретация функции $\Gind$, в частности, введено понятие характеристического вектора конъюнкта, дизъюнкта, кванта, а также понятие матрицы проекции $\G$;
    \item Внедрение алгоритмов локального ЛВВ для квантов и реинжиниринг существующих алгоритмов ЛВВ для идеалов дизъюнктов;
    \item Тесты, проверяющие корректность работы реализации;
    \item Примеры использования полученной реализации; 
    \item Вычислительные эксперименты, результаты которых согласуются с ожиданиями.
\end{enumerate}

\underline{Рекомендации к использованию.} Предложенная программная реализация является модулем объемлющей математической библиотеки, реализующей алгоритмы логико-вероятностного вывода в АБС и предоставляющей доступ к публичному контракту~\cite{50}. Поэтому реализованный модуль может быть переиспользован также в веб-приложении с целью визуализации алгоритма распространения виртуального свидетельства между двумя фрагментами знаний сети. Также программная реализация может быть переиспользована при дальнейшей реализации алгоритмов глобального логико"=вероятностного вывода, а именно распространения влияния свидетельства во все фрагменты знаний сети. Классы и методы могут быть использованы при проведении различных вычислительных экспериментов с целью изучения различных характеристик модели.

\underline{Перспективы дальнейших исследований.} Полученные теоретические результаты могут быть использованы для исследования устойчивости и чувствительности полученных уравнений и создают фундамент для развития теории глобального логико"=вероятностного вывода, в частности, для формализации алгоритмов распространения виртуального свидетельства между двумя фрагментами знаний с интервальными оценками и исследования распространения виртуального свидетельства для третичной глобальной структуры АБС~\cite{284}.

Разработанные примеры могут быть использованы в методических целях.


\setmonofont[Mapping=tex-text]{CMU Typewriter Text}
\bibliography{diploma}
\documentclass[14pt]{matmex-diploma-custom}

\usepackage{tocvsec2}
\usepackage{amssymb}
\usepackage{listings}
\usepackage{xcolor}
\usepackage[font=small,labelsep=period]{caption}
%%% \usepackage{concrete}
% Calligraphy letters
% use: \mathcal
\usepackage[mathscr]{eucal}
% Algorithms
% use: \begin{algorithm}
\usepackage{algorithm}
% use: \begin{algorithmic}
\usepackage[noend]{algpseudocode}
% Maths
% use: for correct control sequence
\usepackage{amsmath}
% Centreing
% use: \centering
\usepackage{varwidth}
% First indent
% use: automatically
\usepackage{indentfirst}

\definecolor{bluekeywords}{rgb}{0,0,1}
\definecolor{greencomments}{rgb}{0,0.5,0}
\definecolor{redstrings}{rgb}{0.64,0.08,0.08}
\definecolor{xmlcomments}{rgb}{0.5,0.5,0.5}
\definecolor{types}{rgb}{0.17,0.57,0.68}

\lstset{
    inputencoding=utf8x, 
    extendedchars=false, 
    keepspaces = true,
    language=[Sharp]C,
    captionpos=b,
   frame=lines, % Oberhalb und unterhalb des Listings ist eine Linie
    showspaces=false,
    showtabs=false,
    breaklines=true,
    showstringspaces=false,
    breakatwhitespace=true,
    basicstyle=\linespread{0.8}
    escapeinside={(*@}{@*)},
    commentstyle=\color{greencomments},
    morekeywords={partial, var, value, get, set},
    keywordstyle=\color{bluekeywords},
    stringstyle=\color{redstrings},
    basicstyle=\small\ttfamily,
}


\renewcommand{\lstlistingname}{Листинг}

\newtheorem{Th}{Теорема}[section]
\newtheorem{Def}{Определение}[section]
\newtheorem{Lem}{Утверждение}[subsection]

\newcommand{\underdot}[1]{\mathop{#1}\limits_{\cdot}}

\newenvironment{Proof} % имя окружения
{\par\noindent{\bf Доказательство.}} % команды для \begin
{\hfill$\scriptstyle\blacksquare$} % команды для \end



\newcommand{\Pc}{\mathbf{P}_\mathrm{c}}
\newcommand{\Pq}{\mathbf{P}_\mathrm{q}}
\newcommand{\In}{\mathbf{I}_n}
\newcommand{\Jn}{\mathbf{J}_n}
\newcommand{\Qmatr}{\mathbf{Q}}
\newcommand{\V}{\mathbf{V}}
\renewcommand{\G}{\mathbf{G}}
\renewcommand{\T}{\mathbf{T}}
\newcommand{\evidenceNumbers}{\langle i, j \rangle}
\newcommand{\Tij}{\T^{\evidenceNumbers}}
\newcommand{\TijTilda}{\widetilde{\T}^{\evidenceNumbers}}
\newcommand{\QijTild}{\widetilde{\Qmatr}^{\evidenceNumbers}}
\newcommand{\VijTilda}{\widetilde{\V}^{\evidenceNumbers}}
\newcommand{\rij}{\mathbf{r}^{\evidenceNumbers}}
\newcommand{\rijTilda}{\widetilde{\mathbf{r}}^{\evidenceNumbers}}
\newcommand{\ev}{\mathrm{ev}}
\newcommand{\Gind}{\mathrm{GInd}}
\newcommand{\Selector}{\mathbf{s}}
\newcommand{\sij}{\mathbf{s}^{\evidenceNumbers}}
\newcommand{\sijTilda}{\widetilde{\mathbf{s}}^{\evidenceNumbers}}
\newcommand{\Pd}{\mathbf{P}_\mathrm{d}}
\newcommand{\Ln}{\mathbf{L}_n}
\newcommand{\On}{\mathbf{F}_n}
\newcommand{\Kn}{\mathbf{K}_n}
\renewcommand{\M}{\mathbf{M}}
\newcommand{\Mij}{\M^{\evidenceNumbers}}
\newcommand{\dij}{\mathbf{d}^{\evidenceNumbers}}
\newcommand{\dijTilda}{\widetilde{\mathbf{d}}^{\evidenceNumbers}}
\newcommand{\MijTilda}{\widetilde{\mathbf{M}}^{\evidenceNumbers}}
\newtheorem{theorem}{Теорема}[section]
\newtheorem{definition}[theorem]{Определение}
 \setcounter{tocdepth}{2}
 \usepackage{lastpage}
 \usepackage{mathtools}
\begin{document}
% Год, город, название университета и факультета предопределены,
% но можно и поменять.
% Если англоязычная титульная страница не нужна, то ее можно просто удалить.
\filltitle{ru}{
    chair              = {Фундаментальная информатика и информационные технологии\\ Информационные технологии},
    title              = {Алгебраические байесовские сети:\\ синтез глобальных структур и алгоритмы логико-вероятностного вывода (проектная работа)},
    type               = {bachelor},
    position           = {студента},
    group              = 14.Б08-мм,
    author             = {Анна Викторовна Шляк},
    supervisorPosition = {проф. каф. инф., д. ф.-м. н., доц.},
    supervisor         = {Тулупьев А.\,Л.},
    reviewerPosition   = {проф. каф. инф., д. ф.-м. н., доц. },
    reviewer           = {Фильченков А. А.},
    chairHeadPosition  = {?},
    chairHead          = {?},
    faculty            = {Математико-механический факультет},
    city               = {Санкт-Петербург},
    year               = {2018}
}
\filltitle{en}{
    chair              = {Fundamental informatics and information technologies \\ Information technologies},
    title              = {Algebraic Bayesian networks: \\global structure synthesis and probabilistic-logic inference algorithms(project work)},
        type               = {bachelor},
    author             = {Anna Shliak},
    supervisorPosition = {Prof. Computer Science Department, Dc. Sc. in Math, Assoc. Prof.},
    supervisor         = {Alexander Tulupyev},
    reviewerPosition   = {As. Prof. Computer Technology Department, Ph. D. Sc. in Math},
    reviewer           = {Andrey Filchenkov},
    chairHeadPosition  = {?},
    chairHead          = {?},
}
\maketitle
\tableofcontents
\section*{Введение}
\underline{Актуальность темы.} В современном мире существует необходимость анализировать и обрабатывать большие объемы информации, однако данных не всегда бывает достаточно и могут появляться различные неопределенности, что значительно осложняет задачу обработки. Подход к обработке знаний с неопределенностью предлагает класс вероятностных графических моделей~(ВГМ), к которым относятся алгебраические байесовские сети~(АБС), байесовские сети доверия~(БСД), марковские сети и многие другие. Зависимости между данными в ВГМ задаются с помощью графов, а степень неопределенности данных характеризуется оценками вероятностей. ВГМ находят широкое применение в различных областях, например, байесовские сети доверия, родственные классу АБС, рассматриваемому в данной работе, используются в медицине~\cite{99}, оценке рисков~\cite{98}, области финансов~\cite{97} и других областях~\cite{96}. 

Алгебраические байесовские сети введены профессором В. И. Городецким~\cite{95, 94} и, благодаря многолетним исследованиям, активно развиваются и также занимают свое место в классе вероятностных графических моделей. АБС состоит из набора случайных элементов, соответствующих высказываниям, которым приписаны оценки вероятностей, и структуры связи между указанными высказываниями, которую представляют в виде графа. 

В АБС используется принцип декомпозиции знаний на фрагменты знаний~(ФЗ). Формирование оценок истинности и обработка данных с неопределенностью в АБС основана на локальном и глобальном логико"=вероятностном выводе~\cite{93}. Новая информация, поступающая во фрагменты знаний, основана на логико-вероятностной модели свидетельств~\cite{109}. Для представления фрагмента знаний в АБС описаны три математические модели фрагмента знаний, построенные над идеалами конъюнктов, идеалами дизъюнктов или множествами пропозиций-квантов~\cite{121}. 
 
 \underline{Степень разработанности темы исследований.} 
 Актуальной задачей является матрично-векторное описание алгоритмов логико"=вероятностного вывода в АБС, потому что матрично"=векторные операции относятся к классическому математическому инструментарию, упрощают программную реализацию и понимание работы алгоритмов, а также предоставляют возможности для дальнейшего исследования математической модели, например, исследование чувствительности уравнений. На сегодняшний день предложены матрично"=векторные уравнения для алгоритмов локального логико"=вероятностного вывода~\cite{91}, однако в них остаются элементы функционального вычисления~(функция $\Gind$). Матрично"=векторный подход для глобального вывода также нуждается в доработке~\cite{70}. Параллельно развитию теории разрабатывается комплекс программ для работы с АБС~\cite{89}, реализующий в себе  алгоритмы ЛВВ, однако, не включающий в себя на данный момент алгоритмы глобального логико"=вероятностного вывода, в частности, в нем отсутствует реализация алгоритма распространения виртуального свидетельства между двумя фрагментами знаний.

Таким образом, \underline{объектом исследования} являются алгебраические байесовские сети,
а \underline{предметом исследования} --- матрично-векторная формализация алгоритма распространения виртуального свидетельства между двумя фрагментами знаний в АБС.

\underline{Целью} данной выпускной квалификационной работы является автоматизация глобального логико-вероятностного вывода в алгебраических
байесовских сетях, а именно алгоритмов распространения виртуального свидетельства между двумя фрагментами знаний. 
Для достижения поставленной цели решаются следующие \underline{задачи}:
\begin{enumerate}
        \item  Развить матрично"=векторную формализацию алгоритмов распространения виртуального свидетельства для различных моделей фрагментов знаний со скалярными оценками;
        \item Предложить матрично"=векторную формализацию для функции $\Gind$;
        \item Осуществить интеграцию и реинженеринг алгоритмов локального логико"=вероятностного вывода для альтернативных моделей ФЗ в рамках комплекса программ;
    \item Реализовать алгоритмы распространения виртуального свидетельства в рамках комплекса программ;
    \item Провести вычислительные эксперименты по распространению виртуальных свидетельств и написать документацию.
\end{enumerate}

\underline{Научная новизна.} 
В данной выпускной квалификационной работе бакалавра предложена матрично-векторная формализация алгоритмов распространения виртуального свидетельства для ФЗ, построенных над идеалами дизъюнктов и наборами пропозиций-квантов, в частности, сформулирована и доказана теорема о матрично-векторном формировании матрицы перехода от вектора вероятностей элементов идеала дизъюнктов к вектору вероятностей элементов виртуального свидетельства, доказана теорема о пропагации виртуального свидетельства между двумя ФЗ, построенными над множествами квантов, введена матрица перехода от вектора вероятностей дизъюнктов к вектору вероятностей квантов. Также получена матрица проекции $\G$, заменяющая функцию $\Gind$, в частности, введены понятия характеристического вектора конъюнкта, дизъюнкта и кванта. 

\underline{Теоретическая и практическая значимость исследования.} Теоретическая значимость работы заключается в создании базы для развития матрично"=векторного подхода в глобальном ЛВВ. Практическая значимость заключается в создании основы для реализации алгоритмов распространения свидетельства во все ФЗ сети, решения задачи визуализации работы с АБС, а также для проведения вычислительных экспериментов.

\underline{Методы исследования.} Для решения поставленных задач в данной области понадобилось изучить методические материалы, поставить проблему, проанализировать ее, спроектировать несколько возможных вариантов решения. Затем были выбраны подходящие средства и технологии программирования, связанные с языком реализации~(C\#), средой разработки~(Rider), сервисом для совместной разработки~(BitBucket). После чего были проведены вычислительные и программные эксперименты с целью обоснования корректности  полученного решения. В качестве методов решения используются теоретические методы~(анализ предыдущих результатов, формализация алгоритмов) и методы объектно"=ориентированного программирования~(ООП).


\underline{На защиту выносятся следующие положения:} 
\begin{enumerate}
\item Матрично"=векторные уравнения распространения виртуального свидетельства между двумя фрагментами знаний для следующих моделей фрагмента знаний: идеал дизъюнктов, множество пропозиций квантов;
\item Матрично"=векторная формализация для функции $\Gind$;
\item Программная реализация алгоритмов распространения виртуального свидетельства для каждой из трех моделей.
\end{enumerate}

\underline{Достоверность полученных результатов.} 
Достоверность предложенных в работе результатов обеспечена корректным применением методов исследования, подтверждена вычислительными экспериментами и примерами работы программы. Полученные результаты не противоречат известным результатам других авторов.

\underline{Апробация результатов.} Результаты исследования докладывались на следующих научных мероприятиях:
\begin{enumerate}
\item Всероссийская научная конференция по проблемам информатики
СПИСОК~(Санкт-Петербург, апрель 2017);
\item VII-Всероссийская научно"=практическая конференция НСМВИТ-2017~(Санкт"=Петербург, июль 2017);
\item Научный семинар лаборатории теоретических и междисциплинарных проблем информатики в СПИИРАН~(Санкт-Петербург, ноябрь 2017).
\end{enumerate}

\underline{Публикации.} Результаты работы вошли в две публикации, была зарегистрирована одна заявка в Роспатенте.

\underline{Сведения о личном вкладе автора.}
Постановка цели и задач, а также выносимые на защиту результаты получены лично автором. 

Личный вклад А.В. Шляк в публикациях с соавторами характеризуется следующим образом: в ~\cite{9} --- краткое описание существующих алгоритмов и примеры вычисления апостериорных вероятностей после пропагации виртуального свидетельства, в ~\cite{70} --- постановка задач формализации глобального ЛВВ.

\underline{Структура и объем работы.} Работа состоит из введения, четырех
глав, заключения, списка используемой литературы и двух приложений. Общий объем работы составляет \pageref{LastPage} страниц. Список используемой литературы содержит 35 источников.

Во введении описана актуальность темы исследования и степень ее разработанности, сформулированы цель и задачи работы и представлены выносимые на защиту результаты.

Первая глава носит обзорный характер и состоит из пяти разделов. В первом разделе описаны общие сведения об АБС. Во втором разделе представлены основные инструменты для работы с АБС. В третьем разделе дан обзор существующих библиотек. В четвертом разделе представлен краткий обзор комплекса программ, в рамках которого осуществлялась разработка, а также приведены и обоснованы  цели и задачи работы. В пятом разделе представлены выводы по данной главе.

Вторая глава описывает основные теоретические результаты, используемые в данной работе. В первом разделе представляются основные элементы теории АБС. Во втором разделе даются основные определения. В третьем разделе рассматриваются виды локального логико"=вероятностного вывода, а в четвертом разделе --- глобальный логико"=вероятностный вывод. В пятом разделе перечислены недостатки имеющихся алгоритмов.

Третья глава содержит основные теоретические результаты, полученные в данной работе. В первом разделе дается краткий обзор полученных результатов. Во втором разделе предложена матрично"=векторная интерпретация функции $\Gind$, а в третьем разделе --- матрично"=векторная формализация матрицы перехода. В четвертом разделе представлены матрично-векторные уравнения распространения виртуального свидетельства для каждой из моделей ФЗ. В пятом разделе описаны выводы по главе.

Четвертая глава описывает программную реализацию алгоритмов из второй и третьей глав. Первый раздел описывает используемые программные технологии. Второй раздел описывает архитектуру комплекса программ. В третьем разделе описаны основные реализованные классы и методы, а в четвертом разделе приведены примеры их использования. В пятом разделе представлены выводы по текущей главе. 

В заключении содержатся итоги работы и дальнейшие перспективы исследования.

В приложении А содержатся вычислительные эксперименты по работе предложенных алгоритмов, а в приложении В --- список публикаций по теме работы.

\small{
Эта работа является частью более широких инициативных проектов, выполняющихся в лаборатории теоретических и междисциплинарных проблем информатики СПИИРАН под руководством А.Л.~Тулупьева; кроме того, разработки были частично поддержаны грантами РФФИ 15-01-09001-a~--- «Комбинированный логико"=вероятностный графический подход к представлению и обработке систем знаний с неопределенностью: алгебраические байесовские сети и родственные модели», 18-01-00626~--- «Методы представления, синтеза оценок истинности и машинного обучения в алгебраических байесовских сетях и родственных моделях знаний с неопределенностью: логико-вероятностный подход и системы графов».}

\normalsize
Работа выполнялась в рамках общей проектной работы вместе с А.А. Золотиным, А.Е. Мальчевской, А.И. Березиным.

\section{Автоматизация алгоритмов вывода в алгебраической байесовской сети}
\subsection{Введение}
Алгебраические байесовские сети относятся к классу вероятностных графических моделей и являются эффективным инструментом для обработки и представления знаний с неопределенностью~\cite{184}.     

 Будем считать, что знания формируют эксперты в определенной предметной области. Эксперты задают оценки вероятностей утверждениям, образующим базу знаний, и характеризуют связи между утверждениями с помощью оценок вероятностей. Из-за этого и возникает неопределенность знаний.

 Представим ситуацию, что экспертам необходимо охарактеризовать несколько утверждений и связи между ними. Пусть характеризуемые утверждения будут атомарными. Так как число взаимосвязей с ростом числа утверждений будет растет экспоненциально,  эффективность работы с этими утверждениями будет быстро падать.
 
 Отличительной особенностью алгебраических байесовских сетей является подход к декомпозиции области знаний на фрагменты знаний~\cite{93}: разобьем множество утверждений на подмножества, называемые фрагментами знаний, и будем характеризовать утверждения в каждом фрагменте знаний отдельно. Это позволит использовать быстрые алгоритмы обработки данных, не предъявляя серьезных требований к вычислительным мощностям. Также этот подход удобен при характеризации высказываний, потому что при ограниченном наборе элементов необходимо охарактеризовать связи лишь между несколькими высказываниями. 
 
 Таким образом, АБС состоит из фрагментов знаний и структуры связей между ними. На данный момент описаны три математические модели фрагмента знаний: идеал конъюнктов, идеал дизъюнктов и множество квантов~\cite{121}. Фрагмент знаний строится над алфавитом, где элементы алфавита соответствуют характеризуемым утверждениям. В данной работе будет рассматриваться вторичная структура АБС~\cite{87}.
 
 
\subsection{Инструменты для работы с алгебраическими байесовскими сетями}
Рассмотрим предоставляемые АБС инструменты для работы с данными, которые содержат неопределенности.

Для того чтобы охарактеризовать утверждение, эксперт может дать оценку вероятности истинности данного утверждения. Оценка может быть скалярной или интервальной. АБС позволяют задавать и обрабатывать как и скалярные, так и интервальные оценки вероятностей~\cite{184}. В данной работе в основном будут рассматриваться скалярные оценки вероятности истинности.

Задачи обработки данных в АБС решаются с помощью аппарата логико-вероятностного вывода~\cite{121}. Логико-вероятностный вывод делится на локальный и глобальный~\cite{51, 93}. Алгоритмы локального вывода работают с отдельным фрагментом знаний, а глобального --- со всей сетью. В данной работе будет кратко рассмотрен локальный ЛВВ и более подробно глобальный вывод. Логико-вероятностный вывод позволяет нам оценить вероятность пропозициональной формулы, основываясь на имеющихся уже оценках в сети, а также позволяет нам получить новые оценки элементов АБС при поступлении новых обуславливающих данных, называемых свидетельствами~\cite{84, 284}.
 
 Также существует аппарат для поддержки и проверки различных видов непротиворечивости в АБС, для того чтобы оценки вероятностей элементов в ФЗ не противоречили друг другу и были согласованы между собой. Подробнее с ним можно познакомиться в ~\cite{184, 121}.
 
\subsection{Библиотеки для работы с алгебраическими байесовскими сетями}
Развитием теории АБС активно занимается научный коллектив ТиМПИ СПИИРАН, в разное время включавший в себя А.Л. Тулупьева, А.В. Сироткина, А.А. Фильченкова и других исследователей. Были опубликованы монографии ~\cite{109, 1, 85}, диссертации~\cite{84, 184, 284}, статьи, посвященные тематике АБС и байесовских сетей в целом~\cite{94}.

Вместе с теорией создавались также библиотеки для работы с АБС, так в 2009 году была создана библиотека AlgBN Modeller j.v.01 на языке Java~\cite{124, 123, 122}. В данной библиотеке есть структуры для хранения фрагментов знаний, машины локального логико-вероятностного вывода и проверки непротиворечивости, а также машины глобального логико-вероятностного вывода. Однако библиотека позволяет осуществить глобальный логико-вероятностный вывод только для АБС, состоящей из ФЗ, построенных над идеалами конъюнктов с заданными оценками вероятностей их элементов.

Позже была создана библиотека AlgBN KPB Reconciler cpp.v.01, разработанная на языке C++~\cite{81}. В этой библиотеке также есть средства для осуществления локального логико"=вероятностного вывода и проверки непротиворечивости, а также структуры для хранения фрагментов знаний, однако алгоритмы глобального логико-вероятностного вывода в ней не реализованы.

Таким образом, в обеих библиотеках отсутствует полноценный функционал для проведения глобального логико-вероятностного вывода в АБС.
\subsection{Цели и задачи исследования}
Так как теория алгебраических байесовских сетей активно развивается, в 2015 году возникла потребность начать разработку комплекса программ AlgBN Math Library~\cite{89, 50}, представляющего собой библиотеку для работы с АБС. Так как алгоритмы ЛВВ приобретают матрично-векторную форму, позволяющую упростить программную реализацию, было принято решение о создании новых библиотек для работы с АБС, использующих современные матрично-векторные алгоритмы. На основе данных библиотек также создается приложение, визуализирующее АБС и позволяющее проводить вывод в сети. Комплекс программ реализован на языке C\# и платформе .Net, являющимися одними из популярнейших современных технологий в разработке программного обеспечения. 

На данный момент в комплексе программ реализованы структуры для хранения фрагментов знаний и АБС, алгоритмы локального логико-вероятностного вывода, а также алгоритмы проверки АБС и ФЗ на непротиворечивость. Реализации алгоритмов глобального логико-вероятностного вывода еще нет.

Что касается теории глобального логико"=вероятностного вывода, существующие алгоритмы обладают рядом недостатков~\cite{70}, к 
которым относится присутствие операций, не относящихся к матрично"=векторным вычислениям, которые усложняют алгоритмы и их программную реализацию.

Таким образом, целью данной выпускной квалификационной работы является автоматизация глобального логико-вероятностного вывода в алгебраических
байесовских сетях, а именно алгоритмов распространения виртуального свидетельства между двумя фрагментами знаний в АБС. 
Для достижения поставленной цели решаются следующие \underline{задачи}:
\begin{enumerate}
        \item  Развить матрично"=векторную формализацию алгоритмов распространения виртуального свидетельства для различных моделей фрагментов знаний с скалярными оценками;
        \item Предложить матрично"=векторную формализацию для функции $\Gind$;
        \item Осуществить интеграцию и реинженеринг алгоритмов локального логико"=вероятностного вывода для альтернативных моделей ФЗ в рамках комплекса программ;
    \item Реализовать алгоритмы распространения виртуального свидетельства в рамках комплекса программ;
\item Провести вычислительные эксперименты по распространению виртуальных свидетельств и написать документацию.
\end{enumerate}
\subsection{Выводы по главе}
В данной главе были рассмотрены основные понятия, используемые в теории алгебраических байесовских сетей и логико"=вероятностном вывода в АБС. Рассмотрены существующие программные реализации, их основные возможности и недостатки. Сформулированы задачи и цель данной выпускной квалификационной работы.

\section{Элементы теории алгебраических байесовских сетей}\
\subsection{Введение}
В данной главе введем основные математические определения и понятия, используемые в теории АБС, и дадим описание операций логико"=вероятностного вывода в АБС. Введем определения фрагмента знаний и алгебраической байесовской сети. Рассмотрим матрично"=векторные уравнения локального апостериорного вывода для каждой из трех моделей и стохастического свидетельства. Также представим имеющийся результат для глобального логико"=вероятностного вывода над идеалами конъюнктов. Более подробное описание введенных понятий, алгоритмов и обоснование формулировок можно найти в ~\cite{ 76, 91, 84,  184, 109, 1, 122, 284, 70}.

\subsection{Математические модели фрагмента знаний и алгебраической байесовской сети}
Зафиксируем алфавит $\mathcal{A} = \{x_i\}^n_{i=0}$ --- конечное множество атомарных пропозициональных формул~(атомов). Над алфавитом определим идеал конъюнктов, идеал дизъюнктов и множество квантов.
\begin{definition}[\cite{1}]
Идеал конъюнктов $C_\mathcal{A} $ --- это множество вида 
\begin{math}
\{x_{i_1} \wedge x_{i_2}\wedge ... \wedge x_{i_k} | \; 0 \leq i_1 < i_2 < ... < i_k \leq  n - 1, 0 \leq k \leq n\}
\end{math}.
\end{definition}

\begin{definition}[\cite{1}]
Квант над алфавитом $\mathcal{A} = \{x_i\}^n_{i=0}$ ---  это конъюнкция, которая для любого атома алфавита содержит либо этот атом, либо его отрицание.
\end{definition}

\begin{definition}[\cite{1}]
Литерал $\tilde{x}_i$ обозначает, что на его месте в формуле может стоять либо $x_i$, либо его отрицание $\bar{x}_i$.
\end{definition}

\begin{definition}[\cite{1}]
	Множество квантов $Q_\mathcal{A} $ --- это множество всех комбинаций вида $\{\tilde{x}_0\tilde{x}_1...\tilde{x}_{n-1}\}$.
\end{definition}

\begin{definition}[\cite{184}]
Идеал дизъюнктов $C_\mathcal{A} $ --- это множество вида 
\begin{math}
\{x_{i_1} \vee x_{i_2}\vee ... \vee x_{i_k} | \; 0 \leq i_1 < i_2 < ... < i_k \leq  n - 1, 0 \leq k \leq n\}
\end{math}.
\end{definition}

Введем правило нумерации на множестве конъюнктов и на множестве квантов.

Каждому кванту $\tilde{x}_0\tilde{x}_1...\tilde{x}_{n-1}$ поставим в соответствие двоичную запись, в которой на $i$-м месте будет стоять $1$, если $i$-тый литерал означен положительно, и $0$ иначе. Аналогично занумеруем конъюнкты: каждому конъюнкту $x_{i_1}x_{i_2}...x_{i_k}$ поставим в соответствие сумму $2^{i_1}+2^{i_2}+...+2^{i_k}$. Тогда, если представить полученную сумму в виде двоичной записи и дополнить лидирующими нулями до $n$ знаков, $i$-тый атом будет входить в конъюнкт тогда и только тогда, когда $i$-тый бит числа равен $1$. Таким образом, получится биективное отображение множества квантов на множество конъюнктов.

Для множества дизъюнктов правило нумерации будет аналогичным правилу для множества конъюнктов, следовательно, существует биективное отображение множества квантов и на множество дизъюнктов.

Теперь введем вектор вероятностей элементов идеала конъюнктов $\Pc$, вектор вероятностей элементов идеала дизъюнктов $\Pd$ и   вектор вероятностей элементов множества квантов $\Pq$: 
\begin{equation*}
\Pc = \begin{pmatrix}
	    p(c_0)  \\ p(c_1) \\ \vdots \\  p(c_{2^n-1})
      \end{pmatrix}, \,
\Pq = \begin{pmatrix}
	    p(q_0)  \\ p(q_1) \\ \vdots \\  p(q_{2^n-1})
      \end{pmatrix}, \,
\Pd = \begin{pmatrix}
	    p(d_0)  \\ p(d_1) \\ \vdots \\  p(d_{2^n-1})
      \end{pmatrix}. 
\end{equation*}

$\Pc$ и $\Pq$ связаны между собой соотношениями $\Pc = \Jn\Pq$ и $\Pq = \In\Pc$, где матрицы перехода $\In$ и $\Jn$ получены с помощью следующих рекуррентных соотношений~\cite{109}:

\begin{equation*}
    \In = \mathbf{I}_1 \otimes ... \otimes \mathbf{I}_{n-1} = \mathbf{I}_1^{[n]},
\end{equation*}
где $\mathbf{I}_1 = \begin{pmatrix*}[r] 1 & -1 \\ 0 & 1 \end{pmatrix*}$.

\begin{equation*}
    \Jn = \mathbf{J}_1 \otimes ... \otimes \mathbf{J}_{n-1} = \mathbf{J}_1^{[n]},
\end{equation*}
где $\mathbf{J}_1 = \begin{pmatrix} 1 & 1 \\ 0 & 1 \end{pmatrix}$, $\otimes$ обозначает кронекерово произведение матриц.

Вектор вероятностей элементов идеала дизъюнктов связан с $\Pc$ и $\Pq$ следующими соотношениями~\cite{76}:

\begin{equation}\label{pdtopq}
\Pq = \Ln(\mathbf{1} -\Pd),
\end{equation} 
где $\Ln = \begin{pmatrix*}[r] 0 & 1 \\ 1 & -1 \end{pmatrix*}^{[n]}$.

\begin{equation}
\Pc = \Kn(\mathbf{1} -\Pd),
\end{equation}
где $\Kn = \begin{pmatrix*}[r] 1 & 0 \\ 1 & -1 \end{pmatrix*}^{[n]}$.

Для удобства будем обозначать $\Pd^{\prime} = \mathbf{1} -\Pd$ и далее работать с вектором $\Pd^{\prime}$.

Дадим определение фрагмента знаний для трех моделей в случае, когда оценки вероятностей элементов скалярные.

\begin{definition}[\cite{1}]
Фрагмент знаний $\mathscr{C}$ над идеалом конъюнктов со скалярными оценками --- это пара вида $(C, \,p)$, где $C$ --- идеал конъюнктов, $p$ --- функция из $C$  в интервал $[0;1]$.
\end{definition}

\begin{definition}[\cite{1}]
	Фрагмент знаний со скалярными оценками непротиворечив тогда и только тогда, когда $\In\Pc \geq \mathbf{0}$.
\end{definition}

\begin{definition}[\cite{121}]
Фрагмент знаний $\mathscr{C}$ над идеалом дизъюнктов со скалярными оценками --- это пара вида $(C, \,p)$, где $C$ --- идеал дизъюнктов, $p$ --- функция из $C$  в интервал $[0;1]$.
\end{definition}

\begin{definition}[\cite{121}]
Фрагмент знаний $\mathscr{C}$ над множеством квантов со скалярными оценками --- это пара вида $(Q, \,p)$, где $Q$ --- множество квантов, лежащее в основе ФЗ, $p$ --- функция из $Q$  в интервал $[0;1]$.
\end{definition}

В качестве функции $p$ можно использовать вероятность истинности пропозиций.
\begin{definition}[\cite{1}]
Алгебраическую байесовскую сеть $\mathcal{N}$ определим как набор фрагментов знаний: $\mathcal{N}^{\circ} = \{\mathscr{C}_i\}_{i=1}^{n}$.
\end{definition}
\subsection{Локальный логико"=вероятностный вывод}
\subsubsection{Задачи локального логико"=вероятностного вывода}
Локальный логико"=вероятностный вывод делится на априорный и апостериорный~\cite{ 184, 121, 109}. Задачей локального априорного вывода является построение оценки истинности пропозициональной формулы, заданной над тем же алфавитом $\mathcal{A}$, что и данный фрагмент знаний $\mathscr{C}$. 
Для описания локального апостериорного вывода введем понятие свидетельства.
\begin{definition}[\cite{184}]
Под свидетельством мы понимаем новые «обуславливающие» данные, которые поступили во фрагмент знаний, и с учетом которых нам требуется пересмотреть все (или некоторые) оценки.  Для обозначения свидетельства будут использоваться угловые скобки --- $\langle ...\rangle$.
\end{definition}

Локальный апостериорный вывод решает 2 задачи: во-первых, оценить вероятности истинности свидетельства при данных оценках вероятности истинности элементов фрагмента знаний, и, во-вторых, оценить условные вероятности истинности элементов фрагмента знаний, предполагая, что свидетельство истинно.
Свидетельства бывают детерминированными, стохастическими и неточными~\cite{184, 109}. В данной работе в основном будут рассматриваться стохастические свидетельства, которые можно трактовать как набор детерминированных свидетельств в сочетании с заданным на них распределением вероятности.
\begin{definition}[\cite{184}]
	Детерминированное свидетельство --- это предположение, что один или несколько атомов получили конкретное означивание. 
\end{definition}

Детерминированное свидетельство обозначим $\langle i,j\rangle$, где $i$ --- индексы положительно означенных атомов, $j$ --- индексы отрицательно означенных атомов.	
\begin{definition}[\cite{184}]
Стохастическое свидетельство --- предположение о том, что над $C^\prime$ --- подыдеале $C$, задан непротиворечивый фрагмент знаний со скалярными оценками,  который определяет вероятности истинности элементов соответствующего подыдеала. Данное свидетельство обозначается $\langle (C^\prime, \Pc) \rangle$.
\end{definition}
\begin{definition}[\cite{184}]
	Неточное свидетельство --- это предположение о том, что над $C^\prime$ --- подыдеале $C$,  задан непротиворечивый фрагмент знаний с интервальными оценками, который определяет вероятности истинности элементов соответствующего подыдеала. Данное свидетельство обозначается $\langle (C^\prime, \Pc^-, \Pc^+) \rangle$.
\end{definition}

\subsubsection{Локальный апостериорный вывод над конъюнктами}

Приведем решение первой и второй задач локального апостериорного вывода, когда  фрагмент знаний $(C, \Pc)$ над идеалом конъюнктов $C$ содержит скалярные оценки и в него поступило стохастическое свидетельство $(C^{\ev}, \Pc^{\ev})$. 

Решение первой задачи~\cite{91}:
\begin{equation} \label{conjStochFirst}
  p(\langle C^{\ev}, \Pc^{\ev} \rangle) =\sum_{i=0}^{2^{n^\prime}-1}  (\mathbf{r}^{\langle \Gind(i,m),\,\Gind(2^{n^\prime}-1-i,m)\rangle},\Pc) \In\Pc^{a}[i],
\end{equation} 
где $\rij = \otimes^{0}_{k=n-1}\rijTilda_k$, \\
\begin{math}
    \rijTilda_k = 
    \begin{cases}
        \mathbf{r}^+ \text{, если $x_k$ входит в $c_i$,}\\
        \mathbf{r}^- \text{, если $x_k$ входит в $c_j$,}\\
        \mathbf{r}^\circ  \text{, иначе;}
    \end{cases}
\end{math}\\
\begin{math}
    \mathbf{r}^+ = \begin{pmatrix} 0 \\ 1 \end{pmatrix},
    \mathbf{r}^- = \begin{pmatrix*}[r] 1 \\ -1 \end{pmatrix*},
    \mathbf{r}^\circ = \begin{pmatrix} 1 \\ 0 \end{pmatrix}
\end{math},\\
$\Gind(i,m)$ --- функция, которая по индексу наибольшего элемента $C^{\ev}$ в алфавите $\mathcal{A}$ и индексу конъюнкта в алфавите $\mathcal{A}^{\ev}$ возвращает индекс соответствующего конъюнкта в алфавите $\mathcal{A}$.

Решение второй задачи апостериорного вывода находится по формуле~\cite{91}:
\begin{equation} \label{conj1}
\Pc^a=\sum_{i=0}^{2^{n^\prime}-1}\dfrac{\T^{\langle \Gind(i,m),\,\Gind(2^{n^\prime}-1-i,m)\rangle }\Pc}{(\mathbf{r}^{\langle \Gind(i,m),\,\Gind(2^{n^\prime}-1-i,m)\rangle},\Pc)}\In\Pc^{\ev}[i],
\end{equation} 
где $\Pc^a$ --- вектор апостериорных вероятностей истинности элементов данного фрагмента знаний,\\
\begin{math}
    \Tij = \TijTilda_{n-1}\otimes \TijTilda_{n-2} \otimes ...\otimes \TijTilda_{0}
\end{math}, \\
\begin{math}
    \TijTilda_k = 
    \begin{cases}
        \T^+ \text{, если $x_k$ входит в $c_i$,}\\
        \T^- \text{, если $x_k$ входит в $c_j$,}\\
        \T^\circ  \text{, иначе;}
    \end{cases}
\end{math} \\ 
\begin{math}
    \T^+ = \begin{pmatrix} 0 & 1 \\ 0 & 1 \end{pmatrix},
    \T^- = \begin{pmatrix*}[r] 1 & -1 \\ 0 & 0 \end{pmatrix*},
\T^\circ = \begin{pmatrix} 1 & 0 \\ 0 & 1 \end{pmatrix}.
\end{math}

\subsubsection{Локальный апостериорный вывод над дизъюнктами}

Приведем имеющиеся результаты для локального ЛВВ для модели ФЗ, построенной над идеалами дизъюнктов~\cite{76, 49}. Рассмотрим решение первой и второй задач  локального апостериорного вывода для стохастических свидетельств.

Решение первой задачи~\cite{49}:
\begin{equation}\label{dis1}
  p(\langle C^{\ev}, \Pd^{\prime \ev} \rangle) =\sum_{i=0}^{2^{n^\prime}-1}  (\mathbf{d}^{\langle \Gind(i,m),\,\Gind(2^{n^\prime}-1-i,m)\rangle},\Pd^{\prime}) \Ln\Pd^{\prime a}[i],
\end{equation}
где
$\dij = \otimes^{0}_{k=n-1}\dijTilda_k$,\\
\begin{math}
    \dijTilda_k = 
    \begin{cases}
        \mathbf{d}^+ \text{, если $x_k$ входит в $c_i$,}\\
        \mathbf{d}^- \text{, если $x_k$ входит в $c_j$,}\\
        \mathbf{d}^\circ  \text{, иначе;}
    \end{cases}
\end{math} \\ 
\begin{math}
    \mathbf{d}^+ = \begin{pmatrix*}[r] 1 \\ -1 \end{pmatrix*},
    \mathbf{d}^- = \begin{pmatrix} 0 \\ 1 \end{pmatrix},
    \mathbf{d}^\circ = \begin{pmatrix} 1 \\ 0 \end{pmatrix},
\end{math}\\
$\Pd^\prime = \mathbf{1} - \Pd$,\\
$\Gind(i,m)$ --- функция, которая по индексу наибольшего элемента $C^{\ev}$ в алфавите $\mathcal{A}$ и индексу дизъюнкта в алфавите $\mathcal{A}_{\ev}$ возвращает индекс соответствующего дизъюнкта в алфавите $\mathcal{A}$.

Решение второй задачи~\cite{49}:
\begin{equation}\label{dis2}
\Pd^{\prime a} =\sum_{i=0}^{2^{n^\prime}-1}\dfrac{\M^{\langle \Gind(i,m),\,\Gind(2^{n^\prime}-1-i,m)\rangle }\Pd^{\prime}}{(\mathbf{d}^{\langle \Gind(i,m),\,\Gind(2^{n^\prime}-1-i,m)\rangle},\Pd^{\prime})}\Ln\Pd^{\prime \ev}[i],
\end{equation}
где 
\begin{math}
    \Mij = \otimes^{0}_{k=n-1}\MijTilda_k
\end{math},\\
\begin{math}
    \MijTilda_k = 
    \begin{cases}
        \M^+ \text{, если $x_k$ входит в $c_i$,}\\
        \M^- \text{, если $x_k$ входит в $c_j$,}\\
        \M^\circ  \text{, иначе;}
    \end{cases}
\end{math} \\ 
\begin{math}
    \M^+ = \begin{pmatrix*}[r] 1 & -1 \\ 0 & 0 \end{pmatrix*},
    \M^- = \begin{pmatrix} 0 & 1 \\ 0 & 1 \end{pmatrix},
    \M^\circ = \begin{pmatrix} 1 & 0 \\ 0 & 1 \end{pmatrix}
\end{math}\\
и $\Pd^{\prime , \evidenceNumbers} = \mathbf{1} - \Pd^{\evidenceNumbers}$.

\subsubsection{Локальный апостериорный вывод над квантами}
Рассмотрим решение первой и второй задачи апостериорного локального ЛВВ, когда во фрагмент знаний со скалярными оценками, построенный над множеством пропозиций-квантов, пришло стохастическое свидетельство. Подробнее про локальный ЛВВ над квантами можно прочитать в~\cite{74}.

Решение первой задачи~\cite{74}:
\begin{equation} \label{quantsFirst}
  p(\langle C^{\ev}, \Pq^{\ev} \rangle) =\sum_{i=0}^{2^{n^\prime}-1}  (\Selector^{\langle \Gind(i,m),\,\Gind(2^{n^\prime}-1-i,m)\rangle},\Pq) \Pq^{a}[i],
\end{equation} 
где вектор-селектор $\sij = \otimes^{0}_{k=n-1}\sijTilda_k$, \\
\begin{math}
    \sijTilda_k = 
    \begin{cases}
        \Selector^+ \text{, если $x_k$ входит в $c_i$,}\\
        \Selector^- \text{, если $x_k$ входит в $c_j$,}\\
        \Selector^\circ  \text{, иначе;}
    \end{cases}
\end{math} \\
и 
\begin{math}
    \Selector^+ = \begin{pmatrix} 0 \\ 1 \end{pmatrix},
    \Selector^- = \begin{pmatrix} 1 \\ 0 \end{pmatrix},
    \Selector^\circ = \begin{pmatrix} 1 \\ 1 \end{pmatrix}
\end{math},\\
$\Gind(i,m)$ --- функция, которая по индексу кванта $i$ в алфавите, над которым построено свидетельство, и индексу $m$ наибольшего элемента $\Pq^a$ в исходном алфавите сопоставляет индекс свидетельства с индексом множества квантов поступившего свидетельства в исходном алфавите.

Теперь рассмотрим решение второй задачи~\cite{74}. Пусть дан фрагмент знаний $(C, \Pq)$ со скалярными оценками и стохастическое свидетельство $(C^{\ev}, \Pq^{\ev})$.

Решение второй задачи апостериорного вывода находится по формуле:
\begin{equation} \label{quants1}
 \Pq^{\evidenceNumbers}=\sum_{i=0}^{2^{n^\prime}-1}\dfrac{{\Selector^{\langle \Gind(i,m),\,\Gind(2^{n^\prime}-1-i,m)\rangle }}\circ \Pq}{(\Selector^{\langle \Gind(i,m),\,\Gind(2^{n^\prime}-1-i,m)\rangle},\Pq)}\Pq^{\ev}[i],
\end{equation}
где $\circ$ обозначает произведение Адамара~(операция покомпонентного произведения векторов одинаковой размерности).


\subsection{Глобальный апостериорный логико"=вероятностный вывод}
Рассмотрим связную ациклическую алгебраическую сеть~\cite{284} со скалярными оценками во всех фрагментах знаний. Предположим, что в один из фрагментов знаний поступило стохастическое свидетельство. Задачей глобального апостериорного вывода является распространение влияния этого свидетельства~(пропагация свидетельства) во все фрагменты знаний сети. Схема глобального апостериорного вывода подробно описана в~\cite{93}.
 Алгоритм пропагации состоит из 3 шагов: 
\begin{enumerate}
	\item  Пропагация свидетельства во фрагмент знаний, в который оно пришло, и оценка апостериорных вероятностей его элементов;
	\item Формирование виртуального свидетельства;
	\item Пропагация виртуального свидетельства в соседний фрагмент знаний.
\end{enumerate}

Аналогичным образом свидетельство пропагируется далее, пока не будут переозначены оценки элементов во всех фрагментах знаний.

Виртуальным свидетельством называется пересечение двух фрагментов знаний~(сепаратор), также являющийся фрагментом знаний~\cite{51, 93}. Так как после переозначивания оценок в начальном фрагменте знаний, конъюнкты, принадлежащие сепаратору, имеют новые оценки, потому что принадлежат начальному фрагменту знаний, но с другой стороны они также принадлежат соседнему фрагменту знаний и имеют другие оценки, то сепаратор можно рассмотреть как новую информацию, поступившую в соседний фрагмент знаний. 

Таким образом, на втором шаге алгоритма из вектора вероятностей начального фрагмента знаний необходимо выделить вектор значений, принадлежащих обоим фрагментам знаний, и принять его за новое свидетельство, которое на третьем шаге пропагируется в соседний фрагмент знаний, оценки которого необходимо переозначить.

Для того чтобы выделить необходимые оценки с помощью матрично-векторных вычислений, требуется сформировать матрицу перехода от вектора оценок вероятностей элементов ФЗ к вектору оценок вероятностей виртуального свидетельства.

Виртуальные свидетельства могут быть двух видов: стохастическое и неточное, потому что пересечение двух фрагментов знаний всегда является фрагментом знаний, но может содержать как и скалярные, так и интервальные оценки вероятностей.

Остановимся подробнее на втором шаге алгоритма и рассмотрим формирование и распространение виртуального свидетельства из одного фрагмента знаний в другой, когда они построены над идеалами конъюнктов. Считаем, что вектор $\Pc^{1, a}$ содержит апостериорные оценки вероятностей, потому что в него ранее было распространено влияние какого-то свидетельства. Распространим влияние этого свидетельства в соседний фрагмент знаний с оценками $\Pc^2$.

Рассмотрим, как выглядит матрица перехода $\Qmatr$ от вектора оценок вероятностей элементов ФЗ к вектору оценок вероятностей виртуального свидетельства. Матрица будет размерности $m \times n$~($m$ --- длина вектора $\Pc^{\ev}$,  $n$ --- длина вектора $\Pc^{1, a}$). Элементы матрицы определяются по следующему правилу~\cite{51, 9, 70}:
\begin{equation*}
    \Qmatr[i,j] = 
    \begin{cases}
        1 \text{, если $\Pc^{1, a}[j] = \Pc^{\ev}[i]$,}\\
        0 \text{, иначе;}\\
    \end{cases}
\end{equation*}

Под $\Pc^{1, a}[i]$ и $\Pc^{\ev}[i]$ подразумеваются сами конъюнкты, а не значения вероятностей. Затем нужно умножить матрицу на $\Pq^{1, a}$. Таким образом, $\Pc^{\ev} = \Qmatr\Jn\Pq^{1, a} = \Qmatr\Pc^{1, a}$.

Алфавит, над которым построено свидетельство, можно найти как $\mathcal{A}_{\ev} = \mathcal{A}_1 \cap \mathcal{A}_2$, где $\mathcal{A}_1$ --- алфавит первого фрагмента знаний, $\mathcal{A}_2$ --- алфавит второго фрагмента знаний.

Таким образом, пропагировать виртуальное свидетельство из одного ФЗ в другой можно с помощью следующего уравнения~\cite{51}:
\begin{equation} \label{conjglob1}
    \Pc^{2,a}=\sum_{i=0}^{2^{n^\prime}-1}\dfrac{\T^{\langle \Gind(i,m),\,\Gind(2^{n^\prime}-1-i,m)\rangle }\Pc^2}{(\mathbf{r}^{\langle \Gind(i,m),\,\Gind(2^{n^\prime}-1-i,m)\rangle},\Pc^2)}\In\Pc^{\ev}[i],
\end{equation}
где $\Pc^{\ev} = \Qmatr\Pc^{1, a}$, $\Pc^{1, a}$ и $\Pc^2$ --- векторы вероятностей элементов идеалов конъюнктов первого и второго фрагментов знаний, $\Pc^{2,a}$ --- вектор апостериорных вероятностей элементов идеала конъюнктов второго фрагмента знаний.

Матрицу перехода $\Qmatr$ можно также сформировать с помощью матрично"=векторных операций, а именно через кронекерово произведение:

\begin{theorem}[\cite{70}]
Вектор оценок виртуального свидетельства $\Pc^{\ev}$ можно вычислить как $\Pc^{\ev} = \Qmatr\Pc^{a,1}$, где $\Qmatr = \QijTild_{n-1} \otimes \QijTild_{n-2} \otimes ... \otimes \QijTild_{0}$,
 $\QijTild_k = \begin{cases}
\Qmatr^+ \text{, если $x_k$ входит в $\mathcal{A}^{\ev}$,} \\
\Qmatr^- \text{, иначе;}
\end{cases}$,
\begin{math}
    \Qmatr^+ = \begin{pmatrix} 1 & 0 \\ 0 & 1 \end{pmatrix},
    \Qmatr^- = \begin{pmatrix} 1 & 0 \end{pmatrix},
\end{math}\\ 
$x_k$ --- $k$-тый элемент в $\mathcal{A}_1$,  $\mathcal{A}_1$ --- алфавит, над которым построен $\Pc^{1,a}$, $\mathcal{A}^{\ev}$ --- алфавит, над которым построен $\Pc^{\ev}$.
\end{theorem}


\subsection{Выводы по главе}
В главе были даны основные определения и понятия, используемые в теории алгебраических байесовских сетей. Представлены определения фрагмента знаний, алгебраической байесовской сети. Рассмотрен логико"=вероятностный вывод в АБС: его виды и задачи, решаемые с помощью ЛВВ, а также основные теоретические результаты, которые ложатся в основу решения задач, рассматриваемых в данной работе.

Стоит отметить, что предложенный в данной главе подход обладает рядом недостатков~\cite{70}, а именно содержит в себе операции, не относящиеся к матрично"=векторным вычислениям. К таким операциям относится функция $\Gind$, для которой ранее не была предложена матрично"=векторная формализация. 

Таким образом, ставится задача улучшить имеющийся результат и заменить функцию $\Gind$ матрично"=векторным аналогом. Также имеющийся результат относится только к одной модели ФЗ и не подходит для двух других моделей, поэтому ставится задача описания алгоритмов глобального ЛВВ для альтернативных моделей фрагмента знаний.


\section{Матрично"=векторная  формализация глобального логико"=вероятностного вывода}\label{cap3}
\subsection{Введение}
В данной главе рассмотрим алгоритм распространения виртуального свидетельства между двумя фрагментами знаний для трех математических моделей ФЗ и получим соответствующие матрично"=векторные уравнения. 

Для того чтобы улучшить результат для модели идеала конъюнктов, представленный в предыдущей главе, и получить уравнения для модели идеала дизъюнктов и модели множества квантов, рассмотрим матрично-векторную формализацию функции $\Gind$. Для этого введем понятие характеристического вектора конъюнкта, дизъюнкта и кванта и матрицы проекции $\G$. Также рассмотрим матрично-векторное формирование матрицы перехода $\V$ от вектора вероятностей элементов идеала дизъюнктов к вектору вероятностей элементов виртуального свидетельства и приведем поясняющие примеры, введем матрицу перехода от вектора дизъюнктов к вектору квантов. Кроме этого докажем теорему, предлагающую матрично"=векторный подход к пропагации виртуального свидетельства в случае пропозиций-квантов. Используя полученные результаты, приведем получившиеся матрично-векторные уравнения для каждой из трех моделей фрагмента знаний. 
\subsection{Матрично"=векторная интерпретация для функции $\Gind$}
Рассмотрим функцию $\Gind(i,m)$ в контексте математической модели фрагмента знаний, заданного идеалом конъюнктов. Для двух других моделей формализация будет аналогичной, потому что сама функция $\Gind(i,m)$ определяется аналогичным образом~\cite{74, 49}, а алфавиты для всех трех моделей одинаковы.

Напомним, что  $\Gind(i,m)$ --- функция, которая по индексу $m$ наибольшего элемента $C^{\ev}$ в алфавите $\mathcal{A}$ и индексу конъюнкта $i$ в алфавите $\mathcal{A}_{\ev}$ возвращает индекс соответствующего конъюнкта в алфавите $\mathcal{A}$~\cite{84}. Так как индексы конъюнктов пронумерованы согласно строго заданному правилу, то у каждого индекса есть свой строго заданный номер, который можно рассмотреть в двоичной системе счисления или как характеристический вектор из нулей и единиц. Аналогично можно рассмотреть индексы квантов и дизъюнктов. 

\begin{definition}
Характеристический вектор конъюнкта --- это вектор из нулей и единиц, в котором единица на $i$-том месте соответствует единице на $i$-том месте в двоичной записи индекса конъюнкта, а ноль на $i$-том месте --- нулю.
\end{definition}

Будем обозначать характеристический вектор  конъюнкта с индексом $i$ как  $\chi^{i}$. Аналогичным будет обозначение для характеристического вектора дизъюнкта или кванта. Так как в данном разделе рассматриваются ФЗ над идеалами конъюнктов, будем далее рассматривать характеристический вектор конъюнкта.

Можно заметить, что функция $\Gind$ проецирует индекс, соответствующий конъюнкту детерминированного свидетельства, на другой индекс, соответствующий эквивалентному данному конъюнкту ФЗ, построенного над алфавитом $\mathcal{A}$. Вместо индексов можно взять соответствующие характеристические векторы и построить проекцию одного на другой.


Пусть $\chi^{i}$ --- характеристический вектор $i$-того конъюнкта свидетельства в алфавите  $\mathcal{A}_{\ev}$, и  пусть $\chi^{j}$ --- характеристический вектор соответствующего ему $j$-того конъюнкта в алфавите  $\mathcal{A}$. $m$ --- мощность алфавита $\mathcal{A}_{\ev}$, $n$ --- мощность алфавита $\mathcal{A}$.

Тогда $\Gind(i,m)$ можно заменить матрицей проекции вектора $\chi^{i}$ на $\chi^{j}$ по следующему правилу:
\begin{equation}\label{G}
\G[i,j] = \begin{cases}
1 \text{, если $\mathcal{A}[n -1-i] = \mathcal{A}_{\ev}[m - 1 -j]$,} \\
0 \text{, иначе};
\end{cases} 
\end{equation}

Такая матрица будет размера $n \times m$, где $n$ --- мощность алфавита $\mathcal{A}$ и $m$ --- мощность алфавита $\mathcal{A}_{\ev}$.

Чтобы получить характеристический вектор конъюнкта $\chi^{j}$ в алфавите $\mathcal{A}$, нужно будет домножить вектор $\chi^{i}$ слева на матрицу $\G$: $\chi^{j} = \G\chi^{i}$.

По построению матрица $\G$ будет содержать в себе нулевые строки, соответствующие элементам алфавита $\mathcal{A}$, которые не входят в алфавит $\mathcal{A}_{\ev}$. Единица на j-той позиции в ненулевой строке i соответствует атому $x_i \in \mathcal{A}$, эквивалентному атому $x_j \in \mathcal{A}_{\ev}$. Умножение нулевых строк на характеристический вектор будет давать нулевой элемент в результирующем векторе,  умножение ненулевых строк даст единицу, если элемент алфавита входит в конъюнкт, и ноль иначе. Таким образом, матрица $\G$ проецирует характеристический вектор конъюнкта, построенного над алфавитом $\mathcal{A}_{\ev}$ на характеристический вектор конъюнкта над алфавитом $\mathcal{A}$.

\subsubsection{Пример использования матрично-векторной интерпретации функции $\Gind$}
Приведем поясняющий пример. Пусть $\mathcal{A} =  \{x_1, x_2, x_3, x_4\}$ и $\mathcal{A}_{\ev} =  \{x_2, x_4\}$. Построим матрицу проекции $\G$ по формуле \ref{G}:

\begin{math}
\G[i,j] = \begin{cases}
1 \text{, если $\mathcal{A}[n -1-i] = \mathcal{A}_{\ev}[m - 1 - j]$,} \\
0 \text{, иначе};
\end{cases}
= \begin{pmatrix}
1 & 0 \\ 0 & 0 \\ 0 & 1\\ 0 & 0
\end{pmatrix}.
\end{math}

Пусть $\chi^{1} = \begin{pmatrix} 0 \\ 1 \end{pmatrix}$ и  соответствует конъюнкту $x_2$ свидетельства. Найдем характеристический вектор индекса конъюнкта $x_2$ в алфавите $\mathcal{A}$, домножив вектор $\chi^{1}$ на матрицу $\G$:

\begin{math}
\begin{pmatrix}
1 & 0 \\ 0 & 0 \\ 0 & 1\\ 0 & 0
\end{pmatrix}
\begin{pmatrix} 0 \\ 1 \end{pmatrix} 
= \begin{pmatrix} 0\\ 0 \\  1 \\ 0 \end{pmatrix}.
\end{math}

Видно, что искомый индекс получен.

\subsubsection{Уравнения для решения второй задачи апостериорного вывода}


Заменим в уравнениях \ref{conj1} и \ref{quants1}, приведенных во второй главе, функцию $\Gind$ на матрицу проекции $\G$. 

Уравнение \ref{conj1} для конъюнктов приобретет следующий вид:
\begin{equation} \label{conjG}
    \Pc^a=\sum_{i=0}^{2^{n^\prime}-1}\dfrac{\T^{\langle \G\chi^{i} ,\,\G\chi^{2^{n^\prime}-1-i} \rangle }\Pc}{(\mathbf{r}^{\langle  \G\chi^{i},\,\G\chi^{2^{n^\prime}-1-i} \rangle},\Pc)}\In\Pc^{\ev}[i].
\end{equation} 

Уравнение \ref{quants1} для квантов будет следующим:
\begin{equation} \label{quantsG}
 \Pq^{\evidenceNumbers}=\sum_{i=0}^{2^{n^\prime}-1}\dfrac{{\Selector^{\langle \G\chi^{i} ,\, \G\chi^{2^{n^\prime}-1-i}\rangle }}\circ \Pq}{(\Selector^{\langle\G \chi^{i} ,\, \G\chi^{2^{n^\prime}-1-i}\rangle},\Pq)}\Pq^{\ev}[i].
\end{equation}

Уравнение \ref{dis2} для дизъюнктов заменится следующим:
\begin{equation}\label{dis3}
 \Pd^{a, \prime} =\sum_{i=0}^{2^{n^\prime}-1}\dfrac{\M^{\langle  \G\chi^{i},\,\G\chi^{2^{n^\prime}-1-i} \rangle  }\Pd^{\prime}}{(\mathbf{d}^{\langle \G\chi^{i} ,\, \G\chi^{2^{n^\prime}-1-i}\rangle },\Pd^{\prime})}\Ln\Pd^{\prime \, \ev}[i].
\end{equation}
\subsection{Формализация формирования матрицы перехода}
\subsubsection{Теорема о построении матрицы перехода}
Рассмотрим фрагменты знаний, построенные над идеалами дизъюнктов. Аналогично случаю для конъюнктов~\cite{51, 9}, построим вектор вероятностей элементов виртуального свидетельства. В виртуальное свидетельство будут входить элементы, стоящие на пересечении двух фрагментов знаний.

Сначала введем матрицу перехода от вектора вероятностей квантов к вектору вероятностей дизъюнктов. Воспользуемся приведенным во второй главе выражением ~\ref{pdtopq} для матрицы перехода $\Ln$ от вектора $\Pd^{\prime}$ к вектору $\Pq$. Так как $\Ln$ имеет обратную матрицу, то домножим выражение для перехода от вектора $\Pd^{\prime}$ к вектору $\Pq$ на обратную матрицу и получим: 
\begin{equation*}
 \Ln^{-1} \Pq  = \mathbf{1} -\Pd,
\end{equation*}
где $\Ln^{-1} = \begin{pmatrix} 1 & 1 \\ 1 & 0 \end{pmatrix}^{[n]}.$

Обозначим $\On = \Ln^{-1}$. Перепишем выражение:
\begin{equation*}\mathbf{1} -\Pd = \On \Pq, \end{equation*} где $\On = \begin{pmatrix} 1 & 1 \\ 1 & 0 \end{pmatrix}^{[n]}.$

Теперь построим матрицу перехода $\V$ от вектора вероятностей  элементов идеала дизъюнктов к вектору вероятностей элементов виртуального свидетельства. Выделим из вектора $\Pd^{\prime 1,a}$, содержащего новые апостериорные оценки, элементы, принадлежащие виртуальному свидетельству.

 От вектора вероятностей квантов можно перейти к вектору вероятностей дизъюнктов с помощью матрицы перехода: $\Pd^{\prime 1,a} = \On\Pq^{1,a}$. При умножении $i$-той строки $\On[i]$ на $\Pq^{1,a}$, получается $i$-тый элемент вектора $\Pd^{\prime 1,a}$. Если выделить в отдельную матрицу строки $\On[i]$, выделяющие из $\Pd^{\prime 1,a}$ элементы виртуального свидетельства, то при умножении эту матрицы на $\Pq^{1,a}$, получился нужный нам вектор оценок вероятностей.

Для того чтобы выделить строки $\On[i]$, нужно домножить $\On$ слева на матрицу $\V$ размерности $m \times n$~($m$ --- длина вектора $\Pd^{\prime \, \ev}$,  $n$ --- длина вектора $\Pd^{\prime 1,a}$). Элементы матрицы определим по следующему правилу:
\begin{equation*}
    \V[i,j] = 
    \begin{cases}
        1 \text{, если $\Pd^{\prime 1,a}[j] = \Pd^{\prime \, \ev}[i]$,}\\
        0 \text{, иначе;}
    \end{cases}
\end{equation*}

При этом под $\Pd^{\prime 1,a}[i]$ и $\Pd^{\prime \, \ev}[i]$ подразумеваются сами дизъюнкты, а не значения вероятностей. Затем нужно умножить матрицу на $\Pq^{1,a}$. Таким образом, $\Pd^{\prime \, \ev} = \V\On\Pq^{1,a} = \V\Pd^{\prime 1,a}$.

Алфавит, над которым построено свидетельство, можно найти как $\mathcal{A}_{\ev} = \mathcal{A}_1 \cap \mathcal{A}_2$, где $\mathcal{A}_1$ --- алфавит первого фрагмента знаний, $\mathcal{A}_2$ --- алфавит второго фрагмента знаний.

Теперь покажем, что матрицу перехода $\V$ можно построить через кронекерово произведение, и обоснуем предложенное построение.

\begin{theorem}\label{disq}
Вектор оценок виртуального свидетельства $\Pd^{\prime \ev}$ можно вычислить как $\Pd^{\prime \ev} = \V \Pd^{\prime a,1}$, где $\V = \VijTilda_{n-1} \otimes \VijTilda_{n-2} \otimes ... \otimes \VijTilda_{0}$ ,
 $\VijTilda_k = \begin{cases}
\V^+ \text{, если $x_k$ входит в $\mathcal{A}^{\ev}$,} \\
\V^- \text{, иначе;}
\end{cases}$, 
\begin{math}
    \V^+ = \begin{pmatrix} 1 & 0 \\ 0 & 1 \end{pmatrix},
    \V^- = \begin{pmatrix} 1 & 0 \end{pmatrix},
\end{math}\\
$x_k$ --- $k$-тый элемент в $\mathcal{A}_1$,  $\mathcal{A}_1$ --- алфавит, над которым построен $\Pd^{\prime 1,a}$, $\mathcal{A}^{\ev}$ --- алфавит, над которым построен $\Pd^{\prime \ev}$.
\end{theorem}
\begin{Proof}
Для упрощения доказательства дополним матрицу $\V^-$ строкой из нулей и будем рассматривать построение матрицы $\V$ через матрицы $\V^+$ и $\V^{\prime-}$, где $\V^{\prime-} = \begin{pmatrix} 1 & 0 \\ 0 & 0 \end{pmatrix}$. Тогда искомая матрица будет квадратной и диагональной по построению и будет содержать строки из нулей. 

Также можно заметить, что $i$-тая единица на диагонали означает, что оценка на $i$-том месте в векторе вероятностей входит в виртуальное свидетельство. Ноль же означает, что оценка не относится к вектору виртуального свидетельства. 

Обоснуем, что позиции нулей и единиц при таком построении действительно показывают, входит ли оценка в виртуальное свидетельство.

Рассмотрим на каких позициях диагонали матрицы $\V$ будут нули.
По построению каждое произведение Кронекера увеличивает размерность искомой матрицы вдвое.
Можно заметить, что нули могут появиться только при участии в произведении матрицы $\V^{\prime-}$, в которой есть единственный $0$, находящийся на диагонали.

Получается, что по построению матрица $\V^{\prime-}$, соответствующая условию $x_m \not \in \mathcal{A}^{\ev}$, в
произведении Кронекера даст нули на всех позициях $\V[i,i]$, где
$i\&2^{m+1} = 2^{m+1}$, где $\&$ обозначает операцию побитового И. Множество таких чисел соответствует множеству чисел, в двоичной записи которых на месте $m$ стоит единица~(Нумерация мест в двоичной записи и нумерация атомов в алфавите начинается с 0). Сравнив с правилом нумерации дизъюнктов, получим, что все такие числа соответствуют номерам дизъюнктов, в которых присутствует атом с номером $m$. 

Следовательно, участие матриц $\V^{\prime-}$ в кронекеровом произведении занулит вероятности в результирующем векторе $\Pd^{\ev}$ всех дизъюнктов $d_m$, для которых не выполняется условие $\forall x_k \in d_m \, x_k \in \mathcal{A}_{\ev}$. А умножение $\Pd^{1,a}$ на столбец матрицы с единицей на диагонали как раз выделит в результирующий вектор нужную оценку вероятности.

Осталось разобраться с размерностью результирующего вектора $\Pd^{\ev}$. Он получается одинаковой размерности с исходным вектором $\Pd^{ 1,a}$ и содержит в себе нулевые элементы, возникающие, как раз, из-за видоизменения матрицы $\V^-$. Тогда при замене матрицы  $\V^{\prime-}$ на $\V^-$ получим вектор $\Pd^{\ev}$ нужной размерности. Так как $\Pd^{\prime}$ линейно выражается через $\Pd$, то результат будет корректным и для векторов $\Pd^{\prime 1,a}$  и $\Pd^{\prime \ev}$.
\end{Proof}

\subsubsection{Пример построения матрицы перехода}
 Рассмотрим на примере формирование матрицы $\V$ с помощью предложенного матрично-векторного алгоритма. Пусть даны два фрагмента знаний над алфавитами $\mathcal{A}_1 = \{x_1, x_2, x_3\}$ и $\mathcal{A}_2 = \{x_1, x_3, x_4\}$. Будем считать, что первый фрагмент знаний содержит апостериорные оценки после пропагации свидетельства и нужно пропагировать свидетельство дальше из первого ФЗ во второй. 
Выделим элементы, относящиеся к виртуальному свидетельству, построив матрицу перехода $\V$:
\begin{equation*}
\Pd^{\prime 1} =  \begin{pmatrix}
1 \\ p(\overline{x}_1) \\ p(\overline{x}_2) \\ p(\overline{x}_2\overline{x}_1) \\ p(\overline{x}_3) \\ p(\overline{x}_3\overline{x}_1) \\ p(\overline{x}_3\overline{x}_2) \\ p(\overline{x}_3\overline{x}_2\overline{x}_1)
\end{pmatrix}, 
\Pd^{\prime 2} =  \begin{pmatrix}
1 \\ p(\overline{x}_1) \\ p(\overline{x}_3) \\ p(\overline{x}_3\overline{x}_1) \\ p(\overline{x}_4) \\ p(\overline{x}_4\overline{x}_1) \\ p(\overline{x}_4\overline{x}_3) \\ p(\overline{x}_4\overline{x}_3\overline{x}_1)
\end{pmatrix},
\Pd^{\prime \ev} =  \begin{pmatrix}
1 \\ p(\overline{x}_1) \\ p(\overline{x}_3) \\ p(\overline{x}_3\overline{x}_1)
\end{pmatrix},
\end{equation*}
$\mathcal{A}_{\ev} = \mathcal{A}_1 \cap \mathcal{A}_2 = \{x_1, x_3\}$.
 
 Тогда $\V = \V^+ \otimes \V^- \otimes \V^+ = 
 \begin{pmatrix} 1 &0 \\ 0 &1 \end{pmatrix} \otimes
  \begin{pmatrix} 1 &0  \end{pmatrix} \otimes
 \begin{pmatrix} 1 &0 \\ 0 & 1 \end{pmatrix} 
 = \\
 \begin{pmatrix} 1 &  0 & 0 & 0 & 0 &  0 & 0 &  0 & 0 \\
                 0 & 1 & 0 & 0 & 0 & 0 & 0 & 0 & 0 \\
                 0 & 0 & 0 & 0 & 1 & 0 & 0 & 0 & 0 \\
                 0 & 0 & 0 & 0 & 0 & 1 & 0 & 0 & 0 \end{pmatrix} $.
                 
                 
Видно, что единицы на $i,j$-тых местах соответствуют выполнению равенства $\Pd^{\prime a,1} [i]= \Pd^{\prime \ev }[j]$, то есть таким построением была получена действительно матрица перехода от вектора $\Pd^{\prime a,1} $ к вектору $\Pd^{\prime \ev }$.

\subsection{Уравнения глобального логико"=вероятностного вывода}
\subsubsection{Матрично-векторное уравнение пропагации виртуального свидетельства над идеалом конъюнктов}

Перепишем приведенное во второй главе уравнение \ref{conjglob1} для распространения виртуального свидетельства между двумя фрагментами знаний с учетом описанной выше матрично"=векторной формализации функции $\Gind$. Заменим функцию $\Gind$ на матрицу проекции $\G$, тогда уравнение примет следующий вид:
\begin{equation} \label{conjglob2}
    \Pc^a=\sum_{i=0}^{2^{n^\prime}-1}\dfrac{\T^{\langle  \G\chi^{i},\,\G\chi^{2^{n^\prime}-1-i} \rangle }\Pc}{(\mathbf{r}^{\langle  \G\chi^{i},\,\G\chi^{2^{n^\prime}-1-i} \rangle},\Pc)}\In\Pc^{\ev}[i],
\end{equation} 
где \begin{math}\G[i,j] = \begin{cases}
1 \text{, если $\mathcal{A}_2[n -1-i] = \mathcal{A}_{\ev}[n^{\prime} - 1  - j]$,} \\
0 \text{, иначе;}
\end{cases}
\end{math}
\\ $n^{\prime}$ --- размерность алфавита $\mathcal{A}_{\ev}$, $n$ --- размерность алфавита $\mathcal{A}_2$.

\subsubsection{Матрично-векторное уравнение пропагации виртуального свидетельства над идеалом дизъюнктов}

Рассмотрим алгоритм пропагации виртуального свидетельства из одного фрагмента знаний в другой. Воспользуемся теоремой \ref{disq} и  вычислим $\Pd^{\prime \, \ev} = \V\Pd^{\prime}$.

Далее, основываясь на уравнении \ref{dis3} и подставив в него выражение для вычисления вектора вероятностей элементов виртуального свидетельства, а также заменив функцию $\Gind$ матрицей $\G$, получим, что пропагировать виртуальное свидетельство из одного ФЗ с апостериорными оценками в соседний ФЗ можно с помощью следующего уравнения:
\begin{equation}\label{disGlob}
    \Pd^{\prime \, 2,a}=\sum_{i=0}^{2^{n^\prime}-1}\dfrac{\M^{\langle\G\chi^{i},\, \G \chi^{2^{n^\prime}-1-i} \rangle }\Pd^{\prime \, 2}}{(\mathbf{d}^{\langle \G \chi^{i},\,\G\chi^{2^{n^\prime}-1-i} \rangle},\Pd^{\prime \, 2})}\Ln\Pd^{\prime \, \ev}[i],
\end{equation}
где \begin{math}\G[i,j] = \begin{cases}
1 \text{, если $\mathcal{A}_2[n -1-i] = \mathcal{A}_{\ev}[n^{\prime} - 1 - j]$,} \\
0 \text{, иначе;}
\end{cases}
\end{math}
\\ $n^{\prime}$ --- размерность алфавита $\mathcal{A}_{\ev}$, $n$ --- размерность алфавита $\mathcal{A}_2$.

\subsubsection{Матрично-векторное уравнение пропагации виртуального свидетельства над множеством квантов}

Рассмотрим распространение виртуального свидетельства для случая, когда фрагменты знаний построены над множеством пропозиций"=квантов со скалярными оценками. Для того чтобы получить необходимое уравнение, необходимо научиться находить значения оценок вероятностей виртуального свидетельства $\Pq^{\ev}$. Под виртуальным свидетельством будем понимать фрагмент знаний, содержащий оценки вероятностей для квантов, согласованных как и с первым, так и со вторым фрагментом знаний.
 \begin{theorem}
Пропагировать виртуальное свидетельство из одного фрагмента знаний во второй, когда ФЗ построены над множествами пропозиций-квантов, можно с помощью следующего уравнения:
\begin{equation}\label{quantsGlob}
  \Pq^2=\sum_{i=0}^{2^{n^\prime}-1}\dfrac{{\Selector_2^{\langle  \G_{\mathcal{A}_2, \mathcal{A}_{\ev}} \chi^{i},\,\G_{\mathcal{A}_2, \mathcal{A}_{\ev}}\chi^{2^{n^\prime}-1-i} \rangle }}\circ \Pq^2}{(\Selector_2^{\langle  \G_{\mathcal{A}_2, \mathcal{A}_{\ev}}\chi^{i},\, \G_{\mathcal{A}_2, \mathcal{A}_{\ev}}\chi^{2^{n^\prime}-1-i}\rangle},\Pq^2)}(\Selector_1^{\langle  \G_{\mathcal{A}_1, \mathcal{A}_{\ev}}\chi^{i},\,\G_{\mathcal{A}_1, \mathcal{A}_{\ev}}\chi^{2^{n^\prime}-1-i} \rangle},\Pq^1),
\end{equation}
где $\Pq^1$ --- вектор апостериорных оценок первого ФЗ, а $\Pq^2$ --- вектор оценок второго ФЗ, куда нужно пропагировать свидетельство, $n^\prime$ --- мощность алфавита свидетельства $\mathcal{A}^{\ev}$, $m$ --- мощность алфавита $\mathcal{A}_1$, а $n$ --- мощность алфавита $\mathcal{A}_2$,\\
\begin{math}
\G_{\mathcal{A}_2, \mathcal{A}_{\ev}}[i,j] = 
\begin{cases}
1 \text{, если $\mathcal{A}_2[n-1 -i] = \mathcal{A}_{\ev}[n^\prime -1- j]$,} \\
0 \text{, иначе;}
\end{cases}
\end{math}\\
\begin{math}\G_{\mathcal{A}_1, \mathcal{A}_{\ev}}[i,j] = \begin{cases}
1 \text{, если $\mathcal{A}_1[m -1-i] = \mathcal{A}_{\ev}[n^\prime -1- j]$,} \\
0 \text{, иначе;}
\end{cases}
\end{math}
\end{theorem}
\begin{Proof}
Рассмотрим задачу нахождения оценок вероятностей элементов виртуального свидетельства $\Pq^{\ev}$. Можно заметить, что вектор $\Pq^{\ev}$ будет содержать оценки вероятностей для квантов, согласованных как и с первым, так и со вторым фрагментом знаний.

Алфавит, над которым построено свидетельство, можно найти как $\mathcal{A}_{\ev} = \mathcal{A}_1 \cap \mathcal{A}_2$, где $\mathcal{A}_1$ --- алфавит первого фрагмента знаний, $\mathcal{A}_2$ --- алфавит второго фрагмента знаний.

Таким образом, виртуальное свидетельство состоит из множества пропозиций-квантов, построенных над алфавитом $\mathcal{A}_{\ev}$. Для каждого кванта необходимо посчитать его вероятность, используя апостериорные оценки из первого фрагмента знаний. Это является первой задачей апостериорного вывода, если мы рассматриваем квант как детерминированное свидетельство $\evidenceNumbers$. Решение первой задачи апостериорного вывода выглядит следующим образом~\cite{74}:
\begin{equation*}
p(\langle i, j \rangle) = (\Selector^{\evidenceNumbers},\Pq^1).\end{equation*}

Значит, воспользуемся матрицей проекции $\G$ и получим значение вероятности каждого $i$-того кванта виртуального свидетельства:
\begin{equation}\label{quantsVirt}
\Pq^\ev[i] =(\Selector^{\langle  \G_{\mathcal{A}_1, \mathcal{A}_{\ev}} \chi^{i},\, \G_{\mathcal{A}_1, \mathcal{A}_{\ev}} \chi^{2^{n^\prime}-1-i}\rangle},\Pq^1),
\end{equation}
где $n^\prime$ --- мощность алфавита свидетельства $\mathcal{A}^{\ev}$, а $m$ --- мощность алфавита $\mathcal{A}_1$
и $\G_{\mathcal{A}_1, \mathcal{A}_{\ev}}[i,j] = \begin{cases}
1 \text{, если $\mathcal{A}_1[m -1-i] = \mathcal{A}_{\ev}[n^\prime - 1 -j]$,} \\
0 \text{, иначе;}
\end{cases}$

 В нижнем индексе матрицы $\G$ явно укажем алфавиты, участвующие в ее формировании, чтобы в дальнейшем различать различно сформированные матрицы в одном уравнении. У вектора-селектора также добавим нижние индексы $1$ и $2$, указывающие, элементы какого алфавита используются при его построении. 
 
Далее воспользуемся уравнением \ref{quantsG} для решения второй задачи апостериорного вывода для стохастического свидетельства, подставим в него выражение \ref{quantsVirt} и получим искомое уравнение. 
\end{Proof}
\subsection{Выводы по главе}
В данной главе была предложена матрично"=векторная формализация для функции $\Gind$ и предложены матрично"=векторные уравнения распространения виртуального свидетельства между двумя соседними фрагментами знаний для двух моделей ФЗ: идеал дизъюнктов, множество пропозиций-квантов. Была сформулирована и доказана теорема о формировании матрицы перехода от вектора оценок элементов ФЗ над идеалом дизъюнктов к вектору оценок виртуального свидетельства, а также доказана теорема о пропагации виртуального свидетельства между двумя ФЗ, построенными над множествами квантов. Введен характеристический вектор конъюнкта, дизъюнкта,  кванта и матрица проекции $\G$, матрица перехода от вектора вероятностей дизъюнктов к вектору вероятностей квантов. Также приведены уравнения локального ЛВВ и уравнение пропагации виртуального свидетельства между двумя ФЗ над идеалами конъюнктов после замены функции $\Gind$  на матрицу $\G$.


\section{Программная реализация}
\subsection{Введение}
В предыдущих главах были описаны основные имеющиеся и полученные теоретические результаты, а в данной главе будет рассмотрена программная реализация. Будет описана архитектура существующего комплекса программ, представлены основные классы и методы для реализации алгоритмов глобального логико-вероятностного вывода для различных моделей ФЗ, а также приведены примеры работы программы.

Полученная в ходе работы программная реализация разработана на языке C\#, в качестве среды разработки использовалась среда JetBrains Rider~\cite{68}, которая обладает меньшей функциональностью в отличие от Visual Studio, но отличается удобством и быстротой работы. Для совместной работы использовался репозиторий на BitBucket~\cite{111} и система контроля версий Git. Для тестирования использовалась библиотека NUnit~\cite{67}, а для работы с матрицами использовалась библиотека Math.Net Numerics~\cite{66}.
\subsection{Архитектура программного комплекса}
Комплекс программ AlgBN Math Library~\cite{89} представляет собой библиотеку для автоматизации логико"=вероятностного вывода в АБС, в которой уже реализованы структуры данных для хранения фрагмента знаний и АБС, а также алгоритмы локального логико-вероятностного вывода и проверки непротиворечивости. Фрагменты знаний можно создавать с бинарными, скалярными и интервальными оценками вероятностей. Рассмотрим наиболее важные структуры, имеющиеся в библиотеке.

 Фрагмент знаний, представленный идеалом конъюнктов с скалярными оценками представлен классом ScalarConjKP, с интервальными оценками --- IntervalConjKP, с бинарными --- BinaryConjKP. Для создания экземпляра любого из этих классов нужно передать в конструктор глобальный индекс ФЗ и массив оценок вероятностей элементов ФЗ. Глобальный индекс --- это число, единицы в двоичной записи которого соответствуют номерам элементов алфавита, над которыми построен ФЗ. Аналогичные классы есть для двух других моделей ФЗ: для дизъюнктов --- BinaryDisjKP, ScalarDisjKP и IntervalDisjKP, для квантов --- BinaryQuantKP, ScalarQuantKP и IntervalQuantKP.

Структура Propagator состоит из классов, позволяющих пропагировать детерминированное, стохастическое или неточное свидетельство в ФЗ с скалярными или интервальными оценками. Каждый из таких классов содержит методы propagate для пропагации свидетельства и getResult для возвращения результата пропагации. Каждая реализация отвечает за пропагацию одного из типов свидетельства в ФЗ с одним видом оценок. Например, StochasticScalarConjunctsLocalPropagator отвечает за пропагацию стохастического свидетельства в ФЗ с скалярными оценками. Структура Propagator позволяет работать с двумя моделями ФЗ --- идеал конъюнктов и идеал дизъюнктов. Для третьей модели функциональность еще не реализована, а алгоритмы, работающие с идеалами дизъюнктов, требуют доработки.

За проверку локальной непротиворечивости отвечает структура In\-ferrer. Структура MatrixTransform реализует матрицы перехода между различными моделями ФЗ. Для решения задач линейного программирования используется библиотека на C++ lp\_solve55, за обращение к библиотеке отвечает класс LP. Кроме того, есть структура Alphabet для работы с алфавитом.

Также в комплексе есть программный модуль ABNGlobal, отвечающий за глобальную непротиворечивость. Так как пропагация виртуального свидетельства относится к глобальному ЛВВ, то расширим эту часть библиотеки своей реализацией. 

Реализация будет использовать существующие структуры для локального ЛВВ. Так как существующие алгоритмы позволяют работать с интервальными оценками, то реализация алгоритмов пропагации виртуального свидетельства будет разработана как и для ФЗ с интервальными оценками, так и для неточных виртуальных свидетельств. Существующую реализацию~\cite{4, 65} алгоритмов  локального логико"=вероятностного вывода для квантов перенесем на существующие в комплексе программ структуры хранения, адаптируем и затем переиспользуем при реализации алгоритма пропагации виртуального свидетельства для данной модели. Реализации Propagator для дизъюнктов также адаптируем и усовершенствуем с целью дальнейшего переиспользования.



\subsection{Описание классов и методов}
Для того чтобы реализовать алгоритмы распространения виртуального свидетельства, была расширена функциональность абстракций фрагмента знаний. Соответственно, для интерфейсов каждой модели~(идеал конъюнктов, идеал дизъюнктов, множество квантов) в случае скалярных и интервальных оценок был добавлен статический класс с тремя методами расширения: \begin{enumerate}
    \item IsIntersect --- проверяет, пересекаются ли между собой два фрагмента знаний, и возвращает true, если пересекаются, и false иначе;
    \item IsEqual --- позволяет сравнивать два фрагмента знаний между собой с заданной точностью и возвращает true, если ФЗ эквивалентны, и false иначе;
    \item GetSeparator --- возвращает пересечение двух фрагментов знаний, если пересечение не пусто, а в случае, если ФЗ не пересекаются, возвращается пустой фрагмент знаний.
\end{enumerate}

Всего было создано 6 таких статических классов.

Рассмотрим подробнее работу метода GetSeparator. В случае конъюнктов и дизъюнктов основную работу по выделению элементов сепаратора делает следующий цикл:

\begin{lstlisting}[caption = Цикл для выделения элементов виртуального свидетельства]
for (int index = 0; index <= KPGlobalIndex; index++)
{
    if ((index & KPGlobalIndex) == index)
    {
        if ((index & separatorGlobalIndex) == index)
        {
            probability[separatorIndex] =
                firstKP.GetPointEstimate(probIndex);
            probIndex++;
            separatorIndex++;
        }
        else
        {
            probIndex++;
        }
    }
}
\end{lstlisting}

Стоит отметить, что при программной реализации алфавит удобнее задавать с помощью глобального индекса. Глобальный индекс сепаратора можно получить с помощью операции побитового И глобальных индексов первого и второго фрагментов знаний, по этому принципу работает метод IsIntersect. Заметим, что количество единиц в глобальном индексе соответствует мощности алфавита. 

Данный цикл также использует операцию побитового И. Он проверяет все возможные числа index, которые могут соответствовать элементам вектора вероятностей, это числа от 0 до KPGlobalIndex --- глобального индекса первого ФЗ. Первое условие проверяет это соответствие, второе условие аналогично проверяет, соответствует ли индекс элементу сепаратора. Если условия верны, то значение вероятности сохраняется в массив вероятностей probability. Индексы текущих элементов в векторе вероятностей фрагмента знаний и векторе вероятностей сепаратора задаются с помощью переменных probIndex и separatorIndex соответственно.
 
 Отметим, что данный цикл реализует функциональность матриц перехода  $\Qmatr$ и $\V$, а сами матрицы строить не нужно, потому что, благодаря глобальным индексам, при реализации сразу понятно, какие элементы формируют виртуальное свидетельство.
 
 Так как у модели ФЗ, представленной квантами, виртуальное свидетельство содержит не просто оценки элементов на пересечении, а согласованные оценки, то тут алгоритм построения виртуального свидетельства будет другим. 
 
 Рассмотрим реализацию метода GetSeparator для построения виртуального свидетельства со скалярными оценками.  Вспомогательный метод GetQuantIndexes возвращает вектор из глобальных индексов соответствующих квантов, составляющих ФЗ.  В цикле для каждого индекса кванта из сепаратора считается его вероятность как вероятность детерминированного свидетельства с помощью метода GetEvidenceProbability объекта класса DeterministicScalarQuantLocalPropagator.
 
 \begin{lstlisting}[caption = Метод построения виртуального видетельства со скалярными оценками над квантами]
public static ScalarQuantKP_I GetSeparator(ScalarQuantKP_I firstKP, long separatorGlobalIndex)
{
    var numberOfElements = Convert
        .ToString(separatorGlobalIndex, 2)
        .Count(c => c == '1');
    var numberOfAtoms = (int)Math.Pow(2, numberOfElements);
    var probability = new double[numberOfAtoms];
    var sepAtomGlobalIndexes = GetQuantIndexes(numberOfAtoms, separatorGlobalIndex);
    var propagator = DeterministicScalarQuantLocalPropagator
    	.Instance(firstKP);
            
    for (int index = 0; index < sepAtomGlobalIndexes.Length; index++)
    {
        var binaryQuantKp = new BinaryQuantKP(separatorGlobalIndex);
        binaryQuantKp.SetQuant(sepAtomGlobalIndexes[index]);
        probability[index] = propagator.evidenceProbaility(binaryQuantKp);
    }
    return new ScalarQuantKP(separatorGlobalIndex, probability);
}
\end{lstlisting}

Кроме того, был создан интерфейс IVirtualEvidencePropagator и 6 его реализаций для каждой модели ФЗ в случае скалярных оценок и в случае интервальных оценок.

Дадим краткое описание классов, реализующих алгоритмы распространения виртуального свидетельства для модели ФЗ, представленной идеалом конъюнктов. 

 VirtualEvidenceScalarConjPropagator --- класс, отвечающий за следующие действия: пропагация стохастического виртуального свидетельства и пропагация неточного виртуального свидетельства во фрагмент знаний со скалярными оценками. Соответственно, класс пропагатора содержит 2 метода для двух видов свидетельства. 
 
 Метод PropagateStochasticEvidence используется в случае стохастического свидетельства. Метод принимает 2 параметра: ScalarConjKP\_I firstKP --- первый фрагмент знаний, ScalarConjKP\_I secondKP --- второй фрагмент знаний, в который нужно пропагировать виртуальное свидетельство. В этом методе сначала создается виртуальное свидетельство, далее оно пропагируется с помощью объекта propagator класса  StochasticScalarConjunctsLocalPropagator и его экземплярных методов propagate и getResult. 
 \begin{lstlisting} [caption = Реализация метода PropagateStochasticEvidence]
 public ScalarConjKP_I PropagateStochasticEvidence(ScalarConjKP_I firstKP, ScalarConjKP_I secondKP)
{
    var evidence = firstKP.GetSeparator(secondKP);
    var propagator = new StochasticScalarConjunctsLocalPropagator(secondKP);
    propagator.propagate(evidence);
    return propagator.getResult();
}\end{lstlisting}
 
Второй метод PropagateImpreciseEvidence нужен для пропагирования неточного виртуального свидетельства во второй фрагмент знаний, содержащий скалярные оценки. Пропагирование виртуального свидетельства осуществляется аналогично с помощью методов реализованного ранее класса ImpreciseScalarConjunctsLocalPropagator.
 \begin{lstlisting} [caption = Реализация метода PropagateImpreciseEvidence]
public IntervalConjKP_I PropagateImpreciseEvidence(IntervalConjKP_I firstKP, ScalarConjKP_I secondKP)
{
    var evidence = firstKP.GetSeparator(secondKP);
    var propagator = new ImpreciseScalarConjunctsLocalPropagator();
    propagator.setPattern(secondKP);
    propagator.propagate(evidence);
    return propagator.getResult();
}
\end{lstlisting}

 Второй класс VirtualEvidenceIntervalConjPropagator  отвечает за распространение виртуального  свидетельства из первого фрагмента знаний во второй ФЗ с интервальными оценками. Содержит в себе два аналогичных метода для пропагации свидетельства в двух различных ситуациях.
 
 Метод PropagateStochasticEvidence используется для распространения стохастического виртуального свидетельства в ФЗ с интервальными оценками. Метод принимает два параметра: ScalarConjKP\_I firstKP --- первый фрагмент знаний, IntervalConjKP\_I secondKP --- второй фрагмент знаний, в который нужно пропагировать виртуальное свидетельство.  Внутри метода создается виртуальное свидетельство и далее оно пропагируется с помощью методов объекта propagator ранее реализованного класса  StochasticIntervalConjunctsLocalPropagator.
\begin{lstlisting}[caption = Реализация метода PropagateStochasticEvidence]
public IntervalConjKP_I PropagateStochasticEvidence(ScalarConjKP_I firstKP, IntervalConjKP_I secondKP)
{
    var evidence = firstKP.GetSeparator(secondKP);
    var propagator = new StochasticIntervalConjunctsLocalPropagator();

    propagator.setPattern(secondKP);
    propagator.propagate(evidence);

return propagator.getResult();
}
\end{lstlisting}

Второй метод PropagateImpreciseEvidence нужен для пропагирования неточного виртуального свидетельства из первого ФЗ во второй. Пропагирование виртуального свидетельства осуществляется аналогично с помощью методов объекта propagator ранее реализованного класса ImpreciseIntervalConjunctsLocalPropagator.
\begin{lstlisting} [caption = Реализация метода PropagateImpreciseEvidence]
public IntervalConjKP_I PropagateImpreciseEvidence(IntervalConjKP_I firstKP, IntervalConjKP_I secondKP)
{
    var evidence = firstKP.GetSeparator(secondKP);
    var propagator = new ImpreciseIntervalConjunctsLocalPropagator();

    propagator.setPattern(secondKP);
    propagator.propagate(evidence);

    return propagator.getResult();
}\end{lstlisting}

Аналогичные классы реализованы для фрагментов знаний, построенных над идеалами дизъюнктов: VirtualEvidenceScalarDisjPropagator и VirtualEvidenceIntervalDisjPropagator, а также классы для ФЗ, построенных над множествами квантов: VirtualEvidenceScalarQuantPropagator и VirtualEvidenceIntervalQuantPropagator.  Каждый класс содержит методы PropagateStochasticEvidence и PropagateImpreciseEvidence для пропагации одного из видов виртуального свидетельства.

	   
\subsection{Примеры работы алгоритмов}
Приведем примеры использования методов реализаций интерфейса IVirtualEvidencePropagator. Рассмотрим примеры, когда фрагменты знаний содержат скалярные оценки, и в первый фрагмент знаний приходит стохастическое свидетельство. Распространим влияние свидетельства в первый ФЗ, сформируем виртуальное свидетельство и пропагируем его во второй ФЗ. Для остальных возможных ситуаций использование будет аналогичным.
\subsubsection{Пример пропагации стохастического виртуального свидетельства в ФЗ с скалярными оценками над идеалом конъюнктов}
Пусть даны два фрагмента знаний над алфавитами $\mathcal{A}_1 = \{x_1, x_2\}$ и  $\mathcal{A}_2 = \{x_2, x_3\}$ и в первый фрагмент знаний поступило стохастическое свидетельство над алфавитом $\mathcal{A}_{\ev} = \{x_1\}$. Пусть ФЗ и свидетельство имеют следующие оценки вероятностей: \\ 
\begin{equation*}
\Pc^1 =  \begin{pmatrix}
1 \\ p(x_1) \\ p(x_2) \\ p(x_2x_1)
\end{pmatrix} = \begin{pmatrix}
1 \\ 0.5\\ 0.7\\ 0.3
\end{pmatrix}, 
\Pc^2 =  \begin{pmatrix}
1 \\ p(x_2) \\ p(x_3) \\ p(x_3x_2)
\end{pmatrix} = \begin{pmatrix}
1 \\ 0.3\\ 0.7\\ 0.1 
\end{pmatrix},
\end{equation*}
\begin{equation*}
\Pc^{\ev} = \begin{pmatrix}  1 \\ p(x_1)	\end{pmatrix} = \begin{pmatrix}
1 \\ 0.4
\end{pmatrix}.
\end{equation*}

После пропагации свидетельства оценки вероятностей элементов первого фрагмента знаний, вычисленные с помощью уравнения~\ref{conjG}, будут следующими:
\begin{equation*}
\Pc^{1,a} = \begin{pmatrix}
1 \\  0.4 \\ 0.72 \\ 0.24
\end{pmatrix}.
\end{equation*}

Виртуальное свидетельство будет построено над алфавитом $\mathcal{A}_{\ev} = \mathcal{A}_1 \cap \mathcal{A}_2 = \{x_1, x_2\} \cap \{x_2, x_3\} = \{x_2\}$ и будет содержать следующий вектор оценок вероятностей элементов:
\begin{equation*}
\Pc^{\ev} = \begin{pmatrix}  1 \\ p(x_2)	\end{pmatrix} = 
\begin{pmatrix} 1 \\ 0.72 \end{pmatrix}.
\end{equation*}

После пропагации виртуального свидетельства во второй фрагмент знаний с помощью уравнения~\ref{conjglob2}, оценки второго ФЗ получаются следующими:
\begin{equation*}
\Pc^{2,a} = \begin{pmatrix}
1 \\ 0.72 \\ 0.48 \\ 0.24
\end{pmatrix}.
\end{equation*}

Данный результат пропагации можно получить с помощью следующего фрагмента кода:

\begin{lstlisting}[caption = Пример пропагации виртуального свидетельства]
// Creating KPs and evidence
var firstKP = new ScalarConjKP(Convert.ToInt64("011", 2), new[] { 1, 0.5, 0.7, 0.3 });
var secondKP = new ScalarConjKP(Convert.ToInt64("110", 2), new[] { 1, 0.3, 0.7, 0.1 });
var evidence = new ScalarConjKP(Convert.ToInt64("001", 2), new[] { 1, 0.4 });
// Creating propagators for stochastic and virtual evidences
var localPropagator = new StochasticScalarConjunctsLocalPropagator();
var virtualEvPropagator = new VirtualEvidenceScalarConjPropagator();

// Propagating stochasic evidence in fisrt KP
localPropagator.setPattern(firstKP);
localPropagator.propagate(evidence);
var firstKPwithAposterioriEst = localPropagator.getResult();

// Propagating virtual evidence in second KP
var secondKPwithAposterioriEst = virtualEvPropagator
    .PropagateStochasticEvidence(firstKPwithAposterioriEst, secondKP);
\end{lstlisting}

Вывод на консоль можно осуществить с помощью следующего кода:
\begin{lstlisting}[caption = Вывод результатов на консоль]
Console.WriteLine("Probability estimates in first KP after propagation:");
firstKPwithAposterioriEst
    .GetPointEstimate()
    .ToList()
    .ForEach(est => Console.Write(" " + est));
 
Console.WriteLine("Probability estimates in second KP after propagation:");
secondKPwithAposterioriEst
    .GetPointEstimate()
    .ToList()
    .ForEach(est => Console.Write(" " + est));
\end{lstlisting}

Результаты вывода на консоль:
\begin{lstlisting}[caption = Результаты]
Probability estimates in first KP after propagation:
 1, 0,4 0,72 0,24
Probability estimates in second KP after propagation:
1, 0,72, 0,48 0,24
\end{lstlisting}
\subsubsection{Пример пропагации стохастического виртуального свидетельства в ФЗ с скалярными оценками над идеалом дизъюнктов}
Рассмотрим фрагменты знаний над алфавитами $\mathcal{A}_1 = \{x_1, x_2\}$ и  $\mathcal{A}_2 = \{x_2, x_3\}$, построенные над идеалами дизъюнктов. Программная реализация принимает на вход и возвращает в качестве результата ФЗ, содержащие вектор вероятностей $\Pd$, в то же время как приведенные в работе формулы работают с векторами $\Pd^{\prime}$, поэтому приведем в данном примере векторы вероятностей как и $\Pd$, так и $\Pd^{\prime}$. Считаем, что в первый фрагмент знаний поступило стохастическое свидетельство над алфавитом $\mathcal{A}_{\ev} = \{x_1\}$. Пусть ФЗ и свидетельство имеют следующие оценки вероятностей: \\ 
\begin{equation*}
\Pd^{\prime 1} =  \begin{pmatrix}
1 \\ p(\overline{x}_1) \\ p(\overline{x}_2) \\ p(\overline{x}_2\overline{x}_1)
\end{pmatrix} = \begin{pmatrix}
1 \\ 0.5\\ 0.3\\ 0.1
\end{pmatrix}, 
\Pd^1 = \mathbf{1} - \Pd^{\prime 1} =  \begin{pmatrix}
0 \\ 0.5\\ 0.7\\ 0.9
\end{pmatrix}, 
\end{equation*}
\begin{equation*}
\Pd^{\prime 2} =  \begin{pmatrix}
1 \\ p(\overline{x}_3) \\ p(\overline{x}_3) \\ p(\overline{x}_3\overline{x}_2)
\end{pmatrix} = \begin{pmatrix}
1 \\ 0.7\\ 0.3\\ 0.1 
\end{pmatrix},
\Pd^2 = \mathbf{1} - \Pd^{\prime 2} =  \begin{pmatrix}
0 \\ 0.3\\ 0.7\\ 0.9 \end{pmatrix},
\end{equation*}
\begin{equation*}
\Pd^{\prime \ev} = \begin{pmatrix}  1 \\ p(\overline{x}_1)	\end{pmatrix} = \begin{pmatrix}
1 \\ 0.6
\end{pmatrix},
\Pd^{\ev} = \mathbf{1} - \Pd^{\prime \ev} =  \begin{pmatrix}
0 \\ 0.4 \end{pmatrix}.
\end{equation*}

После пропагации свидетельства оценки вероятностей элементов первого фрагмента знаний, вычисленные с помощью уравнения~\ref{dis3}, будут следующими:
\begin{equation*}
\Pd^{\prime 1,a} = \begin{pmatrix}
1 \\  0.6 \\ 0.28 \\ 0.12
\end{pmatrix},
\Pd^{1,a}= \mathbf{1} - \Pd^{\prime 1,a} = \begin{pmatrix}
0 \\  0.4 \\ 0.72 \\ 0.88
\end{pmatrix}.
\end{equation*}

Виртуальное свидетельство строится над алфавитом $\mathcal{A}_{\ev} = \mathcal{A}_1 \cap \mathcal{A}_2 = \{x_1, x_2\} \cap \{x_2, x_3\} = \{x_2\}$ и содержит следующий вектор оценок вероятностей элементов:
\begin{equation*}
\Pd^{\prime \ev} = \begin{pmatrix}  1 \\ p(\overline{x}_2)	\end{pmatrix} = 
\begin{pmatrix} 1 \\ 0.28 \end{pmatrix},
\Pd^{\ev} = \mathbf{1} - \Pd^{\prime \ev} = \begin{pmatrix} 0 \\ 0.72 \end{pmatrix}.
\end{equation*}

C помощью уравнения~\ref{disGlob} распространим виртуальное свидетельство во второй фрагмент знаний и получим апостериорные оценки вероятностей:
\begin{equation*}
\Pd^{\prime 2,a} = \begin{pmatrix}
1 \\ 0.28 \\ 0.52 \\ 0.04
\end{pmatrix},
 \Pd^{2,a}  = \mathbf{1} - \Pd^{\prime 2,a} = \begin{pmatrix}
0 \\ 0.72 \\ 0.48 \\ 0.96
\end{pmatrix}.
\end{equation*}

Данный результат пропагации можно получить с помощью следующего фрагмента кода:
\begin{lstlisting}[caption = Пример пропагации виртуального свидетельства]
// Creating KPs and evidence
var firstKP = new ScalarDisjKP(Convert.ToInt64("011", 2), new[] { 0, 0.5, 0.7, 0.9 });
var secondKP = new ScalarDisjKP(Convert.ToInt64("110", 2), new[] { 0, 0.3, 0.7, 0.9 });
var evidence = new ScalarDisjKP(Convert.ToInt64("001", 2), new[] { 0, 0.4 });
// Creating propagators for stochastic and virtual evidences
var localPropagator = new StochasticScalarDisjLocalPropagator(firstKP);
var virtualEvPropagator = new VirtualEvidenceScalarDisjPropagator();
// Propagating stochastic evidence in first KP
var firstKPwithAposterioriEst = localPropagator.PropagateEvidence(evidence); 
// Propagating virtual evidence in second KP
var secondKPwithAposterioriEst = virtualEvPropagator
    .PropagateStochasticEvidence(firstKPwithAposterioriEst, secondKP);
\end{lstlisting}

Вывод на консоль можно осуществить с помощью следующего кода:
\begin{lstlisting}[caption = Вывод результатов на консоль]
Console.WriteLine("Probability estimates in first KP after propagation:");
firstKPwithAposterioriEst
    .GetPointEstimate()
    .ToList()
    .ForEach(est => Console.Write(" " + est));

Console.WriteLine("Probability estimates in second KP after propagation:");
secondKPwithAposterioriEst
    .GetPointEstimate()
    .ToList()
    .ForEach(est => Console.Write(" " + est));
\end{lstlisting}

Результаты вывода на консоль:
\begin{lstlisting}[caption = Результаты]
Probability estimates in first KP after propagation:
 0 0,4 0,72 0,88
Probability estimates in second KP after propagation:
 0 0,72 0,48 0,96
\end{lstlisting}
\subsubsection{Пример пропагации стохастического виртуального свидетельства в ФЗ со скалярными оценками над множеством квантов}
Возьмем два фрагмента знаний над алфавитами $\mathcal{A}_1 = \{x_1, x_2\}$ и  $\mathcal{A}_2 = \{x_2, x_3\}$. Пусть в первый фрагмент знаний поступило стохастическое свидетельство над алфавитом $\mathcal{A}_{\ev} = \{x_1\}$. Будем считать, что ФЗ и свидетельство построены над множеством квантов и имеют следующие оценки вероятностей: 
\begin{equation*}
\Pq^1 =  \begin{pmatrix}
p(\overline{x}_2\overline{x}_1)\\ p(\overline{x}_2x_1) \\ p(x_2\overline{x}_1) \\ p(x_2x_1)
\end{pmatrix} = \begin{pmatrix}
0.1\\ 0.2\\ 0.4\\ 0.3
\end{pmatrix},
\Pq^2 =  \begin{pmatrix}
p(\overline{x}_3\overline{x}_2)\\ p(\overline{x}_3x_2)\\ p(x_3\overline{x}_2) \\ p(x_3x_2)
\end{pmatrix} = \begin{pmatrix}
0.1 \\ 0.2\\ 0.6\\ 0.1 
\end{pmatrix},
\end{equation*}
\begin{equation*}
\Pq^{\ev} = \begin{pmatrix}   p(\overline{x}_1) \\ p(x_1)	\end{pmatrix} = \begin{pmatrix}
0.6 \\ 0.4
\end{pmatrix}.
\end{equation*}

После пропагации свидетельства оценки вероятностей элементов первого фрагмента знаний, вычисленные с помощью уравнения~\ref{quantsG}, будут следующими:
\begin{equation*}
\Pq^{1,a} = \begin{pmatrix}
0.12 \\  0.16 \\ 0.48\\ 0.24
\end{pmatrix}.
\end{equation*}

Виртуальное свидетельство будет построено над алфавитом $\mathcal{A}_{\ev} = \mathcal{A}_1 \cap \mathcal{A}_2 = \{x_1, x_2\} \cap \{x_2, x_3\} = \{x_2\}$ и будет содержать следующий вектор оценок вероятностей квантов:
\begin{equation*}
\Pq^{\ev} = \begin{pmatrix}  p(\overline{x}_2) \\ p(x_2)	\end{pmatrix} = 
\begin{pmatrix} 0.28 \\ 0.72 \end{pmatrix}.
\end{equation*}

Пропагируем виртуальное свидетельство во второй фрагмент знаний с помощью уравнения~\ref{quantsGlob} и получим следующие оценки вероятностей:
\begin{equation*}
\Pq^{2,a} = \begin{pmatrix}
0.04 \\ 0.48 \\ 0.24 \\ 0.24
\end{pmatrix}.
\end{equation*}

Данный результат пропагации можно получить с помощью следующего фрагмента кода:

\begin{lstlisting}[caption = Пример пропагации виртуального свидетельства]
// Creating KPs and evidence
var firstKP = new ScalarQuantKP(Convert.ToInt64("011", 2), new[] { 0.1, 0.2, 0.4, 0.3 });
var secondKP = new ScalarQuantKP(Convert.ToInt64("110", 2), new[] { 0.1, 0.2, 0.6, 0.1 });
var evidence = new ScalarQuantKP(Convert.ToInt64("001", 2), new[] { 0.6, 0.4 });

// Creating propagators for stochastic and virtual evidences
var localPropagator = StochasticScalarQuantLocalPropagator.Instance(firstKP);
var virtualEvPropagator = new VirtualEvidenceScalarQuantPropagator();

// Propagating stochasic evidence in fisrt KP
var firstKPwithAposterioriEst = localPropagator.PropagateEvidence(evidence);
            
// Propagating virtual evidence in second KP
var secondKPwithAposterioriEst = virtualEvPropagator
    .PropagateStochasticEvidence(firstKPwithAposterioriEst, secondKP);
\end{lstlisting}

Вывод на консоль можно осуществить с помощью следующего кода:
\begin{lstlisting}[caption = Вывод результатов на консоль]
Console.WriteLine("Probability estimates in first KP after propagation:");
firstKPwithAposterioriEst
    .GetPointEstimate()
    .ToList()
    .ForEach(est => Console.Write(" " + est));
 
Console.WriteLine("Probability estimates in second KP after propagation:");
secondKPwithAposterioriEst
    .GetPointEstimate()
    .ToList()
    .ForEach(est => Console.Write(" " + est));
\end{lstlisting}

Результаты вывода на консоль:
\begin{lstlisting}[caption = Результаты]
Probability estimates in first KP after propagation:
 0,12 0,16 0,48 0,24
Probability estimates in second KP after propagation:
 0,04 0,48 0,24 0,24
\end{lstlisting}

\subsection{Выводы по главе}
Программная реализация, представленная в данной главе, реализует алгоритмы пропагирования виртуального свидетельства для различных моделей ФЗ, содержащих как скалярные, так и интервальные оценки. В главе описаны основные классы и методы и приведены примеры их использования.

\section*{Заключение}
\underline{Итогами}  квалификационной работы являются программная реализация алгоритма распространения виртуального свидетельства между двумя фрагментами знаний для различных моделей ФЗ в рамках существующего комплекса программ, а также:
\begin{enumerate}
    \item  Матрично"=векторная интерпретация алгоритма пропагации виртуального свидетельства между двумя фрагментами знаний: сформулированы матрично"=векторные уравнения для ФЗ, построенных над множествами квантов и над идеалами дизъюнктов, в частности, доказана теорема формирования матрицы перехода от вектора оценок вероятностей элементов к вектору виртуального свидетельства для модели идеала дизъюнктов и теорема о пропагации виртуального свидетельства между двумя ФЗ, построенными над множествами квантов, предложена матрица перехода от вектора вероятностей дизъюнктов к вектору вероятностей квантов; 
    \item Матрично"=векторная интерпретация функции $\Gind$, в частности, введено понятие характеристического вектора конъюнкта, дизъюнкта, кванта, а также понятие матрицы проекции $\G$;
    \item Внедрение алгоритмов локального ЛВВ для квантов и реинжиниринг существующих алгоритмов ЛВВ для идеалов дизъюнктов;
    \item Тесты, проверяющие корректность работы реализации;
    \item Примеры использования полученной реализации; 
    \item Вычислительные эксперименты, результаты которых согласуются с ожиданиями.
\end{enumerate}

\underline{Рекомендации к использованию.} Предложенная программная реализация является модулем объемлющей математической библиотеки, реализующей алгоритмы логико-вероятностного вывода в АБС и предоставляющей доступ к публичному контракту~\cite{50}. Поэтому реализованный модуль может быть переиспользован также в веб-приложении с целью визуализации алгоритма распространения виртуального свидетельства между двумя фрагментами знаний сети. Также программная реализация может быть переиспользована при дальнейшей реализации алгоритмов глобального логико"=вероятностного вывода, а именно распространения влияния свидетельства во все фрагменты знаний сети. Классы и методы могут быть использованы при проведении различных вычислительных экспериментов с целью изучения различных характеристик модели.

\underline{Перспективы дальнейших исследований.} Полученные теоретические результаты могут быть использованы для исследования устойчивости и чувствительности полученных уравнений и создают фундамент для развития теории глобального логико"=вероятностного вывода, в частности, для формализации алгоритмов распространения виртуального свидетельства между двумя фрагментами знаний с интервальными оценками и исследования распространения виртуального свидетельства для третичной глобальной структуры АБС~\cite{284}.

Разработанные примеры могут быть использованы в методических целях.


\setmonofont[Mapping=tex-text]{CMU Typewriter Text}
\bibliography{diploma}
\documentclass[14pt]{matmex-diploma-custom}

\usepackage{tocvsec2}
\usepackage{amssymb}
\usepackage{listings}
\usepackage{xcolor}
\usepackage[font=small,labelsep=period]{caption}
%%% \usepackage{concrete}
% Calligraphy letters
% use: \mathcal
\usepackage[mathscr]{eucal}
% Algorithms
% use: \begin{algorithm}
\usepackage{algorithm}
% use: \begin{algorithmic}
\usepackage[noend]{algpseudocode}
% Maths
% use: for correct control sequence
\usepackage{amsmath}
% Centreing
% use: \centering
\usepackage{varwidth}
% First indent
% use: automatically
\usepackage{indentfirst}

\definecolor{bluekeywords}{rgb}{0,0,1}
\definecolor{greencomments}{rgb}{0,0.5,0}
\definecolor{redstrings}{rgb}{0.64,0.08,0.08}
\definecolor{xmlcomments}{rgb}{0.5,0.5,0.5}
\definecolor{types}{rgb}{0.17,0.57,0.68}

\lstset{
    inputencoding=utf8x, 
    extendedchars=false, 
    keepspaces = true,
    language=[Sharp]C,
    captionpos=b,
   frame=lines, % Oberhalb und unterhalb des Listings ist eine Linie
    showspaces=false,
    showtabs=false,
    breaklines=true,
    showstringspaces=false,
    breakatwhitespace=true,
    basicstyle=\linespread{0.8}
    escapeinside={(*@}{@*)},
    commentstyle=\color{greencomments},
    morekeywords={partial, var, value, get, set},
    keywordstyle=\color{bluekeywords},
    stringstyle=\color{redstrings},
    basicstyle=\small\ttfamily,
}


\renewcommand{\lstlistingname}{Листинг}

\newtheorem{Th}{Теорема}[section]
\newtheorem{Def}{Определение}[section]
\newtheorem{Lem}{Утверждение}[subsection]

\newcommand{\underdot}[1]{\mathop{#1}\limits_{\cdot}}

\newenvironment{Proof} % имя окружения
{\par\noindent{\bf Доказательство.}} % команды для \begin
{\hfill$\scriptstyle\blacksquare$} % команды для \end



\newcommand{\Pc}{\mathbf{P}_\mathrm{c}}
\newcommand{\Pq}{\mathbf{P}_\mathrm{q}}
\newcommand{\In}{\mathbf{I}_n}
\newcommand{\Jn}{\mathbf{J}_n}
\newcommand{\Qmatr}{\mathbf{Q}}
\newcommand{\V}{\mathbf{V}}
\renewcommand{\G}{\mathbf{G}}
\renewcommand{\T}{\mathbf{T}}
\newcommand{\evidenceNumbers}{\langle i, j \rangle}
\newcommand{\Tij}{\T^{\evidenceNumbers}}
\newcommand{\TijTilda}{\widetilde{\T}^{\evidenceNumbers}}
\newcommand{\QijTild}{\widetilde{\Qmatr}^{\evidenceNumbers}}
\newcommand{\VijTilda}{\widetilde{\V}^{\evidenceNumbers}}
\newcommand{\rij}{\mathbf{r}^{\evidenceNumbers}}
\newcommand{\rijTilda}{\widetilde{\mathbf{r}}^{\evidenceNumbers}}
\newcommand{\ev}{\mathrm{ev}}
\newcommand{\Gind}{\mathrm{GInd}}
\newcommand{\Selector}{\mathbf{s}}
\newcommand{\sij}{\mathbf{s}^{\evidenceNumbers}}
\newcommand{\sijTilda}{\widetilde{\mathbf{s}}^{\evidenceNumbers}}
\newcommand{\Pd}{\mathbf{P}_\mathrm{d}}
\newcommand{\Ln}{\mathbf{L}_n}
\newcommand{\On}{\mathbf{F}_n}
\newcommand{\Kn}{\mathbf{K}_n}
\renewcommand{\M}{\mathbf{M}}
\newcommand{\Mij}{\M^{\evidenceNumbers}}
\newcommand{\dij}{\mathbf{d}^{\evidenceNumbers}}
\newcommand{\dijTilda}{\widetilde{\mathbf{d}}^{\evidenceNumbers}}
\newcommand{\MijTilda}{\widetilde{\mathbf{M}}^{\evidenceNumbers}}
\newtheorem{theorem}{Теорема}[section]
\newtheorem{definition}[theorem]{Определение}
 \setcounter{tocdepth}{2}
 \usepackage{lastpage}
 \usepackage{mathtools}
\begin{document}
% Год, город, название университета и факультета предопределены,
% но можно и поменять.
% Если англоязычная титульная страница не нужна, то ее можно просто удалить.
\filltitle{ru}{
    chair              = {Фундаментальная информатика и информационные технологии\\ Информационные технологии},
    title              = {Алгебраические байесовские сети:\\ синтез глобальных структур и алгоритмы логико-вероятностного вывода (проектная работа)},
    type               = {bachelor},
    position           = {студента},
    group              = 14.Б08-мм,
    author             = {Анна Викторовна Шляк},
    supervisorPosition = {проф. каф. инф., д. ф.-м. н., доц.},
    supervisor         = {Тулупьев А.\,Л.},
    reviewerPosition   = {проф. каф. инф., д. ф.-м. н., доц. },
    reviewer           = {Фильченков А. А.},
    chairHeadPosition  = {?},
    chairHead          = {?},
    faculty            = {Математико-механический факультет},
    city               = {Санкт-Петербург},
    year               = {2018}
}
\filltitle{en}{
    chair              = {Fundamental informatics and information technologies \\ Information technologies},
    title              = {Algebraic Bayesian networks: \\global structure synthesis and probabilistic-logic inference algorithms(project work)},
        type               = {bachelor},
    author             = {Anna Shliak},
    supervisorPosition = {Prof. Computer Science Department, Dc. Sc. in Math, Assoc. Prof.},
    supervisor         = {Alexander Tulupyev},
    reviewerPosition   = {As. Prof. Computer Technology Department, Ph. D. Sc. in Math},
    reviewer           = {Andrey Filchenkov},
    chairHeadPosition  = {?},
    chairHead          = {?},
}
\maketitle
\tableofcontents
\section*{Введение}
\underline{Актуальность темы.} В современном мире существует необходимость анализировать и обрабатывать большие объемы информации, однако данных не всегда бывает достаточно и могут появляться различные неопределенности, что значительно осложняет задачу обработки. Подход к обработке знаний с неопределенностью предлагает класс вероятностных графических моделей~(ВГМ), к которым относятся алгебраические байесовские сети~(АБС), байесовские сети доверия~(БСД), марковские сети и многие другие. Зависимости между данными в ВГМ задаются с помощью графов, а степень неопределенности данных характеризуется оценками вероятностей. ВГМ находят широкое применение в различных областях, например, байесовские сети доверия, родственные классу АБС, рассматриваемому в данной работе, используются в медицине~\cite{99}, оценке рисков~\cite{98}, области финансов~\cite{97} и других областях~\cite{96}. 

Алгебраические байесовские сети введены профессором В. И. Городецким~\cite{95, 94} и, благодаря многолетним исследованиям, активно развиваются и также занимают свое место в классе вероятностных графических моделей. АБС состоит из набора случайных элементов, соответствующих высказываниям, которым приписаны оценки вероятностей, и структуры связи между указанными высказываниями, которую представляют в виде графа. 

В АБС используется принцип декомпозиции знаний на фрагменты знаний~(ФЗ). Формирование оценок истинности и обработка данных с неопределенностью в АБС основана на локальном и глобальном логико"=вероятностном выводе~\cite{93}. Новая информация, поступающая во фрагменты знаний, основана на логико-вероятностной модели свидетельств~\cite{109}. Для представления фрагмента знаний в АБС описаны три математические модели фрагмента знаний, построенные над идеалами конъюнктов, идеалами дизъюнктов или множествами пропозиций-квантов~\cite{121}. 
 
 \underline{Степень разработанности темы исследований.} 
 Актуальной задачей является матрично-векторное описание алгоритмов логико"=вероятностного вывода в АБС, потому что матрично"=векторные операции относятся к классическому математическому инструментарию, упрощают программную реализацию и понимание работы алгоритмов, а также предоставляют возможности для дальнейшего исследования математической модели, например, исследование чувствительности уравнений. На сегодняшний день предложены матрично"=векторные уравнения для алгоритмов локального логико"=вероятностного вывода~\cite{91}, однако в них остаются элементы функционального вычисления~(функция $\Gind$). Матрично"=векторный подход для глобального вывода также нуждается в доработке~\cite{70}. Параллельно развитию теории разрабатывается комплекс программ для работы с АБС~\cite{89}, реализующий в себе  алгоритмы ЛВВ, однако, не включающий в себя на данный момент алгоритмы глобального логико"=вероятностного вывода, в частности, в нем отсутствует реализация алгоритма распространения виртуального свидетельства между двумя фрагментами знаний.

Таким образом, \underline{объектом исследования} являются алгебраические байесовские сети,
а \underline{предметом исследования} --- матрично-векторная формализация алгоритма распространения виртуального свидетельства между двумя фрагментами знаний в АБС.

\underline{Целью} данной выпускной квалификационной работы является автоматизация глобального логико-вероятностного вывода в алгебраических
байесовских сетях, а именно алгоритмов распространения виртуального свидетельства между двумя фрагментами знаний. 
Для достижения поставленной цели решаются следующие \underline{задачи}:
\begin{enumerate}
        \item  Развить матрично"=векторную формализацию алгоритмов распространения виртуального свидетельства для различных моделей фрагментов знаний со скалярными оценками;
        \item Предложить матрично"=векторную формализацию для функции $\Gind$;
        \item Осуществить интеграцию и реинженеринг алгоритмов локального логико"=вероятностного вывода для альтернативных моделей ФЗ в рамках комплекса программ;
    \item Реализовать алгоритмы распространения виртуального свидетельства в рамках комплекса программ;
    \item Провести вычислительные эксперименты по распространению виртуальных свидетельств и написать документацию.
\end{enumerate}

\underline{Научная новизна.} 
В данной выпускной квалификационной работе бакалавра предложена матрично-векторная формализация алгоритмов распространения виртуального свидетельства для ФЗ, построенных над идеалами дизъюнктов и наборами пропозиций-квантов, в частности, сформулирована и доказана теорема о матрично-векторном формировании матрицы перехода от вектора вероятностей элементов идеала дизъюнктов к вектору вероятностей элементов виртуального свидетельства, доказана теорема о пропагации виртуального свидетельства между двумя ФЗ, построенными над множествами квантов, введена матрица перехода от вектора вероятностей дизъюнктов к вектору вероятностей квантов. Также получена матрица проекции $\G$, заменяющая функцию $\Gind$, в частности, введены понятия характеристического вектора конъюнкта, дизъюнкта и кванта. 

\underline{Теоретическая и практическая значимость исследования.} Теоретическая значимость работы заключается в создании базы для развития матрично"=векторного подхода в глобальном ЛВВ. Практическая значимость заключается в создании основы для реализации алгоритмов распространения свидетельства во все ФЗ сети, решения задачи визуализации работы с АБС, а также для проведения вычислительных экспериментов.

\underline{Методы исследования.} Для решения поставленных задач в данной области понадобилось изучить методические материалы, поставить проблему, проанализировать ее, спроектировать несколько возможных вариантов решения. Затем были выбраны подходящие средства и технологии программирования, связанные с языком реализации~(C\#), средой разработки~(Rider), сервисом для совместной разработки~(BitBucket). После чего были проведены вычислительные и программные эксперименты с целью обоснования корректности  полученного решения. В качестве методов решения используются теоретические методы~(анализ предыдущих результатов, формализация алгоритмов) и методы объектно"=ориентированного программирования~(ООП).


\underline{На защиту выносятся следующие положения:} 
\begin{enumerate}
\item Матрично"=векторные уравнения распространения виртуального свидетельства между двумя фрагментами знаний для следующих моделей фрагмента знаний: идеал дизъюнктов, множество пропозиций квантов;
\item Матрично"=векторная формализация для функции $\Gind$;
\item Программная реализация алгоритмов распространения виртуального свидетельства для каждой из трех моделей.
\end{enumerate}

\underline{Достоверность полученных результатов.} 
Достоверность предложенных в работе результатов обеспечена корректным применением методов исследования, подтверждена вычислительными экспериментами и примерами работы программы. Полученные результаты не противоречат известным результатам других авторов.

\underline{Апробация результатов.} Результаты исследования докладывались на следующих научных мероприятиях:
\begin{enumerate}
\item Всероссийская научная конференция по проблемам информатики
СПИСОК~(Санкт-Петербург, апрель 2017);
\item VII-Всероссийская научно"=практическая конференция НСМВИТ-2017~(Санкт"=Петербург, июль 2017);
\item Научный семинар лаборатории теоретических и междисциплинарных проблем информатики в СПИИРАН~(Санкт-Петербург, ноябрь 2017).
\end{enumerate}

\underline{Публикации.} Результаты работы вошли в две публикации, была зарегистрирована одна заявка в Роспатенте.

\underline{Сведения о личном вкладе автора.}
Постановка цели и задач, а также выносимые на защиту результаты получены лично автором. 

Личный вклад А.В. Шляк в публикациях с соавторами характеризуется следующим образом: в ~\cite{9} --- краткое описание существующих алгоритмов и примеры вычисления апостериорных вероятностей после пропагации виртуального свидетельства, в ~\cite{70} --- постановка задач формализации глобального ЛВВ.

\underline{Структура и объем работы.} Работа состоит из введения, четырех
глав, заключения, списка используемой литературы и двух приложений. Общий объем работы составляет \pageref{LastPage} страниц. Список используемой литературы содержит 35 источников.

Во введении описана актуальность темы исследования и степень ее разработанности, сформулированы цель и задачи работы и представлены выносимые на защиту результаты.

Первая глава носит обзорный характер и состоит из пяти разделов. В первом разделе описаны общие сведения об АБС. Во втором разделе представлены основные инструменты для работы с АБС. В третьем разделе дан обзор существующих библиотек. В четвертом разделе представлен краткий обзор комплекса программ, в рамках которого осуществлялась разработка, а также приведены и обоснованы  цели и задачи работы. В пятом разделе представлены выводы по данной главе.

Вторая глава описывает основные теоретические результаты, используемые в данной работе. В первом разделе представляются основные элементы теории АБС. Во втором разделе даются основные определения. В третьем разделе рассматриваются виды локального логико"=вероятностного вывода, а в четвертом разделе --- глобальный логико"=вероятностный вывод. В пятом разделе перечислены недостатки имеющихся алгоритмов.

Третья глава содержит основные теоретические результаты, полученные в данной работе. В первом разделе дается краткий обзор полученных результатов. Во втором разделе предложена матрично"=векторная интерпретация функции $\Gind$, а в третьем разделе --- матрично"=векторная формализация матрицы перехода. В четвертом разделе представлены матрично-векторные уравнения распространения виртуального свидетельства для каждой из моделей ФЗ. В пятом разделе описаны выводы по главе.

Четвертая глава описывает программную реализацию алгоритмов из второй и третьей глав. Первый раздел описывает используемые программные технологии. Второй раздел описывает архитектуру комплекса программ. В третьем разделе описаны основные реализованные классы и методы, а в четвертом разделе приведены примеры их использования. В пятом разделе представлены выводы по текущей главе. 

В заключении содержатся итоги работы и дальнейшие перспективы исследования.

В приложении А содержатся вычислительные эксперименты по работе предложенных алгоритмов, а в приложении В --- список публикаций по теме работы.

\small{
Эта работа является частью более широких инициативных проектов, выполняющихся в лаборатории теоретических и междисциплинарных проблем информатики СПИИРАН под руководством А.Л.~Тулупьева; кроме того, разработки были частично поддержаны грантами РФФИ 15-01-09001-a~--- «Комбинированный логико"=вероятностный графический подход к представлению и обработке систем знаний с неопределенностью: алгебраические байесовские сети и родственные модели», 18-01-00626~--- «Методы представления, синтеза оценок истинности и машинного обучения в алгебраических байесовских сетях и родственных моделях знаний с неопределенностью: логико-вероятностный подход и системы графов».}

\normalsize
Работа выполнялась в рамках общей проектной работы вместе с А.А. Золотиным, А.Е. Мальчевской, А.И. Березиным.

\section{Автоматизация алгоритмов вывода в алгебраической байесовской сети}
\subsection{Введение}
Алгебраические байесовские сети относятся к классу вероятностных графических моделей и являются эффективным инструментом для обработки и представления знаний с неопределенностью~\cite{184}.     

 Будем считать, что знания формируют эксперты в определенной предметной области. Эксперты задают оценки вероятностей утверждениям, образующим базу знаний, и характеризуют связи между утверждениями с помощью оценок вероятностей. Из-за этого и возникает неопределенность знаний.

 Представим ситуацию, что экспертам необходимо охарактеризовать несколько утверждений и связи между ними. Пусть характеризуемые утверждения будут атомарными. Так как число взаимосвязей с ростом числа утверждений будет растет экспоненциально,  эффективность работы с этими утверждениями будет быстро падать.
 
 Отличительной особенностью алгебраических байесовских сетей является подход к декомпозиции области знаний на фрагменты знаний~\cite{93}: разобьем множество утверждений на подмножества, называемые фрагментами знаний, и будем характеризовать утверждения в каждом фрагменте знаний отдельно. Это позволит использовать быстрые алгоритмы обработки данных, не предъявляя серьезных требований к вычислительным мощностям. Также этот подход удобен при характеризации высказываний, потому что при ограниченном наборе элементов необходимо охарактеризовать связи лишь между несколькими высказываниями. 
 
 Таким образом, АБС состоит из фрагментов знаний и структуры связей между ними. На данный момент описаны три математические модели фрагмента знаний: идеал конъюнктов, идеал дизъюнктов и множество квантов~\cite{121}. Фрагмент знаний строится над алфавитом, где элементы алфавита соответствуют характеризуемым утверждениям. В данной работе будет рассматриваться вторичная структура АБС~\cite{87}.
 
 
\subsection{Инструменты для работы с алгебраическими байесовскими сетями}
Рассмотрим предоставляемые АБС инструменты для работы с данными, которые содержат неопределенности.

Для того чтобы охарактеризовать утверждение, эксперт может дать оценку вероятности истинности данного утверждения. Оценка может быть скалярной или интервальной. АБС позволяют задавать и обрабатывать как и скалярные, так и интервальные оценки вероятностей~\cite{184}. В данной работе в основном будут рассматриваться скалярные оценки вероятности истинности.

Задачи обработки данных в АБС решаются с помощью аппарата логико-вероятностного вывода~\cite{121}. Логико-вероятностный вывод делится на локальный и глобальный~\cite{51, 93}. Алгоритмы локального вывода работают с отдельным фрагментом знаний, а глобального --- со всей сетью. В данной работе будет кратко рассмотрен локальный ЛВВ и более подробно глобальный вывод. Логико-вероятностный вывод позволяет нам оценить вероятность пропозициональной формулы, основываясь на имеющихся уже оценках в сети, а также позволяет нам получить новые оценки элементов АБС при поступлении новых обуславливающих данных, называемых свидетельствами~\cite{84, 284}.
 
 Также существует аппарат для поддержки и проверки различных видов непротиворечивости в АБС, для того чтобы оценки вероятностей элементов в ФЗ не противоречили друг другу и были согласованы между собой. Подробнее с ним можно познакомиться в ~\cite{184, 121}.
 
\subsection{Библиотеки для работы с алгебраическими байесовскими сетями}
Развитием теории АБС активно занимается научный коллектив ТиМПИ СПИИРАН, в разное время включавший в себя А.Л. Тулупьева, А.В. Сироткина, А.А. Фильченкова и других исследователей. Были опубликованы монографии ~\cite{109, 1, 85}, диссертации~\cite{84, 184, 284}, статьи, посвященные тематике АБС и байесовских сетей в целом~\cite{94}.

Вместе с теорией создавались также библиотеки для работы с АБС, так в 2009 году была создана библиотека AlgBN Modeller j.v.01 на языке Java~\cite{124, 123, 122}. В данной библиотеке есть структуры для хранения фрагментов знаний, машины локального логико-вероятностного вывода и проверки непротиворечивости, а также машины глобального логико-вероятностного вывода. Однако библиотека позволяет осуществить глобальный логико-вероятностный вывод только для АБС, состоящей из ФЗ, построенных над идеалами конъюнктов с заданными оценками вероятностей их элементов.

Позже была создана библиотека AlgBN KPB Reconciler cpp.v.01, разработанная на языке C++~\cite{81}. В этой библиотеке также есть средства для осуществления локального логико"=вероятностного вывода и проверки непротиворечивости, а также структуры для хранения фрагментов знаний, однако алгоритмы глобального логико-вероятностного вывода в ней не реализованы.

Таким образом, в обеих библиотеках отсутствует полноценный функционал для проведения глобального логико-вероятностного вывода в АБС.
\subsection{Цели и задачи исследования}
Так как теория алгебраических байесовских сетей активно развивается, в 2015 году возникла потребность начать разработку комплекса программ AlgBN Math Library~\cite{89, 50}, представляющего собой библиотеку для работы с АБС. Так как алгоритмы ЛВВ приобретают матрично-векторную форму, позволяющую упростить программную реализацию, было принято решение о создании новых библиотек для работы с АБС, использующих современные матрично-векторные алгоритмы. На основе данных библиотек также создается приложение, визуализирующее АБС и позволяющее проводить вывод в сети. Комплекс программ реализован на языке C\# и платформе .Net, являющимися одними из популярнейших современных технологий в разработке программного обеспечения. 

На данный момент в комплексе программ реализованы структуры для хранения фрагментов знаний и АБС, алгоритмы локального логико-вероятностного вывода, а также алгоритмы проверки АБС и ФЗ на непротиворечивость. Реализации алгоритмов глобального логико-вероятностного вывода еще нет.

Что касается теории глобального логико"=вероятностного вывода, существующие алгоритмы обладают рядом недостатков~\cite{70}, к 
которым относится присутствие операций, не относящихся к матрично"=векторным вычислениям, которые усложняют алгоритмы и их программную реализацию.

Таким образом, целью данной выпускной квалификационной работы является автоматизация глобального логико-вероятностного вывода в алгебраических
байесовских сетях, а именно алгоритмов распространения виртуального свидетельства между двумя фрагментами знаний в АБС. 
Для достижения поставленной цели решаются следующие \underline{задачи}:
\begin{enumerate}
        \item  Развить матрично"=векторную формализацию алгоритмов распространения виртуального свидетельства для различных моделей фрагментов знаний с скалярными оценками;
        \item Предложить матрично"=векторную формализацию для функции $\Gind$;
        \item Осуществить интеграцию и реинженеринг алгоритмов локального логико"=вероятностного вывода для альтернативных моделей ФЗ в рамках комплекса программ;
    \item Реализовать алгоритмы распространения виртуального свидетельства в рамках комплекса программ;
\item Провести вычислительные эксперименты по распространению виртуальных свидетельств и написать документацию.
\end{enumerate}
\subsection{Выводы по главе}
В данной главе были рассмотрены основные понятия, используемые в теории алгебраических байесовских сетей и логико"=вероятностном вывода в АБС. Рассмотрены существующие программные реализации, их основные возможности и недостатки. Сформулированы задачи и цель данной выпускной квалификационной работы.

\section{Элементы теории алгебраических байесовских сетей}\
\subsection{Введение}
В данной главе введем основные математические определения и понятия, используемые в теории АБС, и дадим описание операций логико"=вероятностного вывода в АБС. Введем определения фрагмента знаний и алгебраической байесовской сети. Рассмотрим матрично"=векторные уравнения локального апостериорного вывода для каждой из трех моделей и стохастического свидетельства. Также представим имеющийся результат для глобального логико"=вероятностного вывода над идеалами конъюнктов. Более подробное описание введенных понятий, алгоритмов и обоснование формулировок можно найти в ~\cite{ 76, 91, 84,  184, 109, 1, 122, 284, 70}.

\subsection{Математические модели фрагмента знаний и алгебраической байесовской сети}
Зафиксируем алфавит $\mathcal{A} = \{x_i\}^n_{i=0}$ --- конечное множество атомарных пропозициональных формул~(атомов). Над алфавитом определим идеал конъюнктов, идеал дизъюнктов и множество квантов.
\begin{definition}[\cite{1}]
Идеал конъюнктов $C_\mathcal{A} $ --- это множество вида 
\begin{math}
\{x_{i_1} \wedge x_{i_2}\wedge ... \wedge x_{i_k} | \; 0 \leq i_1 < i_2 < ... < i_k \leq  n - 1, 0 \leq k \leq n\}
\end{math}.
\end{definition}

\begin{definition}[\cite{1}]
Квант над алфавитом $\mathcal{A} = \{x_i\}^n_{i=0}$ ---  это конъюнкция, которая для любого атома алфавита содержит либо этот атом, либо его отрицание.
\end{definition}

\begin{definition}[\cite{1}]
Литерал $\tilde{x}_i$ обозначает, что на его месте в формуле может стоять либо $x_i$, либо его отрицание $\bar{x}_i$.
\end{definition}

\begin{definition}[\cite{1}]
	Множество квантов $Q_\mathcal{A} $ --- это множество всех комбинаций вида $\{\tilde{x}_0\tilde{x}_1...\tilde{x}_{n-1}\}$.
\end{definition}

\begin{definition}[\cite{184}]
Идеал дизъюнктов $C_\mathcal{A} $ --- это множество вида 
\begin{math}
\{x_{i_1} \vee x_{i_2}\vee ... \vee x_{i_k} | \; 0 \leq i_1 < i_2 < ... < i_k \leq  n - 1, 0 \leq k \leq n\}
\end{math}.
\end{definition}

Введем правило нумерации на множестве конъюнктов и на множестве квантов.

Каждому кванту $\tilde{x}_0\tilde{x}_1...\tilde{x}_{n-1}$ поставим в соответствие двоичную запись, в которой на $i$-м месте будет стоять $1$, если $i$-тый литерал означен положительно, и $0$ иначе. Аналогично занумеруем конъюнкты: каждому конъюнкту $x_{i_1}x_{i_2}...x_{i_k}$ поставим в соответствие сумму $2^{i_1}+2^{i_2}+...+2^{i_k}$. Тогда, если представить полученную сумму в виде двоичной записи и дополнить лидирующими нулями до $n$ знаков, $i$-тый атом будет входить в конъюнкт тогда и только тогда, когда $i$-тый бит числа равен $1$. Таким образом, получится биективное отображение множества квантов на множество конъюнктов.

Для множества дизъюнктов правило нумерации будет аналогичным правилу для множества конъюнктов, следовательно, существует биективное отображение множества квантов и на множество дизъюнктов.

Теперь введем вектор вероятностей элементов идеала конъюнктов $\Pc$, вектор вероятностей элементов идеала дизъюнктов $\Pd$ и   вектор вероятностей элементов множества квантов $\Pq$: 
\begin{equation*}
\Pc = \begin{pmatrix}
	    p(c_0)  \\ p(c_1) \\ \vdots \\  p(c_{2^n-1})
      \end{pmatrix}, \,
\Pq = \begin{pmatrix}
	    p(q_0)  \\ p(q_1) \\ \vdots \\  p(q_{2^n-1})
      \end{pmatrix}, \,
\Pd = \begin{pmatrix}
	    p(d_0)  \\ p(d_1) \\ \vdots \\  p(d_{2^n-1})
      \end{pmatrix}. 
\end{equation*}

$\Pc$ и $\Pq$ связаны между собой соотношениями $\Pc = \Jn\Pq$ и $\Pq = \In\Pc$, где матрицы перехода $\In$ и $\Jn$ получены с помощью следующих рекуррентных соотношений~\cite{109}:

\begin{equation*}
    \In = \mathbf{I}_1 \otimes ... \otimes \mathbf{I}_{n-1} = \mathbf{I}_1^{[n]},
\end{equation*}
где $\mathbf{I}_1 = \begin{pmatrix*}[r] 1 & -1 \\ 0 & 1 \end{pmatrix*}$.

\begin{equation*}
    \Jn = \mathbf{J}_1 \otimes ... \otimes \mathbf{J}_{n-1} = \mathbf{J}_1^{[n]},
\end{equation*}
где $\mathbf{J}_1 = \begin{pmatrix} 1 & 1 \\ 0 & 1 \end{pmatrix}$, $\otimes$ обозначает кронекерово произведение матриц.

Вектор вероятностей элементов идеала дизъюнктов связан с $\Pc$ и $\Pq$ следующими соотношениями~\cite{76}:

\begin{equation}\label{pdtopq}
\Pq = \Ln(\mathbf{1} -\Pd),
\end{equation} 
где $\Ln = \begin{pmatrix*}[r] 0 & 1 \\ 1 & -1 \end{pmatrix*}^{[n]}$.

\begin{equation}
\Pc = \Kn(\mathbf{1} -\Pd),
\end{equation}
где $\Kn = \begin{pmatrix*}[r] 1 & 0 \\ 1 & -1 \end{pmatrix*}^{[n]}$.

Для удобства будем обозначать $\Pd^{\prime} = \mathbf{1} -\Pd$ и далее работать с вектором $\Pd^{\prime}$.

Дадим определение фрагмента знаний для трех моделей в случае, когда оценки вероятностей элементов скалярные.

\begin{definition}[\cite{1}]
Фрагмент знаний $\mathscr{C}$ над идеалом конъюнктов со скалярными оценками --- это пара вида $(C, \,p)$, где $C$ --- идеал конъюнктов, $p$ --- функция из $C$  в интервал $[0;1]$.
\end{definition}

\begin{definition}[\cite{1}]
	Фрагмент знаний со скалярными оценками непротиворечив тогда и только тогда, когда $\In\Pc \geq \mathbf{0}$.
\end{definition}

\begin{definition}[\cite{121}]
Фрагмент знаний $\mathscr{C}$ над идеалом дизъюнктов со скалярными оценками --- это пара вида $(C, \,p)$, где $C$ --- идеал дизъюнктов, $p$ --- функция из $C$  в интервал $[0;1]$.
\end{definition}

\begin{definition}[\cite{121}]
Фрагмент знаний $\mathscr{C}$ над множеством квантов со скалярными оценками --- это пара вида $(Q, \,p)$, где $Q$ --- множество квантов, лежащее в основе ФЗ, $p$ --- функция из $Q$  в интервал $[0;1]$.
\end{definition}

В качестве функции $p$ можно использовать вероятность истинности пропозиций.
\begin{definition}[\cite{1}]
Алгебраическую байесовскую сеть $\mathcal{N}$ определим как набор фрагментов знаний: $\mathcal{N}^{\circ} = \{\mathscr{C}_i\}_{i=1}^{n}$.
\end{definition}
\subsection{Локальный логико"=вероятностный вывод}
\subsubsection{Задачи локального логико"=вероятностного вывода}
Локальный логико"=вероятностный вывод делится на априорный и апостериорный~\cite{ 184, 121, 109}. Задачей локального априорного вывода является построение оценки истинности пропозициональной формулы, заданной над тем же алфавитом $\mathcal{A}$, что и данный фрагмент знаний $\mathscr{C}$. 
Для описания локального апостериорного вывода введем понятие свидетельства.
\begin{definition}[\cite{184}]
Под свидетельством мы понимаем новые «обуславливающие» данные, которые поступили во фрагмент знаний, и с учетом которых нам требуется пересмотреть все (или некоторые) оценки.  Для обозначения свидетельства будут использоваться угловые скобки --- $\langle ...\rangle$.
\end{definition}

Локальный апостериорный вывод решает 2 задачи: во-первых, оценить вероятности истинности свидетельства при данных оценках вероятности истинности элементов фрагмента знаний, и, во-вторых, оценить условные вероятности истинности элементов фрагмента знаний, предполагая, что свидетельство истинно.
Свидетельства бывают детерминированными, стохастическими и неточными~\cite{184, 109}. В данной работе в основном будут рассматриваться стохастические свидетельства, которые можно трактовать как набор детерминированных свидетельств в сочетании с заданным на них распределением вероятности.
\begin{definition}[\cite{184}]
	Детерминированное свидетельство --- это предположение, что один или несколько атомов получили конкретное означивание. 
\end{definition}

Детерминированное свидетельство обозначим $\langle i,j\rangle$, где $i$ --- индексы положительно означенных атомов, $j$ --- индексы отрицательно означенных атомов.	
\begin{definition}[\cite{184}]
Стохастическое свидетельство --- предположение о том, что над $C^\prime$ --- подыдеале $C$, задан непротиворечивый фрагмент знаний со скалярными оценками,  который определяет вероятности истинности элементов соответствующего подыдеала. Данное свидетельство обозначается $\langle (C^\prime, \Pc) \rangle$.
\end{definition}
\begin{definition}[\cite{184}]
	Неточное свидетельство --- это предположение о том, что над $C^\prime$ --- подыдеале $C$,  задан непротиворечивый фрагмент знаний с интервальными оценками, который определяет вероятности истинности элементов соответствующего подыдеала. Данное свидетельство обозначается $\langle (C^\prime, \Pc^-, \Pc^+) \rangle$.
\end{definition}

\subsubsection{Локальный апостериорный вывод над конъюнктами}

Приведем решение первой и второй задач локального апостериорного вывода, когда  фрагмент знаний $(C, \Pc)$ над идеалом конъюнктов $C$ содержит скалярные оценки и в него поступило стохастическое свидетельство $(C^{\ev}, \Pc^{\ev})$. 

Решение первой задачи~\cite{91}:
\begin{equation} \label{conjStochFirst}
  p(\langle C^{\ev}, \Pc^{\ev} \rangle) =\sum_{i=0}^{2^{n^\prime}-1}  (\mathbf{r}^{\langle \Gind(i,m),\,\Gind(2^{n^\prime}-1-i,m)\rangle},\Pc) \In\Pc^{a}[i],
\end{equation} 
где $\rij = \otimes^{0}_{k=n-1}\rijTilda_k$, \\
\begin{math}
    \rijTilda_k = 
    \begin{cases}
        \mathbf{r}^+ \text{, если $x_k$ входит в $c_i$,}\\
        \mathbf{r}^- \text{, если $x_k$ входит в $c_j$,}\\
        \mathbf{r}^\circ  \text{, иначе;}
    \end{cases}
\end{math}\\
\begin{math}
    \mathbf{r}^+ = \begin{pmatrix} 0 \\ 1 \end{pmatrix},
    \mathbf{r}^- = \begin{pmatrix*}[r] 1 \\ -1 \end{pmatrix*},
    \mathbf{r}^\circ = \begin{pmatrix} 1 \\ 0 \end{pmatrix}
\end{math},\\
$\Gind(i,m)$ --- функция, которая по индексу наибольшего элемента $C^{\ev}$ в алфавите $\mathcal{A}$ и индексу конъюнкта в алфавите $\mathcal{A}^{\ev}$ возвращает индекс соответствующего конъюнкта в алфавите $\mathcal{A}$.

Решение второй задачи апостериорного вывода находится по формуле~\cite{91}:
\begin{equation} \label{conj1}
\Pc^a=\sum_{i=0}^{2^{n^\prime}-1}\dfrac{\T^{\langle \Gind(i,m),\,\Gind(2^{n^\prime}-1-i,m)\rangle }\Pc}{(\mathbf{r}^{\langle \Gind(i,m),\,\Gind(2^{n^\prime}-1-i,m)\rangle},\Pc)}\In\Pc^{\ev}[i],
\end{equation} 
где $\Pc^a$ --- вектор апостериорных вероятностей истинности элементов данного фрагмента знаний,\\
\begin{math}
    \Tij = \TijTilda_{n-1}\otimes \TijTilda_{n-2} \otimes ...\otimes \TijTilda_{0}
\end{math}, \\
\begin{math}
    \TijTilda_k = 
    \begin{cases}
        \T^+ \text{, если $x_k$ входит в $c_i$,}\\
        \T^- \text{, если $x_k$ входит в $c_j$,}\\
        \T^\circ  \text{, иначе;}
    \end{cases}
\end{math} \\ 
\begin{math}
    \T^+ = \begin{pmatrix} 0 & 1 \\ 0 & 1 \end{pmatrix},
    \T^- = \begin{pmatrix*}[r] 1 & -1 \\ 0 & 0 \end{pmatrix*},
\T^\circ = \begin{pmatrix} 1 & 0 \\ 0 & 1 \end{pmatrix}.
\end{math}

\subsubsection{Локальный апостериорный вывод над дизъюнктами}

Приведем имеющиеся результаты для локального ЛВВ для модели ФЗ, построенной над идеалами дизъюнктов~\cite{76, 49}. Рассмотрим решение первой и второй задач  локального апостериорного вывода для стохастических свидетельств.

Решение первой задачи~\cite{49}:
\begin{equation}\label{dis1}
  p(\langle C^{\ev}, \Pd^{\prime \ev} \rangle) =\sum_{i=0}^{2^{n^\prime}-1}  (\mathbf{d}^{\langle \Gind(i,m),\,\Gind(2^{n^\prime}-1-i,m)\rangle},\Pd^{\prime}) \Ln\Pd^{\prime a}[i],
\end{equation}
где
$\dij = \otimes^{0}_{k=n-1}\dijTilda_k$,\\
\begin{math}
    \dijTilda_k = 
    \begin{cases}
        \mathbf{d}^+ \text{, если $x_k$ входит в $c_i$,}\\
        \mathbf{d}^- \text{, если $x_k$ входит в $c_j$,}\\
        \mathbf{d}^\circ  \text{, иначе;}
    \end{cases}
\end{math} \\ 
\begin{math}
    \mathbf{d}^+ = \begin{pmatrix*}[r] 1 \\ -1 \end{pmatrix*},
    \mathbf{d}^- = \begin{pmatrix} 0 \\ 1 \end{pmatrix},
    \mathbf{d}^\circ = \begin{pmatrix} 1 \\ 0 \end{pmatrix},
\end{math}\\
$\Pd^\prime = \mathbf{1} - \Pd$,\\
$\Gind(i,m)$ --- функция, которая по индексу наибольшего элемента $C^{\ev}$ в алфавите $\mathcal{A}$ и индексу дизъюнкта в алфавите $\mathcal{A}_{\ev}$ возвращает индекс соответствующего дизъюнкта в алфавите $\mathcal{A}$.

Решение второй задачи~\cite{49}:
\begin{equation}\label{dis2}
\Pd^{\prime a} =\sum_{i=0}^{2^{n^\prime}-1}\dfrac{\M^{\langle \Gind(i,m),\,\Gind(2^{n^\prime}-1-i,m)\rangle }\Pd^{\prime}}{(\mathbf{d}^{\langle \Gind(i,m),\,\Gind(2^{n^\prime}-1-i,m)\rangle},\Pd^{\prime})}\Ln\Pd^{\prime \ev}[i],
\end{equation}
где 
\begin{math}
    \Mij = \otimes^{0}_{k=n-1}\MijTilda_k
\end{math},\\
\begin{math}
    \MijTilda_k = 
    \begin{cases}
        \M^+ \text{, если $x_k$ входит в $c_i$,}\\
        \M^- \text{, если $x_k$ входит в $c_j$,}\\
        \M^\circ  \text{, иначе;}
    \end{cases}
\end{math} \\ 
\begin{math}
    \M^+ = \begin{pmatrix*}[r] 1 & -1 \\ 0 & 0 \end{pmatrix*},
    \M^- = \begin{pmatrix} 0 & 1 \\ 0 & 1 \end{pmatrix},
    \M^\circ = \begin{pmatrix} 1 & 0 \\ 0 & 1 \end{pmatrix}
\end{math}\\
и $\Pd^{\prime , \evidenceNumbers} = \mathbf{1} - \Pd^{\evidenceNumbers}$.

\subsubsection{Локальный апостериорный вывод над квантами}
Рассмотрим решение первой и второй задачи апостериорного локального ЛВВ, когда во фрагмент знаний со скалярными оценками, построенный над множеством пропозиций-квантов, пришло стохастическое свидетельство. Подробнее про локальный ЛВВ над квантами можно прочитать в~\cite{74}.

Решение первой задачи~\cite{74}:
\begin{equation} \label{quantsFirst}
  p(\langle C^{\ev}, \Pq^{\ev} \rangle) =\sum_{i=0}^{2^{n^\prime}-1}  (\Selector^{\langle \Gind(i,m),\,\Gind(2^{n^\prime}-1-i,m)\rangle},\Pq) \Pq^{a}[i],
\end{equation} 
где вектор-селектор $\sij = \otimes^{0}_{k=n-1}\sijTilda_k$, \\
\begin{math}
    \sijTilda_k = 
    \begin{cases}
        \Selector^+ \text{, если $x_k$ входит в $c_i$,}\\
        \Selector^- \text{, если $x_k$ входит в $c_j$,}\\
        \Selector^\circ  \text{, иначе;}
    \end{cases}
\end{math} \\
и 
\begin{math}
    \Selector^+ = \begin{pmatrix} 0 \\ 1 \end{pmatrix},
    \Selector^- = \begin{pmatrix} 1 \\ 0 \end{pmatrix},
    \Selector^\circ = \begin{pmatrix} 1 \\ 1 \end{pmatrix}
\end{math},\\
$\Gind(i,m)$ --- функция, которая по индексу кванта $i$ в алфавите, над которым построено свидетельство, и индексу $m$ наибольшего элемента $\Pq^a$ в исходном алфавите сопоставляет индекс свидетельства с индексом множества квантов поступившего свидетельства в исходном алфавите.

Теперь рассмотрим решение второй задачи~\cite{74}. Пусть дан фрагмент знаний $(C, \Pq)$ со скалярными оценками и стохастическое свидетельство $(C^{\ev}, \Pq^{\ev})$.

Решение второй задачи апостериорного вывода находится по формуле:
\begin{equation} \label{quants1}
 \Pq^{\evidenceNumbers}=\sum_{i=0}^{2^{n^\prime}-1}\dfrac{{\Selector^{\langle \Gind(i,m),\,\Gind(2^{n^\prime}-1-i,m)\rangle }}\circ \Pq}{(\Selector^{\langle \Gind(i,m),\,\Gind(2^{n^\prime}-1-i,m)\rangle},\Pq)}\Pq^{\ev}[i],
\end{equation}
где $\circ$ обозначает произведение Адамара~(операция покомпонентного произведения векторов одинаковой размерности).


\subsection{Глобальный апостериорный логико"=вероятностный вывод}
Рассмотрим связную ациклическую алгебраическую сеть~\cite{284} со скалярными оценками во всех фрагментах знаний. Предположим, что в один из фрагментов знаний поступило стохастическое свидетельство. Задачей глобального апостериорного вывода является распространение влияния этого свидетельства~(пропагация свидетельства) во все фрагменты знаний сети. Схема глобального апостериорного вывода подробно описана в~\cite{93}.
 Алгоритм пропагации состоит из 3 шагов: 
\begin{enumerate}
	\item  Пропагация свидетельства во фрагмент знаний, в который оно пришло, и оценка апостериорных вероятностей его элементов;
	\item Формирование виртуального свидетельства;
	\item Пропагация виртуального свидетельства в соседний фрагмент знаний.
\end{enumerate}

Аналогичным образом свидетельство пропагируется далее, пока не будут переозначены оценки элементов во всех фрагментах знаний.

Виртуальным свидетельством называется пересечение двух фрагментов знаний~(сепаратор), также являющийся фрагментом знаний~\cite{51, 93}. Так как после переозначивания оценок в начальном фрагменте знаний, конъюнкты, принадлежащие сепаратору, имеют новые оценки, потому что принадлежат начальному фрагменту знаний, но с другой стороны они также принадлежат соседнему фрагменту знаний и имеют другие оценки, то сепаратор можно рассмотреть как новую информацию, поступившую в соседний фрагмент знаний. 

Таким образом, на втором шаге алгоритма из вектора вероятностей начального фрагмента знаний необходимо выделить вектор значений, принадлежащих обоим фрагментам знаний, и принять его за новое свидетельство, которое на третьем шаге пропагируется в соседний фрагмент знаний, оценки которого необходимо переозначить.

Для того чтобы выделить необходимые оценки с помощью матрично-векторных вычислений, требуется сформировать матрицу перехода от вектора оценок вероятностей элементов ФЗ к вектору оценок вероятностей виртуального свидетельства.

Виртуальные свидетельства могут быть двух видов: стохастическое и неточное, потому что пересечение двух фрагментов знаний всегда является фрагментом знаний, но может содержать как и скалярные, так и интервальные оценки вероятностей.

Остановимся подробнее на втором шаге алгоритма и рассмотрим формирование и распространение виртуального свидетельства из одного фрагмента знаний в другой, когда они построены над идеалами конъюнктов. Считаем, что вектор $\Pc^{1, a}$ содержит апостериорные оценки вероятностей, потому что в него ранее было распространено влияние какого-то свидетельства. Распространим влияние этого свидетельства в соседний фрагмент знаний с оценками $\Pc^2$.

Рассмотрим, как выглядит матрица перехода $\Qmatr$ от вектора оценок вероятностей элементов ФЗ к вектору оценок вероятностей виртуального свидетельства. Матрица будет размерности $m \times n$~($m$ --- длина вектора $\Pc^{\ev}$,  $n$ --- длина вектора $\Pc^{1, a}$). Элементы матрицы определяются по следующему правилу~\cite{51, 9, 70}:
\begin{equation*}
    \Qmatr[i,j] = 
    \begin{cases}
        1 \text{, если $\Pc^{1, a}[j] = \Pc^{\ev}[i]$,}\\
        0 \text{, иначе;}\\
    \end{cases}
\end{equation*}

Под $\Pc^{1, a}[i]$ и $\Pc^{\ev}[i]$ подразумеваются сами конъюнкты, а не значения вероятностей. Затем нужно умножить матрицу на $\Pq^{1, a}$. Таким образом, $\Pc^{\ev} = \Qmatr\Jn\Pq^{1, a} = \Qmatr\Pc^{1, a}$.

Алфавит, над которым построено свидетельство, можно найти как $\mathcal{A}_{\ev} = \mathcal{A}_1 \cap \mathcal{A}_2$, где $\mathcal{A}_1$ --- алфавит первого фрагмента знаний, $\mathcal{A}_2$ --- алфавит второго фрагмента знаний.

Таким образом, пропагировать виртуальное свидетельство из одного ФЗ в другой можно с помощью следующего уравнения~\cite{51}:
\begin{equation} \label{conjglob1}
    \Pc^{2,a}=\sum_{i=0}^{2^{n^\prime}-1}\dfrac{\T^{\langle \Gind(i,m),\,\Gind(2^{n^\prime}-1-i,m)\rangle }\Pc^2}{(\mathbf{r}^{\langle \Gind(i,m),\,\Gind(2^{n^\prime}-1-i,m)\rangle},\Pc^2)}\In\Pc^{\ev}[i],
\end{equation}
где $\Pc^{\ev} = \Qmatr\Pc^{1, a}$, $\Pc^{1, a}$ и $\Pc^2$ --- векторы вероятностей элементов идеалов конъюнктов первого и второго фрагментов знаний, $\Pc^{2,a}$ --- вектор апостериорных вероятностей элементов идеала конъюнктов второго фрагмента знаний.

Матрицу перехода $\Qmatr$ можно также сформировать с помощью матрично"=векторных операций, а именно через кронекерово произведение:

\begin{theorem}[\cite{70}]
Вектор оценок виртуального свидетельства $\Pc^{\ev}$ можно вычислить как $\Pc^{\ev} = \Qmatr\Pc^{a,1}$, где $\Qmatr = \QijTild_{n-1} \otimes \QijTild_{n-2} \otimes ... \otimes \QijTild_{0}$,
 $\QijTild_k = \begin{cases}
\Qmatr^+ \text{, если $x_k$ входит в $\mathcal{A}^{\ev}$,} \\
\Qmatr^- \text{, иначе;}
\end{cases}$,
\begin{math}
    \Qmatr^+ = \begin{pmatrix} 1 & 0 \\ 0 & 1 \end{pmatrix},
    \Qmatr^- = \begin{pmatrix} 1 & 0 \end{pmatrix},
\end{math}\\ 
$x_k$ --- $k$-тый элемент в $\mathcal{A}_1$,  $\mathcal{A}_1$ --- алфавит, над которым построен $\Pc^{1,a}$, $\mathcal{A}^{\ev}$ --- алфавит, над которым построен $\Pc^{\ev}$.
\end{theorem}


\subsection{Выводы по главе}
В главе были даны основные определения и понятия, используемые в теории алгебраических байесовских сетей. Представлены определения фрагмента знаний, алгебраической байесовской сети. Рассмотрен логико"=вероятностный вывод в АБС: его виды и задачи, решаемые с помощью ЛВВ, а также основные теоретические результаты, которые ложатся в основу решения задач, рассматриваемых в данной работе.

Стоит отметить, что предложенный в данной главе подход обладает рядом недостатков~\cite{70}, а именно содержит в себе операции, не относящиеся к матрично"=векторным вычислениям. К таким операциям относится функция $\Gind$, для которой ранее не была предложена матрично"=векторная формализация. 

Таким образом, ставится задача улучшить имеющийся результат и заменить функцию $\Gind$ матрично"=векторным аналогом. Также имеющийся результат относится только к одной модели ФЗ и не подходит для двух других моделей, поэтому ставится задача описания алгоритмов глобального ЛВВ для альтернативных моделей фрагмента знаний.


\section{Матрично"=векторная  формализация глобального логико"=вероятностного вывода}\label{cap3}
\subsection{Введение}
В данной главе рассмотрим алгоритм распространения виртуального свидетельства между двумя фрагментами знаний для трех математических моделей ФЗ и получим соответствующие матрично"=векторные уравнения. 

Для того чтобы улучшить результат для модели идеала конъюнктов, представленный в предыдущей главе, и получить уравнения для модели идеала дизъюнктов и модели множества квантов, рассмотрим матрично-векторную формализацию функции $\Gind$. Для этого введем понятие характеристического вектора конъюнкта, дизъюнкта и кванта и матрицы проекции $\G$. Также рассмотрим матрично-векторное формирование матрицы перехода $\V$ от вектора вероятностей элементов идеала дизъюнктов к вектору вероятностей элементов виртуального свидетельства и приведем поясняющие примеры, введем матрицу перехода от вектора дизъюнктов к вектору квантов. Кроме этого докажем теорему, предлагающую матрично"=векторный подход к пропагации виртуального свидетельства в случае пропозиций-квантов. Используя полученные результаты, приведем получившиеся матрично-векторные уравнения для каждой из трех моделей фрагмента знаний. 
\subsection{Матрично"=векторная интерпретация для функции $\Gind$}
Рассмотрим функцию $\Gind(i,m)$ в контексте математической модели фрагмента знаний, заданного идеалом конъюнктов. Для двух других моделей формализация будет аналогичной, потому что сама функция $\Gind(i,m)$ определяется аналогичным образом~\cite{74, 49}, а алфавиты для всех трех моделей одинаковы.

Напомним, что  $\Gind(i,m)$ --- функция, которая по индексу $m$ наибольшего элемента $C^{\ev}$ в алфавите $\mathcal{A}$ и индексу конъюнкта $i$ в алфавите $\mathcal{A}_{\ev}$ возвращает индекс соответствующего конъюнкта в алфавите $\mathcal{A}$~\cite{84}. Так как индексы конъюнктов пронумерованы согласно строго заданному правилу, то у каждого индекса есть свой строго заданный номер, который можно рассмотреть в двоичной системе счисления или как характеристический вектор из нулей и единиц. Аналогично можно рассмотреть индексы квантов и дизъюнктов. 

\begin{definition}
Характеристический вектор конъюнкта --- это вектор из нулей и единиц, в котором единица на $i$-том месте соответствует единице на $i$-том месте в двоичной записи индекса конъюнкта, а ноль на $i$-том месте --- нулю.
\end{definition}

Будем обозначать характеристический вектор  конъюнкта с индексом $i$ как  $\chi^{i}$. Аналогичным будет обозначение для характеристического вектора дизъюнкта или кванта. Так как в данном разделе рассматриваются ФЗ над идеалами конъюнктов, будем далее рассматривать характеристический вектор конъюнкта.

Можно заметить, что функция $\Gind$ проецирует индекс, соответствующий конъюнкту детерминированного свидетельства, на другой индекс, соответствующий эквивалентному данному конъюнкту ФЗ, построенного над алфавитом $\mathcal{A}$. Вместо индексов можно взять соответствующие характеристические векторы и построить проекцию одного на другой.


Пусть $\chi^{i}$ --- характеристический вектор $i$-того конъюнкта свидетельства в алфавите  $\mathcal{A}_{\ev}$, и  пусть $\chi^{j}$ --- характеристический вектор соответствующего ему $j$-того конъюнкта в алфавите  $\mathcal{A}$. $m$ --- мощность алфавита $\mathcal{A}_{\ev}$, $n$ --- мощность алфавита $\mathcal{A}$.

Тогда $\Gind(i,m)$ можно заменить матрицей проекции вектора $\chi^{i}$ на $\chi^{j}$ по следующему правилу:
\begin{equation}\label{G}
\G[i,j] = \begin{cases}
1 \text{, если $\mathcal{A}[n -1-i] = \mathcal{A}_{\ev}[m - 1 -j]$,} \\
0 \text{, иначе};
\end{cases} 
\end{equation}

Такая матрица будет размера $n \times m$, где $n$ --- мощность алфавита $\mathcal{A}$ и $m$ --- мощность алфавита $\mathcal{A}_{\ev}$.

Чтобы получить характеристический вектор конъюнкта $\chi^{j}$ в алфавите $\mathcal{A}$, нужно будет домножить вектор $\chi^{i}$ слева на матрицу $\G$: $\chi^{j} = \G\chi^{i}$.

По построению матрица $\G$ будет содержать в себе нулевые строки, соответствующие элементам алфавита $\mathcal{A}$, которые не входят в алфавит $\mathcal{A}_{\ev}$. Единица на j-той позиции в ненулевой строке i соответствует атому $x_i \in \mathcal{A}$, эквивалентному атому $x_j \in \mathcal{A}_{\ev}$. Умножение нулевых строк на характеристический вектор будет давать нулевой элемент в результирующем векторе,  умножение ненулевых строк даст единицу, если элемент алфавита входит в конъюнкт, и ноль иначе. Таким образом, матрица $\G$ проецирует характеристический вектор конъюнкта, построенного над алфавитом $\mathcal{A}_{\ev}$ на характеристический вектор конъюнкта над алфавитом $\mathcal{A}$.

\subsubsection{Пример использования матрично-векторной интерпретации функции $\Gind$}
Приведем поясняющий пример. Пусть $\mathcal{A} =  \{x_1, x_2, x_3, x_4\}$ и $\mathcal{A}_{\ev} =  \{x_2, x_4\}$. Построим матрицу проекции $\G$ по формуле \ref{G}:

\begin{math}
\G[i,j] = \begin{cases}
1 \text{, если $\mathcal{A}[n -1-i] = \mathcal{A}_{\ev}[m - 1 - j]$,} \\
0 \text{, иначе};
\end{cases}
= \begin{pmatrix}
1 & 0 \\ 0 & 0 \\ 0 & 1\\ 0 & 0
\end{pmatrix}.
\end{math}

Пусть $\chi^{1} = \begin{pmatrix} 0 \\ 1 \end{pmatrix}$ и  соответствует конъюнкту $x_2$ свидетельства. Найдем характеристический вектор индекса конъюнкта $x_2$ в алфавите $\mathcal{A}$, домножив вектор $\chi^{1}$ на матрицу $\G$:

\begin{math}
\begin{pmatrix}
1 & 0 \\ 0 & 0 \\ 0 & 1\\ 0 & 0
\end{pmatrix}
\begin{pmatrix} 0 \\ 1 \end{pmatrix} 
= \begin{pmatrix} 0\\ 0 \\  1 \\ 0 \end{pmatrix}.
\end{math}

Видно, что искомый индекс получен.

\subsubsection{Уравнения для решения второй задачи апостериорного вывода}


Заменим в уравнениях \ref{conj1} и \ref{quants1}, приведенных во второй главе, функцию $\Gind$ на матрицу проекции $\G$. 

Уравнение \ref{conj1} для конъюнктов приобретет следующий вид:
\begin{equation} \label{conjG}
    \Pc^a=\sum_{i=0}^{2^{n^\prime}-1}\dfrac{\T^{\langle \G\chi^{i} ,\,\G\chi^{2^{n^\prime}-1-i} \rangle }\Pc}{(\mathbf{r}^{\langle  \G\chi^{i},\,\G\chi^{2^{n^\prime}-1-i} \rangle},\Pc)}\In\Pc^{\ev}[i].
\end{equation} 

Уравнение \ref{quants1} для квантов будет следующим:
\begin{equation} \label{quantsG}
 \Pq^{\evidenceNumbers}=\sum_{i=0}^{2^{n^\prime}-1}\dfrac{{\Selector^{\langle \G\chi^{i} ,\, \G\chi^{2^{n^\prime}-1-i}\rangle }}\circ \Pq}{(\Selector^{\langle\G \chi^{i} ,\, \G\chi^{2^{n^\prime}-1-i}\rangle},\Pq)}\Pq^{\ev}[i].
\end{equation}

Уравнение \ref{dis2} для дизъюнктов заменится следующим:
\begin{equation}\label{dis3}
 \Pd^{a, \prime} =\sum_{i=0}^{2^{n^\prime}-1}\dfrac{\M^{\langle  \G\chi^{i},\,\G\chi^{2^{n^\prime}-1-i} \rangle  }\Pd^{\prime}}{(\mathbf{d}^{\langle \G\chi^{i} ,\, \G\chi^{2^{n^\prime}-1-i}\rangle },\Pd^{\prime})}\Ln\Pd^{\prime \, \ev}[i].
\end{equation}
\subsection{Формализация формирования матрицы перехода}
\subsubsection{Теорема о построении матрицы перехода}
Рассмотрим фрагменты знаний, построенные над идеалами дизъюнктов. Аналогично случаю для конъюнктов~\cite{51, 9}, построим вектор вероятностей элементов виртуального свидетельства. В виртуальное свидетельство будут входить элементы, стоящие на пересечении двух фрагментов знаний.

Сначала введем матрицу перехода от вектора вероятностей квантов к вектору вероятностей дизъюнктов. Воспользуемся приведенным во второй главе выражением ~\ref{pdtopq} для матрицы перехода $\Ln$ от вектора $\Pd^{\prime}$ к вектору $\Pq$. Так как $\Ln$ имеет обратную матрицу, то домножим выражение для перехода от вектора $\Pd^{\prime}$ к вектору $\Pq$ на обратную матрицу и получим: 
\begin{equation*}
 \Ln^{-1} \Pq  = \mathbf{1} -\Pd,
\end{equation*}
где $\Ln^{-1} = \begin{pmatrix} 1 & 1 \\ 1 & 0 \end{pmatrix}^{[n]}.$

Обозначим $\On = \Ln^{-1}$. Перепишем выражение:
\begin{equation*}\mathbf{1} -\Pd = \On \Pq, \end{equation*} где $\On = \begin{pmatrix} 1 & 1 \\ 1 & 0 \end{pmatrix}^{[n]}.$

Теперь построим матрицу перехода $\V$ от вектора вероятностей  элементов идеала дизъюнктов к вектору вероятностей элементов виртуального свидетельства. Выделим из вектора $\Pd^{\prime 1,a}$, содержащего новые апостериорные оценки, элементы, принадлежащие виртуальному свидетельству.

 От вектора вероятностей квантов можно перейти к вектору вероятностей дизъюнктов с помощью матрицы перехода: $\Pd^{\prime 1,a} = \On\Pq^{1,a}$. При умножении $i$-той строки $\On[i]$ на $\Pq^{1,a}$, получается $i$-тый элемент вектора $\Pd^{\prime 1,a}$. Если выделить в отдельную матрицу строки $\On[i]$, выделяющие из $\Pd^{\prime 1,a}$ элементы виртуального свидетельства, то при умножении эту матрицы на $\Pq^{1,a}$, получился нужный нам вектор оценок вероятностей.

Для того чтобы выделить строки $\On[i]$, нужно домножить $\On$ слева на матрицу $\V$ размерности $m \times n$~($m$ --- длина вектора $\Pd^{\prime \, \ev}$,  $n$ --- длина вектора $\Pd^{\prime 1,a}$). Элементы матрицы определим по следующему правилу:
\begin{equation*}
    \V[i,j] = 
    \begin{cases}
        1 \text{, если $\Pd^{\prime 1,a}[j] = \Pd^{\prime \, \ev}[i]$,}\\
        0 \text{, иначе;}
    \end{cases}
\end{equation*}

При этом под $\Pd^{\prime 1,a}[i]$ и $\Pd^{\prime \, \ev}[i]$ подразумеваются сами дизъюнкты, а не значения вероятностей. Затем нужно умножить матрицу на $\Pq^{1,a}$. Таким образом, $\Pd^{\prime \, \ev} = \V\On\Pq^{1,a} = \V\Pd^{\prime 1,a}$.

Алфавит, над которым построено свидетельство, можно найти как $\mathcal{A}_{\ev} = \mathcal{A}_1 \cap \mathcal{A}_2$, где $\mathcal{A}_1$ --- алфавит первого фрагмента знаний, $\mathcal{A}_2$ --- алфавит второго фрагмента знаний.

Теперь покажем, что матрицу перехода $\V$ можно построить через кронекерово произведение, и обоснуем предложенное построение.

\begin{theorem}\label{disq}
Вектор оценок виртуального свидетельства $\Pd^{\prime \ev}$ можно вычислить как $\Pd^{\prime \ev} = \V \Pd^{\prime a,1}$, где $\V = \VijTilda_{n-1} \otimes \VijTilda_{n-2} \otimes ... \otimes \VijTilda_{0}$ ,
 $\VijTilda_k = \begin{cases}
\V^+ \text{, если $x_k$ входит в $\mathcal{A}^{\ev}$,} \\
\V^- \text{, иначе;}
\end{cases}$, 
\begin{math}
    \V^+ = \begin{pmatrix} 1 & 0 \\ 0 & 1 \end{pmatrix},
    \V^- = \begin{pmatrix} 1 & 0 \end{pmatrix},
\end{math}\\
$x_k$ --- $k$-тый элемент в $\mathcal{A}_1$,  $\mathcal{A}_1$ --- алфавит, над которым построен $\Pd^{\prime 1,a}$, $\mathcal{A}^{\ev}$ --- алфавит, над которым построен $\Pd^{\prime \ev}$.
\end{theorem}
\begin{Proof}
Для упрощения доказательства дополним матрицу $\V^-$ строкой из нулей и будем рассматривать построение матрицы $\V$ через матрицы $\V^+$ и $\V^{\prime-}$, где $\V^{\prime-} = \begin{pmatrix} 1 & 0 \\ 0 & 0 \end{pmatrix}$. Тогда искомая матрица будет квадратной и диагональной по построению и будет содержать строки из нулей. 

Также можно заметить, что $i$-тая единица на диагонали означает, что оценка на $i$-том месте в векторе вероятностей входит в виртуальное свидетельство. Ноль же означает, что оценка не относится к вектору виртуального свидетельства. 

Обоснуем, что позиции нулей и единиц при таком построении действительно показывают, входит ли оценка в виртуальное свидетельство.

Рассмотрим на каких позициях диагонали матрицы $\V$ будут нули.
По построению каждое произведение Кронекера увеличивает размерность искомой матрицы вдвое.
Можно заметить, что нули могут появиться только при участии в произведении матрицы $\V^{\prime-}$, в которой есть единственный $0$, находящийся на диагонали.

Получается, что по построению матрица $\V^{\prime-}$, соответствующая условию $x_m \not \in \mathcal{A}^{\ev}$, в
произведении Кронекера даст нули на всех позициях $\V[i,i]$, где
$i\&2^{m+1} = 2^{m+1}$, где $\&$ обозначает операцию побитового И. Множество таких чисел соответствует множеству чисел, в двоичной записи которых на месте $m$ стоит единица~(Нумерация мест в двоичной записи и нумерация атомов в алфавите начинается с 0). Сравнив с правилом нумерации дизъюнктов, получим, что все такие числа соответствуют номерам дизъюнктов, в которых присутствует атом с номером $m$. 

Следовательно, участие матриц $\V^{\prime-}$ в кронекеровом произведении занулит вероятности в результирующем векторе $\Pd^{\ev}$ всех дизъюнктов $d_m$, для которых не выполняется условие $\forall x_k \in d_m \, x_k \in \mathcal{A}_{\ev}$. А умножение $\Pd^{1,a}$ на столбец матрицы с единицей на диагонали как раз выделит в результирующий вектор нужную оценку вероятности.

Осталось разобраться с размерностью результирующего вектора $\Pd^{\ev}$. Он получается одинаковой размерности с исходным вектором $\Pd^{ 1,a}$ и содержит в себе нулевые элементы, возникающие, как раз, из-за видоизменения матрицы $\V^-$. Тогда при замене матрицы  $\V^{\prime-}$ на $\V^-$ получим вектор $\Pd^{\ev}$ нужной размерности. Так как $\Pd^{\prime}$ линейно выражается через $\Pd$, то результат будет корректным и для векторов $\Pd^{\prime 1,a}$  и $\Pd^{\prime \ev}$.
\end{Proof}

\subsubsection{Пример построения матрицы перехода}
 Рассмотрим на примере формирование матрицы $\V$ с помощью предложенного матрично-векторного алгоритма. Пусть даны два фрагмента знаний над алфавитами $\mathcal{A}_1 = \{x_1, x_2, x_3\}$ и $\mathcal{A}_2 = \{x_1, x_3, x_4\}$. Будем считать, что первый фрагмент знаний содержит апостериорные оценки после пропагации свидетельства и нужно пропагировать свидетельство дальше из первого ФЗ во второй. 
Выделим элементы, относящиеся к виртуальному свидетельству, построив матрицу перехода $\V$:
\begin{equation*}
\Pd^{\prime 1} =  \begin{pmatrix}
1 \\ p(\overline{x}_1) \\ p(\overline{x}_2) \\ p(\overline{x}_2\overline{x}_1) \\ p(\overline{x}_3) \\ p(\overline{x}_3\overline{x}_1) \\ p(\overline{x}_3\overline{x}_2) \\ p(\overline{x}_3\overline{x}_2\overline{x}_1)
\end{pmatrix}, 
\Pd^{\prime 2} =  \begin{pmatrix}
1 \\ p(\overline{x}_1) \\ p(\overline{x}_3) \\ p(\overline{x}_3\overline{x}_1) \\ p(\overline{x}_4) \\ p(\overline{x}_4\overline{x}_1) \\ p(\overline{x}_4\overline{x}_3) \\ p(\overline{x}_4\overline{x}_3\overline{x}_1)
\end{pmatrix},
\Pd^{\prime \ev} =  \begin{pmatrix}
1 \\ p(\overline{x}_1) \\ p(\overline{x}_3) \\ p(\overline{x}_3\overline{x}_1)
\end{pmatrix},
\end{equation*}
$\mathcal{A}_{\ev} = \mathcal{A}_1 \cap \mathcal{A}_2 = \{x_1, x_3\}$.
 
 Тогда $\V = \V^+ \otimes \V^- \otimes \V^+ = 
 \begin{pmatrix} 1 &0 \\ 0 &1 \end{pmatrix} \otimes
  \begin{pmatrix} 1 &0  \end{pmatrix} \otimes
 \begin{pmatrix} 1 &0 \\ 0 & 1 \end{pmatrix} 
 = \\
 \begin{pmatrix} 1 &  0 & 0 & 0 & 0 &  0 & 0 &  0 & 0 \\
                 0 & 1 & 0 & 0 & 0 & 0 & 0 & 0 & 0 \\
                 0 & 0 & 0 & 0 & 1 & 0 & 0 & 0 & 0 \\
                 0 & 0 & 0 & 0 & 0 & 1 & 0 & 0 & 0 \end{pmatrix} $.
                 
                 
Видно, что единицы на $i,j$-тых местах соответствуют выполнению равенства $\Pd^{\prime a,1} [i]= \Pd^{\prime \ev }[j]$, то есть таким построением была получена действительно матрица перехода от вектора $\Pd^{\prime a,1} $ к вектору $\Pd^{\prime \ev }$.

\subsection{Уравнения глобального логико"=вероятностного вывода}
\subsubsection{Матрично-векторное уравнение пропагации виртуального свидетельства над идеалом конъюнктов}

Перепишем приведенное во второй главе уравнение \ref{conjglob1} для распространения виртуального свидетельства между двумя фрагментами знаний с учетом описанной выше матрично"=векторной формализации функции $\Gind$. Заменим функцию $\Gind$ на матрицу проекции $\G$, тогда уравнение примет следующий вид:
\begin{equation} \label{conjglob2}
    \Pc^a=\sum_{i=0}^{2^{n^\prime}-1}\dfrac{\T^{\langle  \G\chi^{i},\,\G\chi^{2^{n^\prime}-1-i} \rangle }\Pc}{(\mathbf{r}^{\langle  \G\chi^{i},\,\G\chi^{2^{n^\prime}-1-i} \rangle},\Pc)}\In\Pc^{\ev}[i],
\end{equation} 
где \begin{math}\G[i,j] = \begin{cases}
1 \text{, если $\mathcal{A}_2[n -1-i] = \mathcal{A}_{\ev}[n^{\prime} - 1  - j]$,} \\
0 \text{, иначе;}
\end{cases}
\end{math}
\\ $n^{\prime}$ --- размерность алфавита $\mathcal{A}_{\ev}$, $n$ --- размерность алфавита $\mathcal{A}_2$.

\subsubsection{Матрично-векторное уравнение пропагации виртуального свидетельства над идеалом дизъюнктов}

Рассмотрим алгоритм пропагации виртуального свидетельства из одного фрагмента знаний в другой. Воспользуемся теоремой \ref{disq} и  вычислим $\Pd^{\prime \, \ev} = \V\Pd^{\prime}$.

Далее, основываясь на уравнении \ref{dis3} и подставив в него выражение для вычисления вектора вероятностей элементов виртуального свидетельства, а также заменив функцию $\Gind$ матрицей $\G$, получим, что пропагировать виртуальное свидетельство из одного ФЗ с апостериорными оценками в соседний ФЗ можно с помощью следующего уравнения:
\begin{equation}\label{disGlob}
    \Pd^{\prime \, 2,a}=\sum_{i=0}^{2^{n^\prime}-1}\dfrac{\M^{\langle\G\chi^{i},\, \G \chi^{2^{n^\prime}-1-i} \rangle }\Pd^{\prime \, 2}}{(\mathbf{d}^{\langle \G \chi^{i},\,\G\chi^{2^{n^\prime}-1-i} \rangle},\Pd^{\prime \, 2})}\Ln\Pd^{\prime \, \ev}[i],
\end{equation}
где \begin{math}\G[i,j] = \begin{cases}
1 \text{, если $\mathcal{A}_2[n -1-i] = \mathcal{A}_{\ev}[n^{\prime} - 1 - j]$,} \\
0 \text{, иначе;}
\end{cases}
\end{math}
\\ $n^{\prime}$ --- размерность алфавита $\mathcal{A}_{\ev}$, $n$ --- размерность алфавита $\mathcal{A}_2$.

\subsubsection{Матрично-векторное уравнение пропагации виртуального свидетельства над множеством квантов}

Рассмотрим распространение виртуального свидетельства для случая, когда фрагменты знаний построены над множеством пропозиций"=квантов со скалярными оценками. Для того чтобы получить необходимое уравнение, необходимо научиться находить значения оценок вероятностей виртуального свидетельства $\Pq^{\ev}$. Под виртуальным свидетельством будем понимать фрагмент знаний, содержащий оценки вероятностей для квантов, согласованных как и с первым, так и со вторым фрагментом знаний.
 \begin{theorem}
Пропагировать виртуальное свидетельство из одного фрагмента знаний во второй, когда ФЗ построены над множествами пропозиций-квантов, можно с помощью следующего уравнения:
\begin{equation}\label{quantsGlob}
  \Pq^2=\sum_{i=0}^{2^{n^\prime}-1}\dfrac{{\Selector_2^{\langle  \G_{\mathcal{A}_2, \mathcal{A}_{\ev}} \chi^{i},\,\G_{\mathcal{A}_2, \mathcal{A}_{\ev}}\chi^{2^{n^\prime}-1-i} \rangle }}\circ \Pq^2}{(\Selector_2^{\langle  \G_{\mathcal{A}_2, \mathcal{A}_{\ev}}\chi^{i},\, \G_{\mathcal{A}_2, \mathcal{A}_{\ev}}\chi^{2^{n^\prime}-1-i}\rangle},\Pq^2)}(\Selector_1^{\langle  \G_{\mathcal{A}_1, \mathcal{A}_{\ev}}\chi^{i},\,\G_{\mathcal{A}_1, \mathcal{A}_{\ev}}\chi^{2^{n^\prime}-1-i} \rangle},\Pq^1),
\end{equation}
где $\Pq^1$ --- вектор апостериорных оценок первого ФЗ, а $\Pq^2$ --- вектор оценок второго ФЗ, куда нужно пропагировать свидетельство, $n^\prime$ --- мощность алфавита свидетельства $\mathcal{A}^{\ev}$, $m$ --- мощность алфавита $\mathcal{A}_1$, а $n$ --- мощность алфавита $\mathcal{A}_2$,\\
\begin{math}
\G_{\mathcal{A}_2, \mathcal{A}_{\ev}}[i,j] = 
\begin{cases}
1 \text{, если $\mathcal{A}_2[n-1 -i] = \mathcal{A}_{\ev}[n^\prime -1- j]$,} \\
0 \text{, иначе;}
\end{cases}
\end{math}\\
\begin{math}\G_{\mathcal{A}_1, \mathcal{A}_{\ev}}[i,j] = \begin{cases}
1 \text{, если $\mathcal{A}_1[m -1-i] = \mathcal{A}_{\ev}[n^\prime -1- j]$,} \\
0 \text{, иначе;}
\end{cases}
\end{math}
\end{theorem}
\begin{Proof}
Рассмотрим задачу нахождения оценок вероятностей элементов виртуального свидетельства $\Pq^{\ev}$. Можно заметить, что вектор $\Pq^{\ev}$ будет содержать оценки вероятностей для квантов, согласованных как и с первым, так и со вторым фрагментом знаний.

Алфавит, над которым построено свидетельство, можно найти как $\mathcal{A}_{\ev} = \mathcal{A}_1 \cap \mathcal{A}_2$, где $\mathcal{A}_1$ --- алфавит первого фрагмента знаний, $\mathcal{A}_2$ --- алфавит второго фрагмента знаний.

Таким образом, виртуальное свидетельство состоит из множества пропозиций-квантов, построенных над алфавитом $\mathcal{A}_{\ev}$. Для каждого кванта необходимо посчитать его вероятность, используя апостериорные оценки из первого фрагмента знаний. Это является первой задачей апостериорного вывода, если мы рассматриваем квант как детерминированное свидетельство $\evidenceNumbers$. Решение первой задачи апостериорного вывода выглядит следующим образом~\cite{74}:
\begin{equation*}
p(\langle i, j \rangle) = (\Selector^{\evidenceNumbers},\Pq^1).\end{equation*}

Значит, воспользуемся матрицей проекции $\G$ и получим значение вероятности каждого $i$-того кванта виртуального свидетельства:
\begin{equation}\label{quantsVirt}
\Pq^\ev[i] =(\Selector^{\langle  \G_{\mathcal{A}_1, \mathcal{A}_{\ev}} \chi^{i},\, \G_{\mathcal{A}_1, \mathcal{A}_{\ev}} \chi^{2^{n^\prime}-1-i}\rangle},\Pq^1),
\end{equation}
где $n^\prime$ --- мощность алфавита свидетельства $\mathcal{A}^{\ev}$, а $m$ --- мощность алфавита $\mathcal{A}_1$
и $\G_{\mathcal{A}_1, \mathcal{A}_{\ev}}[i,j] = \begin{cases}
1 \text{, если $\mathcal{A}_1[m -1-i] = \mathcal{A}_{\ev}[n^\prime - 1 -j]$,} \\
0 \text{, иначе;}
\end{cases}$

 В нижнем индексе матрицы $\G$ явно укажем алфавиты, участвующие в ее формировании, чтобы в дальнейшем различать различно сформированные матрицы в одном уравнении. У вектора-селектора также добавим нижние индексы $1$ и $2$, указывающие, элементы какого алфавита используются при его построении. 
 
Далее воспользуемся уравнением \ref{quantsG} для решения второй задачи апостериорного вывода для стохастического свидетельства, подставим в него выражение \ref{quantsVirt} и получим искомое уравнение. 
\end{Proof}
\subsection{Выводы по главе}
В данной главе была предложена матрично"=векторная формализация для функции $\Gind$ и предложены матрично"=векторные уравнения распространения виртуального свидетельства между двумя соседними фрагментами знаний для двух моделей ФЗ: идеал дизъюнктов, множество пропозиций-квантов. Была сформулирована и доказана теорема о формировании матрицы перехода от вектора оценок элементов ФЗ над идеалом дизъюнктов к вектору оценок виртуального свидетельства, а также доказана теорема о пропагации виртуального свидетельства между двумя ФЗ, построенными над множествами квантов. Введен характеристический вектор конъюнкта, дизъюнкта,  кванта и матрица проекции $\G$, матрица перехода от вектора вероятностей дизъюнктов к вектору вероятностей квантов. Также приведены уравнения локального ЛВВ и уравнение пропагации виртуального свидетельства между двумя ФЗ над идеалами конъюнктов после замены функции $\Gind$  на матрицу $\G$.


\section{Программная реализация}
\subsection{Введение}
В предыдущих главах были описаны основные имеющиеся и полученные теоретические результаты, а в данной главе будет рассмотрена программная реализация. Будет описана архитектура существующего комплекса программ, представлены основные классы и методы для реализации алгоритмов глобального логико-вероятностного вывода для различных моделей ФЗ, а также приведены примеры работы программы.

Полученная в ходе работы программная реализация разработана на языке C\#, в качестве среды разработки использовалась среда JetBrains Rider~\cite{68}, которая обладает меньшей функциональностью в отличие от Visual Studio, но отличается удобством и быстротой работы. Для совместной работы использовался репозиторий на BitBucket~\cite{111} и система контроля версий Git. Для тестирования использовалась библиотека NUnit~\cite{67}, а для работы с матрицами использовалась библиотека Math.Net Numerics~\cite{66}.
\subsection{Архитектура программного комплекса}
Комплекс программ AlgBN Math Library~\cite{89} представляет собой библиотеку для автоматизации логико"=вероятностного вывода в АБС, в которой уже реализованы структуры данных для хранения фрагмента знаний и АБС, а также алгоритмы локального логико-вероятностного вывода и проверки непротиворечивости. Фрагменты знаний можно создавать с бинарными, скалярными и интервальными оценками вероятностей. Рассмотрим наиболее важные структуры, имеющиеся в библиотеке.

 Фрагмент знаний, представленный идеалом конъюнктов с скалярными оценками представлен классом ScalarConjKP, с интервальными оценками --- IntervalConjKP, с бинарными --- BinaryConjKP. Для создания экземпляра любого из этих классов нужно передать в конструктор глобальный индекс ФЗ и массив оценок вероятностей элементов ФЗ. Глобальный индекс --- это число, единицы в двоичной записи которого соответствуют номерам элементов алфавита, над которыми построен ФЗ. Аналогичные классы есть для двух других моделей ФЗ: для дизъюнктов --- BinaryDisjKP, ScalarDisjKP и IntervalDisjKP, для квантов --- BinaryQuantKP, ScalarQuantKP и IntervalQuantKP.

Структура Propagator состоит из классов, позволяющих пропагировать детерминированное, стохастическое или неточное свидетельство в ФЗ с скалярными или интервальными оценками. Каждый из таких классов содержит методы propagate для пропагации свидетельства и getResult для возвращения результата пропагации. Каждая реализация отвечает за пропагацию одного из типов свидетельства в ФЗ с одним видом оценок. Например, StochasticScalarConjunctsLocalPropagator отвечает за пропагацию стохастического свидетельства в ФЗ с скалярными оценками. Структура Propagator позволяет работать с двумя моделями ФЗ --- идеал конъюнктов и идеал дизъюнктов. Для третьей модели функциональность еще не реализована, а алгоритмы, работающие с идеалами дизъюнктов, требуют доработки.

За проверку локальной непротиворечивости отвечает структура In\-ferrer. Структура MatrixTransform реализует матрицы перехода между различными моделями ФЗ. Для решения задач линейного программирования используется библиотека на C++ lp\_solve55, за обращение к библиотеке отвечает класс LP. Кроме того, есть структура Alphabet для работы с алфавитом.

Также в комплексе есть программный модуль ABNGlobal, отвечающий за глобальную непротиворечивость. Так как пропагация виртуального свидетельства относится к глобальному ЛВВ, то расширим эту часть библиотеки своей реализацией. 

Реализация будет использовать существующие структуры для локального ЛВВ. Так как существующие алгоритмы позволяют работать с интервальными оценками, то реализация алгоритмов пропагации виртуального свидетельства будет разработана как и для ФЗ с интервальными оценками, так и для неточных виртуальных свидетельств. Существующую реализацию~\cite{4, 65} алгоритмов  локального логико"=вероятностного вывода для квантов перенесем на существующие в комплексе программ структуры хранения, адаптируем и затем переиспользуем при реализации алгоритма пропагации виртуального свидетельства для данной модели. Реализации Propagator для дизъюнктов также адаптируем и усовершенствуем с целью дальнейшего переиспользования.



\subsection{Описание классов и методов}
Для того чтобы реализовать алгоритмы распространения виртуального свидетельства, была расширена функциональность абстракций фрагмента знаний. Соответственно, для интерфейсов каждой модели~(идеал конъюнктов, идеал дизъюнктов, множество квантов) в случае скалярных и интервальных оценок был добавлен статический класс с тремя методами расширения: \begin{enumerate}
    \item IsIntersect --- проверяет, пересекаются ли между собой два фрагмента знаний, и возвращает true, если пересекаются, и false иначе;
    \item IsEqual --- позволяет сравнивать два фрагмента знаний между собой с заданной точностью и возвращает true, если ФЗ эквивалентны, и false иначе;
    \item GetSeparator --- возвращает пересечение двух фрагментов знаний, если пересечение не пусто, а в случае, если ФЗ не пересекаются, возвращается пустой фрагмент знаний.
\end{enumerate}

Всего было создано 6 таких статических классов.

Рассмотрим подробнее работу метода GetSeparator. В случае конъюнктов и дизъюнктов основную работу по выделению элементов сепаратора делает следующий цикл:

\begin{lstlisting}[caption = Цикл для выделения элементов виртуального свидетельства]
for (int index = 0; index <= KPGlobalIndex; index++)
{
    if ((index & KPGlobalIndex) == index)
    {
        if ((index & separatorGlobalIndex) == index)
        {
            probability[separatorIndex] =
                firstKP.GetPointEstimate(probIndex);
            probIndex++;
            separatorIndex++;
        }
        else
        {
            probIndex++;
        }
    }
}
\end{lstlisting}

Стоит отметить, что при программной реализации алфавит удобнее задавать с помощью глобального индекса. Глобальный индекс сепаратора можно получить с помощью операции побитового И глобальных индексов первого и второго фрагментов знаний, по этому принципу работает метод IsIntersect. Заметим, что количество единиц в глобальном индексе соответствует мощности алфавита. 

Данный цикл также использует операцию побитового И. Он проверяет все возможные числа index, которые могут соответствовать элементам вектора вероятностей, это числа от 0 до KPGlobalIndex --- глобального индекса первого ФЗ. Первое условие проверяет это соответствие, второе условие аналогично проверяет, соответствует ли индекс элементу сепаратора. Если условия верны, то значение вероятности сохраняется в массив вероятностей probability. Индексы текущих элементов в векторе вероятностей фрагмента знаний и векторе вероятностей сепаратора задаются с помощью переменных probIndex и separatorIndex соответственно.
 
 Отметим, что данный цикл реализует функциональность матриц перехода  $\Qmatr$ и $\V$, а сами матрицы строить не нужно, потому что, благодаря глобальным индексам, при реализации сразу понятно, какие элементы формируют виртуальное свидетельство.
 
 Так как у модели ФЗ, представленной квантами, виртуальное свидетельство содержит не просто оценки элементов на пересечении, а согласованные оценки, то тут алгоритм построения виртуального свидетельства будет другим. 
 
 Рассмотрим реализацию метода GetSeparator для построения виртуального свидетельства со скалярными оценками.  Вспомогательный метод GetQuantIndexes возвращает вектор из глобальных индексов соответствующих квантов, составляющих ФЗ.  В цикле для каждого индекса кванта из сепаратора считается его вероятность как вероятность детерминированного свидетельства с помощью метода GetEvidenceProbability объекта класса DeterministicScalarQuantLocalPropagator.
 
 \begin{lstlisting}[caption = Метод построения виртуального видетельства со скалярными оценками над квантами]
public static ScalarQuantKP_I GetSeparator(ScalarQuantKP_I firstKP, long separatorGlobalIndex)
{
    var numberOfElements = Convert
        .ToString(separatorGlobalIndex, 2)
        .Count(c => c == '1');
    var numberOfAtoms = (int)Math.Pow(2, numberOfElements);
    var probability = new double[numberOfAtoms];
    var sepAtomGlobalIndexes = GetQuantIndexes(numberOfAtoms, separatorGlobalIndex);
    var propagator = DeterministicScalarQuantLocalPropagator
    	.Instance(firstKP);
            
    for (int index = 0; index < sepAtomGlobalIndexes.Length; index++)
    {
        var binaryQuantKp = new BinaryQuantKP(separatorGlobalIndex);
        binaryQuantKp.SetQuant(sepAtomGlobalIndexes[index]);
        probability[index] = propagator.evidenceProbaility(binaryQuantKp);
    }
    return new ScalarQuantKP(separatorGlobalIndex, probability);
}
\end{lstlisting}

Кроме того, был создан интерфейс IVirtualEvidencePropagator и 6 его реализаций для каждой модели ФЗ в случае скалярных оценок и в случае интервальных оценок.

Дадим краткое описание классов, реализующих алгоритмы распространения виртуального свидетельства для модели ФЗ, представленной идеалом конъюнктов. 

 VirtualEvidenceScalarConjPropagator --- класс, отвечающий за следующие действия: пропагация стохастического виртуального свидетельства и пропагация неточного виртуального свидетельства во фрагмент знаний со скалярными оценками. Соответственно, класс пропагатора содержит 2 метода для двух видов свидетельства. 
 
 Метод PropagateStochasticEvidence используется в случае стохастического свидетельства. Метод принимает 2 параметра: ScalarConjKP\_I firstKP --- первый фрагмент знаний, ScalarConjKP\_I secondKP --- второй фрагмент знаний, в который нужно пропагировать виртуальное свидетельство. В этом методе сначала создается виртуальное свидетельство, далее оно пропагируется с помощью объекта propagator класса  StochasticScalarConjunctsLocalPropagator и его экземплярных методов propagate и getResult. 
 \begin{lstlisting} [caption = Реализация метода PropagateStochasticEvidence]
 public ScalarConjKP_I PropagateStochasticEvidence(ScalarConjKP_I firstKP, ScalarConjKP_I secondKP)
{
    var evidence = firstKP.GetSeparator(secondKP);
    var propagator = new StochasticScalarConjunctsLocalPropagator(secondKP);
    propagator.propagate(evidence);
    return propagator.getResult();
}\end{lstlisting}
 
Второй метод PropagateImpreciseEvidence нужен для пропагирования неточного виртуального свидетельства во второй фрагмент знаний, содержащий скалярные оценки. Пропагирование виртуального свидетельства осуществляется аналогично с помощью методов реализованного ранее класса ImpreciseScalarConjunctsLocalPropagator.
 \begin{lstlisting} [caption = Реализация метода PropagateImpreciseEvidence]
public IntervalConjKP_I PropagateImpreciseEvidence(IntervalConjKP_I firstKP, ScalarConjKP_I secondKP)
{
    var evidence = firstKP.GetSeparator(secondKP);
    var propagator = new ImpreciseScalarConjunctsLocalPropagator();
    propagator.setPattern(secondKP);
    propagator.propagate(evidence);
    return propagator.getResult();
}
\end{lstlisting}

 Второй класс VirtualEvidenceIntervalConjPropagator  отвечает за распространение виртуального  свидетельства из первого фрагмента знаний во второй ФЗ с интервальными оценками. Содержит в себе два аналогичных метода для пропагации свидетельства в двух различных ситуациях.
 
 Метод PropagateStochasticEvidence используется для распространения стохастического виртуального свидетельства в ФЗ с интервальными оценками. Метод принимает два параметра: ScalarConjKP\_I firstKP --- первый фрагмент знаний, IntervalConjKP\_I secondKP --- второй фрагмент знаний, в который нужно пропагировать виртуальное свидетельство.  Внутри метода создается виртуальное свидетельство и далее оно пропагируется с помощью методов объекта propagator ранее реализованного класса  StochasticIntervalConjunctsLocalPropagator.
\begin{lstlisting}[caption = Реализация метода PropagateStochasticEvidence]
public IntervalConjKP_I PropagateStochasticEvidence(ScalarConjKP_I firstKP, IntervalConjKP_I secondKP)
{
    var evidence = firstKP.GetSeparator(secondKP);
    var propagator = new StochasticIntervalConjunctsLocalPropagator();

    propagator.setPattern(secondKP);
    propagator.propagate(evidence);

return propagator.getResult();
}
\end{lstlisting}

Второй метод PropagateImpreciseEvidence нужен для пропагирования неточного виртуального свидетельства из первого ФЗ во второй. Пропагирование виртуального свидетельства осуществляется аналогично с помощью методов объекта propagator ранее реализованного класса ImpreciseIntervalConjunctsLocalPropagator.
\begin{lstlisting} [caption = Реализация метода PropagateImpreciseEvidence]
public IntervalConjKP_I PropagateImpreciseEvidence(IntervalConjKP_I firstKP, IntervalConjKP_I secondKP)
{
    var evidence = firstKP.GetSeparator(secondKP);
    var propagator = new ImpreciseIntervalConjunctsLocalPropagator();

    propagator.setPattern(secondKP);
    propagator.propagate(evidence);

    return propagator.getResult();
}\end{lstlisting}

Аналогичные классы реализованы для фрагментов знаний, построенных над идеалами дизъюнктов: VirtualEvidenceScalarDisjPropagator и VirtualEvidenceIntervalDisjPropagator, а также классы для ФЗ, построенных над множествами квантов: VirtualEvidenceScalarQuantPropagator и VirtualEvidenceIntervalQuantPropagator.  Каждый класс содержит методы PropagateStochasticEvidence и PropagateImpreciseEvidence для пропагации одного из видов виртуального свидетельства.

	   
\subsection{Примеры работы алгоритмов}
Приведем примеры использования методов реализаций интерфейса IVirtualEvidencePropagator. Рассмотрим примеры, когда фрагменты знаний содержат скалярные оценки, и в первый фрагмент знаний приходит стохастическое свидетельство. Распространим влияние свидетельства в первый ФЗ, сформируем виртуальное свидетельство и пропагируем его во второй ФЗ. Для остальных возможных ситуаций использование будет аналогичным.
\subsubsection{Пример пропагации стохастического виртуального свидетельства в ФЗ с скалярными оценками над идеалом конъюнктов}
Пусть даны два фрагмента знаний над алфавитами $\mathcal{A}_1 = \{x_1, x_2\}$ и  $\mathcal{A}_2 = \{x_2, x_3\}$ и в первый фрагмент знаний поступило стохастическое свидетельство над алфавитом $\mathcal{A}_{\ev} = \{x_1\}$. Пусть ФЗ и свидетельство имеют следующие оценки вероятностей: \\ 
\begin{equation*}
\Pc^1 =  \begin{pmatrix}
1 \\ p(x_1) \\ p(x_2) \\ p(x_2x_1)
\end{pmatrix} = \begin{pmatrix}
1 \\ 0.5\\ 0.7\\ 0.3
\end{pmatrix}, 
\Pc^2 =  \begin{pmatrix}
1 \\ p(x_2) \\ p(x_3) \\ p(x_3x_2)
\end{pmatrix} = \begin{pmatrix}
1 \\ 0.3\\ 0.7\\ 0.1 
\end{pmatrix},
\end{equation*}
\begin{equation*}
\Pc^{\ev} = \begin{pmatrix}  1 \\ p(x_1)	\end{pmatrix} = \begin{pmatrix}
1 \\ 0.4
\end{pmatrix}.
\end{equation*}

После пропагации свидетельства оценки вероятностей элементов первого фрагмента знаний, вычисленные с помощью уравнения~\ref{conjG}, будут следующими:
\begin{equation*}
\Pc^{1,a} = \begin{pmatrix}
1 \\  0.4 \\ 0.72 \\ 0.24
\end{pmatrix}.
\end{equation*}

Виртуальное свидетельство будет построено над алфавитом $\mathcal{A}_{\ev} = \mathcal{A}_1 \cap \mathcal{A}_2 = \{x_1, x_2\} \cap \{x_2, x_3\} = \{x_2\}$ и будет содержать следующий вектор оценок вероятностей элементов:
\begin{equation*}
\Pc^{\ev} = \begin{pmatrix}  1 \\ p(x_2)	\end{pmatrix} = 
\begin{pmatrix} 1 \\ 0.72 \end{pmatrix}.
\end{equation*}

После пропагации виртуального свидетельства во второй фрагмент знаний с помощью уравнения~\ref{conjglob2}, оценки второго ФЗ получаются следующими:
\begin{equation*}
\Pc^{2,a} = \begin{pmatrix}
1 \\ 0.72 \\ 0.48 \\ 0.24
\end{pmatrix}.
\end{equation*}

Данный результат пропагации можно получить с помощью следующего фрагмента кода:

\begin{lstlisting}[caption = Пример пропагации виртуального свидетельства]
// Creating KPs and evidence
var firstKP = new ScalarConjKP(Convert.ToInt64("011", 2), new[] { 1, 0.5, 0.7, 0.3 });
var secondKP = new ScalarConjKP(Convert.ToInt64("110", 2), new[] { 1, 0.3, 0.7, 0.1 });
var evidence = new ScalarConjKP(Convert.ToInt64("001", 2), new[] { 1, 0.4 });
// Creating propagators for stochastic and virtual evidences
var localPropagator = new StochasticScalarConjunctsLocalPropagator();
var virtualEvPropagator = new VirtualEvidenceScalarConjPropagator();

// Propagating stochasic evidence in fisrt KP
localPropagator.setPattern(firstKP);
localPropagator.propagate(evidence);
var firstKPwithAposterioriEst = localPropagator.getResult();

// Propagating virtual evidence in second KP
var secondKPwithAposterioriEst = virtualEvPropagator
    .PropagateStochasticEvidence(firstKPwithAposterioriEst, secondKP);
\end{lstlisting}

Вывод на консоль можно осуществить с помощью следующего кода:
\begin{lstlisting}[caption = Вывод результатов на консоль]
Console.WriteLine("Probability estimates in first KP after propagation:");
firstKPwithAposterioriEst
    .GetPointEstimate()
    .ToList()
    .ForEach(est => Console.Write(" " + est));
 
Console.WriteLine("Probability estimates in second KP after propagation:");
secondKPwithAposterioriEst
    .GetPointEstimate()
    .ToList()
    .ForEach(est => Console.Write(" " + est));
\end{lstlisting}

Результаты вывода на консоль:
\begin{lstlisting}[caption = Результаты]
Probability estimates in first KP after propagation:
 1, 0,4 0,72 0,24
Probability estimates in second KP after propagation:
1, 0,72, 0,48 0,24
\end{lstlisting}
\subsubsection{Пример пропагации стохастического виртуального свидетельства в ФЗ с скалярными оценками над идеалом дизъюнктов}
Рассмотрим фрагменты знаний над алфавитами $\mathcal{A}_1 = \{x_1, x_2\}$ и  $\mathcal{A}_2 = \{x_2, x_3\}$, построенные над идеалами дизъюнктов. Программная реализация принимает на вход и возвращает в качестве результата ФЗ, содержащие вектор вероятностей $\Pd$, в то же время как приведенные в работе формулы работают с векторами $\Pd^{\prime}$, поэтому приведем в данном примере векторы вероятностей как и $\Pd$, так и $\Pd^{\prime}$. Считаем, что в первый фрагмент знаний поступило стохастическое свидетельство над алфавитом $\mathcal{A}_{\ev} = \{x_1\}$. Пусть ФЗ и свидетельство имеют следующие оценки вероятностей: \\ 
\begin{equation*}
\Pd^{\prime 1} =  \begin{pmatrix}
1 \\ p(\overline{x}_1) \\ p(\overline{x}_2) \\ p(\overline{x}_2\overline{x}_1)
\end{pmatrix} = \begin{pmatrix}
1 \\ 0.5\\ 0.3\\ 0.1
\end{pmatrix}, 
\Pd^1 = \mathbf{1} - \Pd^{\prime 1} =  \begin{pmatrix}
0 \\ 0.5\\ 0.7\\ 0.9
\end{pmatrix}, 
\end{equation*}
\begin{equation*}
\Pd^{\prime 2} =  \begin{pmatrix}
1 \\ p(\overline{x}_3) \\ p(\overline{x}_3) \\ p(\overline{x}_3\overline{x}_2)
\end{pmatrix} = \begin{pmatrix}
1 \\ 0.7\\ 0.3\\ 0.1 
\end{pmatrix},
\Pd^2 = \mathbf{1} - \Pd^{\prime 2} =  \begin{pmatrix}
0 \\ 0.3\\ 0.7\\ 0.9 \end{pmatrix},
\end{equation*}
\begin{equation*}
\Pd^{\prime \ev} = \begin{pmatrix}  1 \\ p(\overline{x}_1)	\end{pmatrix} = \begin{pmatrix}
1 \\ 0.6
\end{pmatrix},
\Pd^{\ev} = \mathbf{1} - \Pd^{\prime \ev} =  \begin{pmatrix}
0 \\ 0.4 \end{pmatrix}.
\end{equation*}

После пропагации свидетельства оценки вероятностей элементов первого фрагмента знаний, вычисленные с помощью уравнения~\ref{dis3}, будут следующими:
\begin{equation*}
\Pd^{\prime 1,a} = \begin{pmatrix}
1 \\  0.6 \\ 0.28 \\ 0.12
\end{pmatrix},
\Pd^{1,a}= \mathbf{1} - \Pd^{\prime 1,a} = \begin{pmatrix}
0 \\  0.4 \\ 0.72 \\ 0.88
\end{pmatrix}.
\end{equation*}

Виртуальное свидетельство строится над алфавитом $\mathcal{A}_{\ev} = \mathcal{A}_1 \cap \mathcal{A}_2 = \{x_1, x_2\} \cap \{x_2, x_3\} = \{x_2\}$ и содержит следующий вектор оценок вероятностей элементов:
\begin{equation*}
\Pd^{\prime \ev} = \begin{pmatrix}  1 \\ p(\overline{x}_2)	\end{pmatrix} = 
\begin{pmatrix} 1 \\ 0.28 \end{pmatrix},
\Pd^{\ev} = \mathbf{1} - \Pd^{\prime \ev} = \begin{pmatrix} 0 \\ 0.72 \end{pmatrix}.
\end{equation*}

C помощью уравнения~\ref{disGlob} распространим виртуальное свидетельство во второй фрагмент знаний и получим апостериорные оценки вероятностей:
\begin{equation*}
\Pd^{\prime 2,a} = \begin{pmatrix}
1 \\ 0.28 \\ 0.52 \\ 0.04
\end{pmatrix},
 \Pd^{2,a}  = \mathbf{1} - \Pd^{\prime 2,a} = \begin{pmatrix}
0 \\ 0.72 \\ 0.48 \\ 0.96
\end{pmatrix}.
\end{equation*}

Данный результат пропагации можно получить с помощью следующего фрагмента кода:
\begin{lstlisting}[caption = Пример пропагации виртуального свидетельства]
// Creating KPs and evidence
var firstKP = new ScalarDisjKP(Convert.ToInt64("011", 2), new[] { 0, 0.5, 0.7, 0.9 });
var secondKP = new ScalarDisjKP(Convert.ToInt64("110", 2), new[] { 0, 0.3, 0.7, 0.9 });
var evidence = new ScalarDisjKP(Convert.ToInt64("001", 2), new[] { 0, 0.4 });
// Creating propagators for stochastic and virtual evidences
var localPropagator = new StochasticScalarDisjLocalPropagator(firstKP);
var virtualEvPropagator = new VirtualEvidenceScalarDisjPropagator();
// Propagating stochastic evidence in first KP
var firstKPwithAposterioriEst = localPropagator.PropagateEvidence(evidence); 
// Propagating virtual evidence in second KP
var secondKPwithAposterioriEst = virtualEvPropagator
    .PropagateStochasticEvidence(firstKPwithAposterioriEst, secondKP);
\end{lstlisting}

Вывод на консоль можно осуществить с помощью следующего кода:
\begin{lstlisting}[caption = Вывод результатов на консоль]
Console.WriteLine("Probability estimates in first KP after propagation:");
firstKPwithAposterioriEst
    .GetPointEstimate()
    .ToList()
    .ForEach(est => Console.Write(" " + est));

Console.WriteLine("Probability estimates in second KP after propagation:");
secondKPwithAposterioriEst
    .GetPointEstimate()
    .ToList()
    .ForEach(est => Console.Write(" " + est));
\end{lstlisting}

Результаты вывода на консоль:
\begin{lstlisting}[caption = Результаты]
Probability estimates in first KP after propagation:
 0 0,4 0,72 0,88
Probability estimates in second KP after propagation:
 0 0,72 0,48 0,96
\end{lstlisting}
\subsubsection{Пример пропагации стохастического виртуального свидетельства в ФЗ со скалярными оценками над множеством квантов}
Возьмем два фрагмента знаний над алфавитами $\mathcal{A}_1 = \{x_1, x_2\}$ и  $\mathcal{A}_2 = \{x_2, x_3\}$. Пусть в первый фрагмент знаний поступило стохастическое свидетельство над алфавитом $\mathcal{A}_{\ev} = \{x_1\}$. Будем считать, что ФЗ и свидетельство построены над множеством квантов и имеют следующие оценки вероятностей: 
\begin{equation*}
\Pq^1 =  \begin{pmatrix}
p(\overline{x}_2\overline{x}_1)\\ p(\overline{x}_2x_1) \\ p(x_2\overline{x}_1) \\ p(x_2x_1)
\end{pmatrix} = \begin{pmatrix}
0.1\\ 0.2\\ 0.4\\ 0.3
\end{pmatrix},
\Pq^2 =  \begin{pmatrix}
p(\overline{x}_3\overline{x}_2)\\ p(\overline{x}_3x_2)\\ p(x_3\overline{x}_2) \\ p(x_3x_2)
\end{pmatrix} = \begin{pmatrix}
0.1 \\ 0.2\\ 0.6\\ 0.1 
\end{pmatrix},
\end{equation*}
\begin{equation*}
\Pq^{\ev} = \begin{pmatrix}   p(\overline{x}_1) \\ p(x_1)	\end{pmatrix} = \begin{pmatrix}
0.6 \\ 0.4
\end{pmatrix}.
\end{equation*}

После пропагации свидетельства оценки вероятностей элементов первого фрагмента знаний, вычисленные с помощью уравнения~\ref{quantsG}, будут следующими:
\begin{equation*}
\Pq^{1,a} = \begin{pmatrix}
0.12 \\  0.16 \\ 0.48\\ 0.24
\end{pmatrix}.
\end{equation*}

Виртуальное свидетельство будет построено над алфавитом $\mathcal{A}_{\ev} = \mathcal{A}_1 \cap \mathcal{A}_2 = \{x_1, x_2\} \cap \{x_2, x_3\} = \{x_2\}$ и будет содержать следующий вектор оценок вероятностей квантов:
\begin{equation*}
\Pq^{\ev} = \begin{pmatrix}  p(\overline{x}_2) \\ p(x_2)	\end{pmatrix} = 
\begin{pmatrix} 0.28 \\ 0.72 \end{pmatrix}.
\end{equation*}

Пропагируем виртуальное свидетельство во второй фрагмент знаний с помощью уравнения~\ref{quantsGlob} и получим следующие оценки вероятностей:
\begin{equation*}
\Pq^{2,a} = \begin{pmatrix}
0.04 \\ 0.48 \\ 0.24 \\ 0.24
\end{pmatrix}.
\end{equation*}

Данный результат пропагации можно получить с помощью следующего фрагмента кода:

\begin{lstlisting}[caption = Пример пропагации виртуального свидетельства]
// Creating KPs and evidence
var firstKP = new ScalarQuantKP(Convert.ToInt64("011", 2), new[] { 0.1, 0.2, 0.4, 0.3 });
var secondKP = new ScalarQuantKP(Convert.ToInt64("110", 2), new[] { 0.1, 0.2, 0.6, 0.1 });
var evidence = new ScalarQuantKP(Convert.ToInt64("001", 2), new[] { 0.6, 0.4 });

// Creating propagators for stochastic and virtual evidences
var localPropagator = StochasticScalarQuantLocalPropagator.Instance(firstKP);
var virtualEvPropagator = new VirtualEvidenceScalarQuantPropagator();

// Propagating stochasic evidence in fisrt KP
var firstKPwithAposterioriEst = localPropagator.PropagateEvidence(evidence);
            
// Propagating virtual evidence in second KP
var secondKPwithAposterioriEst = virtualEvPropagator
    .PropagateStochasticEvidence(firstKPwithAposterioriEst, secondKP);
\end{lstlisting}

Вывод на консоль можно осуществить с помощью следующего кода:
\begin{lstlisting}[caption = Вывод результатов на консоль]
Console.WriteLine("Probability estimates in first KP after propagation:");
firstKPwithAposterioriEst
    .GetPointEstimate()
    .ToList()
    .ForEach(est => Console.Write(" " + est));
 
Console.WriteLine("Probability estimates in second KP after propagation:");
secondKPwithAposterioriEst
    .GetPointEstimate()
    .ToList()
    .ForEach(est => Console.Write(" " + est));
\end{lstlisting}

Результаты вывода на консоль:
\begin{lstlisting}[caption = Результаты]
Probability estimates in first KP after propagation:
 0,12 0,16 0,48 0,24
Probability estimates in second KP after propagation:
 0,04 0,48 0,24 0,24
\end{lstlisting}

\subsection{Выводы по главе}
Программная реализация, представленная в данной главе, реализует алгоритмы пропагирования виртуального свидетельства для различных моделей ФЗ, содержащих как скалярные, так и интервальные оценки. В главе описаны основные классы и методы и приведены примеры их использования.

\section*{Заключение}
\underline{Итогами}  квалификационной работы являются программная реализация алгоритма распространения виртуального свидетельства между двумя фрагментами знаний для различных моделей ФЗ в рамках существующего комплекса программ, а также:
\begin{enumerate}
    \item  Матрично"=векторная интерпретация алгоритма пропагации виртуального свидетельства между двумя фрагментами знаний: сформулированы матрично"=векторные уравнения для ФЗ, построенных над множествами квантов и над идеалами дизъюнктов, в частности, доказана теорема формирования матрицы перехода от вектора оценок вероятностей элементов к вектору виртуального свидетельства для модели идеала дизъюнктов и теорема о пропагации виртуального свидетельства между двумя ФЗ, построенными над множествами квантов, предложена матрица перехода от вектора вероятностей дизъюнктов к вектору вероятностей квантов; 
    \item Матрично"=векторная интерпретация функции $\Gind$, в частности, введено понятие характеристического вектора конъюнкта, дизъюнкта, кванта, а также понятие матрицы проекции $\G$;
    \item Внедрение алгоритмов локального ЛВВ для квантов и реинжиниринг существующих алгоритмов ЛВВ для идеалов дизъюнктов;
    \item Тесты, проверяющие корректность работы реализации;
    \item Примеры использования полученной реализации; 
    \item Вычислительные эксперименты, результаты которых согласуются с ожиданиями.
\end{enumerate}

\underline{Рекомендации к использованию.} Предложенная программная реализация является модулем объемлющей математической библиотеки, реализующей алгоритмы логико-вероятностного вывода в АБС и предоставляющей доступ к публичному контракту~\cite{50}. Поэтому реализованный модуль может быть переиспользован также в веб-приложении с целью визуализации алгоритма распространения виртуального свидетельства между двумя фрагментами знаний сети. Также программная реализация может быть переиспользована при дальнейшей реализации алгоритмов глобального логико"=вероятностного вывода, а именно распространения влияния свидетельства во все фрагменты знаний сети. Классы и методы могут быть использованы при проведении различных вычислительных экспериментов с целью изучения различных характеристик модели.

\underline{Перспективы дальнейших исследований.} Полученные теоретические результаты могут быть использованы для исследования устойчивости и чувствительности полученных уравнений и создают фундамент для развития теории глобального логико"=вероятностного вывода, в частности, для формализации алгоритмов распространения виртуального свидетельства между двумя фрагментами знаний с интервальными оценками и исследования распространения виртуального свидетельства для третичной глобальной структуры АБС~\cite{284}.

Разработанные примеры могут быть использованы в методических целях.


\setmonofont[Mapping=tex-text]{CMU Typewriter Text}
\bibliography{diploma}
\documentclass[14pt]{matmex-diploma-custom}

\usepackage{tocvsec2}
\usepackage{amssymb}
\usepackage{listings}
\usepackage{xcolor}
\usepackage[font=small,labelsep=period]{caption}
%%% \usepackage{concrete}
% Calligraphy letters
% use: \mathcal
\usepackage[mathscr]{eucal}
% Algorithms
% use: \begin{algorithm}
\usepackage{algorithm}
% use: \begin{algorithmic}
\usepackage[noend]{algpseudocode}
% Maths
% use: for correct control sequence
\usepackage{amsmath}
% Centreing
% use: \centering
\usepackage{varwidth}
% First indent
% use: automatically
\usepackage{indentfirst}

\definecolor{bluekeywords}{rgb}{0,0,1}
\definecolor{greencomments}{rgb}{0,0.5,0}
\definecolor{redstrings}{rgb}{0.64,0.08,0.08}
\definecolor{xmlcomments}{rgb}{0.5,0.5,0.5}
\definecolor{types}{rgb}{0.17,0.57,0.68}

\lstset{
    inputencoding=utf8x, 
    extendedchars=false, 
    keepspaces = true,
    language=[Sharp]C,
    captionpos=b,
   frame=lines, % Oberhalb und unterhalb des Listings ist eine Linie
    showspaces=false,
    showtabs=false,
    breaklines=true,
    showstringspaces=false,
    breakatwhitespace=true,
    basicstyle=\linespread{0.8}
    escapeinside={(*@}{@*)},
    commentstyle=\color{greencomments},
    morekeywords={partial, var, value, get, set},
    keywordstyle=\color{bluekeywords},
    stringstyle=\color{redstrings},
    basicstyle=\small\ttfamily,
}


\renewcommand{\lstlistingname}{Листинг}

\newtheorem{Th}{Теорема}[section]
\newtheorem{Def}{Определение}[section]
\newtheorem{Lem}{Утверждение}[subsection]

\newcommand{\underdot}[1]{\mathop{#1}\limits_{\cdot}}

\newenvironment{Proof} % имя окружения
{\par\noindent{\bf Доказательство.}} % команды для \begin
{\hfill$\scriptstyle\blacksquare$} % команды для \end



\newcommand{\Pc}{\mathbf{P}_\mathrm{c}}
\newcommand{\Pq}{\mathbf{P}_\mathrm{q}}
\newcommand{\In}{\mathbf{I}_n}
\newcommand{\Jn}{\mathbf{J}_n}
\newcommand{\Qmatr}{\mathbf{Q}}
\newcommand{\V}{\mathbf{V}}
\renewcommand{\G}{\mathbf{G}}
\renewcommand{\T}{\mathbf{T}}
\newcommand{\evidenceNumbers}{\langle i, j \rangle}
\newcommand{\Tij}{\T^{\evidenceNumbers}}
\newcommand{\TijTilda}{\widetilde{\T}^{\evidenceNumbers}}
\newcommand{\QijTild}{\widetilde{\Qmatr}^{\evidenceNumbers}}
\newcommand{\VijTilda}{\widetilde{\V}^{\evidenceNumbers}}
\newcommand{\rij}{\mathbf{r}^{\evidenceNumbers}}
\newcommand{\rijTilda}{\widetilde{\mathbf{r}}^{\evidenceNumbers}}
\newcommand{\ev}{\mathrm{ev}}
\newcommand{\Gind}{\mathrm{GInd}}
\newcommand{\Selector}{\mathbf{s}}
\newcommand{\sij}{\mathbf{s}^{\evidenceNumbers}}
\newcommand{\sijTilda}{\widetilde{\mathbf{s}}^{\evidenceNumbers}}
\newcommand{\Pd}{\mathbf{P}_\mathrm{d}}
\newcommand{\Ln}{\mathbf{L}_n}
\newcommand{\On}{\mathbf{F}_n}
\newcommand{\Kn}{\mathbf{K}_n}
\renewcommand{\M}{\mathbf{M}}
\newcommand{\Mij}{\M^{\evidenceNumbers}}
\newcommand{\dij}{\mathbf{d}^{\evidenceNumbers}}
\newcommand{\dijTilda}{\widetilde{\mathbf{d}}^{\evidenceNumbers}}
\newcommand{\MijTilda}{\widetilde{\mathbf{M}}^{\evidenceNumbers}}
\newtheorem{theorem}{Теорема}[section]
\newtheorem{definition}[theorem]{Определение}
 \setcounter{tocdepth}{2}
 \usepackage{lastpage}
 \usepackage{mathtools}
\begin{document}
% Год, город, название университета и факультета предопределены,
% но можно и поменять.
% Если англоязычная титульная страница не нужна, то ее можно просто удалить.
\filltitle{ru}{
    chair              = {Фундаментальная информатика и информационные технологии\\ Информационные технологии},
    title              = {Алгебраические байесовские сети:\\ синтез глобальных структур и алгоритмы логико-вероятностного вывода (проектная работа)},
    type               = {bachelor},
    position           = {студента},
    group              = 14.Б08-мм,
    author             = {Анна Викторовна Шляк},
    supervisorPosition = {проф. каф. инф., д. ф.-м. н., доц.},
    supervisor         = {Тулупьев А.\,Л.},
    reviewerPosition   = {проф. каф. инф., д. ф.-м. н., доц. },
    reviewer           = {Фильченков А. А.},
    chairHeadPosition  = {?},
    chairHead          = {?},
    faculty            = {Математико-механический факультет},
    city               = {Санкт-Петербург},
    year               = {2018}
}
\filltitle{en}{
    chair              = {Fundamental informatics and information technologies \\ Information technologies},
    title              = {Algebraic Bayesian networks: \\global structure synthesis and probabilistic-logic inference algorithms(project work)},
        type               = {bachelor},
    author             = {Anna Shliak},
    supervisorPosition = {Prof. Computer Science Department, Dc. Sc. in Math, Assoc. Prof.},
    supervisor         = {Alexander Tulupyev},
    reviewerPosition   = {As. Prof. Computer Technology Department, Ph. D. Sc. in Math},
    reviewer           = {Andrey Filchenkov},
    chairHeadPosition  = {?},
    chairHead          = {?},
}
\maketitle
\tableofcontents
\section*{Введение}
\input{introduction.tex}

\section{Автоматизация алгоритмов вывода в алгебраической байесовской сети}
\subsection{Введение}
\input{cap1/intro1.tex}
\subsection{Инструменты для работы с алгебраическими байесовскими сетями}
\input{cap1/scope.tex}
\subsection{Библиотеки для работы с алгебраическими байесовскими сетями}
\input{cap1/realizations.tex}
\subsection{Цели и задачи исследования}
\input{cap1/modernization.tex}
\subsection{Выводы по главе}
\input{cap1/concl1.tex}

\section{Элементы теории алгебраических байесовских сетей}\
\subsection{Введение}
\input{cap2/intro2.tex}
\subsection{Математические модели фрагмента знаний и алгебраической байесовской сети}
\input{cap2/ABN.tex}
\subsection{Локальный логико"=вероятностный вывод}
\input{cap2/local.tex}
\subsection{Глобальный апостериорный логико"=вероятностный вывод}
\input{cap2/conjGlobal.tex}
\subsection{Выводы по главе}
\input{cap2/concl2.tex}


\section{Матрично"=векторная  формализация глобального логико"=вероятностного вывода}\label{cap3}
\subsection{Введение}
\input{cap3/intro3.tex}
\subsection{Матрично"=векторная интерпретация для функции $\Gind$}
\input{cap3/GIndFormalization.tex}
\subsection{Формализация формирования матрицы перехода}
\input{cap3/Qformalization.tex}
\subsection{Уравнения глобального логико"=вероятностного вывода}
\input{cap3/equations.tex}
\subsection{Выводы по главе}
\input{cap3/concl3.tex}


\section{Программная реализация}
\subsection{Введение}
\input{cap4/intro4.tex}
\subsection{Архитектура программного комплекса}
\input{cap4/architecture.tex}
\subsection{Описание классов и методов}
\input{cap4/Program.tex}
\subsection{Примеры работы алгоритмов}
\input{cap4/code.tex}
\subsection{Выводы по главе}
\input{cap4/concl4.tex}

\section*{Заключение}
\input{conclusion.tex}

\setmonofont[Mapping=tex-text]{CMU Typewriter Text}
\bibliography{diploma}
\input{diploma.bib}

\section*{Приложение А. Вычислительные эксперименты по распространению виртуального свидетельства}
\input{examples/exampleConj.tex}
\input{examples/exampleDisj.tex}
\input{examples/exampleQuants.tex}
\section*{Приложение B. Публикации по теме работы}
\input{publ.tex}
\end{document}
   


\section*{Приложение А. Вычислительные эксперименты по распространению виртуального свидетельства}
\subsection*{Пример использования уравнения для пропагации виртуального свидетельства над идеалом конъюнктов}
 \addcontentsline{toc}{subsection}{Пример использования уравнения для пропагации виртуального свидетельства над идеалами конъюнктов}
Пусть даны два фрагмента знаний над алфавитами $\mathcal{A}_1 = \{x_1, x_2\}$ и  $\mathcal{A}_2 = \{x_1, x_3\}$, и пусть в первый фрагмент знаний поступило стохастическое свидетельство. Пересчитаем оценки в данных ФЗ с учетом поступившего свидетельства, а затем распространим его влияние во второй ФЗ:
\begin{equation*}
\Pc^1 =  \begin{pmatrix}
1 \\ p(x_1) \\ p(x_2) \\ p(x_2x_1)
\end{pmatrix} = \begin{pmatrix}
1 \\ 0.8\\ 0.7\\ 0.6
\end{pmatrix}, 
\Pc^2 =  \begin{pmatrix}
1 \\ p(x_1) \\ p(x_3) \\ p(x_3x_1)
\end{pmatrix} = \begin{pmatrix}
1 \\ 0.4\\ 0.5\\ 0.1 
\end{pmatrix}.
\end{equation*}

Пусть поступившее свидетельство будет над алфавитом
$\mathcal{A}_{\ev} = \{x_2\}$:
\begin{equation*}
\Pc^{\ev} = \begin{pmatrix}  1 \\ p(x_2)	\end{pmatrix} = \begin{pmatrix}
1 \\ 0.4
\end{pmatrix}.
\end{equation*}

Воспользуемся уравнением \ref{conjG}. Посчитаем матрицу $\G$ по формуле \ref{G}:
$\G = \begin{pmatrix} 1 \\ 0 \end{pmatrix}.$

$\chi^0 = \begin{pmatrix} 0 \end{pmatrix}, 
\G \chi^0 =  \begin{pmatrix} 1 \\ 0  \end{pmatrix} \begin{pmatrix} 0 \end{pmatrix} = \begin{pmatrix}  0 \\ 0 \end{pmatrix}$,
$\chi^1 = \begin{pmatrix} 1 \end{pmatrix} $, 
$ \G \chi^1 =  \begin{pmatrix} 1 \\ 0 \end{pmatrix} \begin{pmatrix} 1 \end{pmatrix} = \begin{pmatrix}  1 \\ 0 \end{pmatrix}$,

\begin{equation*}
\T^{\langle 00, 10 \rangle} = \T^- \otimes \T^\circ = 
\begin{pmatrix*}[r] 1 & -1 \\ 0 & 0 \end{pmatrix*} \otimes
\begin{pmatrix} 1 & 0 \\ 0 & 1\end{pmatrix} = 
\begin{pmatrix*}[r] 1 & 0 & -1 & 0 \\ 0 & 1 & 0 & -1 \\ 0 & 0 & 0 & 0 \\ 0 & 0 & 0 & 0 \end{pmatrix*},
\end{equation*}
\begin{equation*}
\mathbf{r}^{\langle 00, 10 \rangle} = \mathbf{r}^- \otimes \mathbf{r}^\circ = \begin{pmatrix*}[r] 1 \\ -1 \end{pmatrix*} \otimes \begin{pmatrix}  1 \\ 0 \end{pmatrix} = \begin{pmatrix*}[r]  1 \\ 0 \\ -1 \\ 0 \end{pmatrix*},
\end{equation*}
\begin{equation*}
\T^{\langle 10, 00 \rangle} = \T^+ \otimes \T^\circ = 
\begin{pmatrix} 0 & 1 \\ 0 & 1 \end{pmatrix} \otimes
\begin{pmatrix} 1 &0 \\ 0 & 1\end{pmatrix}  =
\begin{pmatrix} 0 &0 & 1 & 0 \\ 0 & 0 & 0 & 1 \\
0 & 0 & 1 & 0\\ 0 & 0 & 0 & 1 \end{pmatrix},
\end{equation*}
\begin{equation*}
\mathbf{r}^{\langle 10, 00 \rangle} = \mathbf{r}^+ \otimes \mathbf{r}^\circ = \begin{pmatrix} 0 \\ 1 \end{pmatrix} \otimes \begin{pmatrix}  1 \\ 0 \end{pmatrix} = \begin{pmatrix}  0 \\ 0 \\ 1 \\ 0 \end{pmatrix},
\end{equation*}
\begin{equation*}
\mathbf{I}_1\Pc^{\ev} = \begin{pmatrix*}[r] 1 & -1 \\ 0 & 1 \end{pmatrix*} \begin{pmatrix}1 \\ 0.4\end{pmatrix} = \begin{pmatrix}
0.6 \\ 0.4 \end{pmatrix}.
\end{equation*}

Подставим в формулу \ref{conjG}, посчитаем и получим, что $\Pc^{1,a} = \begin{pmatrix}
1 \\  \frac{26}{35} \\ 0.4 \\ \frac{12}{35}
\end{pmatrix}$.

Теперь найдем вектор виртуального свидетельства. Виртуальное свидетельство будет построено над алфавитом 
$\mathcal{A}_{\ev} = \mathcal{A}_1 \cap \mathcal{A}_2 = \{x_1, x_2\} \cap \{x_1, x_3\} = \{x_1\}$.
\begin{equation*}
\Qmatr = \Qmatr^- \otimes \Qmatr^+ =  \begin{pmatrix} 1 & 0  \end{pmatrix} \otimes
  \begin{pmatrix} 1 & 0 \\ 0 & 1 \end{pmatrix} = \begin{pmatrix}
1 & 0  & 0 & 0 \\ 0 & 1 & 0 & 0
\end{pmatrix},
\end{equation*}
\begin{equation*}
\Pc^{\ev} = \Qmatr\Pc^{1, a} =  \begin{pmatrix}
1 & 0  & 0 & 0 \\ 0 & 1  & 0 & 0
\end{pmatrix} \begin{pmatrix}
1 \\  \frac{26}{35} \\ 0.4 \\ \frac{12}{35}
\end{pmatrix} = \begin{pmatrix}
1 \\ \frac{26}{35}
\end{pmatrix}.
\end{equation*} 

Воспользуемся уравнением \ref{conjglob2} для пропагации виртуального свидетельства:
$\G = \begin{pmatrix} 0 \\ 1\end{pmatrix}.$

$\chi^0 = \begin{pmatrix} 0 \end{pmatrix} $, 
$ \G \chi^0 = \begin{pmatrix} 0 \\ 1 \end{pmatrix} \begin{pmatrix} 0 \end{pmatrix} = \begin{pmatrix}  0 \\ 0 \end{pmatrix}$,
$\chi^1 = \begin{pmatrix} 1 \end{pmatrix} $, 
$ \G \chi^1 =  \begin{pmatrix} 0 \\ 1 \end{pmatrix}\begin{pmatrix} 1 \end{pmatrix} = \begin{pmatrix}  0 \\ 1 \end{pmatrix}$,
\begin{equation*}
\T^{\langle 00, 01 \rangle} = \T^\circ \otimes \T^- = 
\begin{pmatrix} 1 & 0 \\ 0 & 1\end{pmatrix} \otimes
\begin{pmatrix*}[r] 1 & -1 \\ 0 & 0 \end{pmatrix*} = 
\begin{pmatrix*}[r] 1 & -1 & 0  & 0 \\ 0 & 0 & 0 & 0 \\ 0 & 0 & 1 & -1 \\ 0 & 0 & 0 & 0 \end{pmatrix*},
\end{equation*}
\begin{equation*}
\mathbf{r}^{\langle 00, 01 \rangle} = \mathbf{r}^\circ \otimes \mathbf{r}^- = \begin{pmatrix} 1 \\ 0 \end{pmatrix} \otimes \begin{pmatrix*}[r]  1 \\ -1 \end{pmatrix*} = \begin{pmatrix*}[r] 1 \\ -1 \\ 0 \\ 0 \end{pmatrix*},
\end{equation*}
\begin{equation*}
\T^{\langle 01, 00 \rangle} = \T^\circ \otimes \T^+ = 
\begin{pmatrix} 1 &0 \\ 0 & 1\end{pmatrix}  \otimes
\begin{pmatrix} 0 & 1 \\ 0 & 1 \end{pmatrix} =
\begin{pmatrix} 0 &1 & 0 & 0 \\ 0 & 1 & 0 & 0 \\
0 & 0 & 0 & 1\\ 0 & 0 & 0 & 1 \end{pmatrix},
\end{equation*}
\begin{equation*}
\mathbf{r}^{\langle 01, 00 \rangle} = \mathbf{r}^\circ \otimes \mathbf{r}^+ = \begin{pmatrix} 1 \\ 0 \end{pmatrix} \otimes \begin{pmatrix}  0 \\ 1 \end{pmatrix} = \begin{pmatrix}  0 \\ 1 \\ 0 \\ 0 \end{pmatrix},
\end{equation*}
\begin{equation*}
\mathbf{I}_1\Pc^{\ev} = \begin{pmatrix*}[r] 1 & -1 \\ 0 & 1 \end{pmatrix*} \begin{pmatrix}1 \\  \frac{26}{35} \end{pmatrix} = \begin{pmatrix}
\frac{9}{35} \\ \frac{26}{35} \end{pmatrix}.
\end{equation*}

Подставим в формулу выше, посчитаем и получим, что $\Pc^{2,a} = \begin{pmatrix}
1 \\ \frac{26}{35} \\ \frac{5}{14} \\ \frac{13}{70}
\end{pmatrix}$.

\subsection*{Пример использования уравнения для пропагации виртуального свидетельства над идеалом дизъюнктов}
 \addcontentsline{toc}{subsection}{Пример использования уравнения для пропагации виртуального свидетельства над идеалами дизъюнктов}
Пусть дан фрагмент знаний над алфавитом $\mathcal{A}_1 = \{x_1, x_2\}$, и пусть в него поступило стохастическое свидетельство. Пересчитаем оценки в данном ФЗ с учетом поступившего свидетельства. Далее распространим влияние этого свидетельства из данного ФЗ в другой ФЗ над алфавитом  $\mathcal{A}_2 = \{x_1, x_3\}$:
\begin{equation*}
\Pd^{\prime 1} =  \begin{pmatrix}
1 \\ p(\overline {x}_1) \\ p(\overline {x}_2) \\ p(\overline {x}_2\overline {x}_1)
\end{pmatrix} = \begin{pmatrix}
1 \\ 0.2\\ 0.3\\ 0.1
\end{pmatrix},
\Pd^{\prime 2} =  \begin{pmatrix}
1 \\ p(\overline{x}_1) \\ p(\overline{x}_3) \\ p(\overline{x}_3\overline{x}_1)
\end{pmatrix} = \begin{pmatrix}
1 \\ 0.6\\ 0.5\\ 0.2 
\end{pmatrix}.
\end{equation*}

Пусть поступившее свидетельство будет над алфавитом
$\mathcal{A}_{\ev} = \{x_2\}$ со следующим вектором оценок:
\begin{equation*}
\Pd^{\prime \ev} = \begin{pmatrix}  1 \\ p(\overline{x}_2)	\end{pmatrix} = \begin{pmatrix}
1 \\ 0.6
\end{pmatrix}.
\end{equation*}

Воспользуемся уравнением \ref{dis3} и найдем апостериорные оценки элементов первого ФЗ:
$ \G = \begin{pmatrix} 1 \\ 0 \end{pmatrix}.$

$\chi^0 = \begin{pmatrix} 0 \end{pmatrix} $, 
$\G \chi^0 = \begin{pmatrix} 1 \\ 0  \end{pmatrix}\begin{pmatrix} 0 \end{pmatrix} = \begin{pmatrix}  0 \\ 0 \end{pmatrix}$,
$\chi^1 = \begin{pmatrix} 1 \end{pmatrix} $, 
$\G \chi^1  = \begin{pmatrix} 1 \\ 0 \end{pmatrix} \begin{pmatrix} 1 \end{pmatrix}  = \begin{pmatrix}  1 \\ 0 \end{pmatrix}$,
\begin{equation*}
\M^{\langle 00, 10 \rangle} = \M^- \otimes \M^\circ = 
\begin{pmatrix} 0 & 1 \\ 0 & 1 \end{pmatrix} \otimes
\begin{pmatrix} 1 &0 \\ 0 & 1\end{pmatrix}  =
\begin{pmatrix} 0 &0 & 1 & 0 \\ 0 & 0 & 0 & 1 \\
0 & 0 & 1 & 0\\ 0 & 0 & 0 & 1 \end{pmatrix},
\end{equation*}
\begin{equation*}
\mathbf{d}^{\langle 00, 10 \rangle} = \mathbf{d}^- \otimes \mathbf{d}^\circ =  \begin{pmatrix} 0 \\ 1 \end{pmatrix} \otimes \begin{pmatrix}  1 \\ 0 \end{pmatrix} = \begin{pmatrix}  0 \\ 0 \\ 1 \\ 0 \end{pmatrix},
\end{equation*}
\begin{equation*}
\M^{\langle 10, 00 \rangle} = \M^+ \otimes \M^\circ = 
\begin{pmatrix*}[r] 1 & -1 \\ 0 & 0 \end{pmatrix*} \otimes
\begin{pmatrix} 1 & 0 \\ 0 & 1\end{pmatrix} = 
\begin{pmatrix*}[r] 1 & 0 & -1 & 0 \\ 0 & 1 & 0 & -1 \\ 0 & 0 & 0 & 0 \\ 0 & 0 & 0 & 0 \end{pmatrix*},
\end{equation*}
\begin{equation*}
\mathbf{d}^{\langle 10, 00 \rangle} = \mathbf{d}^+ \otimes \mathbf{d}^\circ = 
\begin{pmatrix*}[r] 1 \\ -1 \end{pmatrix*} \otimes \begin{pmatrix}  1 \\ 0 \end{pmatrix} = \begin{pmatrix*}[r]  1 \\ 0 \\ -1 \\ 0 \end{pmatrix*},
\end{equation*}
\begin{equation*}
\mathbf{L}_1\Pd^{\prime \ev} = \begin{pmatrix*}[r] 0 & 1 \\ 1 & -1 \end{pmatrix*} \begin{pmatrix}1 \\ 0.6\end{pmatrix} = \begin{pmatrix}
0.6 \\ 0.4 \end{pmatrix}.
\end{equation*}

Подставим в уравнение \ref{dis3}, посчитаем и получим, что $\Pd^{\prime 1,a} = \begin{pmatrix}
1 \\  \frac{9}{35} \\ 0.6 \\ 0.2
\end{pmatrix}$.

Найдем вектор виртуального свидетельства. Алфавит, над которым построено свидетельство $\mathcal{A}_{\ev} = \mathcal{A}_1 \cap \mathcal{A}_2 = \{x_1, x_2\} \cap \{x_1, x_3\} = \{x_1\}$.
\begin{equation*}
\V = \V^- \otimes \V^+ =  \begin{pmatrix} 1 & 0  \end{pmatrix} \otimes
  \begin{pmatrix} 1 &0 \\ 0 & 1 \end{pmatrix} = \begin{pmatrix}
1 & 0  & 0 & 0 \\ 0 & 1 & 0 & 0
\end{pmatrix},
\end{equation*}
\begin{equation*}
\Pd^{\prime \ev} = \V\Pd^{\prime a} =  \begin{pmatrix}
1 & 0  & 0 & 0 \\ 0 & 1  & 0 & 0
\end{pmatrix} \begin{pmatrix}
1 \\  \frac{9}{35} \\ 0.6 \\ 0.2
\end{pmatrix} = \begin{pmatrix}
1 \\ \frac{9}{35}
\end{pmatrix}.
\end{equation*} 

Для пропагации виртуального свидетельства воспользуемся уравнением \ref{disGlob}:
$\G = \begin{pmatrix} 0 \\ 1\end{pmatrix}.$

$\chi^0 = \begin{pmatrix} 0 \end{pmatrix} $, 
$\G\chi^0  =  \begin{pmatrix} 0 \\ 1 \end{pmatrix} \begin{pmatrix} 0 \end{pmatrix}= \begin{pmatrix}  0 \\ 0 \end{pmatrix}$,
$\chi^1 = \begin{pmatrix} 1 \end{pmatrix} $, 
$\G \chi^1 = \begin{pmatrix} 0 \\ 1 \end{pmatrix}  \begin{pmatrix} 1 \end{pmatrix}= \begin{pmatrix}  0 \\ 1 \end{pmatrix}$,
\begin{equation*}
\M^{\langle 00, 01 \rangle} = \M^\circ \otimes \M^- = 
\begin{pmatrix} 1 &0 \\ 0 & 1\end{pmatrix}  \otimes
\begin{pmatrix} 0 & 1 \\ 0 & 1 \end{pmatrix} =
\begin{pmatrix} 0 &1 & 0 & 0 \\ 0 & 1 & 0 & 0 \\
0 & 0 & 0 & 1\\ 0 & 0 & 0 & 1 \end{pmatrix},
\end{equation*}
\begin{equation*}
\mathbf{d}^{\langle 00, 01 \rangle} = \mathbf{d}^\circ \otimes \mathbf{d}^- = \begin{pmatrix} 1 \\ 0 \end{pmatrix} \otimes \begin{pmatrix}  0 \\ 1 \end{pmatrix} = \begin{pmatrix}  0 \\ 1 \\ 0 \\ 0 \end{pmatrix},
\end{equation*}
\begin{equation*}
\M^{\langle 01, 00 \rangle} = \M^\circ \otimes \M^+ = 
\begin{pmatrix} 1 & 0 \\ 0 & 1\end{pmatrix} \otimes
\begin{pmatrix*}[r] 1 & -1 \\ 0 & 0 \end{pmatrix*} = 
\begin{pmatrix*}[r] 1 & -1 & 0  & 0 \\ 0 & 0 & 0 & 0 \\ 0 & 0 & 1 & -1 \\ 0 & 0 & 0 & 0 \end{pmatrix*},
\end{equation*}
\begin{equation*}
\mathbf{d}^{\langle 01, 00 \rangle} = \mathbf{d}^\circ \otimes \mathbf{d}^+ =
\begin{pmatrix} 1 \\ 0 \end{pmatrix} \otimes 
\begin{pmatrix*}[r] 1 \\ -1 \end{pmatrix*} = 
\begin{pmatrix*}[r]  1 \\ -1 \\ 0 \\ 0 \end{pmatrix*},
\end{equation*}
\begin{equation*}
\mathbf{L}_1\Pd^{\prime \ev} = \begin{pmatrix*}[r] 0 & 1 \\ 1 & -1 \end{pmatrix*} \begin{pmatrix}1 \\  \frac{9}{35} \end{pmatrix} = \begin{pmatrix}
\frac{9}{35} \\ \frac{26}{35} \end{pmatrix}.
\end{equation*}

Подставим в формулу и получим, что $\Pd^{\prime 2,a} = \begin{pmatrix}
1 \\ \frac{9}{35} \\ \frac{9}{14} \\ \frac{3}{35}
\end{pmatrix}$.	

\subsection*{Пример использования уравнения для пропагации виртуального свидетельства над множеством квантов}
 \addcontentsline{toc}{subsection}{Пример использования уравнения для пропагации виртуального свидетельства над множеством квантов}

Пусть даны два фрагмента знаний над алфавитами $\mathcal{A}_1 = \{x_1, x_2\}$ и $\mathcal{A}_2 = \{x_1, x_3\}$ соответственно. Оба ФЗ построены над множествами квантов. Пусть в первый фрагмент знаний поступило стохастическое свидетельство. Пересчитаем оценки в данном ФЗ с учетом поступившего свидетельства и распространим далее его влияние  из первого ФЗ во второй:
\begin{equation*}
\Pq^{ 1} =  \begin{pmatrix}
p(\overline {x}_2\overline {x}_1)\\ p(\overline {x}_2x_1) \\ p(x_2\overline {x}_1) \\ p(x_2x_1)
\end{pmatrix} = \begin{pmatrix}
0.1 \\ 0.2\\ 0.1\\ 0.6
\end{pmatrix},
\Pq^{2} =  \begin{pmatrix}
p(\overline{x}_3\overline{x}_1) \\ p(\overline{x}_3x_1)\\ p(x_3\overline{x}_1) \\ p(x_3x_1)
\end{pmatrix} = \begin{pmatrix}
0.2 \\ 0.3\\ 0.4\\ 0.1 
\end{pmatrix}.
\end{equation*}

Пусть поступившее свидетельство будет над алфавитом
$\mathcal{A}_{\ev} = \{x_2\}$ со следующим вектором оценок:
\begin{equation*}
\Pq^{\ev} = \begin{pmatrix}  p(\overline {x}_2) \\ p(x_2)	\end{pmatrix} = \begin{pmatrix}
0.4 \\ 0.6
\end{pmatrix}.
\end{equation*}

Воспользуемся уравнением \ref{quantsG} и найдем апостериорные оценки элементов первого ФЗ: $\G = \begin{pmatrix} 1 \\ 0 \end{pmatrix}.$

$\chi^0 = \begin{pmatrix} 0 \end{pmatrix} $, 
$\G\chi^0  = \begin{pmatrix} 1 \\ 0  \end{pmatrix}\begin{pmatrix} 0 \end{pmatrix}  = \begin{pmatrix}  0 \\ 0 \end{pmatrix}$,
$\chi^1 = \begin{pmatrix} 1 \end{pmatrix} $, 
$\G \chi^1  =\begin{pmatrix} 1 \\ 0 \end{pmatrix} \begin{pmatrix} 1 \end{pmatrix}  = \begin{pmatrix}  1 \\ 0 \end{pmatrix}$,
\begin{equation*}
\Selector^{\langle 00, 10 \rangle} = \mathbf{s}^- \otimes \mathbf{s}^\circ =  \begin{pmatrix} 1 \\ 0 \end{pmatrix} \otimes \begin{pmatrix}  1 \\ 1 \end{pmatrix} = \begin{pmatrix}  1 \\ 1 \\ 0 \\ 0 \end{pmatrix},
\end{equation*}
\begin{equation*}
\mathbf{s}^{\langle 10, 00 \rangle} = \mathbf{s}^+ \otimes \mathbf{s}^\circ = 
\begin{pmatrix} 0 \\ 1 \end{pmatrix} \otimes \begin{pmatrix}  1 \\ 1 \end{pmatrix} = \begin{pmatrix}  0 \\ 0 \\ 1 \\ 1 \end{pmatrix}.
\end{equation*}

Подставим в уравнение \ref{quantsG}, вычислим и получим, что $\Pq^{1,a} = \begin{pmatrix}
0.2 \\ 0.4 \\  \frac{2}{35}\\ \frac{12}{35}
\end{pmatrix}$.

Найдем вектор виртуального свидетельства. Алфавит, над которым построено свидетельство $\mathcal{A}_{\ev} = \mathcal{A}_1 \cap \mathcal{A}_2 = \{x_1, x_2\} \cap \{x_1, x_3\} = \{x_1\}$.

Вычислим векторы-селекторы и матрицы проекции $\G_{\mathcal{A}_1, \mathcal{A}_{\ev}}$ и $\G_{\mathcal{A}_2, \mathcal{A}_{\ev}}$. Затем для пропагации виртуального свидетельства воспользуемся уравнением \ref{quantsGlob}: $\G_{\mathcal{A}_1, \mathcal{A}_{\ev}} = \begin{pmatrix} 0 \\ 1\end{pmatrix}.$

$\chi^0 = \begin{pmatrix} 0 \end{pmatrix} $, 
$ \G_{\mathcal{A}_1, \mathcal{A}_{\ev}} \chi^0 =\begin{pmatrix} 0 \\ 1 \end{pmatrix}  \begin{pmatrix} 0 \end{pmatrix} = \begin{pmatrix}  0 \\ 0 \end{pmatrix}$,
$\chi^1 = \begin{pmatrix} 1 \end{pmatrix} $, 
$ \G_{\mathcal{A}_1, \mathcal{A}_{\ev}} \chi^1 = \begin{pmatrix} 0 \\ 1 \end{pmatrix} \begin{pmatrix} 1 \end{pmatrix}  = \begin{pmatrix}  0 \\ 1 \end{pmatrix}$,
\begin{equation*}
\mathbf{s}_1^{\langle 00, 01 \rangle} = \mathbf{s}^\circ \otimes \mathbf{s}^- = \begin{pmatrix} 1 \\ 1 \end{pmatrix} \otimes \begin{pmatrix}  1 \\ 0 \end{pmatrix} = \begin{pmatrix}  1 \\ 0 \\ 1 \\ 0 \end{pmatrix},
\end{equation*}
\begin{equation*}
\mathbf{s}_1^{\langle 01, 00 \rangle} = \mathbf{s}^\circ \otimes \mathbf{s}^+ =
\begin{pmatrix} 1 \\ 1 \end{pmatrix} \otimes \begin{pmatrix}  0 \\ 1 \end{pmatrix} = \begin{pmatrix}  0 \\ 1 \\ 0 \\ 1 \end{pmatrix},
\end{equation*}
\begin{equation*}
\G_{\mathcal{A}_2, \mathcal{A}_{\ev}} = \begin{pmatrix} 0 \\ 1\end{pmatrix},
\end{equation*}

$\chi^0 = \begin{pmatrix} 0 \end{pmatrix} $, 
$ \G_{\mathcal{A}_2, \mathcal{A}_{\ev}} \chi^0 = \begin{pmatrix} 0 \\ 1 \end{pmatrix} \begin{pmatrix} 0 \end{pmatrix}  = \begin{pmatrix}  0 \\ 0 \end{pmatrix}$,
$\chi^1 = \begin{pmatrix} 1 \end{pmatrix} $, 
$\G_{\mathcal{A}_2, \mathcal{A}_{\ev}} \chi^1  = \begin{pmatrix} 0 \\ 1 \end{pmatrix} \begin{pmatrix} 1 \end{pmatrix}  = \begin{pmatrix}  0 \\ 1 \end{pmatrix}$,
\begin{equation*}
\mathbf{s}_2^{\langle 00, 01 \rangle} = \mathbf{s}^- \otimes \mathbf{s}^\circ = \begin{pmatrix} 1 \\ 1 \end{pmatrix} \otimes \begin{pmatrix}  1 \\ 0 \end{pmatrix} = \begin{pmatrix}  1 \\ 0 \\ 1 \\ 0 \end{pmatrix},
\end{equation*}
\begin{equation*}
\mathbf{s}_2^{\langle 01, 00 \rangle} = \mathbf{s}^+ \otimes \mathbf{s}^\circ =
\begin{pmatrix} 1 \\ 1 \end{pmatrix} \otimes \begin{pmatrix}  0 \\ 1 \end{pmatrix} = \begin{pmatrix}  0 \\ 1 \\ 0 \\ 1 \end{pmatrix}.
\end{equation*}


Подставим в формулу \ref{quantsGlob} и получим, что $\Pq^{ 2,a} = \begin{pmatrix}
\frac{3}{35}  \\ \frac{39}{70} \\ \frac{6}{35} \\ \frac{13}{70}
\end{pmatrix}$.

\section*{Приложение B. Публикации по теме работы}
По теме выпускной квалификационной работы были подготовлены следующие публикации:
\begin{enumerate}
\item Золотин А.А., Шляк А.В., Тулупьев А.Л. Пропагация виртуального стохастического свидетельства в алгебраических байесовских сетях: алгоритмы и уравнения // Труды VII Всероссийской Научно-практической Конференции Нечеткие Системы, Мягкие Вычисления и Интеллектуальные Технологии (НСМВИТ-2017). Т. 2. С. 96--107.
\item  Шляк А.В., Золотин А.А. Тулупьев А.Л. Пропагация виртуального свидетельства в алгебраических байесовских сетях: алгоритмы и их особенности // Всероссийская научная конференция по проблемам информатики (СПИСОК-2017). (Санкт-Петербург, 25-27 апреля 2017 г.). Санкт-Петербург: СПбГУ, 2017. С.450--457.
\item Шляк А.В. Задачи матрично-векторной формализации
обработки виртуального свидетельства в алгебраической
байесовской сети // Школа-семинар по искусственному интеллекту: сборник научных трудов. Тверь: ТвГТУ, 2018. C. 76--82.
\end{enumerate}
Также комплекс программ был зарегистрирован в Роспатент:
\begin{enumerate}
\item Шляк А. В., Золотин А.А., Тулупьев А.Л. Algebraic Bayesian Net\-work Virtual Evidence Propagators, Version 01 for CSharp (AlgBN VE Propagators cs.v.01). (Свидетельство). Свид. о гос. регистрации программы для ЭВМ. Рег. № 2018612634(21.02.2018). Роспатент.
\end{enumerate}
\end{document}
   


\section*{Приложение А. Вычислительные эксперименты по распространению виртуального свидетельства}
\subsection*{Пример использования уравнения для пропагации виртуального свидетельства над идеалом конъюнктов}
 \addcontentsline{toc}{subsection}{Пример использования уравнения для пропагации виртуального свидетельства над идеалами конъюнктов}
Пусть даны два фрагмента знаний над алфавитами $\mathcal{A}_1 = \{x_1, x_2\}$ и  $\mathcal{A}_2 = \{x_1, x_3\}$, и пусть в первый фрагмент знаний поступило стохастическое свидетельство. Пересчитаем оценки в данных ФЗ с учетом поступившего свидетельства, а затем распространим его влияние во второй ФЗ:
\begin{equation*}
\Pc^1 =  \begin{pmatrix}
1 \\ p(x_1) \\ p(x_2) \\ p(x_2x_1)
\end{pmatrix} = \begin{pmatrix}
1 \\ 0.8\\ 0.7\\ 0.6
\end{pmatrix}, 
\Pc^2 =  \begin{pmatrix}
1 \\ p(x_1) \\ p(x_3) \\ p(x_3x_1)
\end{pmatrix} = \begin{pmatrix}
1 \\ 0.4\\ 0.5\\ 0.1 
\end{pmatrix}.
\end{equation*}

Пусть поступившее свидетельство будет над алфавитом
$\mathcal{A}_{\ev} = \{x_2\}$:
\begin{equation*}
\Pc^{\ev} = \begin{pmatrix}  1 \\ p(x_2)	\end{pmatrix} = \begin{pmatrix}
1 \\ 0.4
\end{pmatrix}.
\end{equation*}

Воспользуемся уравнением \ref{conjG}. Посчитаем матрицу $\G$ по формуле \ref{G}:
$\G = \begin{pmatrix} 1 \\ 0 \end{pmatrix}.$

$\chi^0 = \begin{pmatrix} 0 \end{pmatrix}, 
\G \chi^0 =  \begin{pmatrix} 1 \\ 0  \end{pmatrix} \begin{pmatrix} 0 \end{pmatrix} = \begin{pmatrix}  0 \\ 0 \end{pmatrix}$,
$\chi^1 = \begin{pmatrix} 1 \end{pmatrix} $, 
$ \G \chi^1 =  \begin{pmatrix} 1 \\ 0 \end{pmatrix} \begin{pmatrix} 1 \end{pmatrix} = \begin{pmatrix}  1 \\ 0 \end{pmatrix}$,

\begin{equation*}
\T^{\langle 00, 10 \rangle} = \T^- \otimes \T^\circ = 
\begin{pmatrix*}[r] 1 & -1 \\ 0 & 0 \end{pmatrix*} \otimes
\begin{pmatrix} 1 & 0 \\ 0 & 1\end{pmatrix} = 
\begin{pmatrix*}[r] 1 & 0 & -1 & 0 \\ 0 & 1 & 0 & -1 \\ 0 & 0 & 0 & 0 \\ 0 & 0 & 0 & 0 \end{pmatrix*},
\end{equation*}
\begin{equation*}
\mathbf{r}^{\langle 00, 10 \rangle} = \mathbf{r}^- \otimes \mathbf{r}^\circ = \begin{pmatrix*}[r] 1 \\ -1 \end{pmatrix*} \otimes \begin{pmatrix}  1 \\ 0 \end{pmatrix} = \begin{pmatrix*}[r]  1 \\ 0 \\ -1 \\ 0 \end{pmatrix*},
\end{equation*}
\begin{equation*}
\T^{\langle 10, 00 \rangle} = \T^+ \otimes \T^\circ = 
\begin{pmatrix} 0 & 1 \\ 0 & 1 \end{pmatrix} \otimes
\begin{pmatrix} 1 &0 \\ 0 & 1\end{pmatrix}  =
\begin{pmatrix} 0 &0 & 1 & 0 \\ 0 & 0 & 0 & 1 \\
0 & 0 & 1 & 0\\ 0 & 0 & 0 & 1 \end{pmatrix},
\end{equation*}
\begin{equation*}
\mathbf{r}^{\langle 10, 00 \rangle} = \mathbf{r}^+ \otimes \mathbf{r}^\circ = \begin{pmatrix} 0 \\ 1 \end{pmatrix} \otimes \begin{pmatrix}  1 \\ 0 \end{pmatrix} = \begin{pmatrix}  0 \\ 0 \\ 1 \\ 0 \end{pmatrix},
\end{equation*}
\begin{equation*}
\mathbf{I}_1\Pc^{\ev} = \begin{pmatrix*}[r] 1 & -1 \\ 0 & 1 \end{pmatrix*} \begin{pmatrix}1 \\ 0.4\end{pmatrix} = \begin{pmatrix}
0.6 \\ 0.4 \end{pmatrix}.
\end{equation*}

Подставим в формулу \ref{conjG}, посчитаем и получим, что $\Pc^{1,a} = \begin{pmatrix}
1 \\  \frac{26}{35} \\ 0.4 \\ \frac{12}{35}
\end{pmatrix}$.

Теперь найдем вектор виртуального свидетельства. Виртуальное свидетельство будет построено над алфавитом 
$\mathcal{A}_{\ev} = \mathcal{A}_1 \cap \mathcal{A}_2 = \{x_1, x_2\} \cap \{x_1, x_3\} = \{x_1\}$.
\begin{equation*}
\Qmatr = \Qmatr^- \otimes \Qmatr^+ =  \begin{pmatrix} 1 & 0  \end{pmatrix} \otimes
  \begin{pmatrix} 1 & 0 \\ 0 & 1 \end{pmatrix} = \begin{pmatrix}
1 & 0  & 0 & 0 \\ 0 & 1 & 0 & 0
\end{pmatrix},
\end{equation*}
\begin{equation*}
\Pc^{\ev} = \Qmatr\Pc^{1, a} =  \begin{pmatrix}
1 & 0  & 0 & 0 \\ 0 & 1  & 0 & 0
\end{pmatrix} \begin{pmatrix}
1 \\  \frac{26}{35} \\ 0.4 \\ \frac{12}{35}
\end{pmatrix} = \begin{pmatrix}
1 \\ \frac{26}{35}
\end{pmatrix}.
\end{equation*} 

Воспользуемся уравнением \ref{conjglob2} для пропагации виртуального свидетельства:
$\G = \begin{pmatrix} 0 \\ 1\end{pmatrix}.$

$\chi^0 = \begin{pmatrix} 0 \end{pmatrix} $, 
$ \G \chi^0 = \begin{pmatrix} 0 \\ 1 \end{pmatrix} \begin{pmatrix} 0 \end{pmatrix} = \begin{pmatrix}  0 \\ 0 \end{pmatrix}$,
$\chi^1 = \begin{pmatrix} 1 \end{pmatrix} $, 
$ \G \chi^1 =  \begin{pmatrix} 0 \\ 1 \end{pmatrix}\begin{pmatrix} 1 \end{pmatrix} = \begin{pmatrix}  0 \\ 1 \end{pmatrix}$,
\begin{equation*}
\T^{\langle 00, 01 \rangle} = \T^\circ \otimes \T^- = 
\begin{pmatrix} 1 & 0 \\ 0 & 1\end{pmatrix} \otimes
\begin{pmatrix*}[r] 1 & -1 \\ 0 & 0 \end{pmatrix*} = 
\begin{pmatrix*}[r] 1 & -1 & 0  & 0 \\ 0 & 0 & 0 & 0 \\ 0 & 0 & 1 & -1 \\ 0 & 0 & 0 & 0 \end{pmatrix*},
\end{equation*}
\begin{equation*}
\mathbf{r}^{\langle 00, 01 \rangle} = \mathbf{r}^\circ \otimes \mathbf{r}^- = \begin{pmatrix} 1 \\ 0 \end{pmatrix} \otimes \begin{pmatrix*}[r]  1 \\ -1 \end{pmatrix*} = \begin{pmatrix*}[r] 1 \\ -1 \\ 0 \\ 0 \end{pmatrix*},
\end{equation*}
\begin{equation*}
\T^{\langle 01, 00 \rangle} = \T^\circ \otimes \T^+ = 
\begin{pmatrix} 1 &0 \\ 0 & 1\end{pmatrix}  \otimes
\begin{pmatrix} 0 & 1 \\ 0 & 1 \end{pmatrix} =
\begin{pmatrix} 0 &1 & 0 & 0 \\ 0 & 1 & 0 & 0 \\
0 & 0 & 0 & 1\\ 0 & 0 & 0 & 1 \end{pmatrix},
\end{equation*}
\begin{equation*}
\mathbf{r}^{\langle 01, 00 \rangle} = \mathbf{r}^\circ \otimes \mathbf{r}^+ = \begin{pmatrix} 1 \\ 0 \end{pmatrix} \otimes \begin{pmatrix}  0 \\ 1 \end{pmatrix} = \begin{pmatrix}  0 \\ 1 \\ 0 \\ 0 \end{pmatrix},
\end{equation*}
\begin{equation*}
\mathbf{I}_1\Pc^{\ev} = \begin{pmatrix*}[r] 1 & -1 \\ 0 & 1 \end{pmatrix*} \begin{pmatrix}1 \\  \frac{26}{35} \end{pmatrix} = \begin{pmatrix}
\frac{9}{35} \\ \frac{26}{35} \end{pmatrix}.
\end{equation*}

Подставим в формулу выше, посчитаем и получим, что $\Pc^{2,a} = \begin{pmatrix}
1 \\ \frac{26}{35} \\ \frac{5}{14} \\ \frac{13}{70}
\end{pmatrix}$.

\subsection*{Пример использования уравнения для пропагации виртуального свидетельства над идеалом дизъюнктов}
 \addcontentsline{toc}{subsection}{Пример использования уравнения для пропагации виртуального свидетельства над идеалами дизъюнктов}
Пусть дан фрагмент знаний над алфавитом $\mathcal{A}_1 = \{x_1, x_2\}$, и пусть в него поступило стохастическое свидетельство. Пересчитаем оценки в данном ФЗ с учетом поступившего свидетельства. Далее распространим влияние этого свидетельства из данного ФЗ в другой ФЗ над алфавитом  $\mathcal{A}_2 = \{x_1, x_3\}$:
\begin{equation*}
\Pd^{\prime 1} =  \begin{pmatrix}
1 \\ p(\overline {x}_1) \\ p(\overline {x}_2) \\ p(\overline {x}_2\overline {x}_1)
\end{pmatrix} = \begin{pmatrix}
1 \\ 0.2\\ 0.3\\ 0.1
\end{pmatrix},
\Pd^{\prime 2} =  \begin{pmatrix}
1 \\ p(\overline{x}_1) \\ p(\overline{x}_3) \\ p(\overline{x}_3\overline{x}_1)
\end{pmatrix} = \begin{pmatrix}
1 \\ 0.6\\ 0.5\\ 0.2 
\end{pmatrix}.
\end{equation*}

Пусть поступившее свидетельство будет над алфавитом
$\mathcal{A}_{\ev} = \{x_2\}$ со следующим вектором оценок:
\begin{equation*}
\Pd^{\prime \ev} = \begin{pmatrix}  1 \\ p(\overline{x}_2)	\end{pmatrix} = \begin{pmatrix}
1 \\ 0.6
\end{pmatrix}.
\end{equation*}

Воспользуемся уравнением \ref{dis3} и найдем апостериорные оценки элементов первого ФЗ:
$ \G = \begin{pmatrix} 1 \\ 0 \end{pmatrix}.$

$\chi^0 = \begin{pmatrix} 0 \end{pmatrix} $, 
$\G \chi^0 = \begin{pmatrix} 1 \\ 0  \end{pmatrix}\begin{pmatrix} 0 \end{pmatrix} = \begin{pmatrix}  0 \\ 0 \end{pmatrix}$,
$\chi^1 = \begin{pmatrix} 1 \end{pmatrix} $, 
$\G \chi^1  = \begin{pmatrix} 1 \\ 0 \end{pmatrix} \begin{pmatrix} 1 \end{pmatrix}  = \begin{pmatrix}  1 \\ 0 \end{pmatrix}$,
\begin{equation*}
\M^{\langle 00, 10 \rangle} = \M^- \otimes \M^\circ = 
\begin{pmatrix} 0 & 1 \\ 0 & 1 \end{pmatrix} \otimes
\begin{pmatrix} 1 &0 \\ 0 & 1\end{pmatrix}  =
\begin{pmatrix} 0 &0 & 1 & 0 \\ 0 & 0 & 0 & 1 \\
0 & 0 & 1 & 0\\ 0 & 0 & 0 & 1 \end{pmatrix},
\end{equation*}
\begin{equation*}
\mathbf{d}^{\langle 00, 10 \rangle} = \mathbf{d}^- \otimes \mathbf{d}^\circ =  \begin{pmatrix} 0 \\ 1 \end{pmatrix} \otimes \begin{pmatrix}  1 \\ 0 \end{pmatrix} = \begin{pmatrix}  0 \\ 0 \\ 1 \\ 0 \end{pmatrix},
\end{equation*}
\begin{equation*}
\M^{\langle 10, 00 \rangle} = \M^+ \otimes \M^\circ = 
\begin{pmatrix*}[r] 1 & -1 \\ 0 & 0 \end{pmatrix*} \otimes
\begin{pmatrix} 1 & 0 \\ 0 & 1\end{pmatrix} = 
\begin{pmatrix*}[r] 1 & 0 & -1 & 0 \\ 0 & 1 & 0 & -1 \\ 0 & 0 & 0 & 0 \\ 0 & 0 & 0 & 0 \end{pmatrix*},
\end{equation*}
\begin{equation*}
\mathbf{d}^{\langle 10, 00 \rangle} = \mathbf{d}^+ \otimes \mathbf{d}^\circ = 
\begin{pmatrix*}[r] 1 \\ -1 \end{pmatrix*} \otimes \begin{pmatrix}  1 \\ 0 \end{pmatrix} = \begin{pmatrix*}[r]  1 \\ 0 \\ -1 \\ 0 \end{pmatrix*},
\end{equation*}
\begin{equation*}
\mathbf{L}_1\Pd^{\prime \ev} = \begin{pmatrix*}[r] 0 & 1 \\ 1 & -1 \end{pmatrix*} \begin{pmatrix}1 \\ 0.6\end{pmatrix} = \begin{pmatrix}
0.6 \\ 0.4 \end{pmatrix}.
\end{equation*}

Подставим в уравнение \ref{dis3}, посчитаем и получим, что $\Pd^{\prime 1,a} = \begin{pmatrix}
1 \\  \frac{9}{35} \\ 0.6 \\ 0.2
\end{pmatrix}$.

Найдем вектор виртуального свидетельства. Алфавит, над которым построено свидетельство $\mathcal{A}_{\ev} = \mathcal{A}_1 \cap \mathcal{A}_2 = \{x_1, x_2\} \cap \{x_1, x_3\} = \{x_1\}$.
\begin{equation*}
\V = \V^- \otimes \V^+ =  \begin{pmatrix} 1 & 0  \end{pmatrix} \otimes
  \begin{pmatrix} 1 &0 \\ 0 & 1 \end{pmatrix} = \begin{pmatrix}
1 & 0  & 0 & 0 \\ 0 & 1 & 0 & 0
\end{pmatrix},
\end{equation*}
\begin{equation*}
\Pd^{\prime \ev} = \V\Pd^{\prime a} =  \begin{pmatrix}
1 & 0  & 0 & 0 \\ 0 & 1  & 0 & 0
\end{pmatrix} \begin{pmatrix}
1 \\  \frac{9}{35} \\ 0.6 \\ 0.2
\end{pmatrix} = \begin{pmatrix}
1 \\ \frac{9}{35}
\end{pmatrix}.
\end{equation*} 

Для пропагации виртуального свидетельства воспользуемся уравнением \ref{disGlob}:
$\G = \begin{pmatrix} 0 \\ 1\end{pmatrix}.$

$\chi^0 = \begin{pmatrix} 0 \end{pmatrix} $, 
$\G\chi^0  =  \begin{pmatrix} 0 \\ 1 \end{pmatrix} \begin{pmatrix} 0 \end{pmatrix}= \begin{pmatrix}  0 \\ 0 \end{pmatrix}$,
$\chi^1 = \begin{pmatrix} 1 \end{pmatrix} $, 
$\G \chi^1 = \begin{pmatrix} 0 \\ 1 \end{pmatrix}  \begin{pmatrix} 1 \end{pmatrix}= \begin{pmatrix}  0 \\ 1 \end{pmatrix}$,
\begin{equation*}
\M^{\langle 00, 01 \rangle} = \M^\circ \otimes \M^- = 
\begin{pmatrix} 1 &0 \\ 0 & 1\end{pmatrix}  \otimes
\begin{pmatrix} 0 & 1 \\ 0 & 1 \end{pmatrix} =
\begin{pmatrix} 0 &1 & 0 & 0 \\ 0 & 1 & 0 & 0 \\
0 & 0 & 0 & 1\\ 0 & 0 & 0 & 1 \end{pmatrix},
\end{equation*}
\begin{equation*}
\mathbf{d}^{\langle 00, 01 \rangle} = \mathbf{d}^\circ \otimes \mathbf{d}^- = \begin{pmatrix} 1 \\ 0 \end{pmatrix} \otimes \begin{pmatrix}  0 \\ 1 \end{pmatrix} = \begin{pmatrix}  0 \\ 1 \\ 0 \\ 0 \end{pmatrix},
\end{equation*}
\begin{equation*}
\M^{\langle 01, 00 \rangle} = \M^\circ \otimes \M^+ = 
\begin{pmatrix} 1 & 0 \\ 0 & 1\end{pmatrix} \otimes
\begin{pmatrix*}[r] 1 & -1 \\ 0 & 0 \end{pmatrix*} = 
\begin{pmatrix*}[r] 1 & -1 & 0  & 0 \\ 0 & 0 & 0 & 0 \\ 0 & 0 & 1 & -1 \\ 0 & 0 & 0 & 0 \end{pmatrix*},
\end{equation*}
\begin{equation*}
\mathbf{d}^{\langle 01, 00 \rangle} = \mathbf{d}^\circ \otimes \mathbf{d}^+ =
\begin{pmatrix} 1 \\ 0 \end{pmatrix} \otimes 
\begin{pmatrix*}[r] 1 \\ -1 \end{pmatrix*} = 
\begin{pmatrix*}[r]  1 \\ -1 \\ 0 \\ 0 \end{pmatrix*},
\end{equation*}
\begin{equation*}
\mathbf{L}_1\Pd^{\prime \ev} = \begin{pmatrix*}[r] 0 & 1 \\ 1 & -1 \end{pmatrix*} \begin{pmatrix}1 \\  \frac{9}{35} \end{pmatrix} = \begin{pmatrix}
\frac{9}{35} \\ \frac{26}{35} \end{pmatrix}.
\end{equation*}

Подставим в формулу и получим, что $\Pd^{\prime 2,a} = \begin{pmatrix}
1 \\ \frac{9}{35} \\ \frac{9}{14} \\ \frac{3}{35}
\end{pmatrix}$.	

\subsection*{Пример использования уравнения для пропагации виртуального свидетельства над множеством квантов}
 \addcontentsline{toc}{subsection}{Пример использования уравнения для пропагации виртуального свидетельства над множеством квантов}

Пусть даны два фрагмента знаний над алфавитами $\mathcal{A}_1 = \{x_1, x_2\}$ и $\mathcal{A}_2 = \{x_1, x_3\}$ соответственно. Оба ФЗ построены над множествами квантов. Пусть в первый фрагмент знаний поступило стохастическое свидетельство. Пересчитаем оценки в данном ФЗ с учетом поступившего свидетельства и распространим далее его влияние  из первого ФЗ во второй:
\begin{equation*}
\Pq^{ 1} =  \begin{pmatrix}
p(\overline {x}_2\overline {x}_1)\\ p(\overline {x}_2x_1) \\ p(x_2\overline {x}_1) \\ p(x_2x_1)
\end{pmatrix} = \begin{pmatrix}
0.1 \\ 0.2\\ 0.1\\ 0.6
\end{pmatrix},
\Pq^{2} =  \begin{pmatrix}
p(\overline{x}_3\overline{x}_1) \\ p(\overline{x}_3x_1)\\ p(x_3\overline{x}_1) \\ p(x_3x_1)
\end{pmatrix} = \begin{pmatrix}
0.2 \\ 0.3\\ 0.4\\ 0.1 
\end{pmatrix}.
\end{equation*}

Пусть поступившее свидетельство будет над алфавитом
$\mathcal{A}_{\ev} = \{x_2\}$ со следующим вектором оценок:
\begin{equation*}
\Pq^{\ev} = \begin{pmatrix}  p(\overline {x}_2) \\ p(x_2)	\end{pmatrix} = \begin{pmatrix}
0.4 \\ 0.6
\end{pmatrix}.
\end{equation*}

Воспользуемся уравнением \ref{quantsG} и найдем апостериорные оценки элементов первого ФЗ: $\G = \begin{pmatrix} 1 \\ 0 \end{pmatrix}.$

$\chi^0 = \begin{pmatrix} 0 \end{pmatrix} $, 
$\G\chi^0  = \begin{pmatrix} 1 \\ 0  \end{pmatrix}\begin{pmatrix} 0 \end{pmatrix}  = \begin{pmatrix}  0 \\ 0 \end{pmatrix}$,
$\chi^1 = \begin{pmatrix} 1 \end{pmatrix} $, 
$\G \chi^1  =\begin{pmatrix} 1 \\ 0 \end{pmatrix} \begin{pmatrix} 1 \end{pmatrix}  = \begin{pmatrix}  1 \\ 0 \end{pmatrix}$,
\begin{equation*}
\Selector^{\langle 00, 10 \rangle} = \mathbf{s}^- \otimes \mathbf{s}^\circ =  \begin{pmatrix} 1 \\ 0 \end{pmatrix} \otimes \begin{pmatrix}  1 \\ 1 \end{pmatrix} = \begin{pmatrix}  1 \\ 1 \\ 0 \\ 0 \end{pmatrix},
\end{equation*}
\begin{equation*}
\mathbf{s}^{\langle 10, 00 \rangle} = \mathbf{s}^+ \otimes \mathbf{s}^\circ = 
\begin{pmatrix} 0 \\ 1 \end{pmatrix} \otimes \begin{pmatrix}  1 \\ 1 \end{pmatrix} = \begin{pmatrix}  0 \\ 0 \\ 1 \\ 1 \end{pmatrix}.
\end{equation*}

Подставим в уравнение \ref{quantsG}, вычислим и получим, что $\Pq^{1,a} = \begin{pmatrix}
0.2 \\ 0.4 \\  \frac{2}{35}\\ \frac{12}{35}
\end{pmatrix}$.

Найдем вектор виртуального свидетельства. Алфавит, над которым построено свидетельство $\mathcal{A}_{\ev} = \mathcal{A}_1 \cap \mathcal{A}_2 = \{x_1, x_2\} \cap \{x_1, x_3\} = \{x_1\}$.

Вычислим векторы-селекторы и матрицы проекции $\G_{\mathcal{A}_1, \mathcal{A}_{\ev}}$ и $\G_{\mathcal{A}_2, \mathcal{A}_{\ev}}$. Затем для пропагации виртуального свидетельства воспользуемся уравнением \ref{quantsGlob}: $\G_{\mathcal{A}_1, \mathcal{A}_{\ev}} = \begin{pmatrix} 0 \\ 1\end{pmatrix}.$

$\chi^0 = \begin{pmatrix} 0 \end{pmatrix} $, 
$ \G_{\mathcal{A}_1, \mathcal{A}_{\ev}} \chi^0 =\begin{pmatrix} 0 \\ 1 \end{pmatrix}  \begin{pmatrix} 0 \end{pmatrix} = \begin{pmatrix}  0 \\ 0 \end{pmatrix}$,
$\chi^1 = \begin{pmatrix} 1 \end{pmatrix} $, 
$ \G_{\mathcal{A}_1, \mathcal{A}_{\ev}} \chi^1 = \begin{pmatrix} 0 \\ 1 \end{pmatrix} \begin{pmatrix} 1 \end{pmatrix}  = \begin{pmatrix}  0 \\ 1 \end{pmatrix}$,
\begin{equation*}
\mathbf{s}_1^{\langle 00, 01 \rangle} = \mathbf{s}^\circ \otimes \mathbf{s}^- = \begin{pmatrix} 1 \\ 1 \end{pmatrix} \otimes \begin{pmatrix}  1 \\ 0 \end{pmatrix} = \begin{pmatrix}  1 \\ 0 \\ 1 \\ 0 \end{pmatrix},
\end{equation*}
\begin{equation*}
\mathbf{s}_1^{\langle 01, 00 \rangle} = \mathbf{s}^\circ \otimes \mathbf{s}^+ =
\begin{pmatrix} 1 \\ 1 \end{pmatrix} \otimes \begin{pmatrix}  0 \\ 1 \end{pmatrix} = \begin{pmatrix}  0 \\ 1 \\ 0 \\ 1 \end{pmatrix},
\end{equation*}
\begin{equation*}
\G_{\mathcal{A}_2, \mathcal{A}_{\ev}} = \begin{pmatrix} 0 \\ 1\end{pmatrix},
\end{equation*}

$\chi^0 = \begin{pmatrix} 0 \end{pmatrix} $, 
$ \G_{\mathcal{A}_2, \mathcal{A}_{\ev}} \chi^0 = \begin{pmatrix} 0 \\ 1 \end{pmatrix} \begin{pmatrix} 0 \end{pmatrix}  = \begin{pmatrix}  0 \\ 0 \end{pmatrix}$,
$\chi^1 = \begin{pmatrix} 1 \end{pmatrix} $, 
$\G_{\mathcal{A}_2, \mathcal{A}_{\ev}} \chi^1  = \begin{pmatrix} 0 \\ 1 \end{pmatrix} \begin{pmatrix} 1 \end{pmatrix}  = \begin{pmatrix}  0 \\ 1 \end{pmatrix}$,
\begin{equation*}
\mathbf{s}_2^{\langle 00, 01 \rangle} = \mathbf{s}^- \otimes \mathbf{s}^\circ = \begin{pmatrix} 1 \\ 1 \end{pmatrix} \otimes \begin{pmatrix}  1 \\ 0 \end{pmatrix} = \begin{pmatrix}  1 \\ 0 \\ 1 \\ 0 \end{pmatrix},
\end{equation*}
\begin{equation*}
\mathbf{s}_2^{\langle 01, 00 \rangle} = \mathbf{s}^+ \otimes \mathbf{s}^\circ =
\begin{pmatrix} 1 \\ 1 \end{pmatrix} \otimes \begin{pmatrix}  0 \\ 1 \end{pmatrix} = \begin{pmatrix}  0 \\ 1 \\ 0 \\ 1 \end{pmatrix}.
\end{equation*}


Подставим в формулу \ref{quantsGlob} и получим, что $\Pq^{ 2,a} = \begin{pmatrix}
\frac{3}{35}  \\ \frac{39}{70} \\ \frac{6}{35} \\ \frac{13}{70}
\end{pmatrix}$.

\section*{Приложение B. Публикации по теме работы}
По теме выпускной квалификационной работы были подготовлены следующие публикации:
\begin{enumerate}
\item Золотин А.А., Шляк А.В., Тулупьев А.Л. Пропагация виртуального стохастического свидетельства в алгебраических байесовских сетях: алгоритмы и уравнения // Труды VII Всероссийской Научно-практической Конференции Нечеткие Системы, Мягкие Вычисления и Интеллектуальные Технологии (НСМВИТ-2017). Т. 2. С. 96--107.
\item  Шляк А.В., Золотин А.А. Тулупьев А.Л. Пропагация виртуального свидетельства в алгебраических байесовских сетях: алгоритмы и их особенности // Всероссийская научная конференция по проблемам информатики (СПИСОК-2017). (Санкт-Петербург, 25-27 апреля 2017 г.). Санкт-Петербург: СПбГУ, 2017. С.450--457.
\item Шляк А.В. Задачи матрично-векторной формализации
обработки виртуального свидетельства в алгебраической
байесовской сети // Школа-семинар по искусственному интеллекту: сборник научных трудов. Тверь: ТвГТУ, 2018. C. 76--82.
\end{enumerate}
Также комплекс программ был зарегистрирован в Роспатент:
\begin{enumerate}
\item Шляк А. В., Золотин А.А., Тулупьев А.Л. Algebraic Bayesian Net\-work Virtual Evidence Propagators, Version 01 for CSharp (AlgBN VE Propagators cs.v.01). (Свидетельство). Свид. о гос. регистрации программы для ЭВМ. Рег. № 2018612634(21.02.2018). Роспатент.
\end{enumerate}
\end{document}
   


\section*{Приложение А. Вычислительные эксперименты по распространению виртуального свидетельства}
\subsection*{Пример использования уравнения для пропагации виртуального свидетельства над идеалом конъюнктов}
 \addcontentsline{toc}{subsection}{Пример использования уравнения для пропагации виртуального свидетельства над идеалами конъюнктов}
Пусть даны два фрагмента знаний над алфавитами $\mathcal{A}_1 = \{x_1, x_2\}$ и  $\mathcal{A}_2 = \{x_1, x_3\}$, и пусть в первый фрагмент знаний поступило стохастическое свидетельство. Пересчитаем оценки в данных ФЗ с учетом поступившего свидетельства, а затем распространим его влияние во второй ФЗ:
\begin{equation*}
\Pc^1 =  \begin{pmatrix}
1 \\ p(x_1) \\ p(x_2) \\ p(x_2x_1)
\end{pmatrix} = \begin{pmatrix}
1 \\ 0.8\\ 0.7\\ 0.6
\end{pmatrix}, 
\Pc^2 =  \begin{pmatrix}
1 \\ p(x_1) \\ p(x_3) \\ p(x_3x_1)
\end{pmatrix} = \begin{pmatrix}
1 \\ 0.4\\ 0.5\\ 0.1 
\end{pmatrix}.
\end{equation*}

Пусть поступившее свидетельство будет над алфавитом
$\mathcal{A}_{\ev} = \{x_2\}$:
\begin{equation*}
\Pc^{\ev} = \begin{pmatrix}  1 \\ p(x_2)	\end{pmatrix} = \begin{pmatrix}
1 \\ 0.4
\end{pmatrix}.
\end{equation*}

Воспользуемся уравнением \ref{conjG}. Посчитаем матрицу $\G$ по формуле \ref{G}:
$\G = \begin{pmatrix} 1 \\ 0 \end{pmatrix}.$

$\chi^0 = \begin{pmatrix} 0 \end{pmatrix}, 
\G \chi^0 =  \begin{pmatrix} 1 \\ 0  \end{pmatrix} \begin{pmatrix} 0 \end{pmatrix} = \begin{pmatrix}  0 \\ 0 \end{pmatrix}$,
$\chi^1 = \begin{pmatrix} 1 \end{pmatrix} $, 
$ \G \chi^1 =  \begin{pmatrix} 1 \\ 0 \end{pmatrix} \begin{pmatrix} 1 \end{pmatrix} = \begin{pmatrix}  1 \\ 0 \end{pmatrix}$,

\begin{equation*}
\T^{\langle 00, 10 \rangle} = \T^- \otimes \T^\circ = 
\begin{pmatrix*}[r] 1 & -1 \\ 0 & 0 \end{pmatrix*} \otimes
\begin{pmatrix} 1 & 0 \\ 0 & 1\end{pmatrix} = 
\begin{pmatrix*}[r] 1 & 0 & -1 & 0 \\ 0 & 1 & 0 & -1 \\ 0 & 0 & 0 & 0 \\ 0 & 0 & 0 & 0 \end{pmatrix*},
\end{equation*}
\begin{equation*}
\mathbf{r}^{\langle 00, 10 \rangle} = \mathbf{r}^- \otimes \mathbf{r}^\circ = \begin{pmatrix*}[r] 1 \\ -1 \end{pmatrix*} \otimes \begin{pmatrix}  1 \\ 0 \end{pmatrix} = \begin{pmatrix*}[r]  1 \\ 0 \\ -1 \\ 0 \end{pmatrix*},
\end{equation*}
\begin{equation*}
\T^{\langle 10, 00 \rangle} = \T^+ \otimes \T^\circ = 
\begin{pmatrix} 0 & 1 \\ 0 & 1 \end{pmatrix} \otimes
\begin{pmatrix} 1 &0 \\ 0 & 1\end{pmatrix}  =
\begin{pmatrix} 0 &0 & 1 & 0 \\ 0 & 0 & 0 & 1 \\
0 & 0 & 1 & 0\\ 0 & 0 & 0 & 1 \end{pmatrix},
\end{equation*}
\begin{equation*}
\mathbf{r}^{\langle 10, 00 \rangle} = \mathbf{r}^+ \otimes \mathbf{r}^\circ = \begin{pmatrix} 0 \\ 1 \end{pmatrix} \otimes \begin{pmatrix}  1 \\ 0 \end{pmatrix} = \begin{pmatrix}  0 \\ 0 \\ 1 \\ 0 \end{pmatrix},
\end{equation*}
\begin{equation*}
\mathbf{I}_1\Pc^{\ev} = \begin{pmatrix*}[r] 1 & -1 \\ 0 & 1 \end{pmatrix*} \begin{pmatrix}1 \\ 0.4\end{pmatrix} = \begin{pmatrix}
0.6 \\ 0.4 \end{pmatrix}.
\end{equation*}

Подставим в формулу \ref{conjG}, посчитаем и получим, что $\Pc^{1,a} = \begin{pmatrix}
1 \\  \frac{26}{35} \\ 0.4 \\ \frac{12}{35}
\end{pmatrix}$.

Теперь найдем вектор виртуального свидетельства. Виртуальное свидетельство будет построено над алфавитом 
$\mathcal{A}_{\ev} = \mathcal{A}_1 \cap \mathcal{A}_2 = \{x_1, x_2\} \cap \{x_1, x_3\} = \{x_1\}$.
\begin{equation*}
\Qmatr = \Qmatr^- \otimes \Qmatr^+ =  \begin{pmatrix} 1 & 0  \end{pmatrix} \otimes
  \begin{pmatrix} 1 & 0 \\ 0 & 1 \end{pmatrix} = \begin{pmatrix}
1 & 0  & 0 & 0 \\ 0 & 1 & 0 & 0
\end{pmatrix},
\end{equation*}
\begin{equation*}
\Pc^{\ev} = \Qmatr\Pc^{1, a} =  \begin{pmatrix}
1 & 0  & 0 & 0 \\ 0 & 1  & 0 & 0
\end{pmatrix} \begin{pmatrix}
1 \\  \frac{26}{35} \\ 0.4 \\ \frac{12}{35}
\end{pmatrix} = \begin{pmatrix}
1 \\ \frac{26}{35}
\end{pmatrix}.
\end{equation*} 

Воспользуемся уравнением \ref{conjglob2} для пропагации виртуального свидетельства:
$\G = \begin{pmatrix} 0 \\ 1\end{pmatrix}.$

$\chi^0 = \begin{pmatrix} 0 \end{pmatrix} $, 
$ \G \chi^0 = \begin{pmatrix} 0 \\ 1 \end{pmatrix} \begin{pmatrix} 0 \end{pmatrix} = \begin{pmatrix}  0 \\ 0 \end{pmatrix}$,
$\chi^1 = \begin{pmatrix} 1 \end{pmatrix} $, 
$ \G \chi^1 =  \begin{pmatrix} 0 \\ 1 \end{pmatrix}\begin{pmatrix} 1 \end{pmatrix} = \begin{pmatrix}  0 \\ 1 \end{pmatrix}$,
\begin{equation*}
\T^{\langle 00, 01 \rangle} = \T^\circ \otimes \T^- = 
\begin{pmatrix} 1 & 0 \\ 0 & 1\end{pmatrix} \otimes
\begin{pmatrix*}[r] 1 & -1 \\ 0 & 0 \end{pmatrix*} = 
\begin{pmatrix*}[r] 1 & -1 & 0  & 0 \\ 0 & 0 & 0 & 0 \\ 0 & 0 & 1 & -1 \\ 0 & 0 & 0 & 0 \end{pmatrix*},
\end{equation*}
\begin{equation*}
\mathbf{r}^{\langle 00, 01 \rangle} = \mathbf{r}^\circ \otimes \mathbf{r}^- = \begin{pmatrix} 1 \\ 0 \end{pmatrix} \otimes \begin{pmatrix*}[r]  1 \\ -1 \end{pmatrix*} = \begin{pmatrix*}[r] 1 \\ -1 \\ 0 \\ 0 \end{pmatrix*},
\end{equation*}
\begin{equation*}
\T^{\langle 01, 00 \rangle} = \T^\circ \otimes \T^+ = 
\begin{pmatrix} 1 &0 \\ 0 & 1\end{pmatrix}  \otimes
\begin{pmatrix} 0 & 1 \\ 0 & 1 \end{pmatrix} =
\begin{pmatrix} 0 &1 & 0 & 0 \\ 0 & 1 & 0 & 0 \\
0 & 0 & 0 & 1\\ 0 & 0 & 0 & 1 \end{pmatrix},
\end{equation*}
\begin{equation*}
\mathbf{r}^{\langle 01, 00 \rangle} = \mathbf{r}^\circ \otimes \mathbf{r}^+ = \begin{pmatrix} 1 \\ 0 \end{pmatrix} \otimes \begin{pmatrix}  0 \\ 1 \end{pmatrix} = \begin{pmatrix}  0 \\ 1 \\ 0 \\ 0 \end{pmatrix},
\end{equation*}
\begin{equation*}
\mathbf{I}_1\Pc^{\ev} = \begin{pmatrix*}[r] 1 & -1 \\ 0 & 1 \end{pmatrix*} \begin{pmatrix}1 \\  \frac{26}{35} \end{pmatrix} = \begin{pmatrix}
\frac{9}{35} \\ \frac{26}{35} \end{pmatrix}.
\end{equation*}

Подставим в формулу выше, посчитаем и получим, что $\Pc^{2,a} = \begin{pmatrix}
1 \\ \frac{26}{35} \\ \frac{5}{14} \\ \frac{13}{70}
\end{pmatrix}$.

\subsection*{Пример использования уравнения для пропагации виртуального свидетельства над идеалом дизъюнктов}
 \addcontentsline{toc}{subsection}{Пример использования уравнения для пропагации виртуального свидетельства над идеалами дизъюнктов}
Пусть дан фрагмент знаний над алфавитом $\mathcal{A}_1 = \{x_1, x_2\}$, и пусть в него поступило стохастическое свидетельство. Пересчитаем оценки в данном ФЗ с учетом поступившего свидетельства. Далее распространим влияние этого свидетельства из данного ФЗ в другой ФЗ над алфавитом  $\mathcal{A}_2 = \{x_1, x_3\}$:
\begin{equation*}
\Pd^{\prime 1} =  \begin{pmatrix}
1 \\ p(\overline {x}_1) \\ p(\overline {x}_2) \\ p(\overline {x}_2\overline {x}_1)
\end{pmatrix} = \begin{pmatrix}
1 \\ 0.2\\ 0.3\\ 0.1
\end{pmatrix},
\Pd^{\prime 2} =  \begin{pmatrix}
1 \\ p(\overline{x}_1) \\ p(\overline{x}_3) \\ p(\overline{x}_3\overline{x}_1)
\end{pmatrix} = \begin{pmatrix}
1 \\ 0.6\\ 0.5\\ 0.2 
\end{pmatrix}.
\end{equation*}

Пусть поступившее свидетельство будет над алфавитом
$\mathcal{A}_{\ev} = \{x_2\}$ со следующим вектором оценок:
\begin{equation*}
\Pd^{\prime \ev} = \begin{pmatrix}  1 \\ p(\overline{x}_2)	\end{pmatrix} = \begin{pmatrix}
1 \\ 0.6
\end{pmatrix}.
\end{equation*}

Воспользуемся уравнением \ref{dis3} и найдем апостериорные оценки элементов первого ФЗ:
$ \G = \begin{pmatrix} 1 \\ 0 \end{pmatrix}.$

$\chi^0 = \begin{pmatrix} 0 \end{pmatrix} $, 
$\G \chi^0 = \begin{pmatrix} 1 \\ 0  \end{pmatrix}\begin{pmatrix} 0 \end{pmatrix} = \begin{pmatrix}  0 \\ 0 \end{pmatrix}$,
$\chi^1 = \begin{pmatrix} 1 \end{pmatrix} $, 
$\G \chi^1  = \begin{pmatrix} 1 \\ 0 \end{pmatrix} \begin{pmatrix} 1 \end{pmatrix}  = \begin{pmatrix}  1 \\ 0 \end{pmatrix}$,
\begin{equation*}
\M^{\langle 00, 10 \rangle} = \M^- \otimes \M^\circ = 
\begin{pmatrix} 0 & 1 \\ 0 & 1 \end{pmatrix} \otimes
\begin{pmatrix} 1 &0 \\ 0 & 1\end{pmatrix}  =
\begin{pmatrix} 0 &0 & 1 & 0 \\ 0 & 0 & 0 & 1 \\
0 & 0 & 1 & 0\\ 0 & 0 & 0 & 1 \end{pmatrix},
\end{equation*}
\begin{equation*}
\mathbf{d}^{\langle 00, 10 \rangle} = \mathbf{d}^- \otimes \mathbf{d}^\circ =  \begin{pmatrix} 0 \\ 1 \end{pmatrix} \otimes \begin{pmatrix}  1 \\ 0 \end{pmatrix} = \begin{pmatrix}  0 \\ 0 \\ 1 \\ 0 \end{pmatrix},
\end{equation*}
\begin{equation*}
\M^{\langle 10, 00 \rangle} = \M^+ \otimes \M^\circ = 
\begin{pmatrix*}[r] 1 & -1 \\ 0 & 0 \end{pmatrix*} \otimes
\begin{pmatrix} 1 & 0 \\ 0 & 1\end{pmatrix} = 
\begin{pmatrix*}[r] 1 & 0 & -1 & 0 \\ 0 & 1 & 0 & -1 \\ 0 & 0 & 0 & 0 \\ 0 & 0 & 0 & 0 \end{pmatrix*},
\end{equation*}
\begin{equation*}
\mathbf{d}^{\langle 10, 00 \rangle} = \mathbf{d}^+ \otimes \mathbf{d}^\circ = 
\begin{pmatrix*}[r] 1 \\ -1 \end{pmatrix*} \otimes \begin{pmatrix}  1 \\ 0 \end{pmatrix} = \begin{pmatrix*}[r]  1 \\ 0 \\ -1 \\ 0 \end{pmatrix*},
\end{equation*}
\begin{equation*}
\mathbf{L}_1\Pd^{\prime \ev} = \begin{pmatrix*}[r] 0 & 1 \\ 1 & -1 \end{pmatrix*} \begin{pmatrix}1 \\ 0.6\end{pmatrix} = \begin{pmatrix}
0.6 \\ 0.4 \end{pmatrix}.
\end{equation*}

Подставим в уравнение \ref{dis3}, посчитаем и получим, что $\Pd^{\prime 1,a} = \begin{pmatrix}
1 \\  \frac{9}{35} \\ 0.6 \\ 0.2
\end{pmatrix}$.

Найдем вектор виртуального свидетельства. Алфавит, над которым построено свидетельство $\mathcal{A}_{\ev} = \mathcal{A}_1 \cap \mathcal{A}_2 = \{x_1, x_2\} \cap \{x_1, x_3\} = \{x_1\}$.
\begin{equation*}
\V = \V^- \otimes \V^+ =  \begin{pmatrix} 1 & 0  \end{pmatrix} \otimes
  \begin{pmatrix} 1 &0 \\ 0 & 1 \end{pmatrix} = \begin{pmatrix}
1 & 0  & 0 & 0 \\ 0 & 1 & 0 & 0
\end{pmatrix},
\end{equation*}
\begin{equation*}
\Pd^{\prime \ev} = \V\Pd^{\prime a} =  \begin{pmatrix}
1 & 0  & 0 & 0 \\ 0 & 1  & 0 & 0
\end{pmatrix} \begin{pmatrix}
1 \\  \frac{9}{35} \\ 0.6 \\ 0.2
\end{pmatrix} = \begin{pmatrix}
1 \\ \frac{9}{35}
\end{pmatrix}.
\end{equation*} 

Для пропагации виртуального свидетельства воспользуемся уравнением \ref{disGlob}:
$\G = \begin{pmatrix} 0 \\ 1\end{pmatrix}.$

$\chi^0 = \begin{pmatrix} 0 \end{pmatrix} $, 
$\G\chi^0  =  \begin{pmatrix} 0 \\ 1 \end{pmatrix} \begin{pmatrix} 0 \end{pmatrix}= \begin{pmatrix}  0 \\ 0 \end{pmatrix}$,
$\chi^1 = \begin{pmatrix} 1 \end{pmatrix} $, 
$\G \chi^1 = \begin{pmatrix} 0 \\ 1 \end{pmatrix}  \begin{pmatrix} 1 \end{pmatrix}= \begin{pmatrix}  0 \\ 1 \end{pmatrix}$,
\begin{equation*}
\M^{\langle 00, 01 \rangle} = \M^\circ \otimes \M^- = 
\begin{pmatrix} 1 &0 \\ 0 & 1\end{pmatrix}  \otimes
\begin{pmatrix} 0 & 1 \\ 0 & 1 \end{pmatrix} =
\begin{pmatrix} 0 &1 & 0 & 0 \\ 0 & 1 & 0 & 0 \\
0 & 0 & 0 & 1\\ 0 & 0 & 0 & 1 \end{pmatrix},
\end{equation*}
\begin{equation*}
\mathbf{d}^{\langle 00, 01 \rangle} = \mathbf{d}^\circ \otimes \mathbf{d}^- = \begin{pmatrix} 1 \\ 0 \end{pmatrix} \otimes \begin{pmatrix}  0 \\ 1 \end{pmatrix} = \begin{pmatrix}  0 \\ 1 \\ 0 \\ 0 \end{pmatrix},
\end{equation*}
\begin{equation*}
\M^{\langle 01, 00 \rangle} = \M^\circ \otimes \M^+ = 
\begin{pmatrix} 1 & 0 \\ 0 & 1\end{pmatrix} \otimes
\begin{pmatrix*}[r] 1 & -1 \\ 0 & 0 \end{pmatrix*} = 
\begin{pmatrix*}[r] 1 & -1 & 0  & 0 \\ 0 & 0 & 0 & 0 \\ 0 & 0 & 1 & -1 \\ 0 & 0 & 0 & 0 \end{pmatrix*},
\end{equation*}
\begin{equation*}
\mathbf{d}^{\langle 01, 00 \rangle} = \mathbf{d}^\circ \otimes \mathbf{d}^+ =
\begin{pmatrix} 1 \\ 0 \end{pmatrix} \otimes 
\begin{pmatrix*}[r] 1 \\ -1 \end{pmatrix*} = 
\begin{pmatrix*}[r]  1 \\ -1 \\ 0 \\ 0 \end{pmatrix*},
\end{equation*}
\begin{equation*}
\mathbf{L}_1\Pd^{\prime \ev} = \begin{pmatrix*}[r] 0 & 1 \\ 1 & -1 \end{pmatrix*} \begin{pmatrix}1 \\  \frac{9}{35} \end{pmatrix} = \begin{pmatrix}
\frac{9}{35} \\ \frac{26}{35} \end{pmatrix}.
\end{equation*}

Подставим в формулу и получим, что $\Pd^{\prime 2,a} = \begin{pmatrix}
1 \\ \frac{9}{35} \\ \frac{9}{14} \\ \frac{3}{35}
\end{pmatrix}$.	

\subsection*{Пример использования уравнения для пропагации виртуального свидетельства над множеством квантов}
 \addcontentsline{toc}{subsection}{Пример использования уравнения для пропагации виртуального свидетельства над множеством квантов}

Пусть даны два фрагмента знаний над алфавитами $\mathcal{A}_1 = \{x_1, x_2\}$ и $\mathcal{A}_2 = \{x_1, x_3\}$ соответственно. Оба ФЗ построены над множествами квантов. Пусть в первый фрагмент знаний поступило стохастическое свидетельство. Пересчитаем оценки в данном ФЗ с учетом поступившего свидетельства и распространим далее его влияние  из первого ФЗ во второй:
\begin{equation*}
\Pq^{ 1} =  \begin{pmatrix}
p(\overline {x}_2\overline {x}_1)\\ p(\overline {x}_2x_1) \\ p(x_2\overline {x}_1) \\ p(x_2x_1)
\end{pmatrix} = \begin{pmatrix}
0.1 \\ 0.2\\ 0.1\\ 0.6
\end{pmatrix},
\Pq^{2} =  \begin{pmatrix}
p(\overline{x}_3\overline{x}_1) \\ p(\overline{x}_3x_1)\\ p(x_3\overline{x}_1) \\ p(x_3x_1)
\end{pmatrix} = \begin{pmatrix}
0.2 \\ 0.3\\ 0.4\\ 0.1 
\end{pmatrix}.
\end{equation*}

Пусть поступившее свидетельство будет над алфавитом
$\mathcal{A}_{\ev} = \{x_2\}$ со следующим вектором оценок:
\begin{equation*}
\Pq^{\ev} = \begin{pmatrix}  p(\overline {x}_2) \\ p(x_2)	\end{pmatrix} = \begin{pmatrix}
0.4 \\ 0.6
\end{pmatrix}.
\end{equation*}

Воспользуемся уравнением \ref{quantsG} и найдем апостериорные оценки элементов первого ФЗ: $\G = \begin{pmatrix} 1 \\ 0 \end{pmatrix}.$

$\chi^0 = \begin{pmatrix} 0 \end{pmatrix} $, 
$\G\chi^0  = \begin{pmatrix} 1 \\ 0  \end{pmatrix}\begin{pmatrix} 0 \end{pmatrix}  = \begin{pmatrix}  0 \\ 0 \end{pmatrix}$,
$\chi^1 = \begin{pmatrix} 1 \end{pmatrix} $, 
$\G \chi^1  =\begin{pmatrix} 1 \\ 0 \end{pmatrix} \begin{pmatrix} 1 \end{pmatrix}  = \begin{pmatrix}  1 \\ 0 \end{pmatrix}$,
\begin{equation*}
\Selector^{\langle 00, 10 \rangle} = \mathbf{s}^- \otimes \mathbf{s}^\circ =  \begin{pmatrix} 1 \\ 0 \end{pmatrix} \otimes \begin{pmatrix}  1 \\ 1 \end{pmatrix} = \begin{pmatrix}  1 \\ 1 \\ 0 \\ 0 \end{pmatrix},
\end{equation*}
\begin{equation*}
\mathbf{s}^{\langle 10, 00 \rangle} = \mathbf{s}^+ \otimes \mathbf{s}^\circ = 
\begin{pmatrix} 0 \\ 1 \end{pmatrix} \otimes \begin{pmatrix}  1 \\ 1 \end{pmatrix} = \begin{pmatrix}  0 \\ 0 \\ 1 \\ 1 \end{pmatrix}.
\end{equation*}

Подставим в уравнение \ref{quantsG}, вычислим и получим, что $\Pq^{1,a} = \begin{pmatrix}
0.2 \\ 0.4 \\  \frac{2}{35}\\ \frac{12}{35}
\end{pmatrix}$.

Найдем вектор виртуального свидетельства. Алфавит, над которым построено свидетельство $\mathcal{A}_{\ev} = \mathcal{A}_1 \cap \mathcal{A}_2 = \{x_1, x_2\} \cap \{x_1, x_3\} = \{x_1\}$.

Вычислим векторы-селекторы и матрицы проекции $\G_{\mathcal{A}_1, \mathcal{A}_{\ev}}$ и $\G_{\mathcal{A}_2, \mathcal{A}_{\ev}}$. Затем для пропагации виртуального свидетельства воспользуемся уравнением \ref{quantsGlob}: $\G_{\mathcal{A}_1, \mathcal{A}_{\ev}} = \begin{pmatrix} 0 \\ 1\end{pmatrix}.$

$\chi^0 = \begin{pmatrix} 0 \end{pmatrix} $, 
$ \G_{\mathcal{A}_1, \mathcal{A}_{\ev}} \chi^0 =\begin{pmatrix} 0 \\ 1 \end{pmatrix}  \begin{pmatrix} 0 \end{pmatrix} = \begin{pmatrix}  0 \\ 0 \end{pmatrix}$,
$\chi^1 = \begin{pmatrix} 1 \end{pmatrix} $, 
$ \G_{\mathcal{A}_1, \mathcal{A}_{\ev}} \chi^1 = \begin{pmatrix} 0 \\ 1 \end{pmatrix} \begin{pmatrix} 1 \end{pmatrix}  = \begin{pmatrix}  0 \\ 1 \end{pmatrix}$,
\begin{equation*}
\mathbf{s}_1^{\langle 00, 01 \rangle} = \mathbf{s}^\circ \otimes \mathbf{s}^- = \begin{pmatrix} 1 \\ 1 \end{pmatrix} \otimes \begin{pmatrix}  1 \\ 0 \end{pmatrix} = \begin{pmatrix}  1 \\ 0 \\ 1 \\ 0 \end{pmatrix},
\end{equation*}
\begin{equation*}
\mathbf{s}_1^{\langle 01, 00 \rangle} = \mathbf{s}^\circ \otimes \mathbf{s}^+ =
\begin{pmatrix} 1 \\ 1 \end{pmatrix} \otimes \begin{pmatrix}  0 \\ 1 \end{pmatrix} = \begin{pmatrix}  0 \\ 1 \\ 0 \\ 1 \end{pmatrix},
\end{equation*}
\begin{equation*}
\G_{\mathcal{A}_2, \mathcal{A}_{\ev}} = \begin{pmatrix} 0 \\ 1\end{pmatrix},
\end{equation*}

$\chi^0 = \begin{pmatrix} 0 \end{pmatrix} $, 
$ \G_{\mathcal{A}_2, \mathcal{A}_{\ev}} \chi^0 = \begin{pmatrix} 0 \\ 1 \end{pmatrix} \begin{pmatrix} 0 \end{pmatrix}  = \begin{pmatrix}  0 \\ 0 \end{pmatrix}$,
$\chi^1 = \begin{pmatrix} 1 \end{pmatrix} $, 
$\G_{\mathcal{A}_2, \mathcal{A}_{\ev}} \chi^1  = \begin{pmatrix} 0 \\ 1 \end{pmatrix} \begin{pmatrix} 1 \end{pmatrix}  = \begin{pmatrix}  0 \\ 1 \end{pmatrix}$,
\begin{equation*}
\mathbf{s}_2^{\langle 00, 01 \rangle} = \mathbf{s}^- \otimes \mathbf{s}^\circ = \begin{pmatrix} 1 \\ 1 \end{pmatrix} \otimes \begin{pmatrix}  1 \\ 0 \end{pmatrix} = \begin{pmatrix}  1 \\ 0 \\ 1 \\ 0 \end{pmatrix},
\end{equation*}
\begin{equation*}
\mathbf{s}_2^{\langle 01, 00 \rangle} = \mathbf{s}^+ \otimes \mathbf{s}^\circ =
\begin{pmatrix} 1 \\ 1 \end{pmatrix} \otimes \begin{pmatrix}  0 \\ 1 \end{pmatrix} = \begin{pmatrix}  0 \\ 1 \\ 0 \\ 1 \end{pmatrix}.
\end{equation*}


Подставим в формулу \ref{quantsGlob} и получим, что $\Pq^{ 2,a} = \begin{pmatrix}
\frac{3}{35}  \\ \frac{39}{70} \\ \frac{6}{35} \\ \frac{13}{70}
\end{pmatrix}$.

\section*{Приложение B. Публикации по теме работы}
По теме выпускной квалификационной работы были подготовлены следующие публикации:
\begin{enumerate}
\item Золотин А.А., Шляк А.В., Тулупьев А.Л. Пропагация виртуального стохастического свидетельства в алгебраических байесовских сетях: алгоритмы и уравнения // Труды VII Всероссийской Научно-практической Конференции Нечеткие Системы, Мягкие Вычисления и Интеллектуальные Технологии (НСМВИТ-2017). Т. 2. С. 96--107.
\item  Шляк А.В., Золотин А.А. Тулупьев А.Л. Пропагация виртуального свидетельства в алгебраических байесовских сетях: алгоритмы и их особенности // Всероссийская научная конференция по проблемам информатики (СПИСОК-2017). (Санкт-Петербург, 25-27 апреля 2017 г.). Санкт-Петербург: СПбГУ, 2017. С.450--457.
\item Шляк А.В. Задачи матрично-векторной формализации
обработки виртуального свидетельства в алгебраической
байесовской сети // Школа-семинар по искусственному интеллекту: сборник научных трудов. Тверь: ТвГТУ, 2018. C. 76--82.
\end{enumerate}
Также комплекс программ был зарегистрирован в Роспатент:
\begin{enumerate}
\item Шляк А. В., Золотин А.А., Тулупьев А.Л. Algebraic Bayesian Net\-work Virtual Evidence Propagators, Version 01 for CSharp (AlgBN VE Propagators cs.v.01). (Свидетельство). Свид. о гос. регистрации программы для ЭВМ. Рег. № 2018612634(21.02.2018). Роспатент.
\end{enumerate}
\end{document}
   
