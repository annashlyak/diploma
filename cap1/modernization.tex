Так как теория алгебраических байесовских сетей активно развивается, в 2015 году возникла потребность начать разработку комплекса программ AlgBN Math Library~\cite{89, 50}, представляющего собой библиотеку для работы с АБС. Так как алгоритмы ЛВВ приобретают матрично-векторную форму, позволяющую упростить программную реализацию, было принято решение о создании новых библиотек для работы с АБС, использующих современные матрично-векторные алгоритмы. На основе данных библиотек также создается приложение, визуализирующее АБС и позволяющее проводить вывод в сети. Комплекс программ реализован на языке C\# и платформе .Net, являющимися одними из популярнейших современных технологий в разработке программного обеспечения. 

На данный момент в комплексе программ реализованы структуры для хранения фрагментов знаний и АБС, алгоритмы локального логико-вероятностного вывода, а также алгоритмы проверки АБС и ФЗ на непротиворечивость. Реализации алгоритмов глобального логико-вероятностного вывода еще нет.

Что касается теории глобального логико"=вероятностного вывода, существующие алгоритмы обладают рядом недостатков~\cite{70}, к 
которым относится присутствие операций, не относящихся к матрично"=векторным вычислениям, которые усложняют алгоритмы и их программную реализацию.

Таким образом, целью данной выпускной квалификационной работы является автоматизация глобального логико-вероятностного вывода в алгебраических
байесовских сетях, а именно алгоритмов распространения виртуального свидетельства между двумя фрагментами знаний в АБС. 
Для достижения поставленной цели решаются следующие \underline{задачи}:
\begin{enumerate}
        \item  Развить матрично"=векторную формализацию алгоритмов распространения виртуального свидетельства для различных моделей фрагментов знаний с скалярными оценками;
        \item Предложить матрично"=векторную формализацию для функции $\Gind$;
        \item Осуществить интеграцию и реинженеринг алгоритмов локального логико"=вероятностного вывода для альтернативных моделей ФЗ в рамках комплекса программ;
    \item Реализовать алгоритмы распространения виртуального свидетельства в рамках комплекса программ;
\item Провести вычислительные эксперименты по распространению виртуальных свидетельств и написать документацию.
\end{enumerate}