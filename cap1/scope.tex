Рассмотрим предоставляемые АБС инструменты для работы с данными, которые содержат неопределенности.

Для того чтобы охарактеризовать утверждение, эксперт может дать оценку вероятности истинности данного утверждения. Оценка может быть скалярной или интервальной. АБС позволяют задавать и обрабатывать как и скалярные, так и интервальные оценки вероятностей~\cite{184}. В данной работе в основном будут рассматриваться скалярные оценки вероятности истинности.

Задачи обработки данных в АБС решаются с помощью аппарата логико-вероятностного вывода~\cite{121}. Логико-вероятностный вывод делится на локальный и глобальный~\cite{51, 93}. Алгоритмы локального вывода работают с отдельным фрагментом знаний, а глобального --- со всей сетью. В данной работе будет кратко рассмотрен локальный ЛВВ и более подробно глобальный вывод. Логико-вероятностный вывод позволяет нам оценить вероятность пропозициональной формулы, основываясь на имеющихся уже оценках в сети, а также позволяет нам получить новые оценки элементов АБС при поступлении новых обуславливающих данных, называемых свидетельствами~\cite{84, 284}.
 
 Также существует аппарат для поддержки и проверки различных видов непротиворечивости в АБС, для того чтобы оценки вероятностей элементов в ФЗ не противоречили друг другу и были согласованы между собой. Подробнее с ним можно познакомиться в ~\cite{184, 121}.
 