В главе были даны основные определения и понятия, используемые в теории алгебраических байесовских сетей. Представлены определения фрагмента знаний, алгебраической байесовской сети. Рассмотрен логико"=вероятностный вывод в АБС: его виды и задачи, решаемые с помощью ЛВВ, а также основные теоретические результаты, которые ложатся в основу решения задач, рассматриваемых в данной работе.

Стоит отметить, что предложенный в данной главе подход обладает рядом недостатков~\cite{70}, а именно содержит в себе операции, не относящиеся к матрично"=векторным вычислениям. К таким операциям относится функция $\Gind$, для которой ранее не была предложена матрично"=векторная формализация. 

Таким образом, ставится задача улучшить имеющийся результат и заменить функцию $\Gind$ матрично"=векторным аналогом. Также имеющийся результат относится только к одной модели ФЗ и не подходит для двух других моделей, поэтому ставится задача описания алгоритмов глобального ЛВВ для альтернативных моделей фрагмента знаний.