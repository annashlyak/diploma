\subsubsection{Матрично-векторное уравнение пропагации виртуального свидетельства над идеалом конъюнктов}

Перепишем приведенное во второй главе уравнение \ref{conjglob1} для распространения виртуального свидетельства между двумя фрагментами знаний с учетом описанной выше матрично"=векторной формализации функции $\Gind$. Заменим функцию $\Gind$ на матрицу проекции $\G$, тогда уравнение примет следующий вид:
\begin{equation} \label{conjglob2}
    \Pc^a=\sum_{i=0}^{2^{n^\prime}-1}\dfrac{\T^{\langle  \G\chi^{i},\,\G\chi^{2^{n^\prime}-1-i} \rangle }\Pc}{(\mathbf{r}^{\langle  \G\chi^{i},\,\G\chi^{2^{n^\prime}-1-i} \rangle},\Pc)}\In\Pc^{\ev}[i],
\end{equation} 
где \begin{math}\G[i,j] = \begin{cases}
1 \text{, если $\mathcal{A}_2[n -1-i] = \mathcal{A}_{\ev}[n^{\prime} - 1  - j]$,} \\
0 \text{, иначе;}
\end{cases}
\end{math}
\\ $n^{\prime}$ --- размерность алфавита $\mathcal{A}_{\ev}$, $n$ --- размерность алфавита $\mathcal{A}_2$.

\subsubsection{Матрично-векторное уравнение пропагации виртуального свидетельства над идеалом дизъюнктов}

Рассмотрим алгоритм пропагации виртуального свидетельства из одного фрагмента знаний в другой. Воспользуемся теоремой \ref{disq} и  вычислим $\Pd^{\prime \, \ev} = \V\Pd^{\prime}$.

Далее, основываясь на уравнении \ref{dis3} и подставив в него выражение для вычисления вектора вероятностей элементов виртуального свидетельства, а также заменив функцию $\Gind$ матрицей $\G$, получим, что пропагировать виртуальное свидетельство из одного ФЗ с апостериорными оценками в соседний ФЗ можно с помощью следующего уравнения:
\begin{equation}\label{disGlob}
    \Pd^{\prime \, 2,a}=\sum_{i=0}^{2^{n^\prime}-1}\dfrac{\M^{\langle\G\chi^{i},\, \G \chi^{2^{n^\prime}-1-i} \rangle }\Pd^{\prime \, 2}}{(\mathbf{d}^{\langle \G \chi^{i},\,\G\chi^{2^{n^\prime}-1-i} \rangle},\Pd^{\prime \, 2})}\Ln\Pd^{\prime \, \ev}[i],
\end{equation}
где \begin{math}\G[i,j] = \begin{cases}
1 \text{, если $\mathcal{A}_2[n -1-i] = \mathcal{A}_{\ev}[n^{\prime} - 1 - j]$,} \\
0 \text{, иначе;}
\end{cases}
\end{math}
\\ $n^{\prime}$ --- размерность алфавита $\mathcal{A}_{\ev}$, $n$ --- размерность алфавита $\mathcal{A}_2$.

\subsubsection{Матрично-векторное уравнение пропагации виртуального свидетельства над множеством квантов}

Рассмотрим распространение виртуального свидетельства для случая, когда фрагменты знаний построены над множеством пропозиций"=квантов со скалярными оценками. Для того чтобы получить необходимое уравнение, необходимо научиться находить значения оценок вероятностей виртуального свидетельства $\Pq^{\ev}$. Под виртуальным свидетельством будем понимать фрагмент знаний, содержащий оценки вероятностей для квантов, согласованных как и с первым, так и со вторым фрагментом знаний.
 \begin{theorem}
Пропагировать виртуальное свидетельство из одного фрагмента знаний во второй, когда ФЗ построены над множествами пропозиций-квантов, можно с помощью следующего уравнения:
\begin{equation}\label{quantsGlob}
  \Pq^2=\sum_{i=0}^{2^{n^\prime}-1}\dfrac{{\Selector_2^{\langle  \G_{\mathcal{A}_2, \mathcal{A}_{\ev}} \chi^{i},\,\G_{\mathcal{A}_2, \mathcal{A}_{\ev}}\chi^{2^{n^\prime}-1-i} \rangle }}\circ \Pq^2}{(\Selector_2^{\langle  \G_{\mathcal{A}_2, \mathcal{A}_{\ev}}\chi^{i},\, \G_{\mathcal{A}_2, \mathcal{A}_{\ev}}\chi^{2^{n^\prime}-1-i}\rangle},\Pq^2)}(\Selector_1^{\langle  \G_{\mathcal{A}_1, \mathcal{A}_{\ev}}\chi^{i},\,\G_{\mathcal{A}_1, \mathcal{A}_{\ev}}\chi^{2^{n^\prime}-1-i} \rangle},\Pq^1),
\end{equation}
где $\Pq^1$ --- вектор апостериорных оценок первого ФЗ, а $\Pq^2$ --- вектор оценок второго ФЗ, куда нужно пропагировать свидетельство, $n^\prime$ --- мощность алфавита свидетельства $\mathcal{A}^{\ev}$, $m$ --- мощность алфавита $\mathcal{A}_1$, а $n$ --- мощность алфавита $\mathcal{A}_2$,\\
\begin{math}
\G_{\mathcal{A}_2, \mathcal{A}_{\ev}}[i,j] = 
\begin{cases}
1 \text{, если $\mathcal{A}_2[n-1 -i] = \mathcal{A}_{\ev}[n^\prime -1- j]$,} \\
0 \text{, иначе;}
\end{cases}
\end{math}\\
\begin{math}\G_{\mathcal{A}_1, \mathcal{A}_{\ev}}[i,j] = \begin{cases}
1 \text{, если $\mathcal{A}_1[m -1-i] = \mathcal{A}_{\ev}[n^\prime -1- j]$,} \\
0 \text{, иначе;}
\end{cases}
\end{math}
\end{theorem}
\begin{Proof}
Рассмотрим задачу нахождения оценок вероятностей элементов виртуального свидетельства $\Pq^{\ev}$. Можно заметить, что вектор $\Pq^{\ev}$ будет содержать оценки вероятностей для квантов, согласованных как и с первым, так и со вторым фрагментом знаний.

Алфавит, над которым построено свидетельство, можно найти как $\mathcal{A}_{\ev} = \mathcal{A}_1 \cap \mathcal{A}_2$, где $\mathcal{A}_1$ --- алфавит первого фрагмента знаний, $\mathcal{A}_2$ --- алфавит второго фрагмента знаний.

Таким образом, виртуальное свидетельство состоит из множества пропозиций-квантов, построенных над алфавитом $\mathcal{A}_{\ev}$. Для каждого кванта необходимо посчитать его вероятность, используя апостериорные оценки из первого фрагмента знаний. Это является первой задачей апостериорного вывода, если мы рассматриваем квант как детерминированное свидетельство $\evidenceNumbers$. Решение первой задачи апостериорного вывода выглядит следующим образом~\cite{74}:
\begin{equation*}
p(\langle i, j \rangle) = (\Selector^{\evidenceNumbers},\Pq^1).\end{equation*}

Значит, воспользуемся матрицей проекции $\G$ и получим значение вероятности каждого $i$-того кванта виртуального свидетельства:
\begin{equation}\label{quantsVirt}
\Pq^\ev[i] =(\Selector^{\langle  \G_{\mathcal{A}_1, \mathcal{A}_{\ev}} \chi^{i},\, \G_{\mathcal{A}_1, \mathcal{A}_{\ev}} \chi^{2^{n^\prime}-1-i}\rangle},\Pq^1),
\end{equation}
где $n^\prime$ --- мощность алфавита свидетельства $\mathcal{A}^{\ev}$, а $m$ --- мощность алфавита $\mathcal{A}_1$
и $\G_{\mathcal{A}_1, \mathcal{A}_{\ev}}[i,j] = \begin{cases}
1 \text{, если $\mathcal{A}_1[m -1-i] = \mathcal{A}_{\ev}[n^\prime - 1 -j]$,} \\
0 \text{, иначе;}
\end{cases}$

 В нижнем индексе матрицы $\G$ явно укажем алфавиты, участвующие в ее формировании, чтобы в дальнейшем различать различно сформированные матрицы в одном уравнении. У вектора-селектора также добавим нижние индексы $1$ и $2$, указывающие, элементы какого алфавита используются при его построении. 
 
Далее воспользуемся уравнением \ref{quantsG} для решения второй задачи апостериорного вывода для стохастического свидетельства, подставим в него выражение \ref{quantsVirt} и получим искомое уравнение. 
\end{Proof}