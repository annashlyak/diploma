\section{Основные термины, объекты и результаты в теории логико-вероятностного вывода в теории алгебраических байесовских сетей}
\subsection{Введение}

\subsection{Модель ФЗ}
	\subsubsection{ФЗ, построенные над идеалом конъюнктов}
	Алгебраическая байесовская сеть является одной из моделей, которые представляют данные с неопределенностью, а также вероятностной графической моделью. Неопределенность возникает вследствие того, что конъюнктам в узлах фрагментов знаний, из которых состоит АБС, приписывается некоторая вероятность, степень неточности. 
          
           \begin{Def}
            ~\cite{AVS_2011}
                Алфавит А $= \{ x_{i}\}^{n - 1}_{i = 0}$ -- конечное множество атомарных пропозициональных формул (атомов). Введем нумерацию атомов с нуля.
            \end{Def}


           Над атомами из указанного алфавита определим  наборы пропозициональных формул.
            
            \begin{Def}
            ~\cite{AVS_2011}
                Конъюнкт (цепочка конъюнкций) -- это конъюнкция некоторого числа атомарных переменных вида:
                \begin{eqnarray*}
                    x_{i_{1}} \wedge x_{i_{2}} \wedge \cdots \wedge x_{i_{k}}.
                \end{eqnarray*}
            \end{Def}
            
 
            Теперь введем некоторые определения, связанные с видами фрагментов знаний.
        
            \begin{Def}
                ~\cite{AVS_2011}
                Фрагмент знаний (ФЗ) со скалярными оценками -- это пара вида \begin{math}(C,p)\end{math}, где \begin{math}C\end{math} -- идеал конъюнктов, а \begin{math}p\end{math} --  функция из \begin{math}C\end{math} в интервал  
                \begin{math}
                [0;1]
                \end{math}
            \end{Def}
         
            \begin{Def}
            ~\cite{AVS_2011}
                Фрагмент знаний (ФЗ) с интервальными оценками -- это структура вида \begin{math}(C,p)\end{math}, где \begin{math}C\end{math} -- идеал конъюнктов, а \begin{math}p\end{math} --  функция из \begin{math}C\end{math} в множество интервалов вида 
                \begin{eqnarray*}
                    \{[a;b] : a, b \in [0;1], a \leq b\}.
                \end{eqnarray*}
            \end{Def}
           
           Оценки вероятности истинности, приписанные элементам ФЗ, организуются в один или два вектора $\mathbf{P_{c}}$ и $\{\mathbf{P_{c}^{-}}, \mathbf{P_{c}^{+}}\}$ соответственно. Вектор $\mathbf{P_{c}}$ задается при скалярных оценках. Векторы $\{\mathbf{P_{c}^{-}}, \mathbf{P_{c}^{+}}\}$ задаются при интервальных оценках. В последнем случае в один из векторов ($\mathbf{P_{c}^{-}}$) помещаются нижние границы оценок, а в другой ($ \mathbf{P_{c}^{+}}$) -- верхние границы. 
    
    \begin{eqnarray*}
        \mathbf{P_{c}} = \left( \begin{array}{c}
            1 \\
            p(c_{1}) \\
            \vdots \\
            p(c_{2^{n-1}})
        \end{array}  \right)
    \end{eqnarray*}
    \begin{eqnarray*}
        \mathbf{P_{c}^{-}} = \left( \begin{array}{c}
            1 \\
            p^{-}(c_{1}) \\
            \vdots \\
            p^{-}(c_{2^{n -1}})
        \end{array}  \right); 
        \mathbf{P_{c}^{+}} = \left( \begin{array}{c}
            1 \\
            p^{+}(c_{1}) \\
            \vdots \\
            p^{+}(c_{2^{n -1}})
        \end{array}  \right).
    \end{eqnarray*}
    
            Введем определения непротиворечивости и согласуемости ФЗ~\cite{AVS_2011}.
            
               
            \begin{Def}
            ~\cite{AVS_2011}
                 Пусть задан ФЗ со скалярными оценками вероятности \begin{math}(C,p)\end{math}. Мы говорим, что он непротиворечив(согласован), и обозначаем это как \begin{math}\mathrm{Consistent}[(C,p)]\end{math} тогда и только тогда, когда существует вероятность \begin{math}p_{F}\end{math}, заданная над множеством пропозициональных формул \begin{math}F(A)\end{math}(где \begin{math}A\end{math} -- это алфавит, над которым построен ФЗ), такая что 
                \begin{eqnarray*}
                    \forall c\in C \:  p_{F} = p(c).
                \end{eqnarray*}
            \end{Def}
          
            \begin{Def}
            ~\cite{AVS_2011}
                Пусть задан ФЗ с интервальными оценками вероятности \begin{math}(C,p)\end{math}. Мы говорим, что он \it{непротиворечив}(согласован), и обозначаем это как \begin{math}\mathrm{Consistent}[(C,p)]\end{math} тогда и только тогда, когда для любого конъюнкта \begin{math} c \in C \end{math}  и любого \begin{math} \epsilon \in p(c) \end{math} найдется функция \begin{math} p_{c,\epsilon} : C \to [0,1] \end{math}  такая, что \begin{math} p_{c,\epsilon} = \epsilon \end{math} и \begin{math} (C, p_{c,\epsilon}) \end{math} -- непротиворечивый в смысле предыдущего определения.
            \end{Def}
            
            \begin{Def}
            ~\cite{AVS_2011}
                Пусть задан ФЗ с интервальными оценками вероятностей \begin{math}(C,p)\end{math}. Мы говорим, что он \it{согласуем} тогда и только тогда, когда существует непротиворечивый ФЗ с интервальными оценками \begin{math}(C,p')\end{math} такой, что 
                \begin{eqnarray*}
                    \forall c\in C \:  p'(c) \subseteq p(c).
                \end{eqnarray*}
            \end{Def}
            
            
            Введем основные понятия локального апостериорного логико-ве\-роят\-ност\-но\-го вывода в рамках рассматриваемой модели ФЗ. Для формализации локального логико-вероятностного вывода нам необходимо ввести определение математической модели свидетельства в АБС.

            \begin{Def}
            ~\cite{AVS_2011}
                Задача локального априорного вывода состоит в том, чтобы на основе непротиворечивого фрагмента знаний построить оценки истинности пропозициональной формулы, заданной над тем же алфавитом, что и фрагмент знаний.
            \end{Def}

            % Свидетельство  

            \begin{Def}
            ~\cite{AVS_2011}
                Свидетельство -- новые <<обусловливающие>> данные, которые поступили во ФЗ, и с учетом которых нам требуется пересмотреть все (или только некоторые) оценки. Для обозначения свидетельства будут использоваться угловые скобки -- $\langle \cdots \rangle$.
            \end{Def}
            
            Далее оговорим, что из себя представляют задачи апостериорного логико-вероятностного вывода.
            
            % Задачи апостериорного вывода

            \begin{Def}
            ~\cite{AVS_2011}
                {\it Первая задача апостериорного вывода} состоит в том, чтобы оценить вероятность истинности свидетельства при уже заданных оценках вероятности истинности элементов ФЗ.
            \end{Def}
            
            \begin{Def}
            ~\cite{AVS_2011}
                {\it Вторая задача апостериорного вывода} состоит в том, чтобы оценить условные вероятности истинности элементов ФЗ при предположении, что имеет место быть свидетельство.
            \end{Def}  
  
    
    \subsection{Алгоритмизация логико-вероятностного вывода}
        Прежде чем приступить непосредственно к рассмотрению алгоритмов решения задач локального логико-вероятностного вывода, разберем аппарат, осуществляющий проверку непротиворечивости вероятностных оценок ФЗ и некоторые понятия с ним связанные.
        
       
            
            Матрица $\mathbf{I}_{n}$, упомянутая ранее, имеет чёткую структуру~\cite{AVS_2011}, которую можно описать рекуррентно.
             $$ \mathbf{I}_{1} = \left( \begin{array}{cc}
                 1 & -1 \\
                 0 &  1 \\
                \end{array}  \right) ,\cdots,\; \mathbf{I}_{n} = \mathbf{I}_{1} \otimes \mathbf{I}_{n-1} = \mathbf{I}_{1} \otimes \mathbf{I}_{1}^{[n-1]} = \mathbf{I}_{1}^{[n]}. $$
                
            При этом выполняется соотношение:
            $$ \mathbf{P_{q}} =  \mathbf{I}_{n} \times \mathbf{P_{c}}. $$
            
            Здесь $\otimes$ кронекерово произведение матриц. Также используется обратная матрице $\mathbf{I}_{n}$ матрица $\mathbf{J}_{n}$, которая удовлетворяет условию:
            \begin{eqnarray*}
                \mathbf{P_{c}} =  \mathbf{J}_{n} \times \mathbf{P_{q}}, \\
                \mathbf{J}_{1} = \left( \begin{array}{cc}
                     1 & 1 \\
                     0 & 1 \\
                \end{array}  \right),\; \mathbf{J}_{n} = \mathbf{J}_{1} \otimes \mathbf{J}_{n-1} = \mathbf{J}_{1} \otimes \mathbf{J}_{1}^{[n-1]} = \mathbf{J}_{1}^{[n]}.
            \end{eqnarray*}
            
            Сформулируем определения непротиворечивости фрагментов знаний, описанные ранее на матрично-векторном языке~\cite{AVS_2011}.
            

            \begin{Def}
                Пусть задан ФЗ со скалярными оценками вероятностей $(C, \mathbf{P_{c}})$. мы говорим, что он непротиворечив тогда и только тогда, когда $$ \mathbf{I}_{n} \times \mathbf{P_{c}} \geq \mathbf{0}.$$
            \end{Def}

            \begin{Def}
                Пусть задан ФЗ с интервальными оценками вероятностей $(C, \mathbf{P^{-}_{c}}, \mathbf{P^{+}_{c}})$. Мы говорим, что он непротиворечив тогда и только тогда, когда 
                \begin{eqnarray*}
                    \forall i : 1 \leq i \leq 2^{n} -1 \qquad \forall \epsilon : \mathbf{P^{-}_{c}}[i] \leq \epsilon \leq \mathbf{P^{+}_{c}}[i]  \\ \exists \mathbf{P_{c}} : 
                    (\mathbf{P^{-}_{c}} \leq \mathbf{P_{c}} \leq \mathbf{P^{+}_{c}})\; \&\; (\mathbf{P_{c}}[i] = \epsilon)\; \&\; (\mathbf{I}_{n} \times \mathbf{P_{c}} \geq 0).
                \end{eqnarray*}
            \end{Def}

            \begin{Def}
                Пусть задан ФЗ с интервальными оценками вероятностей $(C, \mathbf{P^{-}_{c}}, \mathbf{P^{+}_{c}})$. Мы говорим, что он согласуем тогда и только тогда, когда  существует непротиворечивый ФЗ с интервальными оценками $(C, \mathbf{P^{-}}, \mathbf{P^{+}})$, такой что $ \mathbf{P^{-}_{c}} \leq \mathbf{P^{-}}$ и $ \mathbf{P^{+}} \leq \mathbf{P^{+}_{c}}$.
            \end{Def}
            
            Обоснование приведенных выше рассуждений, дополнительные факты, утверждения и теоремы и  примеры приведены в~\cite{AVS_2011}.
            
       
        \subsubsection{Априорный вывод}
            
            Рассмотрим локальный априорный вывод над двумя видами ФЗ: скалярным и интервальным.
            
            \paragraph{Априорный вывод над скалярным ФЗ.}
                Опишем решение задачи априорного вывода для случая ФЗ со скалярными оценками вероятности.
                Рассмотрим непротиворечивый ФЗ со скалярными оценками $(C, \mathbf{P_{c}})$. Мы можем выразить вероятность произвольной пропозициональной формулы через вероятности квантов, входящих в её СДНФ.
                
                \begin{Def}
                    Пусть $f$ -- произвольная пропозициональная формула, тогда характеристическим вектором формулы $f$ мы будем называть вектор $\chi_{f}$, состоящий из $2^{n}$ элементов, такой, что
                
                    \begin{eqnarray*}
                        \chi_{f} [i]
                        =
                        \left \{ 
                            \begin{array}{c}
                            0,\; $если$ \; q_{i} \notin S_{f};  \\
                            1,\; $если$\;  q_{i} \in S_{f}
                        \end{array}\
                        \right.
                    \end{eqnarray*}
                    , где $q_{i}$ обозначает $i$-й квант, а  $S_{f}$ -- множество квантов, содержащихся в СДНФ ф-лы $f$.
                \end{Def}
                
                
                Рассмотрим пропозициональную формулу $f$, истинность которой требуется оценить. На основе прежде рассмотренных знаний мы можем сделать вывод, что 
                $$ p(f) = (\chi_{f}, \mathbf{P_{q}}).$$
                
                Используя ранее рассмотренные формулы, мы можем перейти от вектора с вероятностями квантов к вектору с вероятностями конъюнктов и получим
                 \begin{eqnarray*}
                    p(f) = (\chi_{f}, \mathbf{I}_{n} \times \mathbf{P_{c}}) = (\mathbf{I}_{n}^ {\mathbf{T}} \times \chi_{f}, \mathbf{P_{c}}).
                \end{eqnarray*}
                Обозначим $\mathbf{L_{f}} = \mathbf{I}_{n}^{{T}} \times \chi_{f}$, тогда 
                \begin{eqnarray*}
                    p(f) = (\mathbf{L_{f}}, \mathbf{P_{c}}).
                \end{eqnarray*}
 
            
            \paragraph{Априорный вывод над интервальным ФЗ.}
                Перейдем к ФЗ с интервальными оценками вероятности. В данном случае однозначно определить вероятность истинности пропозициональной формулы не удастся, но можно найти максимальную и минимальную оценки:
                 \begin{eqnarray*}
                     p^{-}(f) =
                     \displaystyle \min_{\mathfrak{D}\cup \mathfrak{E}}
                     (\mathbf{L_{f}}, \mathbf{P_{c}}), \qquad
                     p(f)^{+} = 
                     \displaystyle \max _{\mathfrak{D}\cup \mathfrak{E}}
                     (\mathbf{L_{f}}, \mathbf{P_{c}}).
                \end{eqnarray*}
 
        
        \subsubsection{Апостериорный вывод}
            В теории АБС рассматриваются три вида свидетельств: детерминированные, стохастические и неточные.
            
            % Виды свидетельств

            \begin{Def}
                {\it Детерминированное свидетельство} -- некоторое означивание цепочки конъюнкций, рассматриваемое в качестве поступившего свидетельства. Данное свидетельство обозначается \begin{math} \langle c_{i}, c_{j} \rangle \end{math}. Здесь \begin{math}c_{i}, c_{j}\end{math} -- конъюнкты, состоящие из атомов, получивших положительные и отрицательные означивания соответственно.
            \end{Def}             

            \begin{Def}
                {\it Стохастическое свидетельство} -- непротиворечивый ФЗ, заданный над \begin{math} C'\end{math} -- подыдеале \begin{math} C\end{math}, со скалярными оценками, которые определяет вероятности истинности элементов соответствующего подыдеала. Данное свидетельство обозначается \begin{math} \langle ( C', \mathbf{P_{c}}) \rangle \end{math}.
            \end{Def}           

            \begin{Def}
                {\it Неточное свидетельство} -- непротиворечивый ФЗ с \begin{math} C'\end{math} -- подыдеале \begin{math} C\end{math}  интервальными оценками, которые определяет вероятности истинности элементов соответствующего подыдеала. Данное свидетельство обозначается \begin{math} \langle ( C', \mathbf{P_{c}^{-}},  \mathbf{P_{c}^{+}}) \rangle \end{math}.
            \end{Def}
            
            Все рассмотренные выше задачи образуют локальный логико-ве\-роят\-ност\-ный вывод в АБС, который был реализован в программном комплексе. Также большинство задач ЛВВ сводится к решению задач линейного программирования(ЛП), то есть поиску минимума и максимума некоторой целевой функции, при заданных органичениях, что позволяет использовать известные методы решения задач ЛП.
            
\paragraph{Детерминированное свидетельство  и ФЗ со скалярными оценками.}
            
 Таким образом получим решения первой задачи апостериорного вывода для ФЗ над идеалом конъюнктов~\cite{tulupex2015rus}:
\begin{equation*}
    \Pcicj = (\rij, \Pc),
\end{equation*}
причем
\begin{math} \rij = \rijTilda_{n - 1} \otimes \rijTilda_{n - 2} \otimes \cdots \otimes \rijTilda_0 \quad
    \rijTilda_{k} = \begin{cases}
   \mathbf{r}^+ \text{, если $x_k$ входит в $c_i$}\\
   \mathbf{r}^- \text{, если $x_k$ входит в $c_j$}\\
   \mathbf{r}^0 \text{, иначе}
 \end{cases}
\end{math},
\begin{math}
    \mathbf{r}^+ = \rPlusMatrix; \qquad \mathbf{r}^- = \rMinusMatrix; \qquad \mathbf{r}^0 = \rZeroMatrix
\end{math}


 %Слова про то, что рассмотрим только эту формулу. остальные можно найти -- там-то.           
                
                
                
\subsection{Чувствительность первой задачи апостериорного вывода}

 Анализ чувствительности является одним из критериев оценки математической модели \cite{WOS5,WOS2}. Данная оценка характеризует степень изменения результата в зависимости от колебания значений входных данных.
    
    В работе \cite{iiti2017} была сформирована следующая задача линейного программирования~(ЗЛП) для оценки чувствительности первой задачи апостериорного вывода:
            \begin{equation*}
            \varepsilon(\Pc^{\circ}) = \max_{\substack{
            \PcHat\In \geq 0, \Pc^{\circ}\In \geq 0,\\ 
            v(\Pc^{\circ}, \PcHat)\leq\delta,\\ 
            \p\evidence=(\redistributor,\Pc^{\circ}),\\ 
            \pHat\evidence=(\redistributor,\PcHat)}}
            {\{\p-\pHat, \pHat-\p\}}.
            \end{equation*}
            
    С целью исследования поведения оценки чувствительности первой задачи апостериорного вывода для детерминированного свидетельства при колебании входных оценок вероятности истинности элементов исходного ФЗ, были проведены вычислительные эксперименты, описанные далее.
    
\subsection{Выводы по главе}