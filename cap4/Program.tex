Для того чтобы реализовать алгоритмы распространения виртуального свидетельства, была расширена функциональность абстракций фрагмента знаний. Соответственно, для интерфейсов каждой модели~(идеал конъюнктов, идеал дизъюнктов, множество квантов) в случае скалярных и интервальных оценок был добавлен статический класс с тремя методами расширения: \begin{enumerate}
    \item IsIntersect --- проверяет, пересекаются ли между собой два фрагмента знаний, и возвращает true, если пересекаются, и false иначе;
    \item IsEqual --- позволяет сравнивать два фрагмента знаний между собой с заданной точностью и возвращает true, если ФЗ эквивалентны, и false иначе;
    \item GetSeparator --- возвращает пересечение двух фрагментов знаний, если пересечение не пусто, а в случае, если ФЗ не пересекаются, возвращается пустой фрагмент знаний.
\end{enumerate}

Всего было создано 6 таких статических классов.

Рассмотрим подробнее работу метода GetSeparator. В случае конъюнктов и дизъюнктов основную работу по выделению элементов сепаратора делает следующий цикл:

\begin{lstlisting}[caption = Цикл для выделения элементов виртуального свидетельства]
for (int index = 0; index <= KPGlobalIndex; index++)
{
    if ((index & KPGlobalIndex) == index)
    {
        if ((index & separatorGlobalIndex) == index)
        {
            probability[separatorIndex] =
                firstKP.GetPointEstimate(probIndex);
            probIndex++;
            separatorIndex++;
        }
        else
        {
            probIndex++;
        }
    }
}
\end{lstlisting}

Стоит отметить, что при программной реализации алфавит удобнее задавать с помощью глобального индекса. Глобальный индекс сепаратора можно получить с помощью операции побитового И глобальных индексов первого и второго фрагментов знаний, по этому принципу работает метод IsIntersect. Заметим, что количество единиц в глобальном индексе соответствует мощности алфавита. 

Данный цикл также использует операцию побитового И. Он проверяет все возможные числа index, которые могут соответствовать элементам вектора вероятностей, это числа от 0 до KPGlobalIndex --- глобального индекса первого ФЗ. Первое условие проверяет это соответствие, второе условие аналогично проверяет, соответствует ли индекс элементу сепаратора. Если условия верны, то значение вероятности сохраняется в массив вероятностей probability. Индексы текущих элементов в векторе вероятностей фрагмента знаний и векторе вероятностей сепаратора задаются с помощью переменных probIndex и separatorIndex соответственно.
 
 Отметим, что данный цикл реализует функциональность матриц перехода  $\Qmatr$ и $\V$, а сами матрицы строить не нужно, потому что, благодаря глобальным индексам, при реализации сразу понятно, какие элементы формируют виртуальное свидетельство.
 
 Так как у модели ФЗ, представленной квантами, виртуальное свидетельство содержит не просто оценки элементов на пересечении, а согласованные оценки, то тут алгоритм построения виртуального свидетельства будет другим. 
 
 Рассмотрим реализацию метода GetSeparator для построения виртуального свидетельства со скалярными оценками.  Вспомогательный метод GetQuantIndexes возвращает вектор из глобальных индексов соответствующих квантов, составляющих ФЗ.  В цикле для каждого индекса кванта из сепаратора считается его вероятность как вероятность детерминированного свидетельства с помощью метода GetEvidenceProbability объекта класса DeterministicScalarQuantLocalPropagator.
 
 \begin{lstlisting}[caption = Метод построения виртуального видетельства со скалярными оценками над квантами]
public static ScalarQuantKP_I GetSeparator(ScalarQuantKP_I firstKP, long separatorGlobalIndex)
{
    var numberOfElements = Convert
        .ToString(separatorGlobalIndex, 2)
        .Count(c => c == '1');
    var numberOfAtoms = (int)Math.Pow(2, numberOfElements);
    var probability = new double[numberOfAtoms];
    var sepAtomGlobalIndexes = GetQuantIndexes(numberOfAtoms, separatorGlobalIndex);
    var propagator = DeterministicScalarQuantLocalPropagator
    	.Instance(firstKP);
            
    for (int index = 0; index < sepAtomGlobalIndexes.Length; index++)
    {
        var binaryQuantKp = new BinaryQuantKP(separatorGlobalIndex);
        binaryQuantKp.SetQuant(sepAtomGlobalIndexes[index]);
        probability[index] = propagator.evidenceProbaility(binaryQuantKp);
    }
    return new ScalarQuantKP(separatorGlobalIndex, probability);
}
\end{lstlisting}

Кроме того, был создан интерфейс IVirtualEvidencePropagator и 6 его реализаций для каждой модели ФЗ в случае скалярных оценок и в случае интервальных оценок.

Дадим краткое описание классов, реализующих алгоритмы распространения виртуального свидетельства для модели ФЗ, представленной идеалом конъюнктов. 

 VirtualEvidenceScalarConjPropagator --- класс, отвечающий за следующие действия: пропагация стохастического виртуального свидетельства и пропагация неточного виртуального свидетельства во фрагмент знаний со скалярными оценками. Соответственно, класс пропагатора содержит 2 метода для двух видов свидетельства. 
 
 Метод PropagateStochasticEvidence используется в случае стохастического свидетельства. Метод принимает 2 параметра: ScalarConjKP\_I firstKP --- первый фрагмент знаний, ScalarConjKP\_I secondKP --- второй фрагмент знаний, в который нужно пропагировать виртуальное свидетельство. В этом методе сначала создается виртуальное свидетельство, далее оно пропагируется с помощью объекта propagator класса  StochasticScalarConjunctsLocalPropagator и его экземплярных методов propagate и getResult. 
 \begin{lstlisting} [caption = Реализация метода PropagateStochasticEvidence]
 public ScalarConjKP_I PropagateStochasticEvidence(ScalarConjKP_I firstKP, ScalarConjKP_I secondKP)
{
    var evidence = firstKP.GetSeparator(secondKP);
    var propagator = new StochasticScalarConjunctsLocalPropagator(secondKP);
    propagator.propagate(evidence);
    return propagator.getResult();
}\end{lstlisting}
 
Второй метод PropagateImpreciseEvidence нужен для пропагирования неточного виртуального свидетельства во второй фрагмент знаний, содержащий скалярные оценки. Пропагирование виртуального свидетельства осуществляется аналогично с помощью методов реализованного ранее класса ImpreciseScalarConjunctsLocalPropagator.
 \begin{lstlisting} [caption = Реализация метода PropagateImpreciseEvidence]
public IntervalConjKP_I PropagateImpreciseEvidence(IntervalConjKP_I firstKP, ScalarConjKP_I secondKP)
{
    var evidence = firstKP.GetSeparator(secondKP);
    var propagator = new ImpreciseScalarConjunctsLocalPropagator();
    propagator.setPattern(secondKP);
    propagator.propagate(evidence);
    return propagator.getResult();
}
\end{lstlisting}

 Второй класс VirtualEvidenceIntervalConjPropagator  отвечает за распространение виртуального  свидетельства из первого фрагмента знаний во второй ФЗ с интервальными оценками. Содержит в себе два аналогичных метода для пропагации свидетельства в двух различных ситуациях.
 
 Метод PropagateStochasticEvidence используется для распространения стохастического виртуального свидетельства в ФЗ с интервальными оценками. Метод принимает два параметра: ScalarConjKP\_I firstKP --- первый фрагмент знаний, IntervalConjKP\_I secondKP --- второй фрагмент знаний, в который нужно пропагировать виртуальное свидетельство.  Внутри метода создается виртуальное свидетельство и далее оно пропагируется с помощью методов объекта propagator ранее реализованного класса  StochasticIntervalConjunctsLocalPropagator.
\begin{lstlisting}[caption = Реализация метода PropagateStochasticEvidence]
public IntervalConjKP_I PropagateStochasticEvidence(ScalarConjKP_I firstKP, IntervalConjKP_I secondKP)
{
    var evidence = firstKP.GetSeparator(secondKP);
    var propagator = new StochasticIntervalConjunctsLocalPropagator();

    propagator.setPattern(secondKP);
    propagator.propagate(evidence);

return propagator.getResult();
}
\end{lstlisting}

Второй метод PropagateImpreciseEvidence нужен для пропагирования неточного виртуального свидетельства из первого ФЗ во второй. Пропагирование виртуального свидетельства осуществляется аналогично с помощью методов объекта propagator ранее реализованного класса ImpreciseIntervalConjunctsLocalPropagator.
\begin{lstlisting} [caption = Реализация метода PropagateImpreciseEvidence]
public IntervalConjKP_I PropagateImpreciseEvidence(IntervalConjKP_I firstKP, IntervalConjKP_I secondKP)
{
    var evidence = firstKP.GetSeparator(secondKP);
    var propagator = new ImpreciseIntervalConjunctsLocalPropagator();

    propagator.setPattern(secondKP);
    propagator.propagate(evidence);

    return propagator.getResult();
}\end{lstlisting}

Аналогичные классы реализованы для фрагментов знаний, построенных над идеалами дизъюнктов: VirtualEvidenceScalarDisjPropagator и VirtualEvidenceIntervalDisjPropagator, а также классы для ФЗ, построенных над множествами квантов: VirtualEvidenceScalarQuantPropagator и VirtualEvidenceIntervalQuantPropagator.  Каждый класс содержит методы PropagateStochasticEvidence и PropagateImpreciseEvidence для пропагации одного из видов виртуального свидетельства.

	   