\underline{Итогами}  квалификационной работы являются программная реализация алгоритма распространения виртуального свидетельства между двумя фрагментами знаний для различных моделей ФЗ в рамках существующего комплекса программ, а также:
\begin{enumerate}
    \item  Матрично"=векторная интерпретация алгоритма пропагации виртуального свидетельства между двумя фрагментами знаний: сформулированы матрично"=векторные уравнения для ФЗ, построенных над множествами квантов и над идеалами дизъюнктов, в частности, доказана теорема формирования матрицы перехода от вектора оценок вероятностей элементов к вектору виртуального свидетельства для модели идеала дизъюнктов и теорема о пропагации виртуального свидетельства между двумя ФЗ, построенными над множествами квантов, предложена матрица перехода от вектора вероятностей дизъюнктов к вектору вероятностей квантов; 
    \item Матрично"=векторная интерпретация функции $\Gind$, в частности, введено понятие характеристического вектора конъюнкта, дизъюнкта, кванта, а также понятие матрицы проекции $\G$;
    \item Внедрение алгоритмов локального ЛВВ для квантов и реинжиниринг существующих алгоритмов ЛВВ для идеалов дизъюнктов;
    \item Тесты, проверяющие корректность работы реализации;
    \item Примеры использования полученной реализации; 
    \item Вычислительные эксперименты, результаты которых согласуются с ожиданиями.
\end{enumerate}

\underline{Рекомендации к использованию.} Предложенная программная реализация является модулем объемлющей математической библиотеки, реализующей алгоритмы логико-вероятностного вывода в АБС и предоставляющей доступ к публичному контракту~\cite{50}. Поэтому реализованный модуль может быть переиспользован также в веб-приложении с целью визуализации алгоритма распространения виртуального свидетельства между двумя фрагментами знаний сети. Также программная реализация может быть переиспользована при дальнейшей реализации алгоритмов глобального логико"=вероятностного вывода, а именно распространения влияния свидетельства во все фрагменты знаний сети. Классы и методы могут быть использованы при проведении различных вычислительных экспериментов с целью изучения различных характеристик модели.

\underline{Перспективы дальнейших исследований.} Полученные теоретические результаты могут быть использованы для исследования устойчивости и чувствительности полученных уравнений и создают фундамент для развития теории глобального логико"=вероятностного вывода, в частности, для формализации алгоритмов распространения виртуального свидетельства между двумя фрагментами знаний с интервальными оценками и исследования распространения виртуального свидетельства для третичной глобальной структуры АБС~\cite{284}.

Разработанные примеры могут быть использованы в методических целях.
