Рассмотрим функцию $\Gind(i,m)$ в контексте математической модели фрагмента знаний, заданного идеалом конъюнктов. Для двух других моделей формализация будет аналогичной, потому что сама функция $\Gind(i,m)$ определяется аналогичным образом~\cite{74, 49}, а алфавиты для всех трех моделей одинаковы.

Напомним, что  $\Gind(i,m)$ --- функция, которая по индексу $m$ наибольшего элемента $C^{\ev}$ в алфавите $\mathcal{A}$ и индексу конъюнкта $i$ в алфавите $\mathcal{A}_{\ev}$ возвращает индекс соответствующего конъюнкта в алфавите $\mathcal{A}$~\cite{84}. Так как индексы конъюнктов пронумерованы согласно строго заданному правилу, то у каждого индекса есть свой строго заданный номер, который можно рассмотреть в двоичной системе счисления или как характеристический вектор из нулей и единиц. Аналогично можно рассмотреть индексы квантов и дизъюнктов. 

\begin{definition}
Характеристический вектор конъюнкта --- это вектор из нулей и единиц, в котором единица на $i$-том месте соответствует единице на $i$-том месте в двоичной записи индекса конъюнкта, а ноль на $i$-том месте --- нулю.
\end{definition}

Будем обозначать характеристический вектор  конъюнкта с индексом $i$ как  $\chi^{i}$. Аналогичным будет обозначение для характеристического вектора дизъюнкта или кванта. Так как в данном разделе рассматриваются ФЗ над идеалами конъюнктов, будем далее рассматривать характеристический вектор конъюнкта.

Можно заметить, что функция $\Gind$ проецирует индекс, соответствующий конъюнкту детерминированного свидетельства, на другой индекс, соответствующий эквивалентному данному конъюнкту ФЗ, построенного над алфавитом $\mathcal{A}$. Вместо индексов можно взять соответствующие характеристические векторы и построить проекцию одного на другой.


Пусть $\chi^{i}$ --- характеристический вектор $i$-того конъюнкта свидетельства в алфавите  $\mathcal{A}_{\ev}$, и  пусть $\chi^{j}$ --- характеристический вектор соответствующего ему $j$-того конъюнкта в алфавите  $\mathcal{A}$. $m$ --- мощность алфавита $\mathcal{A}_{\ev}$, $n$ --- мощность алфавита $\mathcal{A}$.

Тогда $\Gind(i,m)$ можно заменить матрицей проекции вектора $\chi^{i}$ на $\chi^{j}$ по следующему правилу:
\begin{equation}\label{G}
\G[i,j] = \begin{cases}
1 \text{, если $\mathcal{A}[n -1-i] = \mathcal{A}_{\ev}[m - 1 -j]$,} \\
0 \text{, иначе};
\end{cases} 
\end{equation}

Такая матрица будет размера $n \times m$, где $n$ --- мощность алфавита $\mathcal{A}$ и $m$ --- мощность алфавита $\mathcal{A}_{\ev}$.

Чтобы получить характеристический вектор конъюнкта $\chi^{j}$ в алфавите $\mathcal{A}$, нужно будет домножить вектор $\chi^{i}$ слева на матрицу $\G$: $\chi^{j} = \G\chi^{i}$.

По построению матрица $\G$ будет содержать в себе нулевые строки, соответствующие элементам алфавита $\mathcal{A}$, которые не входят в алфавит $\mathcal{A}_{\ev}$. Единица на j-той позиции в ненулевой строке i соответствует атому $x_i \in \mathcal{A}$, эквивалентному атому $x_j \in \mathcal{A}_{\ev}$. Умножение нулевых строк на характеристический вектор будет давать нулевой элемент в результирующем векторе,  умножение ненулевых строк даст единицу, если элемент алфавита входит в конъюнкт, и ноль иначе. Таким образом, матрица $\G$ проецирует характеристический вектор конъюнкта, построенного над алфавитом $\mathcal{A}_{\ev}$ на характеристический вектор конъюнкта над алфавитом $\mathcal{A}$.

\subsubsection{Пример использования матрично-векторной интерпретации функции $\Gind$}
Приведем поясняющий пример. Пусть $\mathcal{A} =  \{x_1, x_2, x_3, x_4\}$ и $\mathcal{A}_{\ev} =  \{x_2, x_4\}$. Построим матрицу проекции $\G$ по формуле \ref{G}:

\begin{math}
\G[i,j] = \begin{cases}
1 \text{, если $\mathcal{A}[n -1-i] = \mathcal{A}_{\ev}[m - 1 - j]$,} \\
0 \text{, иначе};
\end{cases}
= \begin{pmatrix}
1 & 0 \\ 0 & 0 \\ 0 & 1\\ 0 & 0
\end{pmatrix}.
\end{math}

Пусть $\chi^{1} = \begin{pmatrix} 0 \\ 1 \end{pmatrix}$ и  соответствует конъюнкту $x_2$ свидетельства. Найдем характеристический вектор индекса конъюнкта $x_2$ в алфавите $\mathcal{A}$, домножив вектор $\chi^{1}$ на матрицу $\G$:

\begin{math}
\begin{pmatrix}
1 & 0 \\ 0 & 0 \\ 0 & 1\\ 0 & 0
\end{pmatrix}
\begin{pmatrix} 0 \\ 1 \end{pmatrix} 
= \begin{pmatrix} 0\\ 0 \\  1 \\ 0 \end{pmatrix}.
\end{math}

Видно, что искомый индекс получен.

\subsubsection{Уравнения для решения второй задачи апостериорного вывода}


Заменим в уравнениях \ref{conj1} и \ref{quants1}, приведенных во второй главе, функцию $\Gind$ на матрицу проекции $\G$. 

Уравнение \ref{conj1} для конъюнктов приобретет следующий вид:
\begin{equation} \label{conjG}
    \Pc^a=\sum_{i=0}^{2^{n^\prime}-1}\dfrac{\T^{\langle \G\chi^{i} ,\,\G\chi^{2^{n^\prime}-1-i} \rangle }\Pc}{(\mathbf{r}^{\langle  \G\chi^{i},\,\G\chi^{2^{n^\prime}-1-i} \rangle},\Pc)}\In\Pc^{\ev}[i].
\end{equation} 

Уравнение \ref{quants1} для квантов будет следующим:
\begin{equation} \label{quantsG}
 \Pq^{\evidenceNumbers}=\sum_{i=0}^{2^{n^\prime}-1}\dfrac{{\Selector^{\langle \G\chi^{i} ,\, \G\chi^{2^{n^\prime}-1-i}\rangle }}\circ \Pq}{(\Selector^{\langle\G \chi^{i} ,\, \G\chi^{2^{n^\prime}-1-i}\rangle},\Pq)}\Pq^{\ev}[i].
\end{equation}

Уравнение \ref{dis2} для дизъюнктов заменится следующим:
\begin{equation}\label{dis3}
 \Pd^{a, \prime} =\sum_{i=0}^{2^{n^\prime}-1}\dfrac{\M^{\langle  \G\chi^{i},\,\G\chi^{2^{n^\prime}-1-i} \rangle  }\Pd^{\prime}}{(\mathbf{d}^{\langle \G\chi^{i} ,\, \G\chi^{2^{n^\prime}-1-i}\rangle },\Pd^{\prime})}\Ln\Pd^{\prime \, \ev}[i].
\end{equation}