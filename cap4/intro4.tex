В предыдущих главах были описаны основные имеющиеся и полученные теоретические результаты, а в данной главе будет рассмотрена программная реализация. Будет описана архитектура существующего комплекса программ, представлены основные классы и методы для реализации алгоритмов глобального логико-вероятностного вывода для различных моделей ФЗ, а также приведены примеры работы программы.

Полученная в ходе работы программная реализация разработана на языке C\#, в качестве среды разработки использовалась среда JetBrains Rider~\cite{68}, которая обладает меньшей функциональностью в отличие от Visual Studio, но отличается удобством и быстротой работы. Для совместной работы использовался репозиторий на BitBucket~\cite{111} и система контроля версий Git. Для тестирования использовалась библиотека NUnit~\cite{67}, а для работы с матрицами использовалась библиотека Math.Net Numerics~\cite{66}.