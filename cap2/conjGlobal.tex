Рассмотрим связную ациклическую алгебраическую сеть~\cite{284} со скалярными оценками во всех фрагментах знаний. Предположим, что в один из фрагментов знаний поступило стохастическое свидетельство. Задачей глобального апостериорного вывода является распространение влияния этого свидетельства~(пропагация свидетельства) во все фрагменты знаний сети. Схема глобального апостериорного вывода подробно описана в~\cite{93}.
 Алгоритм пропагации состоит из 3 шагов: 
\begin{enumerate}
	\item  Пропагация свидетельства во фрагмент знаний, в который оно пришло, и оценка апостериорных вероятностей его элементов;
	\item Формирование виртуального свидетельства;
	\item Пропагация виртуального свидетельства в соседний фрагмент знаний.
\end{enumerate}

Аналогичным образом свидетельство пропагируется далее, пока не будут переозначены оценки элементов во всех фрагментах знаний.

Виртуальным свидетельством называется пересечение двух фрагментов знаний~(сепаратор), также являющийся фрагментом знаний~\cite{51, 93}. Так как после переозначивания оценок в начальном фрагменте знаний, конъюнкты, принадлежащие сепаратору, имеют новые оценки, потому что принадлежат начальному фрагменту знаний, но с другой стороны они также принадлежат соседнему фрагменту знаний и имеют другие оценки, то сепаратор можно рассмотреть как новую информацию, поступившую в соседний фрагмент знаний. 

Таким образом, на втором шаге алгоритма из вектора вероятностей начального фрагмента знаний необходимо выделить вектор значений, принадлежащих обоим фрагментам знаний, и принять его за новое свидетельство, которое на третьем шаге пропагируется в соседний фрагмент знаний, оценки которого необходимо переозначить.

Для того чтобы выделить необходимые оценки с помощью матрично-векторных вычислений, требуется сформировать матрицу перехода от вектора оценок вероятностей элементов ФЗ к вектору оценок вероятностей виртуального свидетельства.

Виртуальные свидетельства могут быть двух видов: стохастическое и неточное, потому что пересечение двух фрагментов знаний всегда является фрагментом знаний, но может содержать как и скалярные, так и интервальные оценки вероятностей.

Остановимся подробнее на втором шаге алгоритма и рассмотрим формирование и распространение виртуального свидетельства из одного фрагмента знаний в другой, когда они построены над идеалами конъюнктов. Считаем, что вектор $\Pc^{1, a}$ содержит апостериорные оценки вероятностей, потому что в него ранее было распространено влияние какого-то свидетельства. Распространим влияние этого свидетельства в соседний фрагмент знаний с оценками $\Pc^2$.

Рассмотрим, как выглядит матрица перехода $\Qmatr$ от вектора оценок вероятностей элементов ФЗ к вектору оценок вероятностей виртуального свидетельства. Матрица будет размерности $m \times n$~($m$ --- длина вектора $\Pc^{\ev}$,  $n$ --- длина вектора $\Pc^{1, a}$). Элементы матрицы определяются по следующему правилу~\cite{51, 9, 70}:
\begin{equation*}
    \Qmatr[i,j] = 
    \begin{cases}
        1 \text{, если $\Pc^{1, a}[j] = \Pc^{\ev}[i]$,}\\
        0 \text{, иначе;}\\
    \end{cases}
\end{equation*}

Под $\Pc^{1, a}[i]$ и $\Pc^{\ev}[i]$ подразумеваются сами конъюнкты, а не значения вероятностей. Затем нужно умножить матрицу на $\Pq^{1, a}$. Таким образом, $\Pc^{\ev} = \Qmatr\Jn\Pq^{1, a} = \Qmatr\Pc^{1, a}$.

Алфавит, над которым построено свидетельство, можно найти как $\mathcal{A}_{\ev} = \mathcal{A}_1 \cap \mathcal{A}_2$, где $\mathcal{A}_1$ --- алфавит первого фрагмента знаний, $\mathcal{A}_2$ --- алфавит второго фрагмента знаний.

Таким образом, пропагировать виртуальное свидетельство из одного ФЗ в другой можно с помощью следующего уравнения~\cite{51}:
\begin{equation} \label{conjglob1}
    \Pc^{2,a}=\sum_{i=0}^{2^{n^\prime}-1}\dfrac{\T^{\langle \Gind(i,m),\,\Gind(2^{n^\prime}-1-i,m)\rangle }\Pc^2}{(\mathbf{r}^{\langle \Gind(i,m),\,\Gind(2^{n^\prime}-1-i,m)\rangle},\Pc^2)}\In\Pc^{\ev}[i],
\end{equation}
где $\Pc^{\ev} = \Qmatr\Pc^{1, a}$, $\Pc^{1, a}$ и $\Pc^2$ --- векторы вероятностей элементов идеалов конъюнктов первого и второго фрагментов знаний, $\Pc^{2,a}$ --- вектор апостериорных вероятностей элементов идеала конъюнктов второго фрагмента знаний.

Матрицу перехода $\Qmatr$ можно также сформировать с помощью матрично"=векторных операций, а именно через кронекерово произведение:

\begin{theorem}[\cite{70}]
Вектор оценок виртуального свидетельства $\Pc^{\ev}$ можно вычислить как $\Pc^{\ev} = \Qmatr\Pc^{a,1}$, где $\Qmatr = \QijTild_{n-1} \otimes \QijTild_{n-2} \otimes ... \otimes \QijTild_{0}$,
 $\QijTild_k = \begin{cases}
\Qmatr^+ \text{, если $x_k$ входит в $\mathcal{A}^{\ev}$,} \\
\Qmatr^- \text{, иначе;}
\end{cases}$,
\begin{math}
    \Qmatr^+ = \begin{pmatrix} 1 & 0 \\ 0 & 1 \end{pmatrix},
    \Qmatr^- = \begin{pmatrix} 1 & 0 \end{pmatrix},
\end{math}\\ 
$x_k$ --- $k$-тый элемент в $\mathcal{A}_1$,  $\mathcal{A}_1$ --- алфавит, над которым построен $\Pc^{1,a}$, $\mathcal{A}^{\ev}$ --- алфавит, над которым построен $\Pc^{\ev}$.
\end{theorem}

