\section{Результаты теоретического представления данных и основные алгоритмы в библиотеке ЛВВ}
\subsection{Введение}



\subsection{Описание библиотеки, все части}



\subsection{Алгоритм для проведения апостериорного вывода для детерминированного свидетельства}
        
        Рассмотрим алгоритм решения первой и второй задачи апостериорного вывода с использованием вектора-редистрибъютера \cite{AAZ_matr_vect} для детерминированного свидетельства $\evidence$, которое также может быть записано таким образом $\evidenceNumbers$, где $i$ – индекс положительной части, а $j$ – индекс отрицательной. Стоит заметить, что для алфавита, над которым строится свидетельство, индекс положительной и отрицательной части является по своей сути маской, которая показывает, какие атому входят в соответствующую часть свидетельства, а какие – нет. В индексе, если атому соответствует 1, то этот атом входит в конъюнкцию атомов определенной части свидетельства, если 0, то нет. 
        
        \begin{lstlisting}[label=list1,caption={Функция GetRedistributerVector},escapeinside={(*}{*)}]
            input:  (*$A = \{x_{n-1} , ... , x_{0}\}, c_i, c_j$*)
            output:  (*$r^{\evidenceNumbers}$*)
            1:   function GenerateRedistributerVector
            2:   	(*$r^{o} = (1,1)$*)
            3:   	(*$r^{+} = (0,1)$*)
            4:   	(*$r^{-} = (1,-1)$*)
            5:	    (*$r^{\evidenceNumbers} = 1$*)
            6:   	foreach (atom in A)
            7:	      	if (atom in (*$c_i$*))
            8:			  (*$r^{\evidenceNumbers}$*)= kroneckerV((*$r^{\evidenceNumbers}$*), (*$r^{+}$*))
            9:	      	else if (atom in (*$c_j$*))
            10:			  (*$r^{\evidenceNumbers}$*)= kroneckerV((*$r^{\evidenceNumbers}$*), (*$r^{-}$*))
            11:	      	else
            12:			  (*$r^{\evidenceNumbers}$*)= kroneckerV((*$r^{\evidenceNumbers}$*), (*$r^{o}$*))
            13:	 return (*$r^{\evidenceNumbers}$*)
        \end{lstlisting}
        
        На листинге \ref{list1} представлена вспомогательная функция GetRedistributer\-Vector для вычисления вектора-редистрибъютера для рассматриваемого свидетельства, на вход которой подается алфавит, над которым задан исходный ФЗ, положительно означенная и отрицательно означенная части свидетельства соответственно. Функция возвращает вычисленный вектор. Здесь и далее запись «atom in A» означает, что atom входит в алфавит A, а запись «atom in » -- atom входит в соответствующую конъюнкцию. Функция kronekerV(t,r) вычисляет кронекерово произведение векторов t и r.
        
        \begin{lstlisting}[label=list2,caption={Функция FirstPosteriorTaskSolution},escapeinside={(*}{*)}]
            input:  (*$A = \{x_{n-1} , ... , x_{0}\}, c_i, c_j, \Pc $*)	
            output:  (*$p(\evidence)$*) 
            1:   function FirstPosteriorTaskSolution
            2:   	  (*$r^{\evidenceNumbers}$*)= GenerateRedistributerVector(A, (*$c_i $*) , (*$c_j $*) )
            3:	    res = 0
            4:   	foreach ( (*$p_i $*) in (*$\Pc $*))
            5:	      foreach ( (*$r_i $*) in (*$r^{\evidenceNumbers}$*) )
            6:		res =  (*$p_i $*) *  (*$r_i $*) 
            7:	 return res
        \end{lstlisting}
        
        На листинге \ref{list2} представлена функция FirstPosteriorTaskSolution для решения первой задачи апостериорного вывода для рассматриваемого свидетельства и исходного ФЗ. На вход поступает алфавит, над которым задан исходный ФЗ, положительно означенная и отрицатель-но означенная части свидетельства и вектор вероятностей истинности элементов ФЗ. Функция возвращает вычисленное значение вероятно-сти.
        
        \begin{lstlisting}[label=list3,caption={Функция GetMatrixT},escapeinside={(*}{*)}]
            input:  (*$A = \{x_{n-1} , ... , x_{0}\}, c_i, c_j$*)	
            output: (*$T^{\evidenceNumbers}$*) 
            1:   function GetMatrixT
            2:   	(*$T^{o} = \left( \begin{array}{cc}
                                1& 0\\  0&1\\  \end{array}  \right)$*)
            3:   	(*$T^{+} = \left( \begin{array}{cc}
                                0& 1\\  0&1\\  \end{array}   \right)$*)
            4:   	(*$T^{-} = \left( \begin{array}{cc}
                                1& -1\\  0&0\\  \end{array}   \right)$*)
            5:	    (*$T^{\evidenceNumbers} = 1$*) 
            6:   	foreach (atom in A)
            7:	      	if (atom in (*$c_i$*))
            8:			 (*$T^{\evidenceNumbers}$*) = kroneckerM((*$T^{\evidenceNumbers}$*), (*$T^{+}$*))
            9:	      	else if (atom in (*$c_j$*))
            10:			 (*$T^{\evidenceNumbers}$*) = kroneckerM((*$T^{\evidenceNumbers}$*), (*$T^{-}$*))
            11:	      	else 
            12:			 (*$T^{\evidenceNumbers}$*) = kroneckerM((*$T^{\evidenceNumbers}$*), (*$T^{o}$*))
            13:	 return (*$T^{\evidenceNumbers}$*)
        \end{lstlisting}
        
        В листинге \ref{list3} приведен псевдокод вспомогательной функции Get\-MatrixT для вычисления матрицы T для рассматриваемого сви\-де\-тель\-ства, на вход которой подается алфавит, над которым задан исходный ФЗ, положительно означенная и отрицательно означенная части свидетельства соответственно. Функция возвращает вычисленную матрицу. Функция kronekerM(T, R) вычисляет кронекерово произведение матриц T и R.
        
        \begin{lstlisting}[label=list4,caption={Функция SecondPosteriorTaskSolution},escapeinside={(*}{*)}]
            input:  (*$A = \{x_{n-1} , ... , x_{0}\}, c_i, c_j, \Pc $*)		
            output:  (*$P_{c}^{\evidenceNumbers}$*)
            1:   function SecondPosteriorTaskSolution
            2:   	 (*$r^{\evidenceNumbers}$*)= GenerateRedistributerVector(A, (*$c_i $*) , (*$c_j $*) )
            3:		 (*$p(\evidence)$*) = FirstPosteriorTaskSolution(A, (*$c_i, c_j, \Pc$*) )
            4:		n = length((*$P_{c}$*))
            5:   	foreach (k in 0:(n-1))
            6:	      	 (*$P_{c}^{\evidenceNumbers}[k] = (r^{\evidenceNumbers}[k] * \Pc [k]) / p(\evidence)$*)
            7:	 return  (*$P_{c}^{\evidenceNumbers}$*)
        \end{lstlisting}
        
        На листинге \ref{list4} представлена функция SecondPosteriorTaskSolution для решения второй задачи апостериорного вывода для рассматриваемого свидетельства и исходного ФЗ. На вход поступает алфавит, над которым задан исходный ФЗ, положительно означенная и отрицательно означенная части свидетельства и вектор вероятностей истинности элементов ФЗ. Функция возвращает вычисленный вектор условных вероятностей истинности конъюнктов ФЗ.



\subsection{ЧТО-ТО}
\subsection{Выводы по главе}