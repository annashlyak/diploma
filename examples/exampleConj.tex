\subsection*{Пример использования уравнения для пропагации виртуального свидетельства над идеалом конъюнктов}
 \addcontentsline{toc}{subsection}{Пример использования уравнения для пропагации виртуального свидетельства над идеалами конъюнктов}
Пусть даны два фрагмента знаний над алфавитами $\mathcal{A}_1 = \{x_1, x_2\}$ и  $\mathcal{A}_2 = \{x_1, x_3\}$, и пусть в первый фрагмент знаний поступило стохастическое свидетельство. Пересчитаем оценки в данных ФЗ с учетом поступившего свидетельства, а затем распространим его влияние во второй ФЗ:
\begin{equation*}
\Pc^1 =  \begin{pmatrix}
1 \\ p(x_1) \\ p(x_2) \\ p(x_2x_1)
\end{pmatrix} = \begin{pmatrix}
1 \\ 0.8\\ 0.7\\ 0.6
\end{pmatrix}, 
\Pc^2 =  \begin{pmatrix}
1 \\ p(x_1) \\ p(x_3) \\ p(x_3x_1)
\end{pmatrix} = \begin{pmatrix}
1 \\ 0.4\\ 0.5\\ 0.1 
\end{pmatrix}.
\end{equation*}

Пусть поступившее свидетельство будет над алфавитом
$\mathcal{A}_{\ev} = \{x_2\}$:
\begin{equation*}
\Pc^{\ev} = \begin{pmatrix}  1 \\ p(x_2)	\end{pmatrix} = \begin{pmatrix}
1 \\ 0.4
\end{pmatrix}.
\end{equation*}

Воспользуемся уравнением \ref{conjG}. Посчитаем матрицу $\G$ по формуле \ref{G}:
$\G = \begin{pmatrix} 1 \\ 0 \end{pmatrix}.$

$\chi^0 = \begin{pmatrix} 0 \end{pmatrix}, 
\G \chi^0 =  \begin{pmatrix} 1 \\ 0  \end{pmatrix} \begin{pmatrix} 0 \end{pmatrix} = \begin{pmatrix}  0 \\ 0 \end{pmatrix}$,
$\chi^1 = \begin{pmatrix} 1 \end{pmatrix} $, 
$ \G \chi^1 =  \begin{pmatrix} 1 \\ 0 \end{pmatrix} \begin{pmatrix} 1 \end{pmatrix} = \begin{pmatrix}  1 \\ 0 \end{pmatrix}$,

\begin{equation*}
\T^{\langle 00, 10 \rangle} = \T^- \otimes \T^\circ = 
\begin{pmatrix*}[r] 1 & -1 \\ 0 & 0 \end{pmatrix*} \otimes
\begin{pmatrix} 1 & 0 \\ 0 & 1\end{pmatrix} = 
\begin{pmatrix*}[r] 1 & 0 & -1 & 0 \\ 0 & 1 & 0 & -1 \\ 0 & 0 & 0 & 0 \\ 0 & 0 & 0 & 0 \end{pmatrix*},
\end{equation*}
\begin{equation*}
\mathbf{r}^{\langle 00, 10 \rangle} = \mathbf{r}^- \otimes \mathbf{r}^\circ = \begin{pmatrix*}[r] 1 \\ -1 \end{pmatrix*} \otimes \begin{pmatrix}  1 \\ 0 \end{pmatrix} = \begin{pmatrix*}[r]  1 \\ 0 \\ -1 \\ 0 \end{pmatrix*},
\end{equation*}
\begin{equation*}
\T^{\langle 10, 00 \rangle} = \T^+ \otimes \T^\circ = 
\begin{pmatrix} 0 & 1 \\ 0 & 1 \end{pmatrix} \otimes
\begin{pmatrix} 1 &0 \\ 0 & 1\end{pmatrix}  =
\begin{pmatrix} 0 &0 & 1 & 0 \\ 0 & 0 & 0 & 1 \\
0 & 0 & 1 & 0\\ 0 & 0 & 0 & 1 \end{pmatrix},
\end{equation*}
\begin{equation*}
\mathbf{r}^{\langle 10, 00 \rangle} = \mathbf{r}^+ \otimes \mathbf{r}^\circ = \begin{pmatrix} 0 \\ 1 \end{pmatrix} \otimes \begin{pmatrix}  1 \\ 0 \end{pmatrix} = \begin{pmatrix}  0 \\ 0 \\ 1 \\ 0 \end{pmatrix},
\end{equation*}
\begin{equation*}
\mathbf{I}_1\Pc^{\ev} = \begin{pmatrix*}[r] 1 & -1 \\ 0 & 1 \end{pmatrix*} \begin{pmatrix}1 \\ 0.4\end{pmatrix} = \begin{pmatrix}
0.6 \\ 0.4 \end{pmatrix}.
\end{equation*}

Подставим в формулу \ref{conjG}, посчитаем и получим, что $\Pc^{1,a} = \begin{pmatrix}
1 \\  \frac{26}{35} \\ 0.4 \\ \frac{12}{35}
\end{pmatrix}$.

Теперь найдем вектор виртуального свидетельства. Виртуальное свидетельство будет построено над алфавитом 
$\mathcal{A}_{\ev} = \mathcal{A}_1 \cap \mathcal{A}_2 = \{x_1, x_2\} \cap \{x_1, x_3\} = \{x_1\}$.
\begin{equation*}
\Qmatr = \Qmatr^- \otimes \Qmatr^+ =  \begin{pmatrix} 1 & 0  \end{pmatrix} \otimes
  \begin{pmatrix} 1 & 0 \\ 0 & 1 \end{pmatrix} = \begin{pmatrix}
1 & 0  & 0 & 0 \\ 0 & 1 & 0 & 0
\end{pmatrix},
\end{equation*}
\begin{equation*}
\Pc^{\ev} = \Qmatr\Pc^{1, a} =  \begin{pmatrix}
1 & 0  & 0 & 0 \\ 0 & 1  & 0 & 0
\end{pmatrix} \begin{pmatrix}
1 \\  \frac{26}{35} \\ 0.4 \\ \frac{12}{35}
\end{pmatrix} = \begin{pmatrix}
1 \\ \frac{26}{35}
\end{pmatrix}.
\end{equation*} 

Воспользуемся уравнением \ref{conjglob2} для пропагации виртуального свидетельства:
$\G = \begin{pmatrix} 0 \\ 1\end{pmatrix}.$

$\chi^0 = \begin{pmatrix} 0 \end{pmatrix} $, 
$ \G \chi^0 = \begin{pmatrix} 0 \\ 1 \end{pmatrix} \begin{pmatrix} 0 \end{pmatrix} = \begin{pmatrix}  0 \\ 0 \end{pmatrix}$,
$\chi^1 = \begin{pmatrix} 1 \end{pmatrix} $, 
$ \G \chi^1 =  \begin{pmatrix} 0 \\ 1 \end{pmatrix}\begin{pmatrix} 1 \end{pmatrix} = \begin{pmatrix}  0 \\ 1 \end{pmatrix}$,
\begin{equation*}
\T^{\langle 00, 01 \rangle} = \T^\circ \otimes \T^- = 
\begin{pmatrix} 1 & 0 \\ 0 & 1\end{pmatrix} \otimes
\begin{pmatrix*}[r] 1 & -1 \\ 0 & 0 \end{pmatrix*} = 
\begin{pmatrix*}[r] 1 & -1 & 0  & 0 \\ 0 & 0 & 0 & 0 \\ 0 & 0 & 1 & -1 \\ 0 & 0 & 0 & 0 \end{pmatrix*},
\end{equation*}
\begin{equation*}
\mathbf{r}^{\langle 00, 01 \rangle} = \mathbf{r}^\circ \otimes \mathbf{r}^- = \begin{pmatrix} 1 \\ 0 \end{pmatrix} \otimes \begin{pmatrix*}[r]  1 \\ -1 \end{pmatrix*} = \begin{pmatrix*}[r] 1 \\ -1 \\ 0 \\ 0 \end{pmatrix*},
\end{equation*}
\begin{equation*}
\T^{\langle 01, 00 \rangle} = \T^\circ \otimes \T^+ = 
\begin{pmatrix} 1 &0 \\ 0 & 1\end{pmatrix}  \otimes
\begin{pmatrix} 0 & 1 \\ 0 & 1 \end{pmatrix} =
\begin{pmatrix} 0 &1 & 0 & 0 \\ 0 & 1 & 0 & 0 \\
0 & 0 & 0 & 1\\ 0 & 0 & 0 & 1 \end{pmatrix},
\end{equation*}
\begin{equation*}
\mathbf{r}^{\langle 01, 00 \rangle} = \mathbf{r}^\circ \otimes \mathbf{r}^+ = \begin{pmatrix} 1 \\ 0 \end{pmatrix} \otimes \begin{pmatrix}  0 \\ 1 \end{pmatrix} = \begin{pmatrix}  0 \\ 1 \\ 0 \\ 0 \end{pmatrix},
\end{equation*}
\begin{equation*}
\mathbf{I}_1\Pc^{\ev} = \begin{pmatrix*}[r] 1 & -1 \\ 0 & 1 \end{pmatrix*} \begin{pmatrix}1 \\  \frac{26}{35} \end{pmatrix} = \begin{pmatrix}
\frac{9}{35} \\ \frac{26}{35} \end{pmatrix}.
\end{equation*}

Подставим в формулу выше, посчитаем и получим, что $\Pc^{2,a} = \begin{pmatrix}
1 \\ \frac{26}{35} \\ \frac{5}{14} \\ \frac{13}{70}
\end{pmatrix}$.
