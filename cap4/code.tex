Приведем примеры использования методов реализаций интерфейса IVirtualEvidencePropagator. Рассмотрим примеры, когда фрагменты знаний содержат скалярные оценки, и в первый фрагмент знаний приходит стохастическое свидетельство. Распространим влияние свидетельства в первый ФЗ, сформируем виртуальное свидетельство и пропагируем его во второй ФЗ. Для остальных возможных ситуаций использование будет аналогичным.
\subsubsection{Пример пропагации стохастического виртуального свидетельства в ФЗ с скалярными оценками над идеалом конъюнктов}
Пусть даны два фрагмента знаний над алфавитами $\mathcal{A}_1 = \{x_1, x_2\}$ и  $\mathcal{A}_2 = \{x_2, x_3\}$ и в первый фрагмент знаний поступило стохастическое свидетельство над алфавитом $\mathcal{A}_{\ev} = \{x_1\}$. Пусть ФЗ и свидетельство имеют следующие оценки вероятностей: \\ 
\begin{equation*}
\Pc^1 =  \begin{pmatrix}
1 \\ p(x_1) \\ p(x_2) \\ p(x_2x_1)
\end{pmatrix} = \begin{pmatrix}
1 \\ 0.5\\ 0.7\\ 0.3
\end{pmatrix}, 
\Pc^2 =  \begin{pmatrix}
1 \\ p(x_2) \\ p(x_3) \\ p(x_3x_2)
\end{pmatrix} = \begin{pmatrix}
1 \\ 0.3\\ 0.7\\ 0.1 
\end{pmatrix},
\end{equation*}
\begin{equation*}
\Pc^{\ev} = \begin{pmatrix}  1 \\ p(x_1)	\end{pmatrix} = \begin{pmatrix}
1 \\ 0.4
\end{pmatrix}.
\end{equation*}

После пропагации свидетельства оценки вероятностей элементов первого фрагмента знаний, вычисленные с помощью уравнения~\ref{conjG}, будут следующими:
\begin{equation*}
\Pc^{1,a} = \begin{pmatrix}
1 \\  0.4 \\ 0.72 \\ 0.24
\end{pmatrix}.
\end{equation*}

Виртуальное свидетельство будет построено над алфавитом $\mathcal{A}_{\ev} = \mathcal{A}_1 \cap \mathcal{A}_2 = \{x_1, x_2\} \cap \{x_2, x_3\} = \{x_2\}$ и будет содержать следующий вектор оценок вероятностей элементов:
\begin{equation*}
\Pc^{\ev} = \begin{pmatrix}  1 \\ p(x_2)	\end{pmatrix} = 
\begin{pmatrix} 1 \\ 0.72 \end{pmatrix}.
\end{equation*}

После пропагации виртуального свидетельства во второй фрагмент знаний с помощью уравнения~\ref{conjglob2}, оценки второго ФЗ получаются следующими:
\begin{equation*}
\Pc^{2,a} = \begin{pmatrix}
1 \\ 0.72 \\ 0.48 \\ 0.24
\end{pmatrix}.
\end{equation*}

Данный результат пропагации можно получить с помощью следующего фрагмента кода:

\begin{lstlisting}[caption = Пример пропагации виртуального свидетельства]
// Creating KPs and evidence
var firstKP = new ScalarConjKP(Convert.ToInt64("011", 2), new[] { 1, 0.5, 0.7, 0.3 });
var secondKP = new ScalarConjKP(Convert.ToInt64("110", 2), new[] { 1, 0.3, 0.7, 0.1 });
var evidence = new ScalarConjKP(Convert.ToInt64("001", 2), new[] { 1, 0.4 });
// Creating propagators for stochastic and virtual evidences
var localPropagator = new StochasticScalarConjunctsLocalPropagator();
var virtualEvPropagator = new VirtualEvidenceScalarConjPropagator();

// Propagating stochasic evidence in fisrt KP
localPropagator.setPattern(firstKP);
localPropagator.propagate(evidence);
var firstKPwithAposterioriEst = localPropagator.getResult();

// Propagating virtual evidence in second KP
var secondKPwithAposterioriEst = virtualEvPropagator
    .PropagateStochasticEvidence(firstKPwithAposterioriEst, secondKP);
\end{lstlisting}

Вывод на консоль можно осуществить с помощью следующего кода:
\begin{lstlisting}[caption = Вывод результатов на консоль]
Console.WriteLine("Probability estimates in first KP after propagation:");
firstKPwithAposterioriEst
    .GetPointEstimate()
    .ToList()
    .ForEach(est => Console.Write(" " + est));
 
Console.WriteLine("Probability estimates in second KP after propagation:");
secondKPwithAposterioriEst
    .GetPointEstimate()
    .ToList()
    .ForEach(est => Console.Write(" " + est));
\end{lstlisting}

Результаты вывода на консоль:
\begin{lstlisting}[caption = Результаты]
Probability estimates in first KP after propagation:
 1, 0,4 0,72 0,24
Probability estimates in second KP after propagation:
1, 0,72, 0,48 0,24
\end{lstlisting}
\subsubsection{Пример пропагации стохастического виртуального свидетельства в ФЗ с скалярными оценками над идеалом дизъюнктов}
Рассмотрим фрагменты знаний над алфавитами $\mathcal{A}_1 = \{x_1, x_2\}$ и  $\mathcal{A}_2 = \{x_2, x_3\}$, построенные над идеалами дизъюнктов. Программная реализация принимает на вход и возвращает в качестве результата ФЗ, содержащие вектор вероятностей $\Pd$, в то же время как приведенные в работе формулы работают с векторами $\Pd^{\prime}$, поэтому приведем в данном примере векторы вероятностей как и $\Pd$, так и $\Pd^{\prime}$. Считаем, что в первый фрагмент знаний поступило стохастическое свидетельство над алфавитом $\mathcal{A}_{\ev} = \{x_1\}$. Пусть ФЗ и свидетельство имеют следующие оценки вероятностей: \\ 
\begin{equation*}
\Pd^{\prime 1} =  \begin{pmatrix}
1 \\ p(\overline{x}_1) \\ p(\overline{x}_2) \\ p(\overline{x}_2\overline{x}_1)
\end{pmatrix} = \begin{pmatrix}
1 \\ 0.5\\ 0.3\\ 0.1
\end{pmatrix}, 
\Pd^1 = \mathbf{1} - \Pd^{\prime 1} =  \begin{pmatrix}
0 \\ 0.5\\ 0.7\\ 0.9
\end{pmatrix}, 
\end{equation*}
\begin{equation*}
\Pd^{\prime 2} =  \begin{pmatrix}
1 \\ p(\overline{x}_3) \\ p(\overline{x}_3) \\ p(\overline{x}_3\overline{x}_2)
\end{pmatrix} = \begin{pmatrix}
1 \\ 0.7\\ 0.3\\ 0.1 
\end{pmatrix},
\Pd^2 = \mathbf{1} - \Pd^{\prime 2} =  \begin{pmatrix}
0 \\ 0.3\\ 0.7\\ 0.9 \end{pmatrix},
\end{equation*}
\begin{equation*}
\Pd^{\prime \ev} = \begin{pmatrix}  1 \\ p(\overline{x}_1)	\end{pmatrix} = \begin{pmatrix}
1 \\ 0.6
\end{pmatrix},
\Pd^{\ev} = \mathbf{1} - \Pd^{\prime \ev} =  \begin{pmatrix}
0 \\ 0.4 \end{pmatrix}.
\end{equation*}

После пропагации свидетельства оценки вероятностей элементов первого фрагмента знаний, вычисленные с помощью уравнения~\ref{dis3}, будут следующими:
\begin{equation*}
\Pd^{\prime 1,a} = \begin{pmatrix}
1 \\  0.6 \\ 0.28 \\ 0.12
\end{pmatrix},
\Pd^{1,a}= \mathbf{1} - \Pd^{\prime 1,a} = \begin{pmatrix}
0 \\  0.4 \\ 0.72 \\ 0.88
\end{pmatrix}.
\end{equation*}

Виртуальное свидетельство строится над алфавитом $\mathcal{A}_{\ev} = \mathcal{A}_1 \cap \mathcal{A}_2 = \{x_1, x_2\} \cap \{x_2, x_3\} = \{x_2\}$ и содержит следующий вектор оценок вероятностей элементов:
\begin{equation*}
\Pd^{\prime \ev} = \begin{pmatrix}  1 \\ p(\overline{x}_2)	\end{pmatrix} = 
\begin{pmatrix} 1 \\ 0.28 \end{pmatrix},
\Pd^{\ev} = \mathbf{1} - \Pd^{\prime \ev} = \begin{pmatrix} 0 \\ 0.72 \end{pmatrix}.
\end{equation*}

C помощью уравнения~\ref{disGlob} распространим виртуальное свидетельство во второй фрагмент знаний и получим апостериорные оценки вероятностей:
\begin{equation*}
\Pd^{\prime 2,a} = \begin{pmatrix}
1 \\ 0.28 \\ 0.52 \\ 0.04
\end{pmatrix},
 \Pd^{2,a}  = \mathbf{1} - \Pd^{\prime 2,a} = \begin{pmatrix}
0 \\ 0.72 \\ 0.48 \\ 0.96
\end{pmatrix}.
\end{equation*}

Данный результат пропагации можно получить с помощью следующего фрагмента кода:
\begin{lstlisting}[caption = Пример пропагации виртуального свидетельства]
// Creating KPs and evidence
var firstKP = new ScalarDisjKP(Convert.ToInt64("011", 2), new[] { 0, 0.5, 0.7, 0.9 });
var secondKP = new ScalarDisjKP(Convert.ToInt64("110", 2), new[] { 0, 0.3, 0.7, 0.9 });
var evidence = new ScalarDisjKP(Convert.ToInt64("001", 2), new[] { 0, 0.4 });
// Creating propagators for stochastic and virtual evidences
var localPropagator = new StochasticScalarDisjLocalPropagator(firstKP);
var virtualEvPropagator = new VirtualEvidenceScalarDisjPropagator();
// Propagating stochastic evidence in first KP
var firstKPwithAposterioriEst = localPropagator.PropagateEvidence(evidence); 
// Propagating virtual evidence in second KP
var secondKPwithAposterioriEst = virtualEvPropagator
    .PropagateStochasticEvidence(firstKPwithAposterioriEst, secondKP);
\end{lstlisting}

Вывод на консоль можно осуществить с помощью следующего кода:
\begin{lstlisting}[caption = Вывод результатов на консоль]
Console.WriteLine("Probability estimates in first KP after propagation:");
firstKPwithAposterioriEst
    .GetPointEstimate()
    .ToList()
    .ForEach(est => Console.Write(" " + est));

Console.WriteLine("Probability estimates in second KP after propagation:");
secondKPwithAposterioriEst
    .GetPointEstimate()
    .ToList()
    .ForEach(est => Console.Write(" " + est));
\end{lstlisting}

Результаты вывода на консоль:
\begin{lstlisting}[caption = Результаты]
Probability estimates in first KP after propagation:
 0 0,4 0,72 0,88
Probability estimates in second KP after propagation:
 0 0,72 0,48 0,96
\end{lstlisting}
\subsubsection{Пример пропагации стохастического виртуального свидетельства в ФЗ со скалярными оценками над множеством квантов}
Возьмем два фрагмента знаний над алфавитами $\mathcal{A}_1 = \{x_1, x_2\}$ и  $\mathcal{A}_2 = \{x_2, x_3\}$. Пусть в первый фрагмент знаний поступило стохастическое свидетельство над алфавитом $\mathcal{A}_{\ev} = \{x_1\}$. Будем считать, что ФЗ и свидетельство построены над множеством квантов и имеют следующие оценки вероятностей: 
\begin{equation*}
\Pq^1 =  \begin{pmatrix}
p(\overline{x}_2\overline{x}_1)\\ p(\overline{x}_2x_1) \\ p(x_2\overline{x}_1) \\ p(x_2x_1)
\end{pmatrix} = \begin{pmatrix}
0.1\\ 0.2\\ 0.4\\ 0.3
\end{pmatrix},
\Pq^2 =  \begin{pmatrix}
p(\overline{x}_3\overline{x}_2)\\ p(\overline{x}_3x_2)\\ p(x_3\overline{x}_2) \\ p(x_3x_2)
\end{pmatrix} = \begin{pmatrix}
0.1 \\ 0.2\\ 0.6\\ 0.1 
\end{pmatrix},
\end{equation*}
\begin{equation*}
\Pq^{\ev} = \begin{pmatrix}   p(\overline{x}_1) \\ p(x_1)	\end{pmatrix} = \begin{pmatrix}
0.6 \\ 0.4
\end{pmatrix}.
\end{equation*}

После пропагации свидетельства оценки вероятностей элементов первого фрагмента знаний, вычисленные с помощью уравнения~\ref{quantsG}, будут следующими:
\begin{equation*}
\Pq^{1,a} = \begin{pmatrix}
0.12 \\  0.16 \\ 0.48\\ 0.24
\end{pmatrix}.
\end{equation*}

Виртуальное свидетельство будет построено над алфавитом $\mathcal{A}_{\ev} = \mathcal{A}_1 \cap \mathcal{A}_2 = \{x_1, x_2\} \cap \{x_2, x_3\} = \{x_2\}$ и будет содержать следующий вектор оценок вероятностей квантов:
\begin{equation*}
\Pq^{\ev} = \begin{pmatrix}  p(\overline{x}_2) \\ p(x_2)	\end{pmatrix} = 
\begin{pmatrix} 0.28 \\ 0.72 \end{pmatrix}.
\end{equation*}

Пропагируем виртуальное свидетельство во второй фрагмент знаний с помощью уравнения~\ref{quantsGlob} и получим следующие оценки вероятностей:
\begin{equation*}
\Pq^{2,a} = \begin{pmatrix}
0.04 \\ 0.48 \\ 0.24 \\ 0.24
\end{pmatrix}.
\end{equation*}

Данный результат пропагации можно получить с помощью следующего фрагмента кода:

\begin{lstlisting}[caption = Пример пропагации виртуального свидетельства]
// Creating KPs and evidence
var firstKP = new ScalarQuantKP(Convert.ToInt64("011", 2), new[] { 0.1, 0.2, 0.4, 0.3 });
var secondKP = new ScalarQuantKP(Convert.ToInt64("110", 2), new[] { 0.1, 0.2, 0.6, 0.1 });
var evidence = new ScalarQuantKP(Convert.ToInt64("001", 2), new[] { 0.6, 0.4 });

// Creating propagators for stochastic and virtual evidences
var localPropagator = StochasticScalarQuantLocalPropagator.Instance(firstKP);
var virtualEvPropagator = new VirtualEvidenceScalarQuantPropagator();

// Propagating stochasic evidence in fisrt KP
var firstKPwithAposterioriEst = localPropagator.PropagateEvidence(evidence);
            
// Propagating virtual evidence in second KP
var secondKPwithAposterioriEst = virtualEvPropagator
    .PropagateStochasticEvidence(firstKPwithAposterioriEst, secondKP);
\end{lstlisting}

Вывод на консоль можно осуществить с помощью следующего кода:
\begin{lstlisting}[caption = Вывод результатов на консоль]
Console.WriteLine("Probability estimates in first KP after propagation:");
firstKPwithAposterioriEst
    .GetPointEstimate()
    .ToList()
    .ForEach(est => Console.Write(" " + est));
 
Console.WriteLine("Probability estimates in second KP after propagation:");
secondKPwithAposterioriEst
    .GetPointEstimate()
    .ToList()
    .ForEach(est => Console.Write(" " + est));
\end{lstlisting}

Результаты вывода на консоль:
\begin{lstlisting}[caption = Результаты]
Probability estimates in first KP after propagation:
 0,12 0,16 0,48 0,24
Probability estimates in second KP after propagation:
 0,04 0,48 0,24 0,24
\end{lstlisting}
