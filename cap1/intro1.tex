Алгебраические байесовские сети относятся к классу вероятностных графических моделей и являются эффективным инструментом для обработки и представления знаний с неопределенностью~\cite{184}.     

 Будем считать, что знания формируют эксперты в определенной предметной области. Эксперты задают оценки вероятностей утверждениям, образующим базу знаний, и характеризуют связи между утверждениями с помощью оценок вероятностей. Из-за этого и возникает неопределенность знаний.

 Представим ситуацию, что экспертам необходимо охарактеризовать несколько утверждений и связи между ними. Пусть характеризуемые утверждения будут атомарными. Так как число взаимосвязей с ростом числа утверждений будет растет экспоненциально,  эффективность работы с этими утверждениями будет быстро падать.
 
 Отличительной особенностью алгебраических байесовских сетей является подход к декомпозиции области знаний на фрагменты знаний~\cite{93}: разобьем множество утверждений на подмножества, называемые фрагментами знаний, и будем характеризовать утверждения в каждом фрагменте знаний отдельно. Это позволит использовать быстрые алгоритмы обработки данных, не предъявляя серьезных требований к вычислительным мощностям. Также этот подход удобен при характеризации высказываний, потому что при ограниченном наборе элементов необходимо охарактеризовать связи лишь между несколькими высказываниями. 
 
 Таким образом, АБС состоит из фрагментов знаний и структуры связей между ними. На данный момент описаны три математические модели фрагмента знаний: идеал конъюнктов, идеал дизъюнктов и множество квантов~\cite{121}. Фрагмент знаний строится над алфавитом, где элементы алфавита соответствуют характеризуемым утверждениям. В данной работе будет рассматриваться вторичная структура АБС~\cite{87}.
 
 