Комплекс программ AlgBN Math Library~\cite{89} представляет собой библиотеку для автоматизации логико"=вероятностного вывода в АБС, в которой уже реализованы структуры данных для хранения фрагмента знаний и АБС, а также алгоритмы локального логико-вероятностного вывода и проверки непротиворечивости. Фрагменты знаний можно создавать с бинарными, скалярными и интервальными оценками вероятностей. Рассмотрим наиболее важные структуры, имеющиеся в библиотеке.

 Фрагмент знаний, представленный идеалом конъюнктов с скалярными оценками представлен классом ScalarConjKP, с интервальными оценками --- IntervalConjKP, с бинарными --- BinaryConjKP. Для создания экземпляра любого из этих классов нужно передать в конструктор глобальный индекс ФЗ и массив оценок вероятностей элементов ФЗ. Глобальный индекс --- это число, единицы в двоичной записи которого соответствуют номерам элементов алфавита, над которыми построен ФЗ. Аналогичные классы есть для двух других моделей ФЗ: для дизъюнктов --- BinaryDisjKP, ScalarDisjKP и IntervalDisjKP, для квантов --- BinaryQuantKP, ScalarQuantKP и IntervalQuantKP.

Структура Propagator состоит из классов, позволяющих пропагировать детерминированное, стохастическое или неточное свидетельство в ФЗ с скалярными или интервальными оценками. Каждый из таких классов содержит методы propagate для пропагации свидетельства и getResult для возвращения результата пропагации. Каждая реализация отвечает за пропагацию одного из типов свидетельства в ФЗ с одним видом оценок. Например, StochasticScalarConjunctsLocalPropagator отвечает за пропагацию стохастического свидетельства в ФЗ с скалярными оценками. Структура Propagator позволяет работать с двумя моделями ФЗ --- идеал конъюнктов и идеал дизъюнктов. Для третьей модели функциональность еще не реализована, а алгоритмы, работающие с идеалами дизъюнктов, требуют доработки.

За проверку локальной непротиворечивости отвечает структура In\-ferrer. Структура MatrixTransform реализует матрицы перехода между различными моделями ФЗ. Для решения задач линейного программирования используется библиотека на C++ lp\_solve55, за обращение к библиотеке отвечает класс LP. Кроме того, есть структура Alphabet для работы с алфавитом.

Также в комплексе есть программный модуль ABNGlobal, отвечающий за глобальную непротиворечивость. Так как пропагация виртуального свидетельства относится к глобальному ЛВВ, то расширим эту часть библиотеки своей реализацией. 

Реализация будет использовать существующие структуры для локального ЛВВ. Так как существующие алгоритмы позволяют работать с интервальными оценками, то реализация алгоритмов пропагации виртуального свидетельства будет разработана как и для ФЗ с интервальными оценками, так и для неточных виртуальных свидетельств. Существующую реализацию~\cite{4, 65} алгоритмов  локального логико"=вероятностного вывода для квантов перенесем на существующие в комплексе программ структуры хранения, адаптируем и затем переиспользуем при реализации алгоритма пропагации виртуального свидетельства для данной модели. Реализации Propagator для дизъюнктов также адаптируем и усовершенствуем с целью дальнейшего переиспользования.


