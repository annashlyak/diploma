\section*{Заключение}

   В ходе работы над магистерской диссертацией была достигнута основная цель -- проведен реинжениринг библиотеки для осуществления локального логико-вероятностного вывода над фрагментами знаний AlgBN Math Library. 
    
    Все задачи были выполнены, получены следующие результаты:
   \begin{enumerate}
                \item[1)] разработана система Unit-тестов для библиотеки AlgBN Math Library;
                \item[2)] спроектирована и реализована настройка в AlgBN Math Library для осуществления локального ЛВВ в применении в глобальных структурах;
                \item[3)] внедрен парсер в машину для априорного вывода в AlgBN Math Library;
                \item[4)] реализована система методов для проведения вычислительных экспериментов по вычислению оценки чувствительности первой задачи для детерминированного свидетельства и ФЗ со скалярными оценками в AlgBN Math Library.
   \end{enumerate}
    
    Результаты вошли в 15 публикаций:
    \begin{enumerate}
        \item	\textbf{Мальчевская~Е.А.} Архитектура библиотеки AlgBNModeller: представление альтернативных фрагментов знаний // Региональная информатика (РИ-2016). Юбилейная XV Санкт-Петербургская международная конференция «Региональная информатика (РИ-2016)». (Санкт-Петербург, 26-28 октября 2016 г.): Материалы конференции. СПб: СПОИСУ, 2016. С. 519;
		\item	Харитонов~Н.А., \textbf{Мальчевская~Е.А.}  Локальный априорный вывод в АБС: автоматизация анализа пропозициональной формулы// Региональная информатика (РИ-2016). Юбилейная XV Санкт-Петербургская международная конференция «Региональная информатика (РИ-2016)». (Санкт-Петербург, 26-28 октября 2016 г.): Материалы конференции. СПб: СПОИСУ, 2016. С. 522;
		\item	Золотин~А.А., \textbf{Мальчевская~Е.А.}, Бирилло~А.И., Тулупьев~А.Л. Управление согласованностью оценок вероятностей в локальном апостериорном выводе в алгебраических байесовских сетях // 9-я Российская мультиконференция по проблемам управления, материалы 9-й конференции "Информационные технологии в управлении" (ИТУ-2016) - СПб: АО "Концерн "ЦНИИ "Электроприбор", 2016, С. 52-61;
        \item Золотин А.A., Левенец Д.Г., \textbf{Мальчевская Е.А.}, Зотов М.А., Бирилло А.И., Березин А.И., Иванова А.В., Тулупьев А.Л. Алгоритмы обработки и визуализации алгебраических байесовских сетей // Образовательные технологии и общество. 2017. Т.20. №1. С.446-457;
        \item	\textbf{Мальчевская Е.А.} Имплементация уравнений локального логико- вероятностного вывода в комплексе программ AlgBN Math Library //  Нечеткие системы, мягкие вычисления и интеллектуальные технология (НСМВИТ-2017) труды VII всероссийской научной-практической конференции. 2017. С. 125-134;
        \item	\textbf{Мальчевская Е.А.}, Бирилло А.И., Харитонов Н.А., Золотин А.А. Развитие матрично-векторного подхода в алгоритмах локального априорного вывода в алгебраических байесовских сетях // Нечеткие системы, мягкие вычисления и интеллектуальные технология (НСМВИТ-2017) труды VII всероссийской научной-практической конференции. 2017. С. 92-100; 
        \item	\textbf{Мальчевская Е.А.} Алгоритмизация локального апостериорного логико-вероятностного вывода в алгебраических байесовских сетях // Интеллектуальные системы и технологии: современное состояние и перспективы Сборник научных трудов IV Международной летней школы-семинара по искусственному интеллекту для студентов, аспирантов, молодых ученых и специалистов. 2017. С. 120-127;
        \item	Zolotin, A. A., \textbf{Malchevskaya, E. A.}, Tulupyev, A. L., Sirotkin, A. V. (2017, September). An Approach to Sensitivity Analysis of Inference Equations in Algebraic Bayesian Networks. In International Conference on Intelligent Information Technologies for Industry (pp. 34-42). Springer, Cham;
        \item	\textbf{Мальчевская Е.А.}, Золотин А.А., Тулупьев А.Л. Уравнения апостериорного вывода в фрагментах знаний над идеалом дизъюнктов // Всероссийская научная конференция по проблемам информатики (СПИСОК-2017). (Санкт-Петербург, 25-27 апреля 2017 г.). Санкт-Петербург: СПбГУ, 2017. C. 395-403.
        \item	\textbf{Мальчевская Е.А.} Анализ чувствительности локального логико-вероятностного вывода в математической библиотеке AlgBN Math Library. // Информационная безопасность регионов России (ИБРР-2017). X Санкт-Петербургская межрегиональная конференция. (Санкт-Петербург, 1–3 ноября 2017 г.): Материалы конференции. СПб: СПОИСУ, 2017. С. 423–424;
        \item \textbf{Мальчевская Е.А.}, Золотин А.А., Тулупьев А.Л. Алгоритмы апостериорного вывода в алгебраических байесовских сетях: рафинирование матрично-векторного представления // Нечеткие системы и мягкие вычисления. Промышленные применения. Fuzzy Technologies in the Industry (FTI-2017): Первая Всероссийская научно-практическая конференция (Россия, г. Ульяновск, 14-15 ноября, 2017 г.): сборник научных трудов. – Ульяновск : УлГТУ, 2017. 376-388 с;
        \item	\textbf{Malchevskaya E.}, Kharitonov N., Zolotin A., Birillo A. Algebraic Bayesian Networks: Probabilistic-Logic Inference Algorithms and Storage Structures // Proceedings of the Finnish-Russian University Cooperation in Telecommunications(FRUCT’20) Conference, 2017. P 628-633;
        \item	Тулупьев А.Л., Тулупьева Т.В., Суворова А.В., Абрамов М.В., Золотин А.А., Зотов М.А., Азаров А.А.,\textbf{ Мальчевская Е.А.}, Левенец Д.Г., Торопова А.В., Харитонов Н.А., Бирилло А.И., Сольницев Р.И., Микони С.В., Орлов С.П., Толстов А.В. Мягкие вычисления и измерения. Модели и методы: монография. Том III / под ред. д.т.н., проф. С.В. Прокопчиной. – М.: ИД "НАУЧНАЯ БИБЛИОТЕКА", 2017. – 300 с;
        \item	жНСМВ!
        \item	ТвГТУ
        \end{enumerate}
    
  Результаты были представлены на восьми конференциях: 
  \begin{enumerate}
                \item  РИ’2016: Юбилейная XV Санкт-Петербургская международная конференция «Региональная информатика 2016». Темы докладов: «Архитектура библиотеки AlgBNModeller: представление альтернативных фрагментов знаний» и «Локальный априорный вывод в алгебраических байесовских сетях: автоматизация анализа пропозициональной формулы». Санкт-Петербург, 26-28 октября 2016 г.;
                \item	IITI-2017: 2nd International Scientific Conference “Intelligent infor\-ma\-tion technologies for industry” (IITI’17). Варна, Болгария, 14–16 сентября 2017;
                \item	Всероссийская научная конференция по проблемам информатики СПИСОК–2017. Санкт-Петербург, 25–27 апреля 2017;
                \item	FTI’2017: Первая всероссийская научно-практическая конференция «Нечеткие системы и мягкие вычисления. Промышленные применения» (Fuzzy Technologies in the Industry – FTI-2017). Ульяновск, 14–15 ноября 2017;
                \item	НСМВИТ’2017: VII-я Всероссийская научно-практическая конференция «Нечеткие системы, мягкие вычисления и интеллектуальные технологии» (НСМВИТ-2017). Санкт-Петербург, 3–7 июля 2017;
                \item	ISyT’2017: IV-я Международная летняя школа-семинар по искусственному интеллекту для студентов, аспирантов, молодых ученых и специалистов «Интеллектуальные системы и технологии: современное состояние и перспективы–2017» (ISyT–2017). Санкт-Петербург, 30 июня – 3 июля 2017;
                \item	ИБРР’2017: X Санкт-Петербургская межрегиональная конференция «Информационная безопасность регионов России (ИБРР-2017)». Санкт-Петербург, 1–3 ноября 2017;
                \item   FRUCT’2017: 20th Finnish-Russian University Cooperation in Te\-le\-commu\-ni\-ca\-tions (FRUCT) Conference. Санкт-Петербург, 5-7 апреля 2017;
   \end{enumerate}
    
   Была подана заявка на регистрацию программы для ЭВМ: Мальчевская~Е.А., Тулупьев~А.Л. Algebraic Bayesian Network Local Reconciler, Version 01 for CSharp (AlgBN L Reconciler cs.v.01).

	Работа выполнялась под руководством А.Л.~Тулупьева, на базе лаборатории теоретических и междисциплинарных проблем информатики СПИИРАН в рамках проекта по государственному заданию № 0073-2014-0002; кроме того, разработки были частично поддержаны грантами РФФИ 15-01-09001-a~--- <<Комбинированный логико-ве\-роят\-ност\-ный графический подход к представлению и обработке систем знаний с неопределенностью: алгебраические байесовские сети и родственные модели>>, 18-01-00626~--- <<Методы представления, синтеза оценок истинности и машинного обучения в алгебраических байесовских сетях и родственных моделях знаний с неопределенностью: логико-вероятностный подход и системы графов>>.
    
   % Полученные результаты позволяют в будущем развить функциональность данной библиотеки, реализовывать глобальный логико-ве\-ро\-ятност\-ный вывод, а также интегрироваться с другими подпроектами, которые связаны со структурной компонентой библиотеки.  