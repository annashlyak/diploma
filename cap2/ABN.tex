Зафиксируем алфавит $\mathcal{A} = \{x_i\}^n_{i=0}$ --- конечное множество атомарных пропозициональных формул~(атомов). Над алфавитом определим идеал конъюнктов, идеал дизъюнктов и множество квантов.
\begin{definition}[\cite{1}]
Идеал конъюнктов $C_\mathcal{A} $ --- это множество вида 
\begin{math}
\{x_{i_1} \wedge x_{i_2}\wedge ... \wedge x_{i_k} | \; 0 \leq i_1 < i_2 < ... < i_k \leq  n - 1, 0 \leq k \leq n\}
\end{math}.
\end{definition}

\begin{definition}[\cite{1}]
Квант над алфавитом $\mathcal{A} = \{x_i\}^n_{i=0}$ ---  это конъюнкция, которая для любого атома алфавита содержит либо этот атом, либо его отрицание.
\end{definition}

\begin{definition}[\cite{1}]
Литерал $\tilde{x}_i$ обозначает, что на его месте в формуле может стоять либо $x_i$, либо его отрицание $\bar{x}_i$.
\end{definition}

\begin{definition}[\cite{1}]
	Множество квантов $Q_\mathcal{A} $ --- это множество всех комбинаций вида $\{\tilde{x}_0\tilde{x}_1...\tilde{x}_{n-1}\}$.
\end{definition}

\begin{definition}[\cite{184}]
Идеал дизъюнктов $C_\mathcal{A} $ --- это множество вида 
\begin{math}
\{x_{i_1} \vee x_{i_2}\vee ... \vee x_{i_k} | \; 0 \leq i_1 < i_2 < ... < i_k \leq  n - 1, 0 \leq k \leq n\}
\end{math}.
\end{definition}

Введем правило нумерации на множестве конъюнктов и на множестве квантов.

Каждому кванту $\tilde{x}_0\tilde{x}_1...\tilde{x}_{n-1}$ поставим в соответствие двоичную запись, в которой на $i$-м месте будет стоять $1$, если $i$-тый литерал означен положительно, и $0$ иначе. Аналогично занумеруем конъюнкты: каждому конъюнкту $x_{i_1}x_{i_2}...x_{i_k}$ поставим в соответствие сумму $2^{i_1}+2^{i_2}+...+2^{i_k}$. Тогда, если представить полученную сумму в виде двоичной записи и дополнить лидирующими нулями до $n$ знаков, $i$-тый атом будет входить в конъюнкт тогда и только тогда, когда $i$-тый бит числа равен $1$. Таким образом, получится биективное отображение множества квантов на множество конъюнктов.

Для множества дизъюнктов правило нумерации будет аналогичным правилу для множества конъюнктов, следовательно, существует биективное отображение множества квантов и на множество дизъюнктов.

Теперь введем вектор вероятностей элементов идеала конъюнктов $\Pc$, вектор вероятностей элементов идеала дизъюнктов $\Pd$ и   вектор вероятностей элементов множества квантов $\Pq$: 
\begin{equation*}
\Pc = \begin{pmatrix}
	    p(c_0)  \\ p(c_1) \\ \vdots \\  p(c_{2^n-1})
      \end{pmatrix}, \,
\Pq = \begin{pmatrix}
	    p(q_0)  \\ p(q_1) \\ \vdots \\  p(q_{2^n-1})
      \end{pmatrix}, \,
\Pd = \begin{pmatrix}
	    p(d_0)  \\ p(d_1) \\ \vdots \\  p(d_{2^n-1})
      \end{pmatrix}. 
\end{equation*}

$\Pc$ и $\Pq$ связаны между собой соотношениями $\Pc = \Jn\Pq$ и $\Pq = \In\Pc$, где матрицы перехода $\In$ и $\Jn$ получены с помощью следующих рекуррентных соотношений~\cite{109}:

\begin{equation*}
    \In = \mathbf{I}_1 \otimes ... \otimes \mathbf{I}_{n-1} = \mathbf{I}_1^{[n]},
\end{equation*}
где $\mathbf{I}_1 = \begin{pmatrix*}[r] 1 & -1 \\ 0 & 1 \end{pmatrix*}$.

\begin{equation*}
    \Jn = \mathbf{J}_1 \otimes ... \otimes \mathbf{J}_{n-1} = \mathbf{J}_1^{[n]},
\end{equation*}
где $\mathbf{J}_1 = \begin{pmatrix} 1 & 1 \\ 0 & 1 \end{pmatrix}$, $\otimes$ обозначает кронекерово произведение матриц.

Вектор вероятностей элементов идеала дизъюнктов связан с $\Pc$ и $\Pq$ следующими соотношениями~\cite{76}:

\begin{equation}\label{pdtopq}
\Pq = \Ln(\mathbf{1} -\Pd),
\end{equation} 
где $\Ln = \begin{pmatrix*}[r] 0 & 1 \\ 1 & -1 \end{pmatrix*}^{[n]}$.

\begin{equation}
\Pc = \Kn(\mathbf{1} -\Pd),
\end{equation}
где $\Kn = \begin{pmatrix*}[r] 1 & 0 \\ 1 & -1 \end{pmatrix*}^{[n]}$.

Для удобства будем обозначать $\Pd^{\prime} = \mathbf{1} -\Pd$ и далее работать с вектором $\Pd^{\prime}$.

Дадим определение фрагмента знаний для трех моделей в случае, когда оценки вероятностей элементов скалярные.

\begin{definition}[\cite{1}]
Фрагмент знаний $\mathscr{C}$ над идеалом конъюнктов со скалярными оценками --- это пара вида $(C, \,p)$, где $C$ --- идеал конъюнктов, $p$ --- функция из $C$  в интервал $[0;1]$.
\end{definition}

\begin{definition}[\cite{1}]
	Фрагмент знаний со скалярными оценками непротиворечив тогда и только тогда, когда $\In\Pc \geq \mathbf{0}$.
\end{definition}

\begin{definition}[\cite{121}]
Фрагмент знаний $\mathscr{C}$ над идеалом дизъюнктов со скалярными оценками --- это пара вида $(C, \,p)$, где $C$ --- идеал дизъюнктов, $p$ --- функция из $C$  в интервал $[0;1]$.
\end{definition}

\begin{definition}[\cite{121}]
Фрагмент знаний $\mathscr{C}$ над множеством квантов со скалярными оценками --- это пара вида $(Q, \,p)$, где $Q$ --- множество квантов, лежащее в основе ФЗ, $p$ --- функция из $Q$  в интервал $[0;1]$.
\end{definition}

В качестве функции $p$ можно использовать вероятность истинности пропозиций.
\begin{definition}[\cite{1}]
Алгебраическую байесовскую сеть $\mathcal{N}$ определим как набор фрагментов знаний: $\mathcal{N}^{\circ} = \{\mathscr{C}_i\}_{i=1}^{n}$.
\end{definition}