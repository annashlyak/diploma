\section*{Введение}

\underline{Актуальность темы.} Алгебраические байесовские сети~(АБС), являющиеся одним из классов вероятностных графических моделей, позволяют обрабатывать данные с неопределенностью, которая может порождаться, как необходимостью трансформации высказываний с естественного языка на математический (например высказывания ``скорее всего'', ``наверное'', ``возможно'' удобнее представлять интервальной оценкой вероятности чем скалярной), так и возникновением пробелов в собранных и анализируемых данных. Возможность работать с неопределенностью различного рода в данных является одним из преимуществ алгебраических байесовских сетей в сравнении с другими родственными им вероятностными графическими моделями, в частности с байесовскими сетями доверия, которые не позволяют обрабатывать интервальные оценки вероятности истинности.
   
   Понятие алгебраических байесовских сетей было введено В.И.~Городецким в 1993 году. С того момента теория существенно развивалась, были написаны работы по различным ветвлениям теории: структурному представлению алгебраических байесовских сетей, глобальному и локальному логико-вероятностному выводу.
   %уточняющие и дополняющие срез теории АБС, связанный со структурными представлениями, например первичными и вторичными структурами~\cite{struct1,struct2}. Также были написаны работы, рассматривающие и развивающие подходы к логико-вероятностному выводу в АБС~(поддержание непротиворечивости, априорный вывод, апостериорный вывод)~\cite{consist1,aprior1,apost1,apost2,matr}.
   
   %
   %Расширить
   Упомянутый раньше логико-вероятностный вывод~(ЛВВ) является основным математическим аппаратом, с помощью которого осуществляется обработка существующих данных и получение новых в теории АБС. Эта машина вывода подразделяется на проверку и поддержание непротиворечивости, решение задач априорного и апостериорного вывода. Машина может воздействовать на разных уровнях сети: на локальном и глобальном. `Глобальный'' говорит нам о том, что в область нашего рассмотрения попадает вся сеть и входящие в нее ФЗ, а ``локальный'' о том, что область рассмотрения сужается до одного фрагмента знаний.
        
     Ранее в рамках обширного проекта по работе с АБС AlgBN Web App на языке программирования C\# была спроектирован и реализован комплекс программ, математическая библиотека AlgBN Math Library, позволяющая создавать локальные структуры АБС и выполнять над ними локальный ЛВВ. В дальнейшем планируется использование библиотеки в еще одной составной части проекта -- комплексе программ по работе с различными глобальными структурами АБС (первичной, вторичной, третичной...). Для этого необходимо было спроектировать и реализовать контракт для внешнего доступа и удобного использования функциональности библиотеки.  
     
     Как один из способов анализа алгоритмов локального логико-вероятностного вывода, требовалось посчитать оценку чувствительности при некоторых начальных условиях и выбранной вариации одной величины. Было решено рассмотреть анализ чувствительности первой задачи локального апостериорного вывода с детерминированным свидетельством для фрагмента знаний со скалярными оценками и вариацией оценок вероятности исходного вектора. 
  % Магистерская диссертация базируется на бакалаврской !!!!!
        
        \underline{Объектом исследования} являются алгебраические байесовские сети, а \underline{предметом исследования} -- локальный логико-вероятностный вывод в АБС с использованием матрично-векторных алгоритмов.
        
        \underline{Целью данной магистерской диссертации} является реинжениринг библиотеки для осуществления локального логико-вероятностного вывода над фрагментами знаний AlgBN Math Library. Для достижения поставленной цели были сформированы \underline{задачи}:
            \begin{enumerate}
                \item[1)] разработать систему Unit-тестов для библиотеки AlgBN Math Library;
                \item[2)] спроектировать и реализовать настройку в AlgBN Math Library для осуществления локального ЛВВ в применении в глобальных структурах;
                \item[3)] внедрить парсер в машину для априорного вывода в AlgBN Math Library;
                \item[4)] реализовать систему методов для проведения вычислительных экспериментов по вычислению оценки чувствительности первой задачи для детерминированного свидетельства и ФЗ со скалярными оценками в AlgBN Math Library.
            \end{enumerate}
        
        \underline{Апробация результатов исследования.} Результаты исследования были представлены на восьми научных конференциях.
        
        \underline{Публикации.} По теме магистерской диссертации было опубликовано 15 научных трудов.
        
       \bigskip
         {\small Эта работа является частью более широких инициативных проектов, выполняющихся в лаборатории теоретических и междисциплинарных основ информатики СПИИРАН под руководством А.Л.~Тулупьева; кроме того, разработки были частично поддержаны грантами РФФИ 15-01-09001-a~--- <<Комбинированный логико-вероятностный графический подход к представлению и обработке систем знаний с неопределенностью: алгебраические байесовские сети и родственные модели>>, 18-01-00626~--- <<Методы представления, синтеза оценок истинности и машинного обучения в алгебраических байесовских сетях и родственных моделях знаний с неопределенностью: логико-вероятностный подход и системы графов>>..}