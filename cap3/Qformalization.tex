\subsubsection{Теорема о построении матрицы перехода}
Рассмотрим фрагменты знаний, построенные над идеалами дизъюнктов. Аналогично случаю для конъюнктов~\cite{51, 9}, построим вектор вероятностей элементов виртуального свидетельства. В виртуальное свидетельство будут входить элементы, стоящие на пересечении двух фрагментов знаний.

Сначала введем матрицу перехода от вектора вероятностей квантов к вектору вероятностей дизъюнктов. Воспользуемся приведенным во второй главе выражением ~\ref{pdtopq} для матрицы перехода $\Ln$ от вектора $\Pd^{\prime}$ к вектору $\Pq$. Так как $\Ln$ имеет обратную матрицу, то домножим выражение для перехода от вектора $\Pd^{\prime}$ к вектору $\Pq$ на обратную матрицу и получим: 
\begin{equation*}
 \Ln^{-1} \Pq  = \mathbf{1} -\Pd,
\end{equation*}
где $\Ln^{-1} = \begin{pmatrix} 1 & 1 \\ 1 & 0 \end{pmatrix}^{[n]}.$

Обозначим $\On = \Ln^{-1}$. Перепишем выражение:
\begin{equation*}\mathbf{1} -\Pd = \On \Pq, \end{equation*} где $\On = \begin{pmatrix} 1 & 1 \\ 1 & 0 \end{pmatrix}^{[n]}.$

Теперь построим матрицу перехода $\V$ от вектора вероятностей  элементов идеала дизъюнктов к вектору вероятностей элементов виртуального свидетельства. Выделим из вектора $\Pd^{\prime 1,a}$, содержащего новые апостериорные оценки, элементы, принадлежащие виртуальному свидетельству.

 От вектора вероятностей квантов можно перейти к вектору вероятностей дизъюнктов с помощью матрицы перехода: $\Pd^{\prime 1,a} = \On\Pq^{1,a}$. При умножении $i$-той строки $\On[i]$ на $\Pq^{1,a}$, получается $i$-тый элемент вектора $\Pd^{\prime 1,a}$. Если выделить в отдельную матрицу строки $\On[i]$, выделяющие из $\Pd^{\prime 1,a}$ элементы виртуального свидетельства, то при умножении эту матрицы на $\Pq^{1,a}$, получился нужный нам вектор оценок вероятностей.

Для того чтобы выделить строки $\On[i]$, нужно домножить $\On$ слева на матрицу $\V$ размерности $m \times n$~($m$ --- длина вектора $\Pd^{\prime \, \ev}$,  $n$ --- длина вектора $\Pd^{\prime 1,a}$). Элементы матрицы определим по следующему правилу:
\begin{equation*}
    \V[i,j] = 
    \begin{cases}
        1 \text{, если $\Pd^{\prime 1,a}[j] = \Pd^{\prime \, \ev}[i]$,}\\
        0 \text{, иначе;}
    \end{cases}
\end{equation*}

При этом под $\Pd^{\prime 1,a}[i]$ и $\Pd^{\prime \, \ev}[i]$ подразумеваются сами дизъюнкты, а не значения вероятностей. Затем нужно умножить матрицу на $\Pq^{1,a}$. Таким образом, $\Pd^{\prime \, \ev} = \V\On\Pq^{1,a} = \V\Pd^{\prime 1,a}$.

Алфавит, над которым построено свидетельство, можно найти как $\mathcal{A}_{\ev} = \mathcal{A}_1 \cap \mathcal{A}_2$, где $\mathcal{A}_1$ --- алфавит первого фрагмента знаний, $\mathcal{A}_2$ --- алфавит второго фрагмента знаний.

Теперь покажем, что матрицу перехода $\V$ можно построить через кронекерово произведение, и обоснуем предложенное построение.

\begin{theorem}\label{disq}
Вектор оценок виртуального свидетельства $\Pd^{\prime \ev}$ можно вычислить как $\Pd^{\prime \ev} = \V \Pd^{\prime a,1}$, где $\V = \VijTilda_{n-1} \otimes \VijTilda_{n-2} \otimes ... \otimes \VijTilda_{0}$ ,
 $\VijTilda_k = \begin{cases}
\V^+ \text{, если $x_k$ входит в $\mathcal{A}^{\ev}$,} \\
\V^- \text{, иначе;}
\end{cases}$, 
\begin{math}
    \V^+ = \begin{pmatrix} 1 & 0 \\ 0 & 1 \end{pmatrix},
    \V^- = \begin{pmatrix} 1 & 0 \end{pmatrix},
\end{math}\\
$x_k$ --- $k$-тый элемент в $\mathcal{A}_1$,  $\mathcal{A}_1$ --- алфавит, над которым построен $\Pd^{\prime 1,a}$, $\mathcal{A}^{\ev}$ --- алфавит, над которым построен $\Pd^{\prime \ev}$.
\end{theorem}
\begin{Proof}
Для упрощения доказательства дополним матрицу $\V^-$ строкой из нулей и будем рассматривать построение матрицы $\V$ через матрицы $\V^+$ и $\V^{\prime-}$, где $\V^{\prime-} = \begin{pmatrix} 1 & 0 \\ 0 & 0 \end{pmatrix}$. Тогда искомая матрица будет квадратной и диагональной по построению и будет содержать строки из нулей. 

Также можно заметить, что $i$-тая единица на диагонали означает, что оценка на $i$-том месте в векторе вероятностей входит в виртуальное свидетельство. Ноль же означает, что оценка не относится к вектору виртуального свидетельства. 

Обоснуем, что позиции нулей и единиц при таком построении действительно показывают, входит ли оценка в виртуальное свидетельство.

Рассмотрим на каких позициях диагонали матрицы $\V$ будут нули.
По построению каждое произведение Кронекера увеличивает размерность искомой матрицы вдвое.
Можно заметить, что нули могут появиться только при участии в произведении матрицы $\V^{\prime-}$, в которой есть единственный $0$, находящийся на диагонали.

Получается, что по построению матрица $\V^{\prime-}$, соответствующая условию $x_m \not \in \mathcal{A}^{\ev}$, в
произведении Кронекера даст нули на всех позициях $\V[i,i]$, где
$i\&2^{m+1} = 2^{m+1}$, где $\&$ обозначает операцию побитового И. Множество таких чисел соответствует множеству чисел, в двоичной записи которых на месте $m$ стоит единица~(Нумерация мест в двоичной записи и нумерация атомов в алфавите начинается с 0). Сравнив с правилом нумерации дизъюнктов, получим, что все такие числа соответствуют номерам дизъюнктов, в которых присутствует атом с номером $m$. 

Следовательно, участие матриц $\V^{\prime-}$ в кронекеровом произведении занулит вероятности в результирующем векторе $\Pd^{\ev}$ всех дизъюнктов $d_m$, для которых не выполняется условие $\forall x_k \in d_m \, x_k \in \mathcal{A}_{\ev}$. А умножение $\Pd^{1,a}$ на столбец матрицы с единицей на диагонали как раз выделит в результирующий вектор нужную оценку вероятности.

Осталось разобраться с размерностью результирующего вектора $\Pd^{\ev}$. Он получается одинаковой размерности с исходным вектором $\Pd^{ 1,a}$ и содержит в себе нулевые элементы, возникающие, как раз, из-за видоизменения матрицы $\V^-$. Тогда при замене матрицы  $\V^{\prime-}$ на $\V^-$ получим вектор $\Pd^{\ev}$ нужной размерности. Так как $\Pd^{\prime}$ линейно выражается через $\Pd$, то результат будет корректным и для векторов $\Pd^{\prime 1,a}$  и $\Pd^{\prime \ev}$.
\end{Proof}

\subsubsection{Пример построения матрицы перехода}
 Рассмотрим на примере формирование матрицы $\V$ с помощью предложенного матрично-векторного алгоритма. Пусть даны два фрагмента знаний над алфавитами $\mathcal{A}_1 = \{x_1, x_2, x_3\}$ и $\mathcal{A}_2 = \{x_1, x_3, x_4\}$. Будем считать, что первый фрагмент знаний содержит апостериорные оценки после пропагации свидетельства и нужно пропагировать свидетельство дальше из первого ФЗ во второй. 
Выделим элементы, относящиеся к виртуальному свидетельству, построив матрицу перехода $\V$:
\begin{equation*}
\Pd^{\prime 1} =  \begin{pmatrix}
1 \\ p(\overline{x}_1) \\ p(\overline{x}_2) \\ p(\overline{x}_2\overline{x}_1) \\ p(\overline{x}_3) \\ p(\overline{x}_3\overline{x}_1) \\ p(\overline{x}_3\overline{x}_2) \\ p(\overline{x}_3\overline{x}_2\overline{x}_1)
\end{pmatrix}, 
\Pd^{\prime 2} =  \begin{pmatrix}
1 \\ p(\overline{x}_1) \\ p(\overline{x}_3) \\ p(\overline{x}_3\overline{x}_1) \\ p(\overline{x}_4) \\ p(\overline{x}_4\overline{x}_1) \\ p(\overline{x}_4\overline{x}_3) \\ p(\overline{x}_4\overline{x}_3\overline{x}_1)
\end{pmatrix},
\Pd^{\prime \ev} =  \begin{pmatrix}
1 \\ p(\overline{x}_1) \\ p(\overline{x}_3) \\ p(\overline{x}_3\overline{x}_1)
\end{pmatrix},
\end{equation*}
$\mathcal{A}_{\ev} = \mathcal{A}_1 \cap \mathcal{A}_2 = \{x_1, x_3\}$.
 
 Тогда $\V = \V^+ \otimes \V^- \otimes \V^+ = 
 \begin{pmatrix} 1 &0 \\ 0 &1 \end{pmatrix} \otimes
  \begin{pmatrix} 1 &0  \end{pmatrix} \otimes
 \begin{pmatrix} 1 &0 \\ 0 & 1 \end{pmatrix} 
 = \\
 \begin{pmatrix} 1 &  0 & 0 & 0 & 0 &  0 & 0 &  0 & 0 \\
                 0 & 1 & 0 & 0 & 0 & 0 & 0 & 0 & 0 \\
                 0 & 0 & 0 & 0 & 1 & 0 & 0 & 0 & 0 \\
                 0 & 0 & 0 & 0 & 0 & 1 & 0 & 0 & 0 \end{pmatrix} $.
                 
                 
Видно, что единицы на $i,j$-тых местах соответствуют выполнению равенства $\Pd^{\prime a,1} [i]= \Pd^{\prime \ev }[j]$, то есть таким построением была получена действительно матрица перехода от вектора $\Pd^{\prime a,1} $ к вектору $\Pd^{\prime \ev }$.
