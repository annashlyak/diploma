\subsection*{Пример использования уравнения для пропагации виртуального свидетельства над множеством квантов}
 \addcontentsline{toc}{subsection}{Пример использования уравнения для пропагации виртуального свидетельства над множеством квантов}

Пусть даны два фрагмента знаний над алфавитами $\mathcal{A}_1 = \{x_1, x_2\}$ и $\mathcal{A}_2 = \{x_1, x_3\}$ соответственно. Оба ФЗ построены над множествами квантов. Пусть в первый фрагмент знаний поступило стохастическое свидетельство. Пересчитаем оценки в данном ФЗ с учетом поступившего свидетельства и распространим далее его влияние  из первого ФЗ во второй:
\begin{equation*}
\Pq^{ 1} =  \begin{pmatrix}
p(\overline {x}_2\overline {x}_1)\\ p(\overline {x}_2x_1) \\ p(x_2\overline {x}_1) \\ p(x_2x_1)
\end{pmatrix} = \begin{pmatrix}
0.1 \\ 0.2\\ 0.1\\ 0.6
\end{pmatrix},
\Pq^{2} =  \begin{pmatrix}
p(\overline{x}_3\overline{x}_1) \\ p(\overline{x}_3x_1)\\ p(x_3\overline{x}_1) \\ p(x_3x_1)
\end{pmatrix} = \begin{pmatrix}
0.2 \\ 0.3\\ 0.4\\ 0.1 
\end{pmatrix}.
\end{equation*}

Пусть поступившее свидетельство будет над алфавитом
$\mathcal{A}_{\ev} = \{x_2\}$ со следующим вектором оценок:
\begin{equation*}
\Pq^{\ev} = \begin{pmatrix}  p(\overline {x}_2) \\ p(x_2)	\end{pmatrix} = \begin{pmatrix}
0.4 \\ 0.6
\end{pmatrix}.
\end{equation*}

Воспользуемся уравнением \ref{quantsG} и найдем апостериорные оценки элементов первого ФЗ: $\G = \begin{pmatrix} 1 \\ 0 \end{pmatrix}.$

$\chi^0 = \begin{pmatrix} 0 \end{pmatrix} $, 
$\G\chi^0  = \begin{pmatrix} 1 \\ 0  \end{pmatrix}\begin{pmatrix} 0 \end{pmatrix}  = \begin{pmatrix}  0 \\ 0 \end{pmatrix}$,
$\chi^1 = \begin{pmatrix} 1 \end{pmatrix} $, 
$\G \chi^1  =\begin{pmatrix} 1 \\ 0 \end{pmatrix} \begin{pmatrix} 1 \end{pmatrix}  = \begin{pmatrix}  1 \\ 0 \end{pmatrix}$,
\begin{equation*}
\Selector^{\langle 00, 10 \rangle} = \mathbf{s}^- \otimes \mathbf{s}^\circ =  \begin{pmatrix} 1 \\ 0 \end{pmatrix} \otimes \begin{pmatrix}  1 \\ 1 \end{pmatrix} = \begin{pmatrix}  1 \\ 1 \\ 0 \\ 0 \end{pmatrix},
\end{equation*}
\begin{equation*}
\mathbf{s}^{\langle 10, 00 \rangle} = \mathbf{s}^+ \otimes \mathbf{s}^\circ = 
\begin{pmatrix} 0 \\ 1 \end{pmatrix} \otimes \begin{pmatrix}  1 \\ 1 \end{pmatrix} = \begin{pmatrix}  0 \\ 0 \\ 1 \\ 1 \end{pmatrix}.
\end{equation*}

Подставим в уравнение \ref{quantsG}, вычислим и получим, что $\Pq^{1,a} = \begin{pmatrix}
0.2 \\ 0.4 \\  \frac{2}{35}\\ \frac{12}{35}
\end{pmatrix}$.

Найдем вектор виртуального свидетельства. Алфавит, над которым построено свидетельство $\mathcal{A}_{\ev} = \mathcal{A}_1 \cap \mathcal{A}_2 = \{x_1, x_2\} \cap \{x_1, x_3\} = \{x_1\}$.

Вычислим векторы-селекторы и матрицы проекции $\G_{\mathcal{A}_1, \mathcal{A}_{\ev}}$ и $\G_{\mathcal{A}_2, \mathcal{A}_{\ev}}$. Затем для пропагации виртуального свидетельства воспользуемся уравнением \ref{quantsGlob}: $\G_{\mathcal{A}_1, \mathcal{A}_{\ev}} = \begin{pmatrix} 0 \\ 1\end{pmatrix}.$

$\chi^0 = \begin{pmatrix} 0 \end{pmatrix} $, 
$ \G_{\mathcal{A}_1, \mathcal{A}_{\ev}} \chi^0 =\begin{pmatrix} 0 \\ 1 \end{pmatrix}  \begin{pmatrix} 0 \end{pmatrix} = \begin{pmatrix}  0 \\ 0 \end{pmatrix}$,
$\chi^1 = \begin{pmatrix} 1 \end{pmatrix} $, 
$ \G_{\mathcal{A}_1, \mathcal{A}_{\ev}} \chi^1 = \begin{pmatrix} 0 \\ 1 \end{pmatrix} \begin{pmatrix} 1 \end{pmatrix}  = \begin{pmatrix}  0 \\ 1 \end{pmatrix}$,
\begin{equation*}
\mathbf{s}_1^{\langle 00, 01 \rangle} = \mathbf{s}^\circ \otimes \mathbf{s}^- = \begin{pmatrix} 1 \\ 1 \end{pmatrix} \otimes \begin{pmatrix}  1 \\ 0 \end{pmatrix} = \begin{pmatrix}  1 \\ 0 \\ 1 \\ 0 \end{pmatrix},
\end{equation*}
\begin{equation*}
\mathbf{s}_1^{\langle 01, 00 \rangle} = \mathbf{s}^\circ \otimes \mathbf{s}^+ =
\begin{pmatrix} 1 \\ 1 \end{pmatrix} \otimes \begin{pmatrix}  0 \\ 1 \end{pmatrix} = \begin{pmatrix}  0 \\ 1 \\ 0 \\ 1 \end{pmatrix},
\end{equation*}
\begin{equation*}
\G_{\mathcal{A}_2, \mathcal{A}_{\ev}} = \begin{pmatrix} 0 \\ 1\end{pmatrix},
\end{equation*}

$\chi^0 = \begin{pmatrix} 0 \end{pmatrix} $, 
$ \G_{\mathcal{A}_2, \mathcal{A}_{\ev}} \chi^0 = \begin{pmatrix} 0 \\ 1 \end{pmatrix} \begin{pmatrix} 0 \end{pmatrix}  = \begin{pmatrix}  0 \\ 0 \end{pmatrix}$,
$\chi^1 = \begin{pmatrix} 1 \end{pmatrix} $, 
$\G_{\mathcal{A}_2, \mathcal{A}_{\ev}} \chi^1  = \begin{pmatrix} 0 \\ 1 \end{pmatrix} \begin{pmatrix} 1 \end{pmatrix}  = \begin{pmatrix}  0 \\ 1 \end{pmatrix}$,
\begin{equation*}
\mathbf{s}_2^{\langle 00, 01 \rangle} = \mathbf{s}^- \otimes \mathbf{s}^\circ = \begin{pmatrix} 1 \\ 1 \end{pmatrix} \otimes \begin{pmatrix}  1 \\ 0 \end{pmatrix} = \begin{pmatrix}  1 \\ 0 \\ 1 \\ 0 \end{pmatrix},
\end{equation*}
\begin{equation*}
\mathbf{s}_2^{\langle 01, 00 \rangle} = \mathbf{s}^+ \otimes \mathbf{s}^\circ =
\begin{pmatrix} 1 \\ 1 \end{pmatrix} \otimes \begin{pmatrix}  0 \\ 1 \end{pmatrix} = \begin{pmatrix}  0 \\ 1 \\ 0 \\ 1 \end{pmatrix}.
\end{equation*}


Подставим в формулу \ref{quantsGlob} и получим, что $\Pq^{ 2,a} = \begin{pmatrix}
\frac{3}{35}  \\ \frac{39}{70} \\ \frac{6}{35} \\ \frac{13}{70}
\end{pmatrix}$.
