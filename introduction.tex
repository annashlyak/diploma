\underline{Актуальность темы.} В современном мире существует необходимость анализировать и обрабатывать большие объемы информации, однако данных не всегда бывает достаточно и могут появляться различные неопределенности, что значительно осложняет задачу обработки. Подход к обработке знаний с неопределенностью предлагает класс вероятностных графических моделей~(ВГМ), к которым относятся алгебраические байесовские сети~(АБС), байесовские сети доверия~(БСД), марковские сети и многие другие. Зависимости между данными в ВГМ задаются с помощью графов, а степень неопределенности данных характеризуется оценками вероятностей. ВГМ находят широкое применение в различных областях, например, байесовские сети доверия, родственные классу АБС, рассматриваемому в данной работе, используются в медицине~\cite{99}, оценке рисков~\cite{98}, области финансов~\cite{97} и других областях~\cite{96}. 

Алгебраические байесовские сети введены профессором В. И. Городецким~\cite{95, 94} и, благодаря многолетним исследованиям, активно развиваются и также занимают свое место в классе вероятностных графических моделей. АБС состоит из набора случайных элементов, соответствующих высказываниям, которым приписаны оценки вероятностей, и структуры связи между указанными высказываниями, которую представляют в виде графа. 

В АБС используется принцип декомпозиции знаний на фрагменты знаний~(ФЗ). Формирование оценок истинности и обработка данных с неопределенностью в АБС основана на локальном и глобальном логико"=вероятностном выводе~\cite{93}. Новая информация, поступающая во фрагменты знаний, основана на логико-вероятностной модели свидетельств~\cite{109}. Для представления фрагмента знаний в АБС описаны три математические модели фрагмента знаний, построенные над идеалами конъюнктов, идеалами дизъюнктов или множествами пропозиций-квантов~\cite{121}. 
 
 \underline{Степень разработанности темы исследований.} 
 Актуальной задачей является матрично-векторное описание алгоритмов логико"=вероятностного вывода в АБС, потому что матрично"=векторные операции относятся к классическому математическому инструментарию, упрощают программную реализацию и понимание работы алгоритмов, а также предоставляют возможности для дальнейшего исследования математической модели, например, исследование чувствительности уравнений. На сегодняшний день предложены матрично"=векторные уравнения для алгоритмов локального логико"=вероятностного вывода~\cite{91}, однако в них остаются элементы функционального вычисления~(функция $\Gind$). Матрично"=векторный подход для глобального вывода также нуждается в доработке~\cite{70}. Параллельно развитию теории разрабатывается комплекс программ для работы с АБС~\cite{89}, реализующий в себе  алгоритмы ЛВВ, однако, не включающий в себя на данный момент алгоритмы глобального логико"=вероятностного вывода, в частности, в нем отсутствует реализация алгоритма распространения виртуального свидетельства между двумя фрагментами знаний.

Таким образом, \underline{объектом исследования} являются алгебраические байесовские сети,
а \underline{предметом исследования} --- матрично-векторная формализация алгоритма распространения виртуального свидетельства между двумя фрагментами знаний в АБС.

\underline{Целью} данной выпускной квалификационной работы является автоматизация глобального логико-вероятностного вывода в алгебраических
байесовских сетях, а именно алгоритмов распространения виртуального свидетельства между двумя фрагментами знаний. 
Для достижения поставленной цели решаются следующие \underline{задачи}:
\begin{enumerate}
        \item  Развить матрично"=векторную формализацию алгоритмов распространения виртуального свидетельства для различных моделей фрагментов знаний со скалярными оценками;
        \item Предложить матрично"=векторную формализацию для функции $\Gind$;
        \item Осуществить интеграцию и реинженеринг алгоритмов локального логико"=вероятностного вывода для альтернативных моделей ФЗ в рамках комплекса программ;
    \item Реализовать алгоритмы распространения виртуального свидетельства в рамках комплекса программ;
    \item Провести вычислительные эксперименты по распространению виртуальных свидетельств и написать документацию.
\end{enumerate}

\underline{Научная новизна.} 
В данной выпускной квалификационной работе бакалавра предложена матрично-векторная формализация алгоритмов распространения виртуального свидетельства для ФЗ, построенных над идеалами дизъюнктов и наборами пропозиций-квантов, в частности, сформулирована и доказана теорема о матрично-векторном формировании матрицы перехода от вектора вероятностей элементов идеала дизъюнктов к вектору вероятностей элементов виртуального свидетельства, доказана теорема о пропагации виртуального свидетельства между двумя ФЗ, построенными над множествами квантов, введена матрица перехода от вектора вероятностей дизъюнктов к вектору вероятностей квантов. Также получена матрица проекции $\G$, заменяющая функцию $\Gind$, в частности, введены понятия характеристического вектора конъюнкта, дизъюнкта и кванта. 

\underline{Теоретическая и практическая значимость исследования.} Теоретическая значимость работы заключается в создании базы для развития матрично"=векторного подхода в глобальном ЛВВ. Практическая значимость заключается в создании основы для реализации алгоритмов распространения свидетельства во все ФЗ сети, решения задачи визуализации работы с АБС, а также для проведения вычислительных экспериментов.

\underline{Методы исследования.} Для решения поставленных задач в данной области понадобилось изучить методические материалы, поставить проблему, проанализировать ее, спроектировать несколько возможных вариантов решения. Затем были выбраны подходящие средства и технологии программирования, связанные с языком реализации~(C\#), средой разработки~(Rider), сервисом для совместной разработки~(BitBucket). После чего были проведены вычислительные и программные эксперименты с целью обоснования корректности  полученного решения. В качестве методов решения используются теоретические методы~(анализ предыдущих результатов, формализация алгоритмов) и методы объектно"=ориентированного программирования~(ООП).


\underline{На защиту выносятся следующие положения:} 
\begin{enumerate}
\item Матрично"=векторные уравнения распространения виртуального свидетельства между двумя фрагментами знаний для следующих моделей фрагмента знаний: идеал дизъюнктов, множество пропозиций квантов;
\item Матрично"=векторная формализация для функции $\Gind$;
\item Программная реализация алгоритмов распространения виртуального свидетельства для каждой из трех моделей.
\end{enumerate}

\underline{Достоверность полученных результатов.} 
Достоверность предложенных в работе результатов обеспечена корректным применением методов исследования, подтверждена вычислительными экспериментами и примерами работы программы. Полученные результаты не противоречат известным результатам других авторов.

\underline{Апробация результатов.} Результаты исследования докладывались на следующих научных мероприятиях:
\begin{enumerate}
\item Всероссийская научная конференция по проблемам информатики
СПИСОК~(Санкт-Петербург, апрель 2017);
\item VII-Всероссийская научно"=практическая конференция НСМВИТ-2017~(Санкт"=Петербург, июль 2017);
\item Научный семинар лаборатории теоретических и междисциплинарных проблем информатики в СПИИРАН~(Санкт-Петербург, ноябрь 2017).
\end{enumerate}

\underline{Публикации.} Результаты работы вошли в две публикации, была зарегистрирована одна заявка в Роспатенте.

\underline{Сведения о личном вкладе автора.}
Постановка цели и задач, а также выносимые на защиту результаты получены лично автором. 

Личный вклад А.В. Шляк в публикациях с соавторами характеризуется следующим образом: в ~\cite{9} --- краткое описание существующих алгоритмов и примеры вычисления апостериорных вероятностей после пропагации виртуального свидетельства, в ~\cite{70} --- постановка задач формализации глобального ЛВВ.

\underline{Структура и объем работы.} Работа состоит из введения, четырех
глав, заключения, списка используемой литературы и двух приложений. Общий объем работы составляет \pageref{LastPage} страниц. Список используемой литературы содержит 35 источников.

Во введении описана актуальность темы исследования и степень ее разработанности, сформулированы цель и задачи работы и представлены выносимые на защиту результаты.

Первая глава носит обзорный характер и состоит из пяти разделов. В первом разделе описаны общие сведения об АБС. Во втором разделе представлены основные инструменты для работы с АБС. В третьем разделе дан обзор существующих библиотек. В четвертом разделе представлен краткий обзор комплекса программ, в рамках которого осуществлялась разработка, а также приведены и обоснованы  цели и задачи работы. В пятом разделе представлены выводы по данной главе.

Вторая глава описывает основные теоретические результаты, используемые в данной работе. В первом разделе представляются основные элементы теории АБС. Во втором разделе даются основные определения. В третьем разделе рассматриваются виды локального логико"=вероятностного вывода, а в четвертом разделе --- глобальный логико"=вероятностный вывод. В пятом разделе перечислены недостатки имеющихся алгоритмов.

Третья глава содержит основные теоретические результаты, полученные в данной работе. В первом разделе дается краткий обзор полученных результатов. Во втором разделе предложена матрично"=векторная интерпретация функции $\Gind$, а в третьем разделе --- матрично"=векторная формализация матрицы перехода. В четвертом разделе представлены матрично-векторные уравнения распространения виртуального свидетельства для каждой из моделей ФЗ. В пятом разделе описаны выводы по главе.

Четвертая глава описывает программную реализацию алгоритмов из второй и третьей глав. Первый раздел описывает используемые программные технологии. Второй раздел описывает архитектуру комплекса программ. В третьем разделе описаны основные реализованные классы и методы, а в четвертом разделе приведены примеры их использования. В пятом разделе представлены выводы по текущей главе. 

В заключении содержатся итоги работы и дальнейшие перспективы исследования.

В приложении А содержатся вычислительные эксперименты по работе предложенных алгоритмов, а в приложении В --- список публикаций по теме работы.

\small{
Эта работа является частью более широких инициативных проектов, выполняющихся в лаборатории теоретических и междисциплинарных проблем информатики СПИИРАН под руководством А.Л.~Тулупьева; кроме того, разработки были частично поддержаны грантами РФФИ 15-01-09001-a~--- «Комбинированный логико"=вероятностный графический подход к представлению и обработке систем знаний с неопределенностью: алгебраические байесовские сети и родственные модели», 18-01-00626~--- «Методы представления, синтеза оценок истинности и машинного обучения в алгебраических байесовских сетях и родственных моделях знаний с неопределенностью: логико-вероятностный подход и системы графов».}

\normalsize
Работа выполнялась в рамках общей проектной работы вместе с А.А. Золотиным, А.Е. Мальчевской, А.И. Березиным.