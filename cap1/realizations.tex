Развитием теории АБС активно занимается научный коллектив ТиМПИ СПИИРАН, в разное время включавший в себя А.Л. Тулупьева, А.В. Сироткина, А.А. Фильченкова и других исследователей. Были опубликованы монографии ~\cite{109, 1, 85}, диссертации~\cite{84, 184, 284}, статьи, посвященные тематике АБС и байесовских сетей в целом~\cite{94}.

Вместе с теорией создавались также библиотеки для работы с АБС, так в 2009 году была создана библиотека AlgBN Modeller j.v.01 на языке Java~\cite{124, 123, 122}. В данной библиотеке есть структуры для хранения фрагментов знаний, машины локального логико-вероятностного вывода и проверки непротиворечивости, а также машины глобального логико-вероятностного вывода. Однако библиотека позволяет осуществить глобальный логико-вероятностный вывод только для АБС, состоящей из ФЗ, построенных над идеалами конъюнктов с заданными оценками вероятностей их элементов.

Позже была создана библиотека AlgBN KPB Reconciler cpp.v.01, разработанная на языке C++~\cite{81}. В этой библиотеке также есть средства для осуществления локального логико"=вероятностного вывода и проверки непротиворечивости, а также структуры для хранения фрагментов знаний, однако алгоритмы глобального логико-вероятностного вывода в ней не реализованы.

Таким образом, в обеих библиотеках отсутствует полноценный функционал для проведения глобального логико-вероятностного вывода в АБС.